% makra używane w projekcie

%głupie wyjście
\newif\ifdumb

%inna nazwa rozdziałów
\makeatletter
\renewcommand{\@chapapp}{Opowiadanie}
\makeatother

%wstęp do rozdziału
\newcommand{\info}[4]
{
	\textsl{#1}
	\begin{description}
		\item[Gatunek:] #2
		\item[Długość:] #3 znaków
		\item[Uniwersum:] #4
	\end{description}
	\bigskip
}

%nieokreślone uniwersum
\newcommand{\undefineduniversum}{\emph{Nieokreślone}}
\newcommand{\quadreversumuniversum}{Quadreversum}

%odpowiednio sformatowane myślniki do dialogów.
\newcommand{\ds}{\par ---}
\newcommand{\dm}{---}
\newcommand{\de}{}
\newcommand{\dialog}[3]{
	\par --- #1
	\def\temp{#2}\ifx\temp\empty\else --- #2 \fi
	\def\temp{#3}\ifx\temp\empty\else --- #3 \fi
	\par
}

%lista dialogowa
\newenvironment{dialogue}{}{\par}
 
%podzielenie tekstu
\newcommand{\divider}
{
	\begin{center}
		\ifdumb
			***
		\else
			\FourStarOpen
		\fi
	\end{center}
}

%znacznik
\newcommand{\marker}
{
	\begin{center}
		--- ! ! ! ---
	\end{center}
}

%blok cenzury
\newcommand{\censor}
{
	\ifdumb
		[CENZURA]
	\else
		\textblock\textblock\textblock\textblock
	\fi
}

%didaskalia
\newcommand{\dida}[1]{\textit{#1}}

%podtytuł
\newcommand{\smalltitle}[1]{\textbf{\textsc{#1\\}}}

%wypowiedzi bohaterów
\newcommand{\chardok}{DOKTOR EVILION\\}
\newcommand{\charszam}{DOKTOR SZAMBONURKOLOGII JANUSZ EVILION\\}
\newcommand{\charmik}{MIKOŁAJ\\}
\newcommand{\charprzy}{PRZYDUPAS\\}
\newcommand{\chardoc}{DOCENT PRZYDUPAS\\}
\newcommand{\charnad}{NADAR\\}
\newcommand{\charfer}{FERRO\\}
\newcommand{\charkap}{KAPITAN\\}
\newcommand{\charkos}{KOSMONAUTA\\}

%tekst maszyny
\newcommand{\machine}[1]{\texttt{#1}}

%ilustracja
\newcommand{\illustration}[1]
{
	\begin{center}
		\includegraphics[width=\textwidth]{#1}\bigskip
	\end{center}
}


%dziwny znak
\newcommand{\weirdchar}[1]
{
	\ifdumb
		\texttt{[NIEZROZUMIAŁY ZNAK]}
	\else
		\includegraphics[height=0.9em]{chars/#1.pdf}
	\fi
}

%inny język
\newcommand{\differentlan}[1]{\emph{#1}}

%miła ramka
\newcommand{\letterframe}[1]
{
	\ifdumb
		\begin{em}
			#1
		\end{em}
	\else
		\curlyframe{
		\begin{Fontlukas}
			#1
		\end{Fontlukas}}
	\fi
}

%elegancka ramka
\newcommand{\elegantframe}[1]
{
	\ifdumb
		\begin{em}
			#1
		\end{em}
	\else
		\niceframe{
		\begin{Fontlukas}
			#1
		\end{Fontlukas}}
	\fi
}

%wewnętrzny tekst opowiadania, czy wiersza
\newenvironment{poem}
{
	\begin{sl}
	\begin{quote}
}{
	\end{quote}
	\end{sl}
}

%duża litera na początku
\newcommand{\begincapital}[1]{\lettrine[loversize=0.1, lhang=0.3, lines=3]{#1}{}}
