% makra używane w projekcie

%wstęp do rozdziału
\newcommand{\info}[1]{\textsl{#1}\bigskip}

%odpowiednio sformatowane myślniki do dialogów.
\newcommand{\ds}{\par ---}
\newcommand{\dm}{---}
\newcommand{\de}{}

%lista dialogowa
\newenvironment{dialogue}{}{\par}
 
%podzielenie tekstu
\newcommand{\divider}{
\begin{center}
	\FourStarOpen
\end{center}
}

%znacznik
\newcommand{\marker}{
\begin{center}
	--- ! ! ! ---
\end{center}
}

%blok cenzury
\newcommand{\censor}{\textblock\textblock\textblock\textblock}

%didaskalia
\newcommand{\dida}[1]{\textit{#1}}

%podtytuł
\newcommand{\smalltitle}[1]{\textbf{\textsc{#1\\}}}

%wypowiedzi bohaterów
\newcommand{\chardok}{DOKTOR EVILION\\}
\newcommand{\charszam}{DOKTOR SZAMBONURKOLOGII JANUSZ EVILION\\}
\newcommand{\charmik}{MIKOŁAJ\\}
\newcommand{\charprzy}{PRZYDUPAS\\}
\newcommand{\chardoc}{DOCENT PRZYDUPAS\\}
\newcommand{\charnad}{NADAR\\}
\newcommand{\charfer}{FERRO\\}

%tekst maszyny
\newcommand{\machine}[1]{\textbf{\texttt{#1}}}

%ilustracja
\newcommand{\illustration}[1]{\includegraphics[width=\textwidth]{#1}\bigskip}

%dziwny znak
\newcommand{\weirdchar}[1]{\includegraphics[height=0.9em]{chars/#1.pdf}}

%inny język
\newcommand{\differentlan}[1]{\emph{#1}}
