% makra używane w projekcie

% Głupie wyjście
\newif\ifdumb
%\dumbtrue

%inna nazwa rozdziałów
\makeatletter
\renewcommand{\@chapapp}{Opowiadanie}
\makeatother

%wstęp do rozdziału
\newcommand{\info}[1]{\textsl{#1}\bigskip}

%odpowiednio sformatowane myślniki do dialogów.
\newcommand{\ds}{\par ---}
\newcommand{\dm}{---}
\newcommand{\de}{}

%lista dialogowa
\newenvironment{dialogue}{}{\par}
 
%podzielenie tekstu
\newcommand{\divider}{
\begin{center}
\ifdumb
***
\else
	\FourStarOpen
\fi
\end{center}
}

%znacznik
\newcommand{\marker}{
\begin{center}
	--- ! ! ! ---
\end{center}
}

%blok cenzury
\newcommand{\censor}{
\ifdumb
	[CENZURA]
\else
	\textblock\textblock\textblock\textblock
\fi
}

%didaskalia
\newcommand{\dida}[1]{\textit{#1}}

%podtytuł
\newcommand{\smalltitle}[1]{\textbf{\textsc{#1\\}}}

%wypowiedzi bohaterów
\newcommand{\chardok}{DOKTOR EVILION\\}
\newcommand{\charszam}{DOKTOR SZAMBONURKOLOGII JANUSZ EVILION\\}
\newcommand{\charmik}{MIKOŁAJ\\}
\newcommand{\charprzy}{PRZYDUPAS\\}
\newcommand{\chardoc}{DOCENT PRZYDUPAS\\}
\newcommand{\charnad}{NADAR\\}
\newcommand{\charfer}{FERRO\\}

%tekst maszyny
\newcommand{\machine}[1]{\textbf{\texttt{#1}}}

%ilustracja
\newcommand{\illustration}[1]{\begin{center}
                              	\includegraphics[width=\textwidth]{#1}\bigskip
                              \end{center}}


%dziwny znak
\newcommand{\weirdchar}[1]{
\ifdumb
	\texttt{[NIEZROZUMIAŁY ZNAK]}
\else
	\includegraphics[height=0.9em]{chars/#1.pdf}
\fi
}

%inny język
\newcommand{\differentlan}[1]{\emph{#1}}

%miła ramka
\newcommand{\letterframe}[1]{
\ifdumb
	\begin{em}
	#1
	\end{em}
\else
	\curlyframe{
	\begin{Fontlukas}
	#1
	\end{Fontlukas}}
\fi
}

%elegancka ramka
\newcommand{\elegantframe}[1]{
\ifdumb
	\begin{em}
	#1
	\end{em}
\else
	\niceframe{
	\begin{Fontlukas}
	#1
	\end{Fontlukas}}
\fi
}
