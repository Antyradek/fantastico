\chapter{Bitwa na klawiatury} 
\dida{Ta komedia została znaleziona w Głównej Bibliotece w czasie jej digitalizacji z sześciokątnych płytek pamięci do komputera.
Nie jest dokładnie wiadome, kto ją napisał, a i nikt się do tego nie przyznaje. Może to i lepiej.}

\smalltitle{Bitwa na klawiatury}
Postaci:
\begin{itemize}
	\item Doktor Evilion --- człowiek.
	\item Przydupas --- dziwna hybryda człowieka z czymśtam, stworzona przez Eviliona.
	\item Mikołaj --- potwór.
	\item Nadar --- człowiek.
	\item Ferro --- potwór.
\end{itemize}

\dida{Poniższy scenariusz powinien być zagrany w naszym jedynym teatrze na Księżycu Radości. 
Dla realizmu, powinno się zamiast aktorów użyć rzeczywistych postaci przedstawionych w skrypcie (z wyjątkiem Ferra, gdyż jest beznadziejnym aktorem i nie zagrałby dobrze nawet samego siebie).}

\dida{Akcja rozgrywa się na Potworanie, w najbardziej strzeżonej części zbrojowni, przy maszynie do nadpisywania zasad wszechświata (znaną też jako Maszyna Życzeń).
Na scenie znajdują się dwie konsole od sterowania maszyną, na podwyższeniach, zwrócone przodem do siebie.
Na scenę z jednej strony wchodzi zły Doktor Evilion i jego pomocnik Przydupas. Zaraz potem wbiega Mikołaj i pozostali od drugiej strony.}

\chardok{}
W końcu udało się! Dostałem się!\\
Maszyna Życzeń, jaka wielka ona jest. A tu konsola, zaraz sterować nią będę.\\

\charmik{}
Już po nas. To koniec wszystkiego.\\
Evilion dostał się do sterowania, zrobi z nami cokolwiek chce jego ego.\\

\chardok{}
Przetestujmy ją, możemy życzyć sobie wszystkiego bez liku.\\
Wpisuję w klawiaturę.\\
\texttt{Chcę loda na patyku.}\\

\charprzy{}
Nic się nie stało, mój panie.\\
Czy na pewno poprawnie podłączone sterowanie?\\

\charmik{}
Wszystko poprawnie, wszytko tak ma być.\\
Inaczej nie mielibyśmy już wszyscy żyć.\\
Nikt wam nie powiedział, jak to działa.\\
Jak nadać poprawne sterowania.\\

\charprzy{}
Chyba Maszyna Życzeń to nazwa zła.\\
To nie tak jest, że ona cokolwiek spełnia.\\
Jeśli pozwolisz, przejmę sterowanie...\\

\chardok{}
...panie.
Dla ciebie --- "`panie przejmę sterowanie"'.\\

\charmik{}
To może ja spróbuję... wielmożny panie.\\
Specjalnie dla ciebie, lody bez czekania.\\
\dida{Mikołaj swoimi lodowymi mocami stwarza loda w ręce.}\\
Oto i lód, którego chciałeś.\\
Chwileczkę, zaraz jeszcze więcej dostaniesz.\\
I w klawiaturę wpisać ten niefajny tekst.\\
\texttt{W każdej ziemskiej sekundzie nad głową Eviliona pojawia się milion sklonowanych lodów trzymanych przez potwora Mikołaja w ręce.}\\
Hmm. Nic się nie dzieje. Chyba coś nie tak jest.\\

\charnad{}
Mikołaju!\\
Właśnie odczytałem coś na moim komunikatorze. \\
Spowodowaliśmy niemałą katastrofę. O Boże!\\
Evilion, ale jaki?\\
Na Tirandii rządzi przecież jeszcze król taki.\\
To ta planeta, która ciągle na swą gwiazdę spada.\\
Co im wielki silnik daliśmy, żeby im nie była biada.\\
Teraz góra lodów w mig go zamroziła.\\
Siła nośna to teraz już tylko była.\\

\charmik{}
Zobacz, coś narobił Evilionie.\\
Przez ciebie cała planeta teraz zaraz spłonie.\\

\chardok{}
Ach, to na mnie wina zwalona jest, taka?\\
\texttt{Mikołaj teraz umiera na raka.}\\

\charmik{}
Ja człowiekiem nie jestem, jak pewnie odkryłeś.\\
Komórek w ciele do zraczenia nie mam, jak pewnie zauważyłeś.\\
Ludzi do zabicia na Potworanie o tym imieniu też nie znajdziesz.\\
Ale na Ziemi prawdopodobnie prezentów w tym roku pod choinką już nie odnajdziesz.\\

\chardok{}
Lodowy potworze. \\
Mój ty Boże. \\
Mógłbym napisać, że po prostu znikasz. \\
Ale lubię zabijać, gdy z innymi problemami się borykasz.\\
Potrafisz produkować lód?\\
Zaraz będziesz mieć tu gorąca w bród.\\
I cały się nam roztopisz.\\
\texttt{Klawiatura maszyny ma temperaturę tysięcy stopni.}\\

\dida{Mikołaj opiera się o rozżarzoną klawiaturę, jak nigdy nic.}

\charmik{}
Gorąca nie znoszę, ale co to są tysiące stopni temperatury.\\
Byłem zapraszany na herbatę do lawowych smoków z Karkentury.\\
O wiele ciekawsze jest jednak to.\\
Co zrobisz teraz z własną gorącą klawiaturą?\\

\chardok{}
Zapomniałem. O ja głupi znowu.\\
Zmiana zasad wszechświata dotyczy nas obu.\\
Jestem doktorem nauk. Mam jednak asa w rękawie.\\
Przydupasie, zastąp mnie, takie twoje zadanie.\\

\charprzy{}
Ależ jak, panie?\\
Klawiatura płonie.\\
Rozumiem, że mam zginąć, panie.\\
Ja jednak szykuję odmowę.\\

\chardok{}
Głupi eksperymencie. Aby cię wytworzyć.\\
Lata nauki poszły, abyś mógł trochę pożyć.\\
Myślisz, że ja Evilion, jam idiota?\\
Odporny jesteś na ogień, ty niecnoto.\\
Wpisz zaraz, że ważysz całe tony. Zaczyna coś trybić?\\
Skocz na niego, aby do ziemi natychmiast przybić.\\

\dida{Mikołaj lecący za scenę pod ciężarem Przydupasa.}

\charmik{}
Jestem potworem, podnoszę całe tony.\\
Zaraz się spod niego wygramolę.\\
O nie, widać wpisał także, że odbiera mi siły.\\
Ferro, zastąp mnie, będę przez pewien czas przybity.\\

\dida{Na scenę wchodzi Ferro.}

\charfer{}
Nie jesteś jedynym, który się pastwi nad ofiarami.\\
Co powiesz na minutę milczenia? Potem pogadamy.\\
\texttt{Kto następny w tym teatrze teraz dźwięk jakiś wyda.\\
Ten majestatycznie eksploduje.} ...enter chyba.\\

\dida{Po kilku sekundach dzwoni telefon podstawionej na widowni osoby. Eksploduje bomba pod jej siedzeniem.}

\charfer{}
Co to było? Przecież ta konsola nie jest podłączona!\\
To ma być tylko teatr! Komedia niedorobiona.\\
Czyli to wszystko, co wpisujemy, to się na prawdę stanie?\\
Co za pojeb pisał to wypracowanie!\\

\chardok{}
W kieszeni miałem ognioodporne rękawice.\\
Wielu tu ludzi na widowni widzę!\\
Na zabawy z wami mam teraz branie.\\
\texttt{Umrze ten, kto z miejsca tutaj wstanie}.\\

\charfer{}
Mówi serio, nie wstawajcie proszę.\\
Zaraz się z nim rozprawię, aż spadną mu kalosze.\\
Patrz na mnie, ważę trzy tony.\\
W porównaniu ze mną jesteś jakiś niedorobiony.\\
Lubisz rodeo? Zaraz polubisz.\\
Przez niekontrolowane wstrząsy konsoli zaraz się zgubisz.\\
Wpisuję kolejny tekst, możliwościami maszyna się chlubi.\\
\texttt{Konsola trzęsie się tak, że Doktor Evilion nie polubi.}\\

\chardok{}
Nic się nie dzieje, wiedziałem o tym.\\
Zdziwiony? Wyjaśnię ci to wszystko potem.\\
A póki co, kolej teraz moja.\\
\texttt{Mechanicznych komarów rzuci się na Ferra cała sfora.}\\

\charfer{}
No to jesteśmy po równo, na mnie też nie działa.\\
Nasz język często robi ludziom w uszach "`ała"'.\\
Dlatego też to tylko ludzie.\\
Wymyślili nam te przydomki, żeby rozmawiać na luzie.\\
Mojego imienia to nawet w pełni usłyszeć byś nie mógł.\\
A ty z kolei nie jesteś doktorem. Oj --- czyżbym twoje ego stłukł?\\

\chardok{}
Nie ma znaczenia, zaraz to naprawię.\\
\texttt{Przede mną leży mój doktorat.} Niech tak się stanie.\\

\charfer{}
Nie powiedziałeś jaki, a w nieokreślonym przypadku.\\
Maszyna używa specjalnej przysadki.\\
Kto ją budował? Mikołaj --- radośnie!\\
Spodziewaj się na papierze jakiejś dziedziny sprośnej.\\
Bardzo ciekawe, co ci wyszło tam?\\
Nie pal tego, chcę wiedzieć! Powiedz proszę nam.\\
A wiesz, że doktoratu zniszczenie.\\
Zasad wszechświata wcale nie zmieni?\\

\charszam{}
Głupia maszyna. Teraz mi ukojenie.\\
Da tylko jej całkowite zniszczenie.\\

\charfer{}
Evilion to nazwisko. To by wyjaśniało.\\
Dlatego też komenda Mikołaja, wcale nie zadziałała.\\
Zasady wszechświata, raz zapisane.\\
Nie zważają na dalsze maszyny poczynanie.\\
Jeśli ją zniszczysz, wszystkie zostaną.\\
Zasada zapisana, pozostanie zapisaną.\\
Siedzieć wszyscy tu będziemy.\\
Aż z głodu pomrzemy.\\

\charszam{}
\texttt{Zasada z konsoli, z konsolą padnie.}\\
Rozwiązałem ten problem całkiem ładnie.\\
Teraz tylko wyłączyć twoją.\\
I kontrola nad światem będzie całkowicie moją.\\
\texttt{Druga konsola rozpada się, jakby dostała batem}.\\
To mi da pełną kontrolę nad wszechświatem.\\

\charfer{}
Myślałeś, że ci się to uda tak po prostu?\\
To moja konsola do pierwszego gniazda podłączona była kabli wiorstą.\\
To znaczy, że twoja część teraz pada, nieprawdaż?\\
Pamiętaj, że z potworami nigdy nie wygrasz.\\
A na zakończenie wpiszę komendę przydatną.\\
Ojej, moja konsola wzięła i się zrobiła rozpadłą.\\

\dida{Przydupas wychodzi zza sceny z kluczem francuskim.}

\charprzy{}
W czasie waszej niepoważnej potyczki.\\
Ja poszedłem na dół i zamieniłem wtyczki.\\

\charszam{}
Zaraz, ale wstałeś z miejsca, jak to możliwe?
Jak w takim razie żyjesz? Powiedz skwapliwie.\\

\charprzy{}
Moje miejsce, nie wiem czy wiecie.\\
To zawsze pomagać panu, gdziekolwiek we wszechświecie.\\
Teraz które zasady padły, a które żyją?\\
Powiem panu na ucho, ruszyć proszę szyją.\\

\charfer{}
Teraz coś wam powiem, że aż wyjdą wam gały.\\
Potwory mają słuch doskonały.\\
Nasz docent powiedział, jedną rzecz od niechcenia.\\
Przyklejone na stałe nie muszą być już wasze siedzenia.\\

\dida{Prawdopodobnie wybucha teraz panika na sali i wszyscy uciekają.}

\charszam{}
Co za panika. Wszyscy w pośpiechu wychodzą.\\
Czy ta sztuka jest aż taką niedorobioną?\\
W takim razie, streszczę się zatem.\\
Aby każdy z was mógł zobaczyć mój triumf nad wszechświatem.\\
Co by tu wpisać, teraz jestem bogiem.\\
Może coś prostego, potem się zastanowię.\\
Władam wszystkim, dobry władca dba o swój lud.\\
Dlatego będę dawał wam wszystkiego w bród.\\
Niech między wszystkimi nami będzie zgoda!\\
Już wiem. \texttt{Na patyku dla każdego loda.}\\












