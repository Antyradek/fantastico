\chapter{Bitwa na klawiatury} 
\info{Niepoważny dramat o całkiem poważnej maszynie do zmieniania zasad wszechświata.}

\illustration{img/keyboards.png}

\dida{Ta komedia została znaleziona w Głównej Bibliotece w czasie jej digitalizacji z sześciokątnych płytek pamięci do komputera.
Nie jest dokładnie wiadome, kto ją napisał, a i nikt się do tego nie przyznaje. Może to i lepiej.}

\smalltitle{Bitwa na klawiatury}
Postaci:
\begin{itemize}
	\item Doktor Evilion --- człowiek.
	\item Przydupas --- dziwna hybryda człowieka z czymśtam, stworzona przez Eviliona.
	\item Mikołaj --- potwór.
	\item Nadar --- człowiek.
	\item Ferro --- potwór.
\end{itemize}
Wymagania:
\begin{itemize}
	\item Puste miejsce w pierwszym rzędzie, dla Eviliona.
	\item Kilka osób o imieniu Mikołaj na widowni.
	\item Podstawiona osoba z telefonem i bombą pod siedzeniem.
\end{itemize}

\dida{Poniższy scenariusz powinien być zagrany w naszym jedynym teatrze na Księżycu Radości. 
Dla realizmu, powinno się zamiast aktorów użyć rzeczywistych postaci przedstawionych w skrypcie 
(z wyjątkiem Ferra, gdyż jest beznadziejnym aktorem i nie zagrałby dobrze nawet samego siebie).}

\dida{Akcja rozgrywa się na Potworanie, w najbardziej strzeżonej części zbrojowni, przy naszej maszynie do nadpisywania zasad wszechświata 
(znanej też jako Maszyna Życzeń, albo Matrycogmeracz).
Na scenie znajdują się dwie konsole do sterowania maszyną, po bokach, na podwyższeniach, zwrócone przodem do siebie.
Na scenę wychodzi wpierw Nadar. Doktor Evilion siedzi w pierwszym rzędzie widowni.}\\

\charnad{}
A oto najciekawszy punkt naszego zwiedzania.\\
Wielka maszyna, do życzeń spełniania.\\
Tutaj widać tylko dwie konsole.\\
Większość maszyny znajduje się pod ziemią, na dole.\\
Modyfikuje zasady wszechświata, wedle życzenia.\\
Czy to wartość $\pi$, czy kolor nieba, to bez znaczenia.\\
Jest tak potężna, że nikt jej nigdy nie dotyka.\\
Jednak nie zniszczymy jej, na wypadek nietypowego ryzyka.\\
Nie ma bardziej strzeżonej rzeczy na tej planecie.\\
A my oprowadzamy wycieczki. Humor autora czujecie?\\
Myślę, że nasz świat za długo by się nie ostał,\\
gdyby teraz jakiś wróg do konsoli się dostał.\\

\dida{Z widowni na scenę wbiega Doktor Evilion i jego pomocnik Przydupas. Zaraz potem z boku wbiega Mikołaj.}\\

\chardok{}
W końcu się udało! Dostałem się!\\
Maszyna Życzeń, teraz należy do mnie!\\
A tu konsola, zaraz zacznę nią sterować.\\
I problemy niespotykane będę generować.\\

\charmik{}
Już po nas. To koniec wszystkiego.\\
Evilion zrobi z nami, cokolwiek chce jego ego.\\

\chardok{}
Przetestujmy ją, najpierw coś małego.\\
Coś bez zbytnich problemów, zbędnego.\\
W klawiaturę te słowa wpisywać bez liku.\\
Otóż --- \machine{chcę loda na patyku.}\\

\charprzy{}
Nic się nie stało, mój panie,\\
czy na pewno poprawnie podłączone sterowanie?\\

\charmik{}
Wszystko poprawnie, wszystko tak ma być.\\
Inaczej moglibyśmy już wszyscy nie żyć.\\
Nikt z was nie wie, jak to wszystko działa.\\
Jak nadać maszynie poprawne sterowania.\\

\charprzy{}
Chyba Maszyna Życzeń to nazwa nieodpowiednia.\\
To nie tak jest, że ona cokolwiek spełnia.\\
Jeśli pozwolisz, przejmę sterowanie...\\

\chardok{}
...panie.
Dla ciebie --- ,,panie przejmę sterowanie''.\\

\dida{Mikołaj swoimi lodowymi mocami stwarza loda w ręce.}\\

\charmik{}
To może ja spróbuję... wielmożny panie.\\
Rękami pomacham, jakbym robił pranie.\\
Poczekaj chwilę, aż będą gotowe.\\
Lodowe lody, w smaku lodowe.\\
Oto i przysmak, którego tak pragniesz.\\
Chwileczkę, zaraz jeszcze więcej dostaniesz.\\
I w klawiaturę wpisać ten rymowany tekst.\\
Rozkazywanie maszynie, toż to sztuką jest.\\
Dzieje się tak, że --- \machine{nad głową Eviliona\\
w każdej sekundzie pojawia się ilość takich lodów miliardowa.}\\
Hmm. Nic się nie dzieje. Co to się stało?\\
Żeby tylko się nie zepsuło, bo problemów mamy niemało.\\

\charnad{}
Mikołaju!\\
Właśnie odczytałem coś na moim komunikatorze. \\
Spowodowaliśmy niemałą katastrofę. O Boże!\\
Wpisałeś Evilion, ale jaki?\\
Na Tirandii rządzi przecież jeszcze król taki.\\
To ta planeta, która ciągle na swą gwiazdę spada,\\
co im wielki silnik daliśmy, żeby im nie była biada.\\
Teraz góra lodów zamroziła go w mig,\\
siła nośna natychmiast zniknęła, o tak --- pyk!\\

\charmik{}
Zobacz, coś narobił Evilionie,\\
przez ciebie cała planeta teraz spłonie.\\

\chardok{}
Ach, to na mnie wina zwalona jest, taka?\\
\machine{Mikołaj właśnie umiera na raka.}\\

\dida{Kilka osób z widowni osuwa się na ziemię.}\\

\charmik{}
Ja człowiekiem nie jestem, jak pewnie odkryłeś.\\
Komórek w ciele do zraczenia nie mam, być może nie zauważyłeś.\\
Ludzi do zabicia na Potworanie, o tym imieniu też nie znajdziesz,\\
ale na Ziemi prawdopodobnie, prezentów pod choinką już nigdy nie odnajdziesz.\\
Hej, czy tamta osoba też właśnie umarła?\\
Zobaczcie, jak prosto maszyna mu życie pożarła.\\

\chardok{}
Lodowy potworze,\\
mój ty Boże.\\
Mógłbym napisać, że po prostu znikasz, \\
ale lubię zabijać, gdy z innymi problemami się borykasz.\\
Poza tym ciężko wpisać tekst, żeby rymowała się każda komenda.\\
Po to, żeby osoba czytająca ten dramat, była jak zaklęta.\\
Zatem potrafisz produkować lód?\\
Zaraz będziesz mieć tu gorąca w bród.\\
I ze strachu i roztopów jeszcze bardziej zbielejesz,\\
\machine{klawiatura jest tak gorąca, że zaraz stopniejesz.}\\

\dida{Mikołaj opiera się o rozżarzoną klawiaturę, jak nigdy nic.}\\

\charmik{}
Gorąca nie znoszę, ale co to są tysiące stopni temperatury.\\
Byłem zapraszany na herbatę, do lawowych smoków z Karkentury.\\
A więc myślisz pewnie, że jesteś nade mną górą?\\
Co zrobisz teraz z własną, gorącą klawiaturą?\\

\chardok{}
Zapomniałem. O ja głupi znowu.\\
Zmiana zasad wszechświata, dotyczy nas obu.\\
Jestem doktorem nauk. W rękawie mam jednak asa.\\
Przydupasie, już, zastąp mnie. Hopsasa.\\

\charprzy{}
Ależ jak, panie? Klawiatura płonie.\\
Rozumiem, że mam kompletnie spalić sobie dłonie?\\
Od zawsze chodzę posłusznie, po pana drodze,\\
a szacunku nigdy dostaję --- zatem odchodzę!\\

\chardok{}
Głupi eksperymencie, aby cię wytworzyć,\\
lata nauki poszły, abyś mógł trochę pożyć.\\
Myślisz, że ja Evilion, jam jest idiotą?\\
Dałem ci odporność na ogień, niecnoto.\\
Wpisz zaraz, że ważysz całe tony, zaczyna coś trybić?\\
Wtedy skocz na niego, aby do ziemi natychmiast przybić.\\

\dida{Mikołaj lecący za scenę pod ciężarem Przydupasa.}\\

\charmik{}
Jestem potworem, podnoszę całe tony,\\
zaraz się wygramolę spod jego ciała zony.\\
O nie, widać wpisał także, że siłę mi odbiera,\\
Ferro, zastąp mnie, pokaż, że potwór nie popiera.\\

\dida{Na scenę wchodzi Ferro.}\\

\charfer{}
Nie jesteś jedynym, który się pastwi nad swą ofiarą,\\
co powiesz na minutę milczenia? Z ust nie puść parą.\\
\machine{Kto następny w tym teatrze nie będzie cichy, jak ryba,\\
ten majestatycznie eksploduje.} ...teraz enter chyba.\\

\dida{Po kilku sekundach dzwoni telefon podstawionej na widowni osoby. Eksploduje bomba pod jej siedzeniem.}\\

\charfer{}
Co to było? Przecież ta konsola nie jest do maszyny podłączona!\\
To ma być tylko teatr! Komedia niedorobiona.\\
Czyli to wszystko, co wpiszemy, to się na prawdę stanie?\\
Co za pojeb pisał, całe to wypracowanie!\\

\chardok{}
W kieszeni znalazłem ognioodporne rękawice,\\
wielu tu ludzi na widowni widzę!\\
Na zabawy z wami, mam teraz branie.\\
\machine{Umrze ten, kto z miejsca tutaj wstanie}.\\

\charfer{}
Mówi serio, nie wstawajcie, proszę,\\
zaraz się z nim rozprawię, aż spadną mu kalosze.\\
Lubisz rodeo? Zaraz polubisz.\\
Przez niekontrolowane wstrząsy konsoli, zaraz się zgubisz.\\
Patrz na mnie, ważę trzy tony.\\
W porównaniu ze mną, jesteś jakiś niedorobiony.\\
Jak myślisz, czy mnie to w ogóle ruszy?\\
Lepiej od razu szukaj na swoim ramieniu duszy.\\
Wpisuję kolejny tekst, możliwościami maszyna się chlubi.\\
\machine{Konsole trzęsą się tak, że Doktor Evilion nie polubi.}\\

\chardok{}
Nic się nie dzieje, wiedziałem o tym,\\
zdziwiony? Wyjaśnię ci to wszystko potem.\\
A póki co, kolej teraz moja.\\
\machine{Mechanicznych komarów rzuci się na Ferra cała sfora.}\\

\charfer{}
No to jesteśmy po równo, na mnie też to nie działa.\\
Dlaczego? Nasz język często robi ludziom w uszach ,,ała''.\\
Dlatego też, to tylko ludzie,\\
wymyślili nam te przydomki, żeby rozmawiać na luzie.\\
Mojego imienia, to nawet w pełni usłyszeć byś nie mógł.\\
A ty z kolei, nie jesteś doktorem. Oj --- czyżbym twoje ego stłukł?\\

\chardok{}
Nie ma znaczenia, zaraz to naprawię.\\
\machine{Przede mną leży mój doktorat.} Niech tak się stanie.\\

\charfer{}
Nie powiedziałeś jaki, a w nieokreślonym przypadku,\\
maszyna używa specjalnej przysadki.\\
Kto ją budował? Mikołaj --- radośnie!\\
Spodziewaj się na papierze, jakiejś dziedziny sprośnej.\\
Bardzo ciekawe, co ci wyszło tam?\\
Nie pal tego, chcę wiedzieć! Powiedz proszę nam.\\
Ale wiesz, że doktoratu zniszczenie,\\
zasad wszechświata wcale nie zmieni?\\

\charszam{}
Głupia maszyna. Teraz mi ukojenie.\\
Da tylko jej całkowite zniszczenie.\\

\charfer{}
Evilion to nazwisko, zagadka się wyjaśniła,\\
dlatego też komenda Mikołaja, nie jego namierzyła.\\
Zasady wszechświata, raz zapisane,\\
nie zważają na dalsze maszyny poczynanie.\\
Jeśli ją zniszczysz, wszystkie zostaną,\\
zasada zapisana, pozostanie zapisaną.\\
Siedzieć wtedy tu wszyscy będziemy,\\
aż całkowicie z głodu pomrzemy.\\

\charszam{}
Już widzę, dlaczego używać tej maszyny jest tak niebezpiecznie,\\
sala z konsolą powinna zamknięta być wiecznie.\\
Trzeba ten śmietnik, jakoś ogarnąć.\\
I zwycięstwo nad światem do siebie przygarnąć.\\
\machine{Zasada z konsoli, z konsolą padnie.}\\
Rozwiązałem ten problem całkiem ładnie,\\
teraz tylko wyłączyć twoją.\\
I kontrola nad światem będzie całkowicie moją.\\
\machine{Druga konsola rozpada się, jakby dostała batem}.\\
To mi da pełną kontrolę nad całym wszechświatem.\\

\charfer{}
Myślałeś że ci się to uda, tak po prostu?\\
To moja konsola do pierwszego gniazda, podłączona była kabli wiorstą,\\
to znaczy, że twoja część teraz pada, wiarę dasz?\\
Pamiętaj że z potworami szans na zwycięstwo nie masz.\\
A na zakończenie wpiszę komendę wspaniałą, niczym angielski poem.\\
Ojej, moja konsola wzięła i się zrobiła złomem.\\

\dida{Przydupas wychodzi zza sceny z kluczem francuskim.}\\

\chardoc{}
W czasie waszej niepoważnej potyczki,\\
ja poszedłem na dół i zamieniłem wtyczki.\\

\charszam{}
Zaraz, ale wstałeś z miejsca, jak to możliwe?\\
Jak w takim razie żyjesz? Powiedz skwapliwie.\\

\chardoc{}
Moje miejsce, nie wiem czy wiecie,\\
to zawsze pomagać panu, gdziekolwiek we wszechświecie.\\
Teraz, które zasady padły, a które żyją?\\
Powiem panu na ucho, ruszyć proszę szyją.\\

\charfer{}
Teraz coś wam powiem, że aż wyjdą wam gały,\\
zapomnieliście, że potwory mają słuch doskonały.\\
Nasz docent powiedział, jedną rzecz, od niechcenia.\\
Przyklejone na stałe nie muszą być już wasze siedzenia.\\

\dida{Prawdopodobnie wybucha teraz panika na sali i wszyscy uciekają.}\\

\charszam{}
Co za panika. Wszyscy w pośpiechu wychodzą.\\
Czy ta sztuka jest aż taką niedorobioną?\\
W takim razie będę musiał streścić się zatem,\\
aby każdy z was doświadczył mojego triumfu nad wszechświatem.\\
Co by tu wpisać, teraz jestem bogiem,\\
może coś prostego, potem się zastanowię.\\
Nic mi nie zagraża, więc czemu mam was zabijać?\\
Nie mogę się ciągle o ten sam cel obijać.\\
Co mi po waszej śmierci, dobry władca dba o swój lud,\\
dlatego będę dawał wam wszystkiego w bród.\\
Jak zrobić, żeby między nami wszystkimi była zgoda?\\
Już wiem, \machine{na patyku dla każdego loda!}\\











