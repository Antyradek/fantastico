\chapter{Bitwa na klawiatury} 
\dida{Ta dramatokomedia została znaleziona w Głównej Bibliotece w czasie jej digitalizacji z sześciokątnych płytek pamięci do komputera.
Nie jest dokładnie wiadome, kto ją napisał, a i nikt się do tego nie przyznaje. Może to i lepiej.}

\smalltitle{Bitwa na klawiatury}
Postaci:
\begin{itemize}
	\item Doktor Evilion --- człowiek.
	\item Przydupas --- dziwna hybryda człowieka z czymśtam, stworzona przez Eviliona.
	\item Mikołaj --- potwór.
	\item Nadar --- człowiek.
	\item Ferro --- potwór.
\end{itemize}

\dida{Poniższy scenariusz powinien być zagrany w naszym jedynym teatrze na Księżycu Radości. 
Dla realizmu, powinno się zamiast aktorów użyć rzeczywistych postaci przedstawionych w skrypcie (z wyjątkiem Ferra, gdyż jest beznadziejnym aktorem i nie zagrałby dobrze nawet samego siebie).}

\dida{Akcja rozgrywa się na Potworanie, w najbardziej strzeżonej części zbrojowni, przy maszynie do nadpisywania zasad wszechświata (znaną też jako Maszyna Życzeń).
Na scenie znajdują się dwie konsole od sterowania maszyną, na podwyższeniach, zwrócone przodem do siebie.
Na scenę z jednej strony wchodzi zły Doktor Evilion i jego pomocnik Przydupas. Zaraz potem wbiega Mikołaj i pozostali od drugiej strony.}

\chardokt{}
W końcu udało się! Dostałem się!\\
Maszyna Życzeń, jaka wielka ona jest. A tu konsola, zaraz sterować nią będę.\\

\charmik{}
Już po nas. To koniec wszystkiego.\\
Evilion dostał się do sterowania, zrobi z nami cokolwiek chce jego ego.\\

\chardokt{}
Przetestujmy ją, możemy życzyć sobie wszystkiego bez liku.\\
Wpisuję w klawiaturę.\\
\texttt{Chcę loda na patyku.}\\

\charprzy{}
Nic się nie stało, mój panie.\\
Czy na pewno poprawnie podłączone sterowanie?\\

\charmik{}
Wszystko poprawnie, wszytko tak ma być.\\
Inaczej nie mielibyśmy już wszyscy żyć.\\
Nikt wam nie powiedział, jak to działa.\\
Jak nadać poprawne sterowania.\\

\charprzy{}
Chyba Maszyna Życzeń to nazwa zła.\\
To nie tak jest, że ona cokolwiek spełnia.\\
Jeśli pozwolisz, przejmę sterowanie...\\

\chardokt{}
...panie.
Dla ciebie --- "`panie przejmę sterowanie"'.\\

\charmik{}
To może ja spróbuję... wielmożny panie.\\
Specjalnie dla ciebie, lody bez czekania.\\
\dida{Mikołaj swoimi lodowymi mocami stwarza loda w ręce.}\\
Oto i lód, którego chciałeś.\\
Chwileczkę, zaraz jeszcze więcej dostaniesz.\\
I w klawiaturę wpisać ten niefajny tekst.\\
\texttt{W każdej ziemskiej sekundzie nad głową Eviliona pojawia się milion sklonowanych lodów trzymanych przez potwora Mikołaja w ręce.}\\
Hmm. Nic się nie dzieje. Chyba coś nie tak jest.\\

\charnad{}
Mikołaju!\\
Właśnie odczytałem coś na moim komunikatorze. \\
Spowodowaliśmy niemałą katastrofę. O Boże!\\
Evilion, ale jaki?\\
Na Tirandii rządzi przecież jeszcze król taki.\\
To ta planeta, która ciągle na swą gwiazdę spada.\\
Co im wielki silnik daliśmy, żeby im nie była biada.\\
Teraz góra lodów w mig go zamroziła.\\
Siła nośna to teraz już tylko była.\\

\charmik{}
Zobacz, coś narobił Evilionie.\\
Przez ciebie cała planeta teraz zaraz spłonie.\\

\chardokt{}
Ach, to na mnie wina zwalona jest, taka?\\
\texttt{Mikołaj teraz umiera na raka.}\\

\charmik{}
Ja człowiekiem nie jestem, jak pewnie odkryłeś.\\
Komórek w ciele do zraczenia nie mam, jak pewnie zauważyłeś.\\
Ludzi do zabicia na Potworanie o tym imieniu też nie znajdziesz.\\
Ale na Ziemi prawdopodobnie prezentów w tym roku pod choinką już nie odnajdziesz.\\

\chardokt{}
Lodowy potworze. \\
Mój ty Boże. \\
Mógłbym napisać, że po prostu znikasz. \\
Ale lubię zabijać, gdy z innymi problemami się borykasz.\\
Potrafisz produkować lód?\\
Zaraz będziesz mieć tu gorąca w bród.\\
I cały się nam roztopisz.\\
\texttt{Klawiatura maszyny ma temperaturę tysięcy stopni.}\\

\dida{Mikołaj opiera się o rozżarzoną klawiaturę, jak nigdy nic.}

\charmik{}
Gorąca nie znoszę, ale co to są tysiące stopni temperatury.\\
Byłem zapraszany na herbatę do lawowych smoków z Karkentury.\\
O wiele ciekawsze jest jednak to.\\
Co zrobisz teraz z własną gorącą klawiaturą?\\

\chardokt{}
Zapomniałem. O ja głupi znowu.\\
Zmiana zasad wszechświata dotyczy nas obu.\\
Jestem doktorem nauk. Mam jednak asa w rękawie.\\
Przydupasie, zastąp mnie, takie twoje zadanie.\\

\charprzy{}
Ależ jak, panie?\\
Klawiatura płonie.\\
Rozumiem, że mam zginąć, panie.\\
Ja jednak szykuję odmowę.\\

\chardokt{}
Głupi eksperymencie. Aby cię wytworzyć.\\
Lata nauki poszły, abyś mógł trochę pożyć.\\
Myślisz, że ja Evilion, jam idiota?\\
Odporny jesteś na ogień, ty niecnoto.\\
Wpisz zaraz, że ważysz całe tony. Zaczyna coś trybić?\\
Skocz na niego, aby do ziemi natychmiast przybić.\\

\dida{Mikołaj lecący za scenę pod ciężarem Przydupasa.}

\charmik{}
Jestem potworem, podnoszę całe tony.\\
Zaraz się spod niego wygramolę.\\
O nie, widać wpisał także, że odbiera mi siły.\\
Ferro, zastąp mnie, będę przez pewien czas przybity.\\

\dida{Na scenę wchodzi Ferro.}

\charfer{}
Nie jesteś jedynym, który się pastwi nad ofiarami.\\
Co powiesz na minutę milczenia? Potem pogadamy.\\
\texttt{Kto w tym teatrze teraz dźwięk jakiś wyda.\\
Ten majestatycznie eksploduje.} ...enter chyba.\\

\dida{Po kilku sekundach dzwoni telefon podstawionej na widowni osoby. Eksploduje bomba pod jej siedzeniem.}

\charfer{}
Co to było? Przecież ta konsola nie jest podłączona!\\
To ma być tylko teatr! Komedia niedorobiona.\\
Czyli to wszystko, co wpisujemy, to się na prawdę stanie?\\
Co za pojeb pisał to wypracowanie!\\

\chardokt{}
W kieszeni miałem ognioodporne rękawice.\\
Wielu tu ludzi na widowni widzę!\\
Na zabawy z wami mam teraz branie.\\
\texttt{Umrze ten, kto z miejsca tutaj wstanie}.\\









