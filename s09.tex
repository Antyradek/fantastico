\chapter{Duży dom}

\info{Opowiadanie, które miało być osobnym opowiadaniem, ale wszyscy mieli je gdzieś, więc powstało jako grafomianiowy tekst z błędami stylistycznymi. Jako Wielka Improwizacja, był tylko zarys początkowej fabuły a od połowy pisany bez zbytniego myślenia.}

Był sobie raz profesor i jego żona i miał kolegę, który też miał swoją żonę, którzy bardzo się nawzajem lubili.
I ten profesor zawsze marzył o wielkim domu pośrodku lasu, takim z horrorów.
Nikt nie wie, dlaczego ale akurat się złożyło że kupił taki dom po okazyjnej cenie od miasta, gdy został zlicytowany po tajemniczym zniknięciu poprzednich właścicieli.

No więc ponieważ dom był za duży dla jednej osoby, poprosił swoich przyjaciół o przeprowadzkę do tego domu, żeby razem tam mieszkali i było im tam fajnie.
Kolega nazywał się Maciej i miał żonę o imieniu Klara, która lubiła kwiaty.
On nie za bardzo lubił kwiaty i dlatego był jej mężem, bo mąż zawsze jest przeciwieństwem żony.
Profesor z kolei miał żonę Aleksandrę, która była taką głupiutką dziewuszką, która wolałaby siedzieć w wannie, zamiast łazić po nawiedzonych domach. A Profesor był prawdziwym Profesorem i kupił ten dom, żeby przeprowadzać szalone eksperymenty z dala od cywilizacji ludzkiej.

Gdy jechali polną drogą przez las, niebezpieczeństwo czuć było w powietrzu.
Zatrzymali się na małej polanie, na której stał ten wielki dom.
Był wielki, ciemny i wyglądał jak z horrorów. 
Jednak nie rozpadał się, choć wyglądał jak ruina.
Miał nawet większość płytek na dachu i wszystkie szyby w oknach!

Profesor wyciągnął zardzewiałe klucze do drzwi i utorował sobie przez zarośla drogę do drzwi.
Otworzył z wielkim trudem i weszli do środka.
Mały salonik z zakurzonymi meblami prowadził schodami do długiego korytarza z tysiącem drzwi, na których końcu znajdowało się wielkie, półkoliste okno na zachodzące słońce.

Pora była, aby odkryć sekrety tego domu.
Aleksandra (ta głupia) rozpędziła się przez korytarz, otwierając wszystkie drzwi po kolei, a te które były zamknięte, omijała, aby poszukać swojej wymarzonej łazienki z marmurowymi podłogami i wanną wypełnioną olejkiem do kąpieli. Ale takiej nie znalazła, niestety. Znalazła jedynie jakiś zardzewiały kran, pod którym się umyła, zamiast wanny. 

Reszta osób poszukała sobie sypialni, bo późno się robiło.
Pierwsze drzwi prowadziły do pokoju z mnóstwem pozasychanych roślin.
To było dziwne, pełno doniczek i drutów na ścianach, po których miały się pnąć pnącza.
Małe szafki w których znajdywało się pełno środków do pielęgnacji roślin.
Łóżko wyglądało, jakby było zrobione z żywopłotu.
Oczywiście większość tych roślin była wyschła i martwa.
Klara, żona kolegi Profesora, bardzo chciała mieć ten pokój, gdyż lubiła kwiaty.
Natomiast jej mąż nie lubił kwiatów i poszukał innego pokoju.

Drugie drzwi skrywały jakieś muzeum starych rzeczy. Wszystko pozakurzane.
Profesor się rozpromienił z radości, jak zobaczył te rzeczy, których nikt inny nie potrafił nazwać.
Były tam dziwne maszynki robiące różne dziwne rzeczy.
Z sufitu zwisał model Układu Słonecznego, w którym nikt nie potrafił znaleźć planet Układu Słonecznego.
Łóżko wyglądało jak wehikuł czasu, albo sterowiec z epoki steampunku.
Profesor bardzo chętnie wziął ten pokój.

Trzecie drzwi skrywały pokój pełen luster i kurzu.
Wszystko było odbijające się i takie ładne.
Narzuta na łóżko była zrobiona z masy perłowej, włosów jednorożca i kurzu.
Podłoga pokryta była dywanem i kurzem.
Na ścianach tapeta z diamentów i kurz.
Na suficie kryształowy żyrandol z kurzu.
Aleksandra zaraz wpadła i zarezerwowała ten pokój, rzucając się na łóżko.
W sekundzie pokój zmienił się w komorę gazową, ale zamiast Cyklonu B, był Cyklon Kurz.
Wszyscy z wyjątkiem Aleksandry uciekli i poszli dalej.

Czwarty pokój to biblioteka.
Maciej z chęcią wziął go dla siebie.
Nic w nim nie było, prócz książek.
Nawet łóżko było zrobione z książek.
Maciejowi to nie przeszkadzało.

Kolejne drzwi były pozamykane, więc poszli wszyscy do swoich pokoi spać.

\divider{}

Po zachodzie słońca, Klara siedziała w swoim łóżku i czytała książkę.
Oderwała się od niej dopiero, jak zobaczyła że jej kartki oświetlały też inne, kolorowe światła, niż jedynie wątła żaróweczka nad łóżkiem.
Rozejrzała się i zobaczyła, że w pokoju jest bardzo wiele kwiatów i każdy z nich świeci, jakby wsadzić do środka żarówkę. Przypomniała sobie, że niektóre z kwitnących roślin wyglądały jeszcze przed paroma godzinami jak suche patyki. Zastanawiała się nad tym, jakie rośliny potrafią tak szybko rosnąć, gdy przypomniała sobie, że żadna z roślin Ziemi nie potrafi świecić, co ją jeszcze bardziej zdziwiło.
Postanowiła więc oderwać się od swojej książki.

Aleksandra zatopiła się w swoim łóżku jak Titanic w górze lodowej.
Nigdy nie chciała z niego wyjść.
Choćby dziwne rzeczy działy się wokół, nic jej by nie zmusiło do wyjścia.

Maciej na chybił-trafił wybrał książkę i otworzył na losowej stronie.
Była czarna i bez tytułu na okładce, więc normalnie nigdy by takiej nie wybrał, gdyby nie to, że miał zamknięte oczy.
Jak otworzył stronę, to zaraz znalazł się na skraju przepaści nad jakimś morzem.
Na niebie świeciły dwa księżyce, a morze było pomarańczowe jak kisiel.

Klara zbadała kwiat i zauważyła w nim rzędy ostrych zębów, jak gdyby chciał ją pożreć.
Popatrzyła na łóżko, na którym wiły się pnącza, szukając swojej poprzedniej ofiary.
Coś chciało ją złapać za buta.
Chyba będzie potrzebowała innego pokoju.

Aleksandra nadal spała na swoim łóżku.
Widziała się we wszystkich lustrach w pokoju.
Tylko w jednym lustrze się nie widziała, co ją zdziwiło wielce, ale nadal postanowiła spać w łóżku.

Maciej poczuł, że nie może oddychać i że w rękach nadal trzyma książkę.
Zamknął ją więc bardzo szybko i wrócił w mgnieniu oka do normalności.
Spróbował na innej stronie i znalazł się u szczytu wulkanu, który akurat wybuchał.
Więc tą książkę też zamknął, bo inaczej by się poparzył lawą.

Klara wyszła z pokoju i weszła na korytarz.
Z każdego pokoju świeciło inne światło.
W dodatku ostatnie okno świeciło jasno jak w dzień. Za nim znajdowało się morze i dwie planety na niebie.
Bardzo dziwne.

Aleksandra nie mogła spać, ciągle myślała o lustrze bez odbicia.
Postanowiła więc jednak wstać, co też uczyniła.
Podeszła do lustra i dotknęła go, a ono zostawiło jej na ręce wodę.
W dodatku było lodowato zimne.

Kolejna książka okazała być się próżnią kosmiczną, w której Maciej nie mógł oddychać.
Postanowił więc nie otwierać już więcej książek w strachu o swoje życie.
Że otworzy jakąś i znowu umrze.
Więc chciał zobaczyć, co tam u innych.
Wyszedł na korytarz i zobaczył Klarę wyglądającą za okno.

Klara zobaczyła wyglądającego z pokoju Macieja i pokazała mu okno.
Za oknem było morze i dwie planety na niebie, okno jakby lewitowało w powietrzu.
Po otwarciu powiał ciepły wiatr i nie mogli oddychać, bo nie było za oknem tlenu.
Dotknęli morza, które było ciepłe jak ciepła woda.
\begin{dialogue}
	\ds{} Ciekawe, czy w tym morzu żyją jakieś ryby \dm{} Klara zapytała się swojego męża.
	\ds{} Być może \dm{} mąż klary odpowiedział.
\end{dialogue}
Potem poszli razem do pokoju Aleksandry, bo bali się że jak jest tak dziwnie wszędzie, to będzie głupia i sobie zrobi jakąś głupią krzywdę.

Aleksandra poświeciła komórką w lustro, które okazało się być oknem na lodowiec.
W środku byli zamarznięci ludzie, którzy w przerażeniu patrzyli się w jej stronę, jakby to na nią się patrzyli, ale nie mogli się patrzyć, bo byli zamarznięci, a jak Aleksandra sobie poszła, to nadal się patrzyli w ten sam sposób. To zastanowiło Aleksandrę, ale że jako była głupia, to sobie nic z tego nie robiła i dalej poszła spać do swojego zakurzonego łóżka.

Gdy Maciej i Klara weszli do Aleksandry, to zobaczyli jak śpi w łóżku i nic sobie nie robi z kapiącego lustra, które topniało pod wpływem ciepła. Maciej zbadał lustro i chciał zajrzeć za nie i się okazało, że to normalne lustro i z tyłu jest ściana. Więc podniósł je i popatrzył z różnych stron, a jak chodził, to zmieniał się obraz i widział także resztę lodowca i więcej zamarzniętych ludzi.
Para obudziła Aleksandrę za pomocą kapania na nią zimną wodą z lodowca, a ona wstała i razem poszli do Profesora, żeby im to wytłumaczył.

Jak wyszli na zewnątrz, to Aleksandra przypomniała sobie, że potrzebuje łazienki i poszła dalej otwierać resztę drzwi, które okazały się nie być już więcej zamknięte.
Nagle polała się na nich woda, jakby zza jednych drzwi lał się ocean wody.
Złapali się za klamki, żeby ich nie porwała.
I nie mogli zamknąć tych drzwi z wodą, bo strumień wody był tak silny.
Zupełnie jakby te drzwi znajdowały się na dnie morza, albo coś.
I wtedy coś wielkiego od drugiej strony drzwi zostało zassane przez wodę i zablokowało przepływ.
Ze strony korytarza wyglądało jak brzuch wieloryba, ale było bardziej gąbczaste jak gąbka.
Postanowili zamknąć drzwi, żeby więcej nie poleciała woda i poszli do pierwszego pokoju profesora.

Gdy weszli do niego, to okazało się, że profesor siedzi skulony w kącie łóżka, a wszystkie te maszyny działają i latają. Do tego ten Układ Słoneczny pod sufitem obracał się tak szybko, że nie mogli przejść obok, bo by ich pociął na kawałki.
Profesor siedział skulony w kącie łóżka i co chwila odbijał jakimś prętem lecące w niego mechanizmy, jak grając w golfa. Trzeba było mu pomóc.
Każdy więc poszedł z powrotem do swojego pokoju, żeby znaleźć coś, co mogłoby mu pomóc się uratować.

Klara wzięła doniczkę z jakimś mało zabójczym kwiatkiem i butelkę wspomagacza wzrostu.
Postawiła ją w pokoju Profesora i wlała butelkę do kwiatka i zaczął on szybko rosnąć i przyjmować na siebie ciosy planet. W ten sposób usunęli bezpiecznie Układ Słoneczny z sufitu i pokój stał się bardziej bezpieczny.
Jednak kwiat okazał się niebezpieczny i także chciał zjeść Klarę i Profesora.
Jednak mądra Klara wlała mu do doniczki trochę specyfiku do zabijania roślin i zabiła ją i już nie było niebezpiecznie.
Jednak w pokoju nadal latały niebezpieczne mechanizmy.

Aleksandra nie miała w pokoju nic, co by mogło pomóc jej mężowi, tylko same lustra.
Więc postanowiła iść dalej spać, bo co innego by mogła robić.

Maciej wertował wszystkie książki i wziął kilka z nich pod pachę i poszedł z powrotem do pokoju Profesora.
Otworzył tą o podróżach kosmicznych i pokój wypełnił się próżnią i wszystkie latające maszyny spadły na ziemię, bo nie miały powietrza do latania.
Potem otworzył książkę o wulkanach i podłoga się zamieniła w lawę i zabiła wszystkie latające maszyny, które spadły na ziemię.
Potem poszedł do Profesora i zapytał, czy wszystko w porządku, a on się bardzo ucieszył z ratunku.
Poszli więc razem z nim, żeby wytłumaczył, co się dzieje w tym domu.

Profesor, korzystając ze swojej profesorskiej natury profesora, zaczął uważnie studiować każde drzwi, otwierając na na oścież. Każde było inne, jedne pokazywały puszczę, inne jakieś kosmiczne miasta, a jeszcze inne laboratoria dziwnych kosmitów. Ledwo uszli z życiem, gdy zobaczyli kosmiczną wojnę, jak otwarli drzwi w jakiejś ruinie pośrodku wojny. W końcu Profesor klasnął w dłonie i oznajmił, że wie jaki jest problem i że każde drzwi prowadzą do innych drzwi w jakimś innym miejscu. Zupełnie jak w tej animacji z potworami, ale tutaj było bardziej losowo.

Więc poszukali odpowiedzi, idąc do jakiegoś domu za jakimiś drzwiami.
Była tam śnieżna i świąteczna atmosfera. Dom wyglądał, jak niedokończony od wewnątrz, 
jakby właścicielom zabrakło pieniędzy na dokończenie piwnicy. 
W kącie pokoju stała choinka.
W niedokończonej piwnicy stał zastawiony na wigilię stół ze smacznymi rzeczami, ale nigdzie nie było mieszkańców. 
Za oknem słychać było inny dom, z którego co chwila dochodziły jakieś śmiechy.
Okazało się, że drzwi z których wyszli prowadzą do małego składzika.
Bardzo miła atmosfera. 
Postanowili skorzystać z gościnności i zjeść trochę jedzenia, bo nie jedli od całego dnia.
Wtedy Maciejowi zadzwonił poranny budzik o szóstej rano, dokładnie wtedy drzwi do ich domu się zamknęły i zostali sami w obcym domu, bez możliwości powrotu.

Bali się, że zaraz wrócą mieszkańcy i usiedli w salonie, żeby im się wytłumaczyć, jak trafili do ich domu.
Jednak minął cały dzień i nikt się nie pojawił, a za oknem nadal była noc i śmiechy.
Więc jak się zrobiła północ na zegarku Macieja, to drzwi ponownie się otwarły i mogli wrócić do swojego domu.

Aleksandra była ciekawa prezentów i wbrew prośbom męża, otworzyła jeden.
I już było wiadomo, gdzie podziali się mieszkańcy.
W pudełku były wysuszone zwłoki jakiegoś człowieka.
W pozostałych prezentach również.
Bardzo się przestraszyli i wrócili do domu, zamykając za sobą drzwi.

Spróbowali kolejnych drzwi. Zobaczyli stół wypełniony owocami, wiele z nich było nieznanych.
Był też tam stary, stojący zegar z kukułką o trzynastu godzinach na cyferblacie i chodził znacznie szybciej, niż ten Macieja. Spróbowali wziąć jakiegoś owoca ze stołu, ale okazało się, że te owoce żyją i się na nich rzuciły, gryząc po ubraniach. 
Wtedy uciekli, bojąc się jeść jakiekolwiek z nich.

Za kolejnymi drzwiami było miasto kosmitów z kosmitami.
Jednak jak szli po tym mieście, to nikt nie zwracał na nich uwagi.
Mieli humanoidalne kształty bez twarzy, trochę strasznie, a trochę dziwnie.
Ich świat był bardzo rozwinięty i mało kolorowy.
Był też nieco zimniejszy i Profesor obliczył, że ci ludzie patrzą na świat temperaturą i dlatego ich nie widzą. Przyniósł trochę lodu z pokoju Aleksandry i kosmici zwracali na nich uwagę.
Traktowali lód jak walutę, więc nasi bohaterzy kupili w tym mieście wiele różnych fajnych rzeczy.
Jednak tak się guzdrali, że zamknęły im się drzwi i zostali w tym świecie na kolejny dzień.

Jednak tutaj już nie było tak strasznie.
Przespali się w jakimś hotelu, za który nawet nie musieli płacić, bo nikt ich nie widział.
Profesor poszedł do muzeum technologii i nauczył się wiele ciekawej nauki o tym świecie.
Jednak szybko zrozumiał, że ludzie są bardziej rozwinięci od tych tutaj, nawet jeśli nie robią tak fajnych budynków. Postanowili więc wrócić do domu gdy nastała północ.

Tak badali pokój za pokojem przez kolejne drzwi i nauczyli się, gdzie co jest.
Zeszło im na to miesiąc, żeby się prawdziwie zadomowić.
Codziennością się stały śniadania z żywych owoców w polewie kisielu z jakiejś książki oraz kąpiele w gorącej kawie olbrzyma z zamku z waty cukrowej. Było to bardzo miłe życie, o jakim marzył Profesor i jakie przyjmowała jego żona, jeśli raz na jakiś czas odwiedzali marmurową łaźnię w prywatnym domu jakiegoś króla, który był smokiem i który nie lubił, jak mu obcy kosmici używali wanny, gdy nie patrzył.

\divider{}

Maciej nie był taki chętny, bo trapiło go jedno, bardzo ważne pytanie.
Gdzie podziali się poprzedni właściciele? 
Mogli utopić się w lawie, a mogli zamarznąć w lodowcu, ale Maciej czuł, że dom ma jakiś związek z tym, co się stało z właścicielami. Postanowił poszukać odpowiedzi na własną rękę w jakiś pokojach.
Szukał kogoś, kto dokładnie wyjaśni mu, czym jest ten dom i dlaczego jest taki dziwny.

Zaczął od książek w swoim pokoju, ale szybko stwierdził, że książki pokazują sztuczne miejsca, które nie istniały na prawdę. 
Więc jakby znalazł tam kogoś, to nie mógłby mu tam powiedzieć prawdziwej prawdy od tym domu, bo by to była sztuczna prawda. 
Jak powiedział o tym Profesorowi, to ten się zaśmiał, bo wiedział od dawna że książki są sztuczne i że nie są kluczem do zagadki. 

Więc drugą rzeczą były pokoje, ale ich było mało i były dokładnie zbadane i nie chciał się za bardzo wychylać w te światy, bojąc się że coś się stanie. Nikt nie potrafił mu pomóc.

Trzecią rzeczą były okna, szafki i wszystkie małe otworki w domu, które nie były aż tak dokładnie zbadane.
Jak otworzył szufladę w pokoju Klary, to wypadła z niej kupa śniegu, jakby lawina śnieżna leciała przez tą szufladę. Zaglądał do butelek i jedna wyglądała obiecująco, był tam pokój jakichś profesorów przy mapach.
\begin{dialogue}
	\ds{} Przepraszam, mam jakiś problem z domem \dm{} grzecznie się zapytał.
\end{dialogue}
Wtedy ci profesorowie wolno obrócili głowę w jego stronę, pokazując ciemne zupełnie oczy i nagle się rzucili na niego. Ale że dziurka w butelce była mała, to nie zmieścili się wszyscy, tylko ich długie języki wylatywały z butelki jak sznurki. Maciej czym prędzej zakręcił butelkę z powrotem.

W kolejnych butelkach były nieprzydatne rzeczy, raz ulatywał jakiś dym w różowym kolorze, raz widział zniszczone apokalipsą miasto, raz klasę dzieci, którzy też rzucili się na niego z językami, raz ciekła z butelki woda, która leciała na sufit, jakby miała odwróconą grawitację, raz z butelki wyfrunął dżin i powiedział, że nie umie spełniać życzeń i nie pomoże, a do tego umie tylko rapować.

Więc Maciej i rapujący dżin poszli poszukać czegoś w pokoju Profesora.
Od pierwszego dnia, wszystkie zabawki na półkach były poprzywiązywane sznurkami do szaf, więc nie były groźne, chociaż groźnie się rzucały na widok Macieja.
Nie było nigdzie żadnych dziurek, a za oknem była całkowicie czarna czerń.
Gdy otworzył okno, ciemny dym wyleciał na podłogę i wezwał dżina na bitwę raperską.
Dżin przeprosił Macieja i poleciał razem z dymem z powrotem za okno. Potem było słychać uciążliwe rapy o jakichś skrzyniach i pompach ropy naftowej, których Maciej nie miał ochoty słuchać, więc sobie poszedł do innego pokoju.

Postanowił zbadać korytarz.
Oprócz pingwinów, grających w makao w lodówce i dyskoteki żelków w cukiernicy nie znalazł niczego ciekawszego. Czyżby nikt nie wiedział, co to za miejsce i dlaczego ten dom jest taki dziwny?
\begin{dialogue}
	\ds{} Hej, szukasz czegoś? \dm{} Maciej usłyszał głos pod swoimi nogami. Schylił się do mysiej dziury w ścianie.
	\ds{} Tak, czy wiesz może coś o tym domu? To dziwne miejsce \dm{} zapytał się.
	\ds{} Tak, wiem o tym domu całkiem dużo. Mam nadzieję że pomogłem. \dm{} Po czym poszedł sobie.
\end{dialogue}
Niestety nie pomógł wcale a wcale, przez co Maciej bardzo się na niego zdenerwował, że nie chce mu powiedzieć. Napisał karteczkę o tym, że jest na niego bardzo zły i wetknął w dziurkę.

Dalej Maciej poszedł poszukać czegoś w pokoju Aleksandry. 
Popatrzył ponownie na lustro i zastanowił się, na co patrzą ci zamarznięci ludzie.
Podniósł lustro ze ściany odsunął się troszkę.
Jego oczom ukazała się duża postać, kilkukrotnie większa, niż pozostali.
Może to był Yeti?
Ale po dokładniejszym przypatrzeniu się, zobaczył że postać ma ogon i kolce na plecach i na ogonie.
Na rękach trójkątne kleszcze i pazury na nogach.
Nie wyglądał miło i przyjemnie, w dodatku jego oczy nie były całkowicie zamarznięte, a uważnie patrzyły się na Macieja.
Postanowił on zapytać się potwora, czy wie coś o tym miejscu, ale tamten nie mógł mówić, bo miał zamarzniętą paszczę. Tylko trochę się poruszył i lód popękał wokół niego.
Przerażony Maciej schował lustro szybko pod pierzynę, mocząc przy okazji łóżko Aleksandry.

Powiedział o tym Profesorowi, który bardzo się ucieszył i postanowił szybko odmrozić straszne straszydło z lodu. Jednak Maciej się na to nie zgodził, więc Profesor zrobił to, jak Maciej nie patrzył.
Wylał trochę gorącej wody, trzymając lustro pod odpowiednim kątem, żeby odmrozić paszczę potworowi.
Spowodował, że tamten zaczął móc mówić. 
Zapytał się, kim jest, a on powiedział jakieś dziwne słowo i dokończył po polsku, że są jakby policją wszechświata i że mu się zamarzło na misji jak ratował jakichś ludzi. I nie uratował ich, bo zamarzli.
Jednak mijał ciągle się z odpowiedzią na to, czym jest dokładnie ten dom, chociaż wiedział doskonale.
Poprosił o całkowite rozmrożenie aż do powierzchni lodu i wtedy im powie.
Tego nie chciał wykonać nawet Profesor, co potwór skwitował tylko, że trudno.

Pewnej nocy Aleksandra, śpiąc w swoim łóżku, jak już wyschło, usłyszała głos potwora z lustra.
Wzięła lustro i przesunęła tak, żeby go widzieć. Tamten poprosił ją o rozmrożenie, a ponieważ była głupia, to się zgodziła. Wymyśliła, że wsadzi lustro do wody z morza na końcu korytarza i pomacha, aby przyspieszyć proces topnienia, chcąc wytopić na tyle dużą dziurę, żeby uwolnić potwora z pułapki i wytopić lód do powierzchni. Jak pomyślała, tak zrobiła i za chwilę zobaczyła potwora w całej okazałości. Był nawet przystojny, jak na potworze standardy. Powiedział, żeby wszyscy się zebrali razem. 

Więc Aleksandra wzięła potwora do pokazania swojemu mężowi, który bardzo się ucieszył, że tamten nie chce ich zabić tak od razu, przywołał do siebie Klarę i Macieja, którzy bardzo się zezłościli na Aleksandrę za to, że jest taka głupia. 
Byli razem w pokoju luster, potwór wskazał swoim kleszczem, żeby położyli lustro na podłodze, ale tylko Profesor zrozumiał jego zamiar.
\begin{dialogue}
	\ds{} Dlaczego chcesz, żebyśmy wskoczyli do twojego świata? \dm{} zapytał.
	\ds{} Ciszej, nie tak głośno \dm{} ryknął, szepcząc. 
\end{dialogue}	
Jednak było już za późno. Usłyszeli, jak dom trzeszczy, zobaczyli jak wnęki drzwi tracą kształt prostokątów, jak podłoga lekko się wygina.
\begin{dialogue}
	\ds{} Wskakujcie, szybko! \dm{} Złapał Aleksandrę za nogę i wciągnął do lustra. Profesor skoczył samemu, chociaż nikt go nie podejrzewał o uwierzenie dziwnemu potworowi z lustra.
	\ds{} Nie wskoczę. Nie ufam ci \dm{} Maciej odpowiedział i poszedł na drugą stronę pokoju.
	\ds{} Chcę was uratować!
	\ds{} Ja mu wierzę. \dm{} Klara wskoczyła do lustra za resztą trójki.
	\ds{} Niby przed czym? \dm{} Maciej nie dawał za wygraną.
	\ds{} Przed domem!
\end{dialogue}
Na te słowa poczuł mocne szarpnięcie podłogi. Wszystkie lustra w pokoju poczęły pękać jeden za drugim.
Zobaczył twarze swoich przyjaciół, ale nie ruszył się z miejsca.
Poczęły się otwierać dziury w podłodze, do których omal nie wpadł.
Przyjaciele i potwór wołały, żeby wskoczył za nimi, ale on nadal nie był wzruszony. 
Aż pękło i to lustro, ostatnia deska ratunku rozsypała się w trójkątne kawałeczki, jak zbite lustro.
Dopiero wtedy ruszył się z miejsca.

Biegł, przeskakując ziejące dziurami podłogi. 
Wyskoczył na korytarz, gdzie kolejne drzwi zatrzaskiwały się z łoskotem, chciał pobiec do wyjścia, ale cały korytarz wypełniony był deskami, które obracały się jak wielki mikser do mielenia.
Pobiegł w stronę okna, które nieco się obróciło w dół i morze zaczęło wlewać się do domu.
Nie wiem, co chciał zrobić, czy wyskoczyć przez okno do ciepłego morza w którym i tak nie dało się oddychać, czy co, ale nie udało mu się, bo podłoga tak sprytnie się obróciła, że przewrócił się.
Zaraz potem okno schowało się w suficie i nastała całkowita ciemność. 
Czuł, jak mielą go deski, niczym całe mnóstwo dziwnych palców twardych dzieci.
Czy to był koniec? Czuł, że będzie długo umierał.

Włączył swoją komórkę, żeby sobie poświecić i zobaczył wysuszone ciała, które wyglądały na poprzednich właścicieli domu. I umierał szczęśliwy, że rozwiązał zagadkę tajemniczego domu. W końcu, a jednocześnie trochę się smucił, że umierał.

A pozostali żyli długo i szczęśliwie, to potwór z lustra mówił prawdę i uratował ich.
Podrzucił z powrotem na Ziemię, w miejsce, gdzie znajdował się wcześniej dom, a teraz ziała czarna dziura w ziemi. Jak odlatywał, to machali mu chusteczkami na pożegnanie.








