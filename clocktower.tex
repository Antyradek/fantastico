\chapter{Bicie dzwonu, śmierć komu?}

\info{Bombastyczna rodzinka spotyka na swojej drodze tajemniczy zegar. Okazuje się, że problem społeczny na tej planecie jest dużo głębszy, niż się pierwotnie wydawało.}{Urban science fantasy}{34 000}{\quadreversumuniversum{}}

\illustration{img/clock.png}

\begincapital{P}rzylecieli swoim małym statkiem pod wieczór.
Zwabieni dużą aktywnością uniwersalności w tej galaktyce.
Ukryli rakietę w pobliskich górkach, wzięli najpotrzebniejsze narzędzia i poszli rozejrzeć się po zabudowaniach.
Ta cywilizacja nie powinna była istnieć.

Ale istniała i należało to zbadać.

Było pochmurno.
Mglista i zimna noc wgryzała się w grube i ciemne ubrania, jakie dla niepoznaki nosili.
Niebieski i wczesnowiosenny księżyc oświetlał świat trupią łuną, niczym wielka lampa nad stołem w prosektorium.
Będąc tam, wolałeś trzymać się w cieniu, bo wydawało się, jakby to światło mogło wyssać z ciebie życie.
Ale i cień okazywał się nieprzyjemnym atramentem, który potrafiłby w każdej chwili złapać za nogę i wciągnąć w siebie na zawsze. 
Przeskakując między strachem i koszmarem, nie znalazłbyś bezpiecznej wyspy.

Był też drugi księżyc, niewiele większy i cały czerwony.
Nadawał miastu krwisty poblask. 
Z prosektorium przenosił do ciemni fotograficznej, gdzie wisiały kościste zdjęcia rentgenowskie.
Była wczesna wiosna, ale mgliste odczucie towarzyszyło jednak późnej jesieni.
Leżące gdzieniegdzie kupy śniegu wyglądały w tym świetle jak zwłoki rozszarpanych zwierząt.
Tam, gdzie światła księżyców się spotykały, panował nieprzyjemny fiolet, niczym żałobna szata rozpostarta na ziemi, ciągnąca się od horyzontu po horyzont.

Źródło uniwersalności, nieprawidłowej materii wszechświata, znajdowało się w centrum miasta.
Czwórka przybyszów posuwała się zabłoconymi uliczkami, omijając kałuże o niezbadanej zawartości.
Starali się pływać w cienistym atramencie, aby nie zwracać na siebie zbytniej uwagi.
Elektroniczny czujnik uniwersalności pikał coraz szybciej, niczym EKG leżącego na stole operacyjnym pacjenta.
Napawało to nadzieją, pomimo że każda zdrowa na umyśle osoba omijałaby uniwersalność o lata świetlne, gdyby wiedziała czym jest. I czym może być.

Gdy pikacz ogłosił, że serce operowanego wyrwało się z jego klatki piersiowej, spostrzegli że są na głównym placu. 
To jedyne miejsce w okolicy, gdzie błoto było tak zadeptane, jakby przeryło je stado ziemskich dzików.
Połyskujące kałuże patrzyły się na rodzinę swoimi czerwonymi, księżycowymi źrenicami.
Nie pompowały jednak umysłów skutecznie strachem, bowiem i tak przyćmione były czernią, bijącą od wieży.
Wielkiej wieży zegarowej, górującej nad półkolistymi budynkami, niczym sęp nad swoją przyszłą ofiarą.

Zegar jaśniał żółtawym światłem, trzeci księżyc na niebie.
Miał osiem godzin i jedną wskazówkę, która poruszała się w lewo, odwrotnie niż jesteśmy przyzwyczajeni.
Złoty okrąg wokół tarczy odbijał światło trupa i jego krew, pozornie niwelując ich złowrogość.
Pomimo mrocznego zarysu wieży wokół, żółte oko emanowało sercowym ciepłem i domowością.
Jednak nauczona doświadczeniem rodzina Nocnych wolała pływać w czarnych dziurach, zamiast ufać tej dziwacznej i niebezpiecznej strukturze, jaką była uniwersalność.

Na plac powoli zaczęli schodzić się mieszkańcy.
Ubrani byli podobnie do gości, snuli się, omijając błotne otchłanie.
Zwiewne, czarne stroje wydawały się meduzami, unoszącymi się bezwładnie po dnie oceanu.
Nie rozmawiając ze sobą, niemal lewitowali tuż nad ziemią, dryfowali na śmierć.
Wpatrywali się w słoneczny cyferblat, jak skazaniec wpatruje się w swojego kata.
Wskazówka sterczała w kierunku dolnej godziny, niczym palec kostuchy wybierający swoją następną ofiarę.
Czekali.

Smukłe budynki z ciemnego metalu, na nich spiczaste zdobienia. 
Były jak sterczące w górę pazury potworów z głębin atramentowych kałuż.
Małe okienka, przepuszczające czerwone światło pochodni, wyglądały jak oczy czyhających w cieniu drapieżników.
Wszystkim z ust buchała para i unosiła się bardzo wolno, nie rozmywając się, niczym uchodzące w zaświaty dusze.
Dymne kolumny sterczały z wierzchołka każdego domu, jak nitki marionetek sterowane przez wielkiego lalkarza, albo raczej w tym przypadku, zegarmistrza.
Ewidentne było, że ci ludzie zbierali się na tym placu co noc.
Co noc oczekiwali na niewiadomą.

Przestraszone dziewczynki przytuliły się do kolan swojego taty, a żona gładziła je po zakapturzonych głowach.
\begin{dialogue}
	\ds{} Boję się \dm{} wyszeptała siedmioletnia Żywia do swojej mamusi.
	\ds{} Ale ty jesteś strachliwa \dm{} odpowiedziała jej starsza o rok Nadzieja.
	\ds{} Ciszej, bo nas odkryją \dm{} szepnął im ojciec rodziny. \dm{} Zobaczcie, chyba coś się dzieje.
\end{dialogue}

Wskazówka przesunęła się lekko w prawo, lądując dokładnie na dolnym znaku. 
Coś zgrzytnęło, coś metalicznie stuknęło we wnętrzu wieży.
Złoty okrąg tarczy błysnął.
Absolutna cisza dzwoniła przez chwilę w uszach.
Wtedy dał się słyszeć mechaniczny głos, niczym wybijany na stalowym mechanizmie.
Był pozbawiony nawet iskierki żywota.

\begin{poem}
	Posłuchajcie uważnie moje dziatki \\
	wybitej o północy zagadki. \\
	Przed wschodem słońca znajdźcie rozwiązanie, \\
	albo komuś coś się dzisiaj stanie. \\
	Bim-bom. \\
	Z piecem dom. \\
	Trucizna w pudełku. \\
	Obok kowadełka. \\
	Dodana do soli. \\
	Nie zrobi tego powoli. \\
	Ofiara się nie ukryje. \\
	Zabije to zabije. \\
\end{poem}

Wskazówka poruszyła się odrobinę dalej.
I była tam wyróżniająca się grupka mieszkańców.
Mieli brudnawe stroje, chusty na twarzach i pałki w rękach. Trochę śmierdzieli.
Wpasowywali się w mroczny krajobraz świata aż za dobrze.
Jakoś pozostali ludzie trzymali się od nich na dystans.
Po wymianie krótkich, niezrozumiałych zdań, pobiegli wszyscy w jakimś kierunku, a raczej podążyli za sobą nawzajem, bo nie widać było, żeby mieli jakąkolwiek zbiorową świadomość.
Większość orbitujących z dala obywateli mimowolnie udała się za nimi, co uczynili także kosmiczni przybysze.

Prosektoryjna cisza zniknęła, trupy ożyły, rozpoczęły się rozmowy i dyskusje.
Ludzie rozprawiali o tym, czego może dotyczyć dzisiejsza zagadka.
Fioletowy całun śmierci został rozdarty przez chaotyczne biegi, głośne rozmowy i przepychanki wśród dzieci.
Czerwone oko na nieboskłonie nadal usilnie próbowało emanować złowrogością, ale nikt już nie zwracał na nie uwagi.

Członkowie rodziny z kosmosu uważnie podsłuchiwali rozmów wokół siebie.
Wyglądało na to, że zegar zadawał jakąś zagadkę codziennie i codziennie losowa osoba umierała pod wpływem tajemniczej siły mechanizmu.
Śladów morderstwa nigdy nikt nie znajdywał, a w prawej ręce ofiary pojawiał się zwój papieru z rozwiązaniem, zapisanym tym upiornym rymem.
Nigdy w historii miasta nie udało się rozwiązać zagadki na czas i zawsze ktoś płacił za to życiem.
Właściwie nie wiedziałeś, czy ta noc nie będzie twoją ostatnią. 
Żywia i Nadzieja bały się, że którąś z nich może trafić, albo gorzej, jednego z ich rodziców.

Maria Nocna bazgrała notatki na swoim elektronicznym komunikatorze.
Uniwersalność. 
To była nieprawidłowa materia, coś jak nowotwór wszechświata.
Ten twór nie słuchał się zasad fizyki, nie można było nim sterować, często nie miał sensu.
Uniwersalność przyjmowała bardzo różne formy, od tortów weselnych, po latające po wszechświecie dziadki w wannach pełnych wody.
A tutaj był zegar, zadający zagadki.

Największa koncentracja gapiów wypadła przed czymś, co mogło być domem piekarza.
Piec i sól. Wyglądało na to, że się zgadza.
W błocie, na kolanach, siedział związany właściciel przybytku.
\begin{dialogue}
	\ds{} Gdzie jest ta trucizna, którą chcesz dodać rano do chleba?
\end{dialogue}

Przywódca bandy groził mu wałkiem do ciasta. 
Był smukły, ubrany lepiej od reszty, jego twarz pozbawiona była wyrazu i pomimo, że księżyce były na niebie,
zawsze wyglądała, jakby była złowieszczo oświetlona od dołu.
Szrama tu i ówdzie potęgowała ten efekt.
Łysa głowa błyskała purpurową bielą, jak kość nagiej czaszki.
Do kompletu brakowało mu jeszcze kosy.
	
\begin{dialogue}		
	\ds{} Nie wmówisz nam, że zagadka nie jest o tobie. Tym razem pokonamy Zegar.
	\ds{} Przysięgam! \dm{} 
		piekarz płakał. \dm{} 
		To nie o mnie tej nocy chodzi! Nikogo nie zamierzam otruć.
	\ds{} Nie przyznasz się? To zaraz znajdziemy dowód. \dm{} 
		Pstryknął palcami i reszta jego bandy zaczęła przewracać piekarnię do góry nogami. \ds{} 
		Zapytam się jeszcze raz...
	\ds{} ...nic trującego nie mam, tylko składniki do wypieków. Zegar przecież nigdy nie mówi wprost, to nie ja.
	\ds{} Czyżby? Jest piec? Jest. Jest kowadełko do krojenia? Jest. Są ofiary jedzące rano twój chleb? Są.
	\ds{} Wystarczy że nie upiekę dzisiaj rano niczego i nikt nie będzie mógł się otruć.
	\ds{} A więc potwierdzasz, że gdybyś coś upiekł, to ktoś by się otruł, tak?
	\ds{} Co to a logika?
\end{dialogue}
Szepty przeszły po zbiorowisku.
\begin{dialogue}
	\ds{} Nie pyskuj. \dm{} 
		Rozejrzał się nerwowo. \dm{} 
		Bo zaraz przewałkuję ci tym facjatę! \dm{} 
		Zagroził. \dm{} 
		A wy co? Dalej szukać tej trucizny.
\end{dialogue}
Tymczasem ktoś przyniósł jakieś znalezione pudełko.
Bezkosiarski kostuch otworzył, powąchał i uśmiechnął się, odsłaniając szereg migoczących czerwienią zębów, niczym umoczonych w krwi tętniczej.
\begin{dialogue}
	\ds{} A co to w takim razie jest, proszę piekarza? \dm{} Podsunął mu znalezisko pod nos.
	\ds{} Sól.
	\ds{} Sól? A słyszałeś może także określenie ,,biała śmierć?''
\end{dialogue}
Piekarz przewrócił oczyma i westchnął.
\begin{dialogue}
	\ds{} Zagadka mówiła, że trucizna miała być dodana do soli, zobaczymy zaraz, ile w tym prawdy. 
		\dm{} Wsadził mu pudełko w twarz. 
		\dm{} Jedz tę swoją truciznę!
\end{dialogue}
Poruszenie wśród ludzi.
Ktoś podniósł rękę, ktoś chrząknął, ktoś się cofnął o parę kroków, ktoś zemdlał.
Chyba nie za bardzo im się podobało takie traktowanie innych. Każdy z nich wiedział, że następnym razem sam mógł skończyć na miejscu piekarza.
\begin{dialogue}
	\ds{} Dzisiejsza noc jest specjalna. 
		\dm{} Kostuch sączył ludziom truciznę do uszu. 
		\dm{} Dzisiaj bowiem rozwiązujemy Zagadkę. Dzisiaj uwalniamy się spod władzy Zegara.
\end{dialogue}
Kilka osób popatrzyło się na wieżę, wieża popatrzyła się z powrotem na nich. Błysnęła złotym okręgiem tarczy. Mam was na oku.
\begin{dialogue}
	\ds{} Jeśli puścimy piekarza wolno, zginie dzisiaj więcej niż jedna osoba, zginą wszyscy, którzy kupują poranny chleb od tego truciciela.
\end{dialogue}
Ktoś chciał protestować, ktoś chciał ocalić torturowanego biedaka i swoje poranne śniadanie.
Ale kilka osób odwróciło się w jego kierunku z takim samym wyrazem twarzy, jak u łysola, więc zaraz ulotnił się ze zbiorowiska.

I tym sposobem torturowanemu wmuszono na raz cały kilogram soli.
Niedługo potem dostał drgawek, co triumfalnie ogłoszono jako dowód, że w soli była trucizna.
Nie wszyscy w to uwierzyli, ale rzekomy truciciel pilnowany był do końca, żeby nikt przypadkiem nie mógł zweryfikować niepowtarzalności zdania białogłowego.

Zostawili jego truchło w zimnym błocie, a księżyce przykryły scenę fioletowym całunem.
Zegar nadal tykał, ale też nikt nie wiedział, co miałoby się z nim stać po rozwiązaniu zagadki.
Zatrzymać się? Wybić pochwalną melodyjkę? Zapaść pod ziemię? A może dać punkt i kontynuować wiersze następnej nocy?
Nie przeszkadzało to bandzie w ogłoszeniu sukcesu i świętowaniu zwycięstwa.

Słońce wstawało, zalewając miasto bielą.
Biały karzeł tego systemu planetarnego był małą gwiazdką, przebijającą się ledwo przez chmury.
Jasność osuszała atrament i zwijała w kłębek fioletową kołdrę.
Błoto zamykało swe ślepia, a metalowe domy poczęły się skrzyć.
Noc nie chowała się, jak zwykle, po kątach. Uciekła w całości za horyzont, wygoniona światłą nadzieją.
Nawet mroczna wieża zrobiła się mniej mroczna, chociaż nadal rzucała na miasto cień przypominający wszystkim o jej istnieniu.
Jedynie piekarz na środku ulicy był trochę nie na miejscu.
Ale czy warto było się nim, przejmować? 
Dzisiaj ten, jutro ktoś inny.

I wtedy zegar uderzył dzwonem tak mocno, że gdyby ludzie mieli w oknach szyby, to już by ich nie mieli.
Idący przystanęli na chwilę, pomacali się po sobie, jakby chcieli sprawdzić, czy nadal żyją.
Ktoś sobie zbadał puls.
Po kilku sekundach, rozluźnieni, kontynuowali swoje wędrówki.
Nawet słychać było gdzieniegdzie jakieś śmiechy.
Zaczęła się ta spokojniejsza część doby.

Grupka osób biegła niespiesznie, jakby chciała zdążyć na pociąg, który i tak ma kilkugodzinne opóźnienie.
Człowiek na czele trzymał w ręce zwój papieru.
Do grupki dołączali się też inni, zaciekawieni rozwiązaniem zagadki.
Rodzinka Nocnych skorzystała z okazji.

Tego poranka zegar końcowym uderzeniem zabił jakąś prostytutkę z domu publicznego na skraju miasta.
Zemdlała w trakcie pracy i nie obudziła się już. W jej ręce znaleziono ten zwój.
I tak ginął ktoś każdego dnia.

Przybiegli do domu kowala.
Ów siedział przed domem i popatrzył się na przybyszów w taki sposób, w jaki trener patrzy się na swojego zawodnika, który przybiegł na metę ostatni.
\begin{dialogue}
	\ds{} Moja żona umarła dzisiaj rano. \dm{} 
		Nie widział, co było zapisane na papirusie, a jednak doskonale znał jego treść. \dm{} 
		Została otruta naszyjnikiem z masy solnej. Ktoś jej go podarował. \dm{}
		Skupił swój wzrok na właścicielu kartki, tak jak lupą ogniskuje się światło słońca w celu podpalenia czyjegoś domu. \dm{} 
		Na pewno nie był to piekarz, którego zamordowaliście dzisiaj w nocy. Więc kto?
	\ds{} Ja to zrobiłem. \dm{} 
		Aptekarz wyszedł z domu obok. \dm{} 
		Ta jędza wpajała naszym dzieciom w szkole, że Zegar jest jakimś rodzajem boga. Że trzeba mu oddawać cześć. Bóg jest tylko jeden, ta wieża jest jego dokładnym przeciwieństwem! \dm{} 
		Kilka osób mimowolnie popatrzyło się na omawianą budowlę, złoty okrąg wokół tarczy błysnął, jakby chciał zaprzeczyć jego słowa.
	\ds{} A kowadło? A pudełko? \dm{} 
		pytali jeden przez drugiego.
	\ds{} Kowadełko. \dm{} Aptekarz wskazał palcem na ucho. \dm{} Pudełko. \dm{} Wskazał na swoją głowę. \dm{} I trucizna w środku.
\end{dialogue}
Ludzie czytali z papirusu i milcząco przytakiwali.
\begin{dialogue}
	\ds{} Zaczynają na głównym placu budować kapliczki. \dm{} 
		Nakręcał się dalej. \dm{}
		Składają dary, prosząc o łatwą zagadkę. Nie możemy pozwolić, żeby tak dalej było! To on. To Zegar zabił wszystkie trzy osoby. Waszymi i moimi rękami. Nakręcany diabeł.
	\ds{} Zabiłeś w taki sam sposób. Jak śmiesz nas pouczać? \dm{} 
		ktoś z tłumu wykrzyknął.
	\ds{} Zabiłem, żeby nasze dzieci nie zabijały w imię mechanicznego szatana, śmierć za więcej śmierci. Każdy z was zrobiłby to na moim miejscu.
	\ds{} Sam jesteś szatanem. 
	\ds{} Przynajmniej nie torturowałem piekarza, zrobiłem to humanitarnie, dla dobra nas wszystkich.
	\ds{} Chwila, chwila. Jeśli znałeś rozwiązanie, to sam mogłeś pokonać zegar, tak?
	\ds{} Ja ten... tego...
	\ds{} A może zrobił sobie z zegara wymówkę, żeby celowo ją zabić?
\end{dialogue}
I zaczęli się przepychać.
Nauczona doświadczeniem rodzinka Nocnych powoli się wycofywała.
Wrzaski słychać było nawet za rogiem, ciekawe czy kolejna osoba znowu pożegna się z życiem.

Rafał Nocny umieścił w kwadratowym urządzeniu różaniec, buteleczkę wody święconej i kasetkę z piórem ze skrzydła Archanioła Gabriela, patrona telekomunikacji.
Bogofon zatrzeszczał, zawibrował i na ekranie ukazała się trójwymiarowa postać.
To był przystojny młodzieniec, miał niemal przezroczyste włosy, błyszczące w ostrym świetle, oraz białą szatę ze srebrnymi akcentami z pereł i diamentów.
Wyglądał jak doskonały człowiek, może nawet aż za bardzo doskonały.
Za nim znajdowały się rzędy kolorowych kul, wokół których chodzili inni aniołowie.
\begin{dialogue}
	\ds{} Niebiański departament symulacji alternatywnych wersji wszechświata, w czym mogę pomóc? 
		\dm{} Rozległ się anielski głos, niczym grany na trąbach w akompaniamencie tysiącosobowego chóru. Pracownik raju popatrzył się swoimi niebieskimi oczyma wprost na dusze rozmówców.
		\dm{} Och, to wy, hejka. I jak tam wasza ludzka egzystencja, moi dzielni wojownicy uniwersalności? Nasze perpetuum mobile w waszej rakiecie nie zawodzi?
	\ds{} Znaleźliśmy zegar 
		\dm{} wypaliła niespodziewanie Nadzieja.
	\ds{} O, to miło. 
		\dm{} Uśmiechnął się nieco drwiąco, pokazując idealnie białe zęby. 
		\dm{} Na samej Ziemi znajduje się kilka miliardów różnych zegarów, rybko. Chciałabyś może doprecyzować, czy mam je wszystkie wymienić?
	\ds{} Ale ten zadaje zagadki 
		\dm{} dodała Maria Nocna.
	\ds{} I morduje ludzi 
		\dm{} wspomniał Rafał.
	\ds{} I gada po polsku. 
		\dm{} Żywia się ożywiła.
	\ds{} Hmmm... 
		\dm{} Anioł zmarszczył czoło na którym nie pojawiła się ani jedna zmarszczka.
		\dm{} Jak na uniwersalność, to strasznie nieuniwersalne to wasze znalezisko. I jeszcze ten polski język. Spauzujcie na kilka pulsów, coś mi świta.
\end{dialogue}
Rodzinka popatrzyła się po sobie pytająco. Zza kamery dał się słyszeć szelest przerzucanych ksiąg.
\begin{dialogue}
	\ds{} Hitler wygrywa wojnę, to nie.
	\ds{} Ludzie znoszą jaja? To też nie.
	\ds{} Derdenole chomipują fizultanowe bugysty? Kilisto.
	\ds{} Nasz ulubieniec wygrywa wybory. Fajne, ale nie.
	\ds{} Atomy poruszają się pod wpływem miłości? Nic z tego nie wynikło.
	\ds{} Wymuszone ćwiczenia inteligencji. Może to?
\end{dialogue}
To, na co trafiła rodzina Nocnych tym razem, to nie była surowa uniwersalność, tylko jeden z wyciekłych niebiańskich eksperymentów.
Anioł obrócił swój bogofon, pokazując resztki słoja do symulacji wszechświatów.
Jeszcze nie posprzątali do końca nieskończonego laboratorium po katastrofie roku zerowego.
Uniwersalność jest umieszczana w tych słojach przed rozpoczęciem symulacji, to coś jak komórki macierzyste.
Kiedy główny zbiornik tej surowej losowości wyciekł i zalał wszechświat, trochę normalnych symulacji też popękało.
\begin{dialogue}
	\ds{} Na trop nakierował mnie ten wasz język. W Niebie często posługujemy się słownictwem Narodu Zapasowego. Dużo bardziej przyszłościowy niż łacina, wiecie? I tak fajnie trzeszczy, odstrasza diabły.
		\dm{} Niebiański urzędnik trochę zboczył z tematu.
		\dm{} W każdym razie... ,,celem tego eksperymentu było zbadanie, jak rozwija się cywilizacja w środowisku w którym co noc wymuszona jest ścisła współpraca pomiędzy obcymi jednostkami w rozwiązywaniu codziennej zagadki''
		\dm{} czytał.
		\dm{} Kto wpadł na pomysł, żeby to było poprzez wiecznie wiszącą nad nimi groźbę śmierci? Kto stworzył ten eksperyment? Zobaczmy. Hmmm... tu pisze że ja.
	\ds{} I jaki był wynik? 
		\dm{} Rafał się zapytał.
	\ds{} To ja powinienem się was zapytać.
\end{dialogue}
Rafał odwrócił bogofon w stronę miasta. Anioł prześwietlał przez chwilę zabudowania swoim wzrokiem.
\begin{dialogue}
	\ds{} Do stu Judaszy, to się na mnie patrzy...
	\ds{} A no. Na wszystkich się tak patrzy. I wszystkich może w każdej chwili zamordować.
	\ds{} Jak w takim zagadkowym świecie radzą sobie mieszkańcy? 
		\dm{} Dało się słyszeć tarcie pióra o papirus.
	\ds{} Dzisiaj zginęły trzy osoby. Może cztery. Tylko jedna z nich była ofiarą zegara. Odpowiedz sobie sam na to pytanie.
	\ds{} No widzicie, a gdyby ludzie współpracowali ze sobą, zamiast się nienawidzić, to może rozwiązaliby zagadkę na czas i nikt by nie ucierpiał, co nie?
		\dm{} Chrobotanie zwiększyło wysokość. \dm{} A jak wspaniale przećwiczyliby sobie przy okazji mózgi, pomyślcie. Świat geniuszy.
	\ds{} Nie wynaleźli jeszcze elektryczności, a błota jest tu więcej niż twardego gruntu. Jest też jakaś grupa dresów, co udaje zbawicieli świata. 
		Wczoraj zatorturowali niewinnego człowieka na oczach niewzruszonych gapiów, w imię rozwiązania zagadki.
	\ds{} Tak. Była możliwość, że tak się będzie działo. \dm{} Anioł popatrzył na wszystkich ludzi z wyrzutem. \dm{} Zapisuję eksperyment jako nieudany. Jak zwykle. Teraz najlepsza część, apokalipsa. Możecie zniszczyć ten zegar, albo lepiej i cały układ.
	\ds{} Rozwiązując zagadkę?
		\dm{} Żywia miała siostrę w głosie.
	\ds{} Myślałem o bombie sacroteriowej, tej co na raz znika całe systemy gwiezdne. Ale jak wam się aż tak nudzi, to manualnie też można.
	\ds{} Mielibyśmy zamordować tych wszystkich ludzi? Tak po prostu? \dm{} Rafał się oburzył.
	\ds{} To nie są ludzie. Nie mają powodu istnienia. Nie posiadają dusz, jedynym celem ich ,,życia'' była symulacja. To tylko trochę skonsolidowanej uniwersalności. Wielkie mi halo.
	\ds{} I to niby my jesteśmy tymi złymi?
	\ds{} Ech, znowu zaczynacie. 
		\dm{} Westchnął organową muzyką.
		\dm{} Każda z tych kul za moimi plecami ma septyliardy istnień. 
		\dm{} Nawet się nie odwrócił. 
		\dm{} I jest przecież ta jedna symulacja, która odtwarza nasz aktualny wszechświat, w tym siebie samą aż do nieskończoności. Na prawdę chcesz kolejny raz przez to mentalnie przechodzić?
		\dm{} Przeczytał mu w myśli odpowiedź na to pytanie.
		\dm{} Zresztą, w tej galaktyce nikt inny nie mieszka, róbcie z nimi co chcecie.
		\dm{} Zatrzasnął księgę. 
		\dm{} Dla nas już nie istnieją. Macie przecież wolność, którą tak chętnie sobie zerwaliście z drzewa. Pobawcie się w lokalnego boga.
\end{dialogue}

Wychodząc ze ślepej uliczki, Rafał zapytał się przechodnia, czy wie może kto jeszcze poległ w szarpaninie przed domem kowala. 
Okazało się, że zadźgano dwie osoby, a jedną zadeptano. I na koniec kowal powiesił się ze smutku. To było razem siedem osób, dużo.
Kobieta się zaśmiała, dzisiaj i tak wypadło mocno poniżej średniej.

Wyglądało na to, że to wcale nie zegar był największym mordercą w tym mieście.
Tarcza na wieży przytakująco odbiła światło białej gwiazdy.

W nocy odbyło się kolejne zebranie na głównym placu.
Niebieski sierp księżyca świecił teraz znacznie słabiej, dawał na niebie miejsce dla swojego czerwonego przeciwieństwa.
Ta czerwień nie była już łuną krwi, jak poprzednio, była ostrzeżeniem.
Była alarmem na tonącej łodzi podwodnej.
Uwaga, dzisiaj wszyscy zginiecie!

\begin{poem}
	Posłuchajcie uważnie moje dziatki \\
	wybitej o północy zagadki. \\
	Przed wschodem słońca znajdźcie rozwiązanie, \\
	albo komuś coś się dzisiaj stanie. \\
	Ding-dong. \\
	Budowniczy wsiąkł. \\
	Szukacie, a znajdziecie. \\
	Wszystkich jego dzieci. \\
	Dokąd sobie poszedł? \\
	Wziął ze sobą kalosze. \\
	I oto historia jest cała. \\
	Znajdźcie tego bałwana. \\
\end{poem}

Zgraja dresiarstwa nadal stała w miejscu, czekając, co powie im ich jedyny mózg.
W tej zagadce trzeba było znaleźć jakiegoś budowniczego, który sobie gdzieś poszedł.
Zwołano więc z miasta wszystkie rodziny, w których byli jacyś konstruktorzy, pytając się, czy żadnych członków im nie brakuje.
Jak jeden zaprzeczyli.

Ktoś podsunął pomysł, że budowniczy nie musiał mieć rodziny, mógł sam wychowywać dziecko.
Więc poproszono, właściwie zagoniono, wszystkie dzieci w miasteczku na główny plac, wraz z rodzicami.
Ale szybko porzucono ten pomysł gdy okazało się, że musiałoby przyjść pół miasta.
A zegar nieubłaganie tykał dalej.

A może dzieci symbolizowały zbudowane konstrukcje? 
No, ale jak znaleźć budowniczych każdej, nawet najmniejszej rzeczy w mieście?

Trzeba odszukać kalosze. Jak będą kalosze, to budowniczy będzie obok nich. Wspaniałe rozwiązanie.
I tak sprzeczali się i sprzeczali.
I nic z tego nie wychodziło.

Nocni poszukali w mieście jakiegoś muzeum.
Znaleźli jedno, które robiło także za bibliotekę.
Było strasznie zatłoczone, bowiem każdej nocy wiele osób przychodziło tutaj, aby spróbować wynaleźć swoją własną broń na zegar.
Biblioteka posiadała spis wszystkich zagadek i ich rozwiązań od początku istnienia miasta, to jest przez jakieś dwa tysiące lat, od czasu katastrofalnego w skutkach wycieku uniwersalności i uwolnienia się tej symulacji ze słoja.
Zaglądając do pierwszych rękopisów, jakoś nie było widać, żeby ten świat w jakikolwiek sposób się w historii zmienił. 
Żadnych nowych badań, żadnych przydatnych technologii.
Świat był do bólu statyczny, może dlatego że mieszkańcy co noc tracili czas na walkę z zagadkami, zamiast zajmować się nauką i samodoskonaleniem?
Może jeśli wiesz, że każda noc może być twoją ostatnią, to nie ma sensu nic budować?
Zabawne, przez tyle lat ci ludzie nie potrafili się zjednoczyć.
Ale z drugiej strony, czy jakakolwiek cywilizacja we wszechświecie tego dokonała?

Rodzice wczytali się w księgi historyczne.
\begin{dialogue}
	\ds{} A może chodzi o prawdziwego bałwana? \dm{} przerwała im znudzona Nadzieja, widząc jak inne dzieci lepią na ulicy struktury z wszechobecnych kawałków rozszarpanych trucheł.
	\ds{} Ale kto konkretnie miałby być budowniczym bałwana, córciu? Każdy mógłby być, zobacz ile tutaj dzieci \dm{} Maria Nocna się zaśmiała.
	\ds{} Nie, bałwan ma być budowniczym.
		\dm{} Głupi dorośli, znowu nie rozumieją.
	\ds{} Bałwany nie budują bałwanów, głuptasku...
	\ds{} Mogą budować jak są duże. I mogą też wsiąkać.
	\ds{} Ale o co ci chodzi?
	\ds{} Chyba wiem o co chodzi Nadziei. 
		\dm{} Tata zamknął kilkusetletni tom. 
		\dm{} Załóżmy tak. Gdyby topiący się pod wpływem wiosny bałwan rozpadł się na kilka mniejszych kawałków, to te odpadłe bryły byłyby formalnie jego dziećmi. Prawda? A ponieważ jak śnieg się topi, to przy okazji robi pod sobą wielką kałużę błota i może całkowicie sam w niej zniknąć, zostawiając swoje niestopione fragmenty wokół.
	\ds{} I kalosze! \dm{} Żywia zawołała. \dm{} I miotłę, i garnek, i marchewkę!
	\ds{} Nie wiem, czy na tej planecie rosną marchewki.
	\ds{} Utopiły się w błocie, musimy znaleźć i odbudować bałwana! \dm{} Dzieci wybiegły na ulicę, przerażeni rodzice za nimi. Tyle razy powtarzali żeby nie biegać po obcych planetach, ale jak do ściany.
\end{dialogue}

Przez resztę nocy dziewczynki szukały po całym mieście truchła wspomnianego bałwana, a rodzice szukali dziewczynek.
Dopiero, gdy się rozjaśniło, spotkali się, zupełnie przypadkowo nad kałużą, która idealnie odpowiadała ich przewidywaniom.
\begin{dialogue}
	\ds{} Co my mówiliśmy o oddalaniu się w taki sposób?
	\ds{} A gdyby was ta zgraja dopadła? Pomyśleliście o tym? \dm{} rodzice rzucali rodzicielskie gadki.
	\ds{} Nie ma czasu, trzeba rozwiązać zagadkę \dm{} Żywia próbowała się tłumaczyć.
	\ds{} A gdybyśmy was nie znaleźli, to co byście zrobiły na tej obcej planecie?
	\ds{} Nadzieja, zobacz, tu wystaje garnek.
	\ds{} Byliśmy na tylu niebezpiecznych światach, a wy się nadal niczego nie nauczyłyście?
	\ds{} Nie mogę dosięgnąć, ty spróbuj, masz dłuższą rękę.
	\ds{} Możecie zapomnieć o wizycie w Pałacu Nadiru, powiemy królowi Freonowi, jak się zachowywałyście.
	\ds{} A patykiem?
	\ds{} I nie dostaniecie już więcej kryonitowych deserów od niego.
	\ds{} Czekaj, coś miękkiego tam.
	\ds{} Ferro już nie będzie z wami latał...
\end{dialogue}

Uderzenie dzwonu zatrzęsło światem.
Koniec czasu uciszył wszystkich.
Rodzice spojrzeli na swoje pociechy, a potem na kałużę otoczoną kilkoma białymi grudkami.
Rafał zanurzył rękę w błocie.
Wyciągnął parę brudnych kaloszy.
Westchnął.
Niedługo to trwało, aż przybiegła do nich jakaś obca osoba. W ręku trzymała zwój papieru.
Studiowała długi czas zastaną scenę, patrząc to na rodzinę, to na kałużę, to na ubłocone kalosze, to na swój zwój.
Potem gdzieś pobiegła.
A nasza rodzinka, nie mając ochoty na bycie celebrytami, ulotniła się, jak to już od dawna miała dobrze wyćwiczone.

Nocni, nauczeni porannym doświadczeniem, wzięli ze swojej rakiety worek nanobotów i rozesłali po całym mieście, aby zmapowały im teren. 
Każda kałuża, każde okienko w każdym domu miało być wpisane do pokładowego komputera.
Zegar także próbowały zbadać, ale nie znalazły w nim żadnego otworu. 
Wieża okazała się całkowicie zamkniętą bryłą.
Ta zabawa pochłonęła im cały dzień.

Zobaczymy, jak nędzna, uniwersalna mechaniczność poradzi sobie z doskonałą formą elektroniki.
Teraz Nocni będą wiedzieć wszystko i będą jednocześnie wszędzie.
Żadne sztuczki z szukaniem igieł w stogach siana, albo kaloszy w kałużach błota, nie przejdą.

Trzecia noc była tak nieprzyjemna, jak iloczyn dwóch poprzednich.
Gęste chmury nie przepuszczały żadnego światła, nawet tej wściekłej, awaryjnej żarówki na niebie.
Jedynie żółta tarcza zegara delikatnie oświetlała miasto.
Teraz to ja jestem waszym słońcem.

Oberwanie chmury, błotne ulice zamieniły się w potoki.
Mglista firana zakleiła miasto, niczym pająk oplatający siecią swoją ofiarę.
Prawie wszystkie nanoboty ugrzęzły w bagnie lub zostały spłukane w żarłocznym, wszechobecnym cieniu.
Nie zemną takie numery.

Czas się zbliżał.
Drgające połyski okręgu zdawały się wskazywać, że rechocze.
Zarys wieży zniknął w atramentowej chmurze, była tylko świecąca tarcza, unosząca się nad ludźmi, niczym ich nowa i jedyna gwiazda.
Na zawsze.
Mieszkańcy stali na głównym placu, przyklejeni do deszczowej pajęczyny, nie spodziewali się, że i tym razem będzie jakkolwiek inaczej.
Legenda o poprzedniej rozwiązanej zagadce zgasła, jak świeczka rzucona w błotną otchłań. 
Zgasła, połknięta przez horyzont zdarzeń cienia.
Nikt nie uwierzył, nikt nie miał nadziei.
Nawet Rafał Nocny.

Zegar zadzwonił.

\begin{poem}
	Posłuchajcie uważnie moje dziatki \\
	wybitej o północy zagadki. \\
	Przed wschodem słońca znajdźcie rozwiązanie, \\
	albo komuś coś się dzisiaj stanie. \\
	Dzyń-dzoń. \\
	Nie dostrzeżecie go. \\
	Jest pewien byt, \\
	co wszystkich morduje w mig. \\
	I to nie jestem ja. \\
	Chociaż reszta się zgadza. \\
	Dzwoni dzwonkami. \\
	Zagadki rozdaje. \\
\end{poem}

\begin{dialogue}
	\ds{} Inni ludzie! \dm{} inni ludzie powiedzieli to jednocześnie.
	\ds{} Bóg? \dm{} ktoś zaproponował.
	\ds{} Jakiś Inny zegar?
	\ds{} Strach?
	\ds{} Wiatr.
	\ds{} A my sami?
\end{dialogue}
Ale kolejne pomysły grzęzły w pajęczynie bezsensowności coraz bardziej.
Ktoś zrezygnowanie usiadł w błocie.
Ktoś zemdlał z beznadziejności i się utopił.
Ktoś celowo sobie żyły podciął.
Widać śmierć mogła przyjść nie tylko od innych ludzi, czy zegara, ale także od ciebie samego.

I musieli wszyscy tak stać.
I nic nie mogli zrobić.
I nawet jak próbowali, to kończyło się to fiaskiem.
Bo wszyscy byli splątani deszczową pajęczyną, która sklejała ich mózgi.
A oko wielkiego pająka bacznie im się przyglądało.
Nie pozwolę wam umrzeć, jeszcze nie, cierpcie.

Tylko Nocni potrafili się z tej pajęczyny zerwać.
Nie dlatego, że mieli doświadczenie, inteligencję, czy chęć do życia, a dlatego że byli oszustami.
Bo za rogiem czekała na nich rakieta i kilka sacroteriowych bomb.
Bo ich historia w każdej chwili mogła się zakończyć widowiskowym \emph{deus ex machina}.
Oni jedyni byli w stanie po prostu sobie pójść i zostawić tych ludzi i tego mechanicznego demona w cholerę.

I między innymi dlatego postanowili zostać i się nie poddawać.
Bo jakby poszli, to dzisiejsza pajęczyna zmieniłaby się w jutrzejsze skamieliny.
A za kolejne dwa tysiąclecia ludzie nadal spotykaliby się na tym placu i nadal próbowaliby rozwiązać kolejną zagadkę.

Nawet nie dało się zauważyć poranka.
Z cebra nie przestawało się lać.
Tylko wieża pokazywała poranną godzinę. Albo taką chciała nam przekazać.

Gdy każdy zwymiotował wszystkie swoje pomysły i u nikogo nie było już kwasu żołądko...
\begin{dialogue}
	\ds{} ...pisarz opowiadania.
\end{dialogue}
Wszystkie oczy zwróciły się na małą Żywię, która była zaskoczona tak samo, jak oni.
\begin{dialogue}
	\ds{} No bo to pisarz to pisze zagadki i dzwoni dzwonem, jak pisze ,,bim-bom''.
\end{dialogue}
Niektórzy ludzie się zaśmiali, trochę przez łzy, a trochę przez deszcz.
Jednak dopiero po minucie zauważyli, że nie ma nad nimi ich mechanicznego słońca.
Rozglądali się, szukając tak znienawidzonej przez wszystkich tarczy, lecz niebo było całkowicie czarne.
Lekki dźwięk pękania coraz bardziej zagłuszał szum deszczu.
\begin{dialogue}
	\ds{} Wszyscy uciekać! \dm{} krzyknął Rafał, ciągnąc za sobą swoją rodzinę.
\end{dialogue}
Nie wiadomo dlaczego, ale inni ludzie również zaczęli w pośpiechu opuszczać plac.
Potykali się o swoje nogi.
Gubili kalosze w kałużach.
Nawoływali się nawzajem.
Próbowali świecić pochodniami, które natychmiast gasły.

Wtem było wielkie uderzenie i fala błota znikąd.  
Porwała wszystkich biegnących.
Rzuciła w wąskie uliczki, wpadła do domów, gasząc latarnie, rozerwała pajęczynę na strzępy.
Przytłumiony dźwięk dzwonu rozszedł się, po raz pierwszy, po ziemi.

I tym razem nikogo nie zabił.


















