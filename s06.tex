\chapter{Bankiet w Kuli}

\info{Wioska jest atakowana przez literackie demony. Tylko dzielny, acz niedoświadczony wojownik jest wstanie je pokonać, używając do tego
swojego miecza kreatywności i worka z alternatywnym wszechświatem.}

Gdy tylko czubek głowy Wielkiego Neofantasora wyłonił się zza gór, odpalono armaty załadowane gorącym atramentem.
Katapulty wystrzeliły litery, a papierowe samoloty wzniosły się w przestworza.
Wszystko na darmo, gigant ani drgnął.

Wojownicy z wioski, uzbrojeni w pióra i kałamarze, ruszyli do boju.
Atakowali wroga, używając całego swojego leksykalnego przeszkolenia. Opowiadania cyberpunkowe, fantasy i fantastyka naukowa cięły powietrze.
Horrory, wiersze i dzieła detektywistyczne rozbijały się o jego ciało.
Jednak niewzruszony Neofantasor pozostawał niewzruszony.
Stanął przed miasteczkiem i wysypał z rękawa, niczym wytrzepując piasek z buta po całodniowym siedzeniu na plaży, dziesiątkę straszliwych demonów.
Jego posłańcy szybko i sprawnie rozprawili się z całą obroną wsi, rzucając po jednej gwiazdce, w każdego z wojaków.
Nikt z obrońców nie przeżył, wieś stała się bezbronna.

Ale jak to w baśniach zwykle bywa, znalazł się młody syn doktora --- Antyrax.
Trochę ułomny językowo i bez jakiegokolwiek doświadczenia, postanowił zginąć dzisiaj w walce.
Nie miał, jak wszyscy, zbroi, pióra, czy zapasu atramentu.
Posiadał jedynie szklany miecz, wypełniony płynną kreatywnością, dmuchawkę z serum śmiechu, lateksowe buty i worek z alternatywnym wszechświatem.

Najbliższa demonica, Obudzona, została z zaskoczenia kopnięta z gumowego buta prosto w zadek i nawet nie poczuła tego subtelnego ataku.
Dopiero środek rozśmieszający, wstrzyknięty centralnie w szyję, zwrócił jej uwagę.
Wtedy Antyrax sięgnął do swojego worka i wyciągnął... rzecz.
Nieśpiąca popatrzyła na zagubionego wojownika, to na dziwaczny przedmiot, to znowu na wojownika. 
Zamrugała, nie rozpoznając zupełnie, co to jest i czy się tego bać.
Ostatni pisarz też obracał w dłoni i oglądał znalezisko z każdej ze stron, ale i on nie wiedział, co właściwie właśnie wydobył.
Wywalił nieokreśloność za siebie i spróbował ponownie.

Po kilku minutach wyrzucania różnych, nieokreślonych obiektów ze swojego wszechświata, spostrzegł że grupka demonów go otoczyła i z zaciekawieniem obserwuje jego zmagania z samym sobą.
Nie umiał używać własnego wszechświata. Co za młot. 
Postanowił więc wykonać swoją powinność w tradycyjny sposób, zamachnął się i wbił swój miecz w najbliższego neofantasorowego sługę, że ten aż odskoczył z bólu. 
Stare, sprawdzone metody zawsze działają.
Niedługo cieszył się z sukcesu, zaraz wszystkie straszydła razem wzięły i dmuchnęły w niego strumieniem przecinków --- jego największą słabością.
Śmiercionośna chmura przypominała bożonarodzeniowy śnieg, lecz przynosiła ze sobą same rózgi.

Pierwszy lecący znak interpunkcyjny ominął, podskakując na lateksowych butach, niczym akrobata nad wanną krokodyli. 
Drugi skontrował kreatywną bronią, jak kucharz rozplatający marchewkę na pół.
Trzeci złapał w zęby, korzystając z faktu że w książkach wszystko jest możliwe.
Jednak tysiąc kolejnych pocisków odrzuciło go na kilkaset metrów w tył.
Upadł obolały w błoto, po przeciwnej stronie wioski, nad którą chyba właśnie przeleciał. Jakimś cudem nic sobie nie połamał. Błotne kąpiele podobno dobrze działają na skórę.
Lecz to nie był koniec kontrataku, spadający błąd ortograficzny prawie rozciął go na pół, Antyrax zdążył wykonać unik, lecz przez przypadek puścił szklaną broń. 
Pękła na gadzilion drobnych kawałeczków, zmiażdżona przez spadające ,,ó''.
Kreatywność wsiąkła w mokry grunt, tworząc nowe królestwo ziemnych mrówko-androidów, zasilanych parą z wody basenowej.

Niedoszły bohater był teraz bezbronny, niczym cenna starożytna waza, na której wystawę wpuszczono wycieczkę przedszkolnych dzieci. 
Tak bezbronnym człowiekiem żaden z demonów się już więcej nie interesował. Pisarz wstał i zobaczył, jak jego osada jest równana z ziemią.
Cień Wielkiego Neofantasora, wychylającego się zza góry, wyglądał jak dłoń dziecka, które zgarnia klocki z dywanu, aby je zaraz obślinić i połknąć.
Czy to był już kres jego egzystencji i wszystkiego co znał?

\begin{dialogue}
\ds{} Nie, to nie miało się tak skończyć. \dm{} Sięgnął po worek. \dm{} Ja tu jestem pisarzem i to ja ustalam epilog. 
\end{dialogue}

Wtedy jego lateksowym butom wyrosły śmigła, wspomagane programem w C, i poniosły go prosto pod nos Obudzonej, zostawiając za sobą linuksowy ślad.
Zaciekawiona demonica rozpoznała niedojdę. Tego samego niedojdę. Pamiętała że wcześniej go zmiażdżyła.
Antyrax nie miał już żadnej broni, tylko swój wór. Zajrzał do środka i długo nie wyciągał ręki.
\begin{dialogue}
\ds{} Mam coś specjalnie dla ciebie, Nieśniąco.
\end{dialogue}

\divider{}

\begin{dialogue}
\ds{} Kolejny dzień w pracy, kolejny dzień robienia bezużytecznych czynności dla bezużytecznej korporacji w tym bezużytecznym świecie \dm{} pomyślałem.
\end{dialogue}

Wziąłem komórkę i zacząłem przeglądać maile. Kupa elektronicznej poczty czekała na spuszczenie w wirtualnym sedesie.
Spam z reklamą talerzy, zaproszenie do znajomych na Facebooku, oczywiście od obcej osoby, powiadomienie o komentarzu na YouTube,
uwaga o mailu na innej skrzynce emailowej, przypomnienie o opłacie za subskrypcję tej bezużytecznej gry komputerowej i tysiąc innych bezużyteczności.
O, znowu rozpętałem gównoburzę na Mirko i zablokowali mi konto Twittera za niepoprawną politycznie myślozbrodnię.
Bezużyteczność do kwadratu.

Z nadmiaru bezużyteczności zdrzemnąłem się w ubraniu na godzinę. Obudziło mnie głośne gruchanie z okna.
Czyżby jakaś użyteczność w końcu mnie spotkała?
Na oknie siedział biały gołąbek, w dzióbku trzymał kopertę. Było to tak abstrakcyjne, jak literacka wojna w wiosce pisarzy, wojna na średniki i znaki interpunkcyjne.
Nie będąc pewnym, czy nadal nie śnię, odebrałem przesyłkę i przyjrzałem się tekturowemu prostokątowi.
Ptasi listonosz chwilę potem zniknął, oddalając się bezszelestnie.

Całość była wykonana z papieru czerpanego.
Adresowana była elegancką kursywą do mnie, do Mateusza Mechalycznego, zamieszkałego na ulicy Szerokiej w Gdańsku.
Województwo Pomorskie, Polska, Ziemia, Układ Słoneczny, Galaktyka Droga Mleczna, Gromada Lokalna, Czwarta Kwadra.
Z tyłu widniała pieczęć z wosku pszczelego, wypukły obraz kuli i jakiś napis z niezrozumiałych znaków.
Przecież był 2017 rok, kto dzisiaj wysyła papierowe listy, zamiast maili?
Zatem to był kawał. To musiał być kawał. Pytanie jeszcze, po co ktoś mi go wyciął?

Ostrożnie otworzyłem prostokąt, trzymając go obcęgami, spodziewając się że coś strasznego z niego zaraz na mnie wyskoczy.
Nic takiego się jednak nie stało.
Pisany odręcznie list był tym, co zwykle znajduje się w kopertach. Nie, w sumie częściej dzisiaj znajduje się rachunki.

\curlyframe{
\begin{Fontlukas}
Szanowny Panie Mateuszu Mechalyczny.

Mam zaszczyt zaprosić Pana na uroczysty bankiet z okazji wyboru na jednego z nowych, obiecujących agentów ALOPP.
To wielka dla mnie radość, móc gościć ludzi z tej organizacji na pokładzie mojej \weirdchar{kula}.
Mam najskrytszą nadzieję, że zostanie Pan przyjęty i będzie mieć Pan we współpracy z \weirdchar{monster} udział w walce o wspólne dobro.

Zapraszam do siebie dnia 14 listopada, roku Pańskiego 2017.
Myślę, że marina w Głównym Mieście Gdańsku jest doskonałym miejscem na lądowanie \weirdchar{kula}. Tam się spotkajmy w lokalne południe.
Po obiedzie wybierzemy się w podróż na Felicję, gdzie pozna Pan swoich przyszłych, mam nadzieję, członków przybranej rodziny.

Przypominam, że w \weirdchar{kula}, oprócz najwyższej kultury osobistej,
od zawsze obowiązywał rokokowy styl ubioru.
Wszyscy goście powinni przywiązać niezwykłą dbałość o szczegóły swojego wyglądu.
Uprzejmie proszę także, aby nie posiadać na pokładzie żadnych urządzeń użytkujących elektryczność.

Z Bogiem. \\
--- Profesor \weirdchar{profesor}
\end{Fontlukas}}

Kilka dziwnych, okrągłych znaków zostało wtrąconych pomiędzy litery. 
Zaczynało się robić ciekawie. Autor tego dowcipu chciał, abym za trzy dni, w XVIII wiecznym stroju pałacowym,
udał się w sam środek miasta w dniach szczytu, nie zabierając ze sobą żadnej elektroniki.
Potem tajemniczo miałem przejść tajemniczy test na zostanie tajemniczym członkiem jakiejś tajemniczej społeczności. Cóż za tajemnicza tajemnica.
Trzeba odtajemniczyć ten tajemniczy list.

Szybkie szukanie Felicji w internecie wskazało jedną stronę o teoriach spiskowych.
Miała być planetką, stworzoną przez kosmitów, na której hodowano ludzi, aby przeprowadzać na nich straszliwe eksperymenty. Standard.
Jeśli to prawda, może zabrakło im tam królików doświadczalnych i porywają kolejne ofiary?
Ale wtedy przecież nie dawali by mi wolnej ręki do odmowy.

Na tym samym forum podano: ALOPP jest pozaziemską organizacją terrorystyczną, zrzeszającą Ziemian w celu mordowania mieszkańców własnego globu.
Ale od czego był to skrót, to nikt nie wiedział. Brzmiało ciekawie, pomordowałbym sobie niektórych ludzi.

,,Czwarta Kwadra'' dawała za dużo losowych wyników, aby wywnioskować z nich, o co mogło Profesorowi chodzić.
Ale pewnie były także trzy inne... jakieś kwadry.

Lokalne południe w Gdańsku, czyli dwunasta godzina czasu słonecznego, uwzględniając jeszcze czas letni, to trochę przed trzynastą według podziału strefowego.
Przyjdę o 12:00, najwyżej ciutkę poczekam.

Wikipedia natomiast wskazała, że gołębie pocztowe w żadnym wypadku nie mogłyby doręczyć listu bezpośrednio do odbiorcy.
Ich mechanika polega na wracaniu do macierzystego gołębnika, z dowolnego miejsca na świecie. I tylko tyle. Zupełnie jak sadełko z brzucha.
Listów z pewnością nie wsadzano im do dzióbka, a przywiązywało się je do nóżek.
Rozejrzałem się po pokoju, czy przypadkiem nie miałem w nim gołębnika, aby hodować tych lotnych listonoszy, ale nie.
Moja teoria o dowcipie zaczęła się lekko sypać. Jej drobinki wpadały mi do oczu, co wcale nie było miłe.

Kilka razy, w różnych częściach świata, widziano w tym samym momencie kuliste UFO i ludzi w strojach rodem z Wersalu. I nie było to w trakcie kręcenia jeszcze jednego odcinka Doktora Who.
Podobno zdjęcie zrobione kuli nigdy nie wychodziło poprawnie, a większość świadków magicznie zapominała o zdarzeniu chwilę po odlocie tajemniczej struktury.
Nieliczni pamiętali i rozpowiadali to dziwo, ale nikt im oczywiście później nie wierzył.
Kulę widywano w niezwiązanych ze sobą miejscach, nie ograniczała się, jak na filmach, tylko do USA, przelatywała przez centra miast, pływała pod wodą, cumowała do Międzynarodowej Stacji Kosmicznej, straszyła samoloty, ślizgała się po biegunowych lodach i toczyła bitwy z wojskami wszystkich krajów Ziemi. 
Zupełnie, jak w słabej jakości opowiadaniach fantastycznych, pisanych przez losowych ludzi na losowych stronach internetowych. 

Następnego dnia zabrałem list do znajomego chemika.
Potwierdził on moje obawy, list wykonany był oryginalną techniką sprzed kilkuset lat.
Skład chemiczny papieru i atramentu odpowiadał tym, używanym dawno temu.
W dodatku narzędzie pisania z pewnością było ptasim piórem.
Na myśl o bezużyteczności otaczającej mnie rzeczywistości, postanowiłem pojutrze zrobić coś użytecznego.

\divider{}

Pod naporem nietypowości i kreatywności dzieła, Obudzona zaczęła krzyczeć, zwijać się w konwulsjach i palić mrocznym ogniem, jak mrówka na płycie kuchennej.
Zmieniła się w mały księżyc i poleciała, obracając się jak frisbee, z powrotem w kierunku Wielkiego Neofantasora.

Triumf Antyraxa nie trwał długo, za chwilę, od tyłu, złapała go Dedirid.
Jej kosmata ręka owinęła się wokół delikatnego światostwórcy, jak czarny worek na zwłoki.
Poczęła go zacieśniać, niczym nawiedzony lekarz mierzący ciśnienie.
Zaraz opróżni wnętrzności naszego bohatera, jak tubkę pasty do zębów.
Wywijając się rybio, pisarz zanurkował do worka i nieprędko wyszedł.
Demonica przez godziny nachylała się nad otworem, aby capnąć go, jak tylko wystawi głowę.
Gdy w końcu coś się wysunęło, porwała to z ochotą.
Była to jednak kolejna opowieść, zabójczo eksperymentalna, niesamowicie abstrakcyjna.
Nietypowość poparzyła jej łapska.

\divider{}

Mateusz uzyskał wymaganą górę w wypożyczalni kostiumów.
Zastanawiał się, czy nie podkraść jakiegoś szustokora z muzeum, ale to chyba nie było by zbyt poprawne zachowanie.
Bał się, czy mierna jakość kaftana, spowodowana nieoryginalnością, nie będzie zwracać nadmiernej uwagi w świecie najprawdziwszych atłasowych pasów i perłowych guzików.
Postanowił kupić więc kilka ozdób ze sztucznej biżuterii, które wyglądały dość kosztownie, a stworzone były z byle-czego, i doszyć w losowych miejscach. 
Miał nadzieję, że Profesor i inni goście nie zauważą różnicy.

Z pończochami nie było żadnego problemu, znalazł je w damskim sklepie.
Tak samo coś, co można było podciągnąć pod starodawną koszulę.
Musiała być flanelowa, z wystającymi rękawami.
Ogarnął także puder.

Peruka, cóż. Przynajmniej miał trójkątną czapkę piracką zza zakurzonej szafy, jeszcze po poprzednich lokatorach.
Najgorzej, że zazwyczaj chodził na łyso, gdyż rodzice nie obdarzyli go mocnymi włosami.
Potrzebował więc na szybko przykleić coś sobie na łeb.
Liczył w głowie, ile lat może dostać za kradzież włosowej czapeczki sędziemu, gdy spostrzegł wyprzedaż starych futer.
Używając magii nożyczek, kleju i starego mopa, wygenerował coś, co po przykryciu jego trójkątnym nakryciem wyglądało dość znośnie.

U zegarmistrza kupił za grosze kopertę zegarka, pustą w środku, z brakującymi wskazówkami, całość zaczepioną na łańcuszku.
Mechanizm nie musiał działać, ważne aby otoczka się zgadzała.

O dziwo, to buty przysporzyły mu najwięcej problemu.
Niby lakierki z paskami nie są niczym bardzo skomplikowanym, a jednak nikt ich nie produkuje.
Może właśnie dlatego, że były modne trzysta lat temu?
Wpadł na pomysł, aby kupić coś podobnego i przerobić.
Znalazł buty dla zakładów pogrzebowych, gdyż tylko te odpowiednio się błyszczały, i przyszył im klamry od spodni.
Z daleka nie było widać różnicy.

W domu ubrał się i przejrzał w lustrze. Przeraził się swojej naiwności.
Połączenie Napoleona Bonapartego, Ludwika XIV i informatyka z Gdańska.
Muszą zrozumieć.

O jedenastej godzinie, owego wielkiego dnia, wdział pełny strój.
Nie mógł się przecież tak pokazać w mieście.
Pończochy zatem przykrył spodniami dresowymi.
Na elegancki szustokor nałożył nieco za dużą bluzę z kapturem.
Stopy wsadził do szmacianych worków, aby błyszczące lakierki nie zwracały zbytniej uwagi.
Pseudo-perukę schował do plecaka.

To nie mogło pójść tak łatwo.
Z daleka zobaczył kordon policji i wojska, stojący w Zielonej Bramie, niczym gęsty las blokujący drogę do magicznej polany.
Blokował wstęp każdemu wychodzącemu z Długiego Targu.
Ucieszył się i kamień spadł mu z serca. Oznaczało to, że jednak nie padł ofiarą żartu.
Znalazł w końcu promyk użyteczności w oceanie bezużyteczności. Każda normalna osoba, wiedząc że wielka latająca kula-zapominajka wylądowała w centrum miasta,
ewakuował by się z niego jak najdalej. 
Mateusz w każdym razie nie był normalny, i może właśnie dlatego został zaproszony na najbardziej nienormalną ucztę we wszechświecie.

Do mariny spróbował dostać się okrężną drogą, przebiegł przez Krowi Most na Wyspę Spichrzów.
Klucząc uliczkami, zbliżył się do portu, niestety, tutaj też była blokada.
Widział jedynie kawałek wody w basenie jachtowym, nietypowe fale odbijały się od brzegów, coś się tam jednak działo.
Popatrzył smutno w kanał i pomyślał, że chyba pozostanie mu wskoczyć do wody i popłynąć wpław, pod mostem omijając strażników.
Ale przecież pewnie nie wpuściliby go mokrego do rakiety.
No i co z pudrem, który już sobie wcześniej pieczołowicie nałożył?
Głupi. Podziurawią go pociskami, jak durszlak, gdy tylko zobaczą kogoś taplającego się w wodzie pod ich nosami.
To był koniec.

Szustokor był bardzo gruby, rozpiął więc bluzę żeby się nie ugotować, już było mu wszystko jedno, czy ktoś go zauważy.
Był tak blisko, a jednocześnie tak daleko. 
Całość miała prysnąć, jak bańka wypuszczona przez sprzedawcę mydlanego płynu, 
goniona przez upośledzone dziecko z ADHD, na rynku w centrum turystycznego miasta.
Czuł się jak rybka w siatce, wrzucona do oceanu.
Zaraz będzie wyraźnie widział swoją straconą szansę, jak w głównej gablocie muzeum porażek życiowych.
Co robić? Co robić?

Wybawienie przyszło nieoczekiwanie, niczym słabo napisane \differentlan{deus-ex-machina}.
Oto bowiem mama z małą dziewczynką podpłynęły do niego skuterem wodnym, oferując szybką podwózkę.
Myśląc, że to pomyłka, zdjął wierzchnie zakrycie, pokazując swój strój w połowie okazałości.
Kobieta jednak nie uciekła, nie przestraszyła się dziwaka, tylko się uśmiechnęła.

\begin{dialogue}
\ds{} Chyba się teraz nie poddasz? \dm{} zapytała.
\ds{} Skąd... kim...
\ds{} Kula miał wielu gości. \dm{} Położyła rękę na piersi. \dm{} W momencie jak zobaczyłam, że wrócił do Gdańska, wiedziałam. Ktoś może potrzebować pomocy.
\ds{} Ja... \dm{} Główny bohater milczał przez chwilę. \dm{} Ha, ha. Prawie się nabrałem.
To niezwykłe. Jak wynajęliście żołnierzy, żeby zastawili miasto specjalnie dla mnie? Głupim, przecież to profesjonalni aktorzy.
\end{dialogue}

Tajemnicza osoba przewróciła oczyma, zdjęła swoją córkę na chodnik i zeskoczyła, zawiesiła mu klucze od motorówki na kryształowym guziku i poszła, nie odzywając się więcej, ciągnąc dziecko za rękę.

Mateusz w rokokowym stroju wersalskim zasuwał na skuterze wodnym kanałem Nowej Motławy, ozdoby szustokora mieniły się w pełnym słońcu tak samo, jak latające wokół niego
krople wody i stalowe kule, wystrzelone z mostu przez żołnierzy zabezpieczających lądowanie wielkiej białej kuli w centrum miasta.
Schował się na chwilę pod mostem, jak zagoniony watahą wilków królik w norze, a gdy wypłynął z drugiej strony, wtedy ją zobaczył.

Była gigantyczna, jak budynek, wypolerowana, biała i doskonale kulista. Tak abstrakcyjna, że nawet nie wyglądała na nadmuchany balon.
Dołem dotykała lekko powierzchni wody, tworząc promieniste fale.
Odbijała w sobie cały Gdańsk.
Przybysz ujrzał w niej tyciego siebie na łódeczce-zabawce, malutkie domy, żołnierzyków, spichrze, basenik, niebo, helikopter jak muchę i blask naszej gwiazdy, niczym żarówkę w lampie.
Już nikt nie strzelał, już tylko wszyscy patrzyli. I bali się.
On się nie bał. 
Przyszedł tu na bankiet.
Przyszedł we francuskim stroju.
Przyszedł tu, bo został zaproszony.

Wdrapał się ze skutera na pomost i poprawił perukastą strukturę, wtedy też właz w dolnej części zaczął się otwierać.
Tak jak się spodziewał, był to dźwięk szczęku łańcuchów, a nie elektrycznego silnika. Klapa, jak w małych samolotach, robiła także za schody.
Ze środka powiał zapach kurzu, wosku i lekkiej stęchlizny.
U dołu rozwinął się elegancki, czerwony dywan.
W przejściu stanął On.
Nosił strój wspanialszy, niż Mechalyczny mógł sobie kiedykolwiek wyobrazić, tak inny od jego własnego, a przecież z tego samego okresu historycznego.
Przy jego ozdobach, sztuczna biżuteria Mateusza, rzeczywiście wyglądała na sztuczną.
Jego najprawdziwsza peruka przyćmiła wielkością cały futrzany twór z głowy gościa.
Lakierki błyszczały się tak samo, jak jego statek kosmiczny z którego się wyłonił.
W ręku trzymał laskę z białą kulką, pomniejszoną wersją tego, co znajdowało się tuż za nim.

\begin{dialogue}
\ds{} Jestem Profesor Kula. Miło mi pana gościć na moim statku.
\end{dialogue}

\divider{}

Antyrax wyszedł z worka, gdy z Dedirid została już tylko kupka popiołu.
Jeszcze ośmiu.
Tym razem demony nie bardzo chciały go atakować.
Pisarz więc wskazał jednego z nich palcem, niczym sędzia nowoskazanego na śmierć.
Lenna. Pora na mikropomysły.
Sięgnął do czeluści alternatywnego wszechświata i od razu złapał to, czego szukał.

\divider{}

\begin{dialogue}
\ds{} Waćpan Mateusz Mechalyczny, jak mniemam \dm{} powitałem gościa. Waćpan Mateusz Mechalyczny ostrożnie, acz żwawo, podszedł, ukłonił się, i schował za framugą włazu,
znikając przed przeszywającym wzrokiem miasta.
\ds{} Proszę wybaczyć mi mój ubiór i maniery, panie Profesorze Kula. \dm{} Zniżył głowę ponownie, prawie do ziemi. \dm{}
Musiałem przedrzeć się przez kordon wojska i ominąć grad pocisków, aby przybyć do pańskiego statku.
\ds{} Nazywam się Kula, mości Mateuszu Mechalyczny, nie Kula \dm{} poprawiłem, zamykając korbą właz. \dm{} A ta kula nosi nazwę Kula. 
I nie jest jakimś statkiem kosmicznym, a Kulą.
\ds{} Kula... \dm{} niepewnie odpowiedział.
\ds{} Nie Kula, Kula. Moje nazwisko, nazwa tego miejsca, typ urządzenia, i bryła geometryczna. Kula, Kula, Kula i kula. To trzy różne słowa, zupełnie inaczej wymawiane.
Zupełnie inaczej zapisywane.
\end{dialogue}

Mój gość podrapał się po głowie, ścierając puder.

\begin{dialogue}
\ds{} Ignoruj go, on mówi i słyszy na częstotliwościach poza zakresem naszych uszu. \dm{} Katarzyna Kosmata zjechała po falistej poręczy schodów i przysunęła do nas, nawet się nie witając.
\dm{} Jestem Kasia, cześć.
\ds{} Droga Katarzyno Kosmata! \dm{} skarciłem ją. \dm{} Maniery! Niech panna nie prezentuje złego przykładu naszemu gościowi. Panie Mateuszu, mam zaszczyt przedstawić panu...
\end{dialogue}

Gość jednak utopił wzrok w olbrzymiej sukni Kosmatej, nachalnie gapiąc się na każdy jej detal.
A już się radowałem, że chociaż on będzie wiedział, czym jest odpowiednie wychowanie. Nadaremnie.
Najgorsze w tej sytuacji było to, że ona sama wręcz go do tego zachęcała. 
Zamiast wyjaśnić, kim jest, zaczęła tłumaczyć skąd i jaka część jej modnego nakrycia pochodzi.

Najpierw zawiesił oczy na jej biuście, wodząc źrenicami to w lewo to w prawo.
Najprawdopodobniej podziwiał plecionkę z anielskich włosów, którą obszyta była góra.
Niebiańskie włosy są całkowicie przezroczyste, gdy odpadną od właściciela, więc
aby stworzyć ten element ubioru, trzeba było najpewniej zbierać je z perłowych podłóg w całym raju, niczym ptak szukający materiału na gniazdo.
\begin{dialogue}
\ds{} Trafiłam przez to na dywanik boskiego ministra od poprawnego zachowania, chciał to podciągnąć pod brak szacunku dla zarządu Nieba, ale wybroniłam się tym, że wszyta w suknię świętość
będzie dodatkowo ochraniać mnie przed demonami, czy jakoś tak.
\end{dialogue}

Następnie zszedł niżej, aby przyjrzeć się lepiej pasowi.
Katarzyna gustowała się w niezwykle kosztownych ubiorach, lecz jej pas był uszyty ze zwyczajnego, ziemskiego jedwabiu.
Może chciała tym pokazać, jakoby jej suknia była w równym stopniu wykonana ze składników z calutkiego wszechświata?
\begin{dialogue}
\ds{} Ten jedwab pochodzi od jedwabników karmionych nektarem jedynie z najrzadszych gatunków orchidei, podlewanych krystaliczną wodą źródlaną z Himalajów
\dm{} wyjaśniła cały sekret.
\end{dialogue}

Po pasie, przyszedł czas na szyję. Modnisia założyła tym razem kolię ze zmutowanych pereł Khaliniskali...
czy to była Rezurma? Nie pamiętam, kto ostatnio przejmował stolicę i nazwę tej przeklętej... Planety Wojny, jak ją wielu nazywa.
Kulki mieniły się i pulsowały wszystkimi kolorami tęczy. Od podczerwieni, po nadfiolet. Kuły w oczy nie tylko pięknem, ale też i wysoką energią fotonów odbijanego światła.
Te perły można było znaleźć w małżach, żyjących w skażonym jeziorze, na północy pustynnego kontynentu Terb. Chwila, teraz to już nie był już Terb... no na północy tego największego kontynentu.
Zdaje się, że to albo Czarna Armia, albo Komodowa utopiła tam kiedyś beczki z kancerogennym żelem, w celu wewnętrznego wyniszczenia przybrzeżnego miasta Hirten... wtedy to było Hirten.
Nie udało się, mieszkańcy wyczuli podstęp i zamiast umrzeć na nowotwory od picia zatrutej wody, poumierali z pragnienia.
W każdym razie flora i fauna w tym zbiorniku wodnym przeszła nieprzyjemne zmiany fizyczne.

Buty. Był to wspólny wytwór czterech Khrnzrhkh.
Najpierw poprosiła Chrrkrhkrrkk o stworzenie lodowej podstawy.
Potem pewnie Iłiścirr obudował to swoją czarną rkkizniisi, Buffsirr dodał czerwone paciorki z buffzerda, a Fluszszrisss utwardził ogniem.
\begin{dialogue}
\ds{} Te buty zostały zrobione przez potworów, Mikołaj wyrzeźbił kopyto z zamrożonej wody, 
Psychit zalał ektoplazmą, Pyrroq stworzył rubinowe klejnoty-bomby, a Plazma wypalił w ognistej kuli. \dm{}
Katarzyna Kosmata właśnie spowodowała, że kolejna osoba będzie nazywać Khrnzaalk potworami, zamiast porządnie w ich własnym języku.
\end{dialogue}

Zahaczył o wachlarz.
Ten był z półprzezroczystych łusek białego cyrkowca, zszytych razem w heksagonalny wzór.
Te smoki wyginęły doszczętnie po ataku czerwonych kartaczy na ich wyspę.
Właściwie, jedyne pozostałe istoty tego gatunku można teraz znaleźć w zoo w Capitalu.
Szkoda ich, miały cudowną kulturę.
Cyrkowe baśnie do dziś opowiada się małym smoczkom do legowiska, a cyrkowi malarze są niedoścignionym przykładem talentu w wielu cywilizacjach.
Kartacze to zwykłe zwierzęta. Żeby chociaż te barbarzyńskie pasowce ich rozbiły, ale nie. 
Największy i najbardziej bezmózgi gatunek smoków zaatakował, rozszarpał i pożarł najwspanialszych.

Ostatecznie Mateusz popatrzył się na jej twarz.
Nie, nie na twarz, a na makijaż.
Oczywiście, Katarzyna Kosmata nie mogła spocząć na wyrywaniu łusek prawie wymarłym gatunkom.
Jej puder był stworzony ze zmielonych ciosów mamuta. Biała twarz śmierci całej populacji, uczennica kostuchy.
Jak ona odkopała wyschnięte zwłoki z syberyjskiego błota i wybieliła, tego nie wiem.
\begin{dialogue}
\ds{} Przekonałam Chronosa, jednego z potworów, żeby odwrócił trochę czas i przywrócił im świeżość.
\end{dialogue}

Nikomu się nie udało namówić kiedykolwiek Pfiishuss do jakiegokolwiek używania swojej mocy! Jak ona to zrobiła?

To jednak nie był koniec podziwiania, poszedł wzrokiem wyżej.
Jej fryzura była przeogromna. A wszystko to z naturalnych włosów. 
Wiem na pewno, że Floria... znaczy Hhurnna przywiązuje podobną uwagę do wyglądu. 
Myślę, że nie odmówiłaby Katarzynie podkręcenia jej cebulek włosowych w celu przyspieszenia wzrostu czupryny.
\begin{dialogue}
\ds{} W drogeriach sprzedają taki super szampon. Nic więcej nie potrzeba. Może na twoją łysinkę też pomoże. 
\end{dialogue}

Na jej głowie osiadły wielobarwne motyle. 
Co jakiś czas, któryś wzbijał się w powietrze, robił pętlę wokół jej głowy i lądował z powrotem.
Były to najprawdziwsze owady, hodowane i tresowane w tajnej placówce pod motylarnią w Burggarten.
Ciekawiło mnie, jak je zdobyła. Znając Katarzynę, pewnie jak gdyby nigdy nic weszła przez ukryte wejście do podziemi, w tej pełnej sukni, z naładowanym szyfratorem w ręce, i powiedziała:
,,Dajcie mnie tych tresowanych motyli na głowę, bo zaraz mam bankiet we wielkiej, latającej kuli.''
Być może tylko po to w ogóle przyjechała dzisiaj do Wiednia.
Przybyła po żywe ozdoby, i żeby przyprawić o zawał serca całą Austrię.

Więc tym razem postanowiła wsiąść do Riesenrad i pojechać wagonikiem na sam szczyt, gdzie wcześniej specjalnie umówiła się ze mną, abym podleciał po nią Kulą.
Oczywiście, w momencie jak przyleciałem, wybuchła panika. 
Całość się zatrzymała, uwięzieni na dużej wysokości ludzie próbowali uciekać po konstrukcji koła przed wielką białą kulą, cumującą właśnie do najwyższej budki. 
Katarzyna otwarła drzwiczki, i robiąc krok nad przepaścią, weszła do sfery, niczym królowa do swej karocy.
Pomachała wachlarzem pozostałym, ledwo żywym ze strachu osobom w wagoniku, i odlecieliśmy.
Jeśli w przyszłości ponownie ją zaproszę, pewnie stanie na szczycie Empire State Building, a ja będę robił za King Konga.
I też będę potem uciekał przez myśliwcami. Ciekawe, czy skończyłoby się jak w filmie.
Czy można się uzależnić od amnezji, którą pokryty jest statek?
Uzależnić od siania paniki w tłumach, które i tak za chwilę o wszystkim zapomną?

Przeczyściłem gardło.
\begin{dialogue}
\ds{} Znaczy... witam... bardzo mi miło, dzień dobry... eee... Kasiu-ażyno. \dm{} Stał bez ruchu kilka pulsów, aż zdecydował się delikatnie ująć jej dłoń i pocałować.
Ważne, że się starał. \dm{} Ja Mateusz... jestem.
\end{dialogue}

Katarzyna zarumieniła się. Widać została w niej szczypta kultury osobistej. A może to był makijaż?

Mój gość dostał oczopląsu, jego wzrok skakał od ozdoby do ozdoby. 
Każdej falki, każdego wgłębienia musiał dotknąć, niczym sprawdzając czy rzeczywiście wykonane są z hebanu i masy perłowej. 
Poprowadziłem ich po schodach, do salonu w którym nakryty był już stół dla czterech osób.
Aż przysiadł z wrażenia, łamiąc kark przy zadzieraniu głowy.
\begin{dialogue}
\ds{} Na nasz bankiet zaproszonych zostało w sumie trzech gości \dm{} oznajmiłem zgromadzonym. \dm{} Zatem dołączy do nas jeszcze jedna osoba.
Będzie to Nadar Nocny, aktualnie bada, albo szabruje, wrak Titanica. Podróż potrwa około dwa i pół kilopulsa, to jest niecałe dwie godziny.
\dm{} Poprawiłem żabot. \dm{}
Pan Nocny jest dość... ekscentryczny. Dla jednych jest najlepszym przyjacielem, a inni go nienawidzą.
Szczerze powiedziawszy, nie popieram jego charakteru, ale obawiam się że może pan, mości Mateuszu, naleźć w nim swoją bratnią duszę.
\ds{} No dobrze, gdzie są ukryte kamery? \dm{} Gość niespodziewanie wypalił.
\ds{} Proszę pana, zaręczam że w całym tym miejscu nie znajduje się ani jedno obrzydliwe cyfrowe urządzenie. 
Ta strefa jest wolna od nieprzyjemnych pól magnetycznych i elektrycznych.
\ds{} On chyba nadal nie wierzy, Profesorze \dm{} Katarzyna zaproponowała. 
\ds{} Nadal nie wierzy w Kulę? Pomimo, że sam w niej stoi? \dm{} Uśmiechnąłem się. \dm{} 
Nigdy nie widziałem takiego zaparcia przy obronie własnych idei. Panie Mechalyczny, ręczę że będzie pan wspaniałym agentem.
\end{dialogue}

Położyłem rękę na lasce. Lekko uderzyłem małym i wskazującym palcem, by obniżyć lot.
Następnie przycisnąłem w dół otwartą dłonią, w celu pokonania siły wyporu, jakbym wmuszał pustą boję pod lustro wody.
Zaczęliśmy się wtedy zanurzać coraz głębiej i głębiej w Atlantyku, zostawiając za sobą pióropusz tęczowych rozbryzgów.

Tymczasem zacząłem oprowadzać naszego gościa po Kuli.
Wycieczkę rozpoczęliśmy, wracając do głównego włazu na najniższym piętrze.
Ta otwierana w dół, mająca od wewnątrz kształt schodów, wykrzywiona płyta, była jedyną, niepokrytą karmazynowym futrem, częścią pancerza.
Zamiast tego posiadała czerwony dywan i wysuwaną poręcz, automatycznie rozwijane przy kontakcie z podłożem.
Operowana za pomocą skomplikowanego systemu łańcuchowo-sprężynowego na korbę. Zawsze przypominała mi zamkowe wrota nad fosą.

Nie zmieniając wysokości, przeszliśmy do garderoby. Zakurzona pieczara z ubraniowymi stalaktytami i buciarskimi stalagmitami.
To właśnie tutaj trzymałem awaryjne suknie, habity, koszule i trzewiki, w razie gdyby któremuś z gości zdarzyło się nie mieć wystarczająco odświętnego ubioru do uczestniczenia w uczcie.
Mateusz zwrócił mi uwagę na grube stroje, wiszące w kącie. Wedle jego wizji były to kostiumy nurkowe.
Opisałem, że pomimo mylącego dla niektórych wyglądu, w rzeczywistości nadawały się zarówno do schodzenia pod wodę, jak i w próżnię kosmiczną.
Są integralną częścią Kuli, wyjaśniałem, trochę jak ściany i meble. 
Czerpią z białego pojazdu energię do podtrzymywania życia. 
Będąc w takim kombinezonie, nigdy nie zabraknie nosicielowi tlenu i pożywienia.
Na szczęście nie zauważył, iż jeden z haków był pusty. 
Nie chciałem się tłumaczyć, że zgubiłem kawałek wyposażenia swojej rakiety.

Zapytany o śluzę ciśnień, aby bezpiecznie wychodzić na zewnątrz, opowiedziałem mu o niewidocznej tarczy rozciągniętej na włazie. 
Chroniła ona wnętrze przed różnorakimi hazardami zewnętrznymi, takimi jak kosmiczny mróz, uniwersalność, demony, czy brak kultury osobistej.
Była jak kotara z paciorków, oddzielająca biuro alfonsa od reszty burdelu.
Chwila, czy ja właśnie porównałem moją sferę do domu publicznego?

Nie był przekonany, co do pewności działania, więc kręcąc jeszcze raz korbą, otworzyłem ponownie właz. 
Płynęliśmy aktualnie tuż przy samym dnie morskim, zostawiając za sobą chmurę wzburzonego dna oceanicznego, niczym ślad dymu z palącego się samolotu.
Falista, lekko wypukła powierzchnia wody, utworzyła się na wysokości framugi. 
Mateusz z niedowierzaniem zamoczył rękę w wodnych głębinach, wyciągając garść osadzającego się piasku.
I meduzę.

Kolejne piętro posiadało same pokoje gościnne. 
Gość uprzejmie podziękował za apartamencik, ale nalegał, abyśmy szli dalej.

W centralnej części Kuli znajdowała się łaźnia, muzeum i mój gabinet. Do tego ostatniego nikogo nie wpuszczam.
Ludzie snują różne domysły na temat tego, co może się kryć za dębowymi drzwiami. 
Zasilanie całej Kuli, zwyczajny pokój, jakieś kosmiczne artefakty, prawda o moim pochodzeniu?
Nikt z nich nigdy nie miał racji, a ja nikomu nigdy środka nie pokażę.

Łaźnia. W wyłożonym terakotą pomieszczeniu panował standardowy zaduch. 
Gość zdziwił się niemiłosiernie, widząc tutaj basen, jacuzzi, saunę fińską, masażery wodne, a także mały wodospad.
Pośrodku stał wielki piec na węgiel, świecił się na bordowo, rozpalony ognistym wnętrzem. Serce mechanicznej bestii.
Bez niego zimna pustka kosmosu szybko by nas dopadła.
Mateusz powiedział, że w XVIII wieku nie używano łaźni i że po stylu wnętrza spodziewał się co najwyżej wychodka w kącie. 
Zaśmiałem się na myśl, iż wziął Kulę za stuprocentowy wycinek pałacu w Wersalu.
Kultura idealna nie istnieje, zacząłem wykład, z każdej cywilizacji należy wyciągnąć najlepsze części. 
I tak, na przykład łącząc rzymskie starożytne łaźnie, francuski późnobarokowy wystrój, średniowieczne królewskie dania i słowiańską mowę przyszłości, 
stworzyłem tę właśnie latającą wyspę kultury idealnej.

Teraz najciekawsza część statku, wystawa.
Moje muzeum zawiera artefakty z różnych zakątków wszechświata. Gość zapytał o wartość zebranych przedmiotów.
\begin{dialogue}
\ds{} Nie wszystko da się sprowadzić do liczby pieniędzy \dm{} dałem mu wykład \dm{} i nie wszystko ma tak zwaną cenę. 
Jeszcze się o tym nie raz przekonasz.
\end{dialogue}

Równo ułożona siatka gablot przynosiła na myśl szachownicę.
Wskazałem skałę przyczepioną widełkami do podstawy i począłem wyjaśniać:
To jest kawałek meteorytu, który uderzył w księżyc planety Tos. Wartość tego kamienia jest równoważna wartości losowego kamienia polnego z Ziemi, 
znajduje się tutaj ze względu na historię, jaką ze sobą niesie.
Otóż, uderzenie tego kosmicznego głazu było tak silne, że wybiło ich naturalnego satelitę z orbity, popychając ją w kierunku Tosa.
Po stu latach ciągłego zbliżania się do powierzchni, w końcu zahaczyła o atmosferę, gwałtownie zwolniła i zderzyła się z powierzchnią.
Każdy organizm, większy od jednokomórkowca, został zniszczony, czy to w morzu ognia, czy w ciemności popiołów.

Co ciekawe, mieszkańcy tego świata byli na tyle rozwinięci naukowo, że doskonale widzieli i rozumieli zbliżającą się katastrofę.
Jednak ich technologia nadal była za słaba, aby móc jej uniknąć.
Przewidzieli dzień swojego końca co do dnia, a koniec faktycznie nastąpił.

Tak, wiem że to smutne, ale cóż począć? Gorsze rzeczy zdarzały się w zbiorowej historii życia. 
Tylko pierwotne grzyby przetrwały globalny Armageddon.
Toksyczna atmosfera, brak słońca i wysoka temperatura post-apokaliptycznego świata wręcz przyspieszyły ich ewolucję.
Na przykład, tutaj jest dziób pewnego latającego ptakochomora. To fungus i ptak jednocześnie, ładnie świeci w ciemności.
Da się go spożywać, niestety nie jest bardzo wysublimowany w smaku. 
Ma posmak śmierci, jak każda istota wychowana na truchłach poprzedników, lub też gorzka czekolada.

Przeszliśmy dalej. Kamień z lodowej strony Kryonii, nic niezwykłego. 
No może poza tym, że musiał być wydobyty spod kilku kilometrów litego lądolodu.
Co w tej planecie jest wspaniałego? Otóż obraca się ona wokół swojej gwiazdy, jak Księżyc wokół Ziemi. 
Wiecznie zwrócona tą samą półkulą.
Mieszkańcy nie mają zatem dni oraz nocy, a słońce permanentnie jest w tej samej części nieba, niczym lampa w wielkim terrarium.
Nocna część jest zamarzniętą pustynią, dzienna od zawsze ma pośrodku szalejące tornado.
Może kiedyś zobaczysz Pałac Nadiru, położony w centrum wiecznej zmarzliny, jest cudownym dziełem sztuki glacjalnej.
Potężna iglica z kryształowych łuków, kopuł, balkonów i kolumn. Nieporównywalna z niczym innym.
Podświetlona trytowym światłem na przeróżne kolory. 
Jest plan by odpowiednio ustawić bieguny planety, jeden w Nadirze, drugi w Zenicie, by dodać przepiękne zorze polarne na obu stronach.
Freon, wielki lodowy król, rządzi swoim państwem dobrze i sprawiedliwie.
Szkoda, że duża grupa jego ludu tego całkowicie nie rozumie. 
Demokraci, socjaliści, libertarianie, i reszta niepoliczalnych ruchów społecznych chce go dosłownie zwalić z tronu i pogrążyć cywilizację w chaosie.

\begin{dialogue}
\ds{} Skąd wie pan, że byłoby gorzej, niż jest teraz? \dm{} zapytał.
\ds{} Może kiedyś zostanie waćpan zaproszony do pałacu Freona i wtedy, na własne oczy zobaczy pan, że na pewno nie mogłoby być lepiej, niż jest obecnie \dm{} odpowiedziałem. \dm{}
Zresztą, prędzej czy później to i tak się pewnie stanie. Król się starzeje i nie znalazł jeszcze na swój tron odpowiedniego następcy. Nikt inny nie może go zastąpić.
Więc albo rozkaże wybrać kogoś głosem ludu, albo znajdzie kolejnego godnego władcę spoza planety. 
To najprawdopodobniej doprowadziłoby do wojny domowej, rozumiecie, nikt nie chciałby być rządzony przez obcego kosmitę z kosmosu, nie ważne jak dobrze by królował.
\ds{} Jak spoza planety? 
\ds{} To jedna z tych cywilizacji, zwanych zapoznanymi. Na tyle rozwinięta technologicznie i kulturalnie, przede wszystkim kulturalnie, 
że ma dostęp do warstw wszechświata. Warstwy to takie jakby obszary ,,pod'', ,,nad'' i ,,z boku'' czasoprzestrzeni.
Pozwalają na szybką i dowolną podróż w każde miejsce, do każdej galaktyki, do każdego układu, używając minimalnej ilości paliwa.
\ds{} Jak to? Czyli taka na przykład Kryonia może w dowolnej chwili przelecieć jakąś warstwą i zaatakować Ziemię? 
\dm{} Zląkł się. \dm{} I czy moja planeta także jest zapoznana?
\ds{} Powiedziałem, rozwinięta kulturalnie cywilizacja. Czy waćpan jest absolutnie pewien, że ludzkość nie napadłaby obcego globu, gdyby lot do niego byłby tak łatwy, jak do Księżyca?
No właśnie. Poza tym, Ziemi bronią jeszcze Khrnzrhki.
\ds{} Kto? 
\ds{} Potwory \dm{} westchnąłem. I ja też się w końcu poddałem. \dm{} Robią za policję wszechświata, dbają o pokój na wszystkich zapoznanych światach i poza nimi. 
ALOPP, do którego pan został zaproszony, to skrót od Akademii Ludzkiej Otoczonej Protekcją Potworów. 
Jako agent tej struktury, będziesz im waść pomagał, będziesz dbał o pokojowy bieg historii, zatrzymywał wojny, walczył ze złem w różnych postaciach. 
To niebezpieczna i bardzo ciężka praca, ale jakże ciekawa.
Raz ucieczka na skuterze grawitacyjnym przed stadem czerwonych kartaczy, a innym razem zasiadanie w sali obradowej Pałacu Nadiru. 
Znaczy, oczywiście jeśli okażesz się godny.
\ds{} Godny?
\ds{} Chodzi o charakter. Do Akademii należy odpowiedzialność przed przyszłością. 
Każdemu z was może zdarzyć się musieć dokonać wyboru decydującego o milionach istot. 
Dlatego kandydaci są poddawani testowi osobowości, żeby mieć pewność że zawsze wybiorą większe dobro. 
Test ma kilka faz, sprawdza reakcję na zaistniałe przeciwności losu.
\ds{} Kilka faz? Jak mam je pozdawać? \dm{} Zaczął panikować.
\ds{} Spokojnie, pierwszą ma pan już za sobą. Przecież jest pan tutaj z nami. 
Celem było przetestowanie, co dana osoba zrobi w absolutnie nietypowej sytuacji.
Można było wyrzucić ten list, można było zgłosić go władzom, można było pokazać w internecie, a można było, jak pan, potraktować go poważnie.
Jest także fragment odpowiadający za twoje zaangażowanie, posłuszeństwo, wykonywanie rozkazów, siłę psychiczną itp.
\end{dialogue}

Kontynuowałem oprowadzanie.
Zapytałem go, czy potwierdzi, iż wskazany przeze mnie kawałek steampunkowych mechanizmów wygląda intrygująco.
Kupka chaosu wyglądała jak zezłomowany sterowiec, razem z załogą.
Poprosiłem, żeby powiedział, co mu to przypomina. Pomogłem, informując jakoby to nie był żaden zegar.
Nie zgadł, jak mógłby zgadnąć?
\begin{dialogue}
\ds{} Otóż, jest to mózg reprezentanta pewnej wybitnie nieprzyjemnej nacji robotów.
I mówiąc roboty, nie mam na myśli ludzików zasilanych na prąd, jak to się przyjęło w ziemskiej kulturze.
Mam na myśli wszelkie żywe istoty zbudowane z nieżywych składników. Pozornie zwykła materia, lecz natchniona myślą.
\ds{} Proszę mi wybaczyć, ale z daleka to dla mnie kupka śmieci.
\ds{} Bo nią jest!
Te... struktury, powstały z ludzkiego złomu, jako sztuczne ciała dla głodnych demonów.
To jest tak, że niektóre upadłe anioły są za słabe, aby pożywiać się duszami prawdziwych istot, 
zatem muszą się zadowalać martwymi fragmentami maszyn, najlepiej tymi, które towarzyszyły różnym osobom przez jak najwięcej lat ich życia.
\dm{} Położyłem rękę na gablotce ze szkła wymiarowego. \dm{}
Ludzki złom. Wszystko, co codziennie było wam tak bliskie, jak własne części ciała, ale jednak nadal martwe i wymienne. 
Protezy kończyn, wózki do poruszania się, kule do chodzenia, rozruszniki serc, inhalatory, tego typu rzeczy.
Te roboty mogą więc także składać się z metalu i układów logicznych, ale nie są napędzane energią elektryczną, lecz szatańską!
\end{dialogue}
Na te słowa Mateusz zrobił krok w tył.
\begin{dialogue}
\ds{} Szatańską \dm{} powtórzyłem z grodzą. \dm{} Szatańska opętana kupa śmieci.
Powstały, jako wcielenie najczystszego zła, zasilanie parą z palonych zwłok, stworzone z ludzkich odrzutów, zlepione śliną i cyrografem.
Może się pan przyjrzeć, ten element przykładowo, jest wykonany ze sztucznej szczęki.
\ds{} Dziwna ta szczęka.
\ds{} Nie, no. Nie ludzkiej sztucznej szczęki, czy ludzie mają po dziesięć półkolistych zębów, jak te tutaj?
Wiele istot we wszechświecie ma przecież zęby i większość z nich, tak jak wasz gatunek, czasami potrzebuje zamiennika.
\end{dialogue}
Mechalyczny przysunął się z powrotem, chociaż nie krył obrzydzenia. Ciekawość brała górę.
\begin{dialogue}
\ds{} To rurka, od kuli od podpierania się, tym razem ziemskiej kuli. 
Właściciel był jakimś wielkim gangsterem, skazali go bodajże za morderstwo na własnych dzieciach, powiesił się w więzieniu, oczekując na śmierć.
Im większy grzesznik, tym dla takiej diabelnej istoty smaczniejszy.
A to jest gumowy wężyk, był kiedyś w rozruszniku serca... smoka.
\ds{} Mam wrażenie, że wciąż się porusza. To znaczy że nadal żyje?
\ds{} Absolutna racja, nadal żyje, lecz akurat nie pamiętam imienia demona, który go zasila.
\ds{} A... a to nie jest trochę niebezpieczne tak go tutaj trzymać?
\ds{} Tylko trochę. W najgorszym razie, gdyby uciekł, i tak pierwsze co by ten demon zrobił, to czmychnął jak najdalej od tego świątecznego miejsca. 
Poza tym, jest zamknięty w gablocie wymiarowego szkła, przez wymiarowe szkło nic się nie przebije.
\end{dialogue}

Powiodłem wzrokiem tam, gdzie wskazywał Mateusz.
\begin{dialogue}
\ds{} Ta nie jest ani atomowa, ani termojądrowa, zwyczajna na proch. \dm{} Otrzepałem kurz ze starego, pokrytego wyschniętym smarem pocisku. \dm{} 
Kiedyś zadarliśmy troszkę za bardzo z wojskiem Stanów Zjednoczonych. Nadal zadzieramy.
Mocno nadszarpnęli nam ochronne powłoki i w końcu ta mała bombka przebiła się przez pancerz i wpadła prosto do pieczonego dzika.
Było blisko, jakby eksplodowała, długo bym musiał czyścić ściany z resztek jedzenia.
\ds{} Czyli goście przeżyliby wybuch?
\ds{} W Matrycy jest zapisana gwarancja na odbudowanie dowolnej istoty, w razie gdyby coś jej się stało na pokładzie. 
Innymi słowy, nie da się tutaj umrzeć, gdyż twoje ciało zaraz zostałoby automatycznie złożone w jedną całość.
Również dusza po utracie fizycznego nośnika nie ucieknie, któraśtam warstwa pancerza blokuje ektoplazmę, byłaby zamknięta tutaj, jak w śnieżnej kuli.
Z definicji śmierć nie jest utratą ciała, lecz duszy. Dusza zwiewa do dolnej warstwy i leci z nurtem energii, gdy nie ma postaci do zamieszkania. 
W związku z tym szybkie przywrócenie organizmu do życia przyjmie duszę z powrotem.
Niektórzy już umarli i nawet o tym nie wiedzą...
\ds{} Matryca? To brzmi jak wielka szafa z tysiącami szufladek, w jakiejś zakurzonej bibliotece.
\ds{} Widzę, że nie masz pojęcia, jak zbudowany jest nasz wszechświat. Nie wiesz pewnie, czym są kwadry, warstwy, Wielki Filtr, główna pompa w centrum, 
Niebiańskie Słoje Symulacji... polecam książkę z mojej biblioteczki.
A tak na szybko, Matryca ma zapisane prawa sterujące całością. Od prędkości światła, po wymiary prześcieradła na twoim łóżku piętro niżej, ale nie historię.
Nie wygląda, jak szafa z szufladkami, a jak foremka do robienia kostek lodu.
\ds{} Po moim doświadczeniu z wojskiem w Gdańsku, wnioskuję że Kula często jest atakowana.
\ds{} Prawie zawsze, gdy składam wizytę na Ziemi \dm{} powiedziałem oczywistość. \dm{} Piekielni amerykanie. 
Ulali swoje pociski z żelaza, wydobywanego przez niewolników w Afryce.
Wypełnili je prochem wyciągniętym z fajerwerków, które miały być wystrzelone na szatańskie święto Halloween. 
Na koniec pokropili zapalniki krwią z abordowanych dzieci.
Takie coś znacznie prościej przebija kadłub stworzony z sacroterii.
\ds{} Przepraszam, czego?
\ds{} Materii pod bezpośrednim sterowaniem Matrycy, na górze jest książka o tym.
\end{dialogue}

Przy okazji, wytłumaczyłem mu ochrony zastosowane w tym latającym pałacyku. 
Nie lubię porównywać mojej kuli do cebuli. Albo ogrów.
Były trzy powłoki, z tym że trzecia to już fizyczny pancerz z sacroterii.
Pierwsza zatrzymuje wszystkie szybko poruszające się obiekty.
Druga chroni przed naporem niepożądanych substancji, już ją widział jak blokowała oceaniczną głębię przed wdarciem się do środka.

Mateusz zapytał się, jak to możliwe, że nie w internecie prawie żadnych informacji o Kuli, pomimo że często ląduje, jak gdyby nigdy nic, w centrach miast.
\begin{dialogue}
\ds{} Dlaczego ludzie nie pamiętają... to dobre pytanie.
Otóż Kula pokryta jest amnezją. Każdy, kto na nią spojrzy, nawet pośrednio, zapomina.
Wspomnień nie tracą tylko ci, którzy wierzą. 
Wierzą w Profesora Kulę, wierzą w bankiety na niebiosach, wierzą w złocone wnętrze.
\dm{} Objąłem rękami całe otoczenie. \dm{} 
Na pewno nie będzie to dla pana zaskoczeniem, że większość osób uważa Kulę zwykle za balon meteorologiczny, fatamorganę, dowcip, sztuczkę magiczną, nowoczesny samolot wojska, itp.
Pan uwierzył w prawdę, dlatego pan tutaj jest.
\end{dialogue}

Wycieczkę przerwał dźwięk otwieranego włazu i chlapanie wody.
Poszliśmy zatem przywitać trzeciego gościa. 
Mateusz był bardzo podekscytowany i pobiegł przodem, niczym kot słyszący właściciela w pobliżu szafki z puszkami tuńczyka.

\divider{}

Lenna popatrzyła się na Antyraxa i pokiwała w aprobacie głową.
Potem sama skierowała swoje kroki w kierunku Neofantasora.

Demony były już chyba przerażone, gdyż teraz poczęły wszystkie uciekać.
Jednak Antyrax był szybszy. Złapał jednego z nich za nogę (a właściwie to jego lateksowy but złapał nogą nogę), przyciągnął do siebie, i włożył mu swój worek na głowę.
Można było powiedzieć, że zakrywa mu twarz, niczym przed egzekucją ścięcia toporem.
Piotr Lektor zaczął się dusić, trująca abstrakcja wgryzała się w jego demoniczne płuca, a niezwykły but, z siłą wolnego oprogramowania, ściskał mu szyję,
jak o dwa rozmiary za mała, automatycznie nadmuchiwana kamizelka ratunkowa, w tonącym w oceanicznej pustce samolocie. 

\begin{dialogue}
\ds{} Dość, wystarczy. Dam ci te dwie gwiazdki! \dm{} Z worka słychać było jedynie stłumione jęki. \dm{} Trzy! Niech będą trzy gwiazdki. I komentarz.
\end{dialogue}

Antyrax jednak nie odpuszczał. Ruchy Piotra Lektora stawały się coraz wolniejsze i wolniejsze, jak babci poślizgniętej na oblodzonym chodniku.

\divider{}

Nadar. Czemu to akurat jego musiał ten Kula zaprosić?
Planowaliśmy eleganckie przyjęcie, a ta niewychowana świnia pewnie pociągnie w swoje odmęty i Mateusza.

Jak tylko usłyszałam szczęk łańcuchów, przerwałam robienie makijażu i wystawiłam głowę z pokoju.
Zobaczyłam nowego gościa, zbiegającego po schodach do szatni, biegł tak szybko, że spłoszył mi motyle z głowy.
Nie spieszyło mi się powitać Nadara równie prędko, ale ciekawiło mnie ujrzeć reakcję Mechalycznego, gdy po raz pierwszy zobaczy tego szaleńca.
Z trudem przecisnęłam się w tej sukni przez drzwi na korytarz i ostrożnie podeszłam do pierwszego schodka w dół.
Oczywiście wdzianko zahaczyło o balustradę, od razu zrobiłam hyc i resztę drogi koziołkowałam, 
wywijając podwójne salto, jak telewizor przez przypadek upuszczony na schody w trakcie robienia przeprowadzki.
Wylądowałam na głowie, z nogami majtającymi się w powietrzu.
Wyglądałam jak polująca na grzybową faunę sesteria z Tosa, z tym wyjątkiem że nie strzelałam jadem. A chciałabym.

Wiedziałam, co w tej chwili usłyszę i nie zawiodłam się.

\begin{dialogue}
\ds{} Ale dupa, co nie? \dm{} Nadar zamykał korbą właz, gapiąc się na moje machające w górze nogi.
\ds{} No, nawet... \dm{} Mateusz okazał się równie niewychowany. Nie wierzę, że się z nim zaprzyjaźniłam.
\end{dialogue}

Natychmiast przybiegł Kula i pomógł mi się postawić do pionu. Był czerwony ze złości.
Ale czy dlatego, że właśnie ze statku z sykiem uchodziła kultura, czy dlatego że świństwo uzyskało nowego członka?

\begin{dialogue}
\ds{} To ty! \dm{} Kula trzymał laskę w górze, niczym śmiercionośny laser krojący Nocnego na pół. \dm{} 
To ty wziąłeś czwarty kombinezon z mojej garderoby! 
Szukałem go po całym wszechświecie. Ten kostium jest integralną częścią Kuli, generuje go Matryca tak samo, jak meble, dywany i ozdoby.
Jest niereplikowalny. Nie wolno go zabierać!
\ds{} Przecież nie zabrałem, a pożyczyłem. Zresztą i tak zawsze się kurzył w tej twojej jaskiniowej szafie. \dm{}
Nadar uznał to za wystarczające wytłumaczenie, rozpiął strój. Pod spodem miał swoje standardowe dresy. \dm{} 
A na przeprosiny mam prezent. Wyłowiłem ci, Profesorze, zestaw kieliszków i butelkę najdoskonalszego wina, prosto z kapitańskiego mostka.
Mieli ją wypić na ukończony rejs, ale wiadomo co się stało. Niech więc Kula ukończy swoją własną wyprawę i nie uderzy w żadną lodową kometę po drodze.
\end{dialogue}

Kula w jednym pulsie zmienił się z czerwonego z powrotem w białego, jak rozwydrzone dziecko z którego uniwersalny wampir wysysa całą krew.
Przypomniała mi się walka z tym straszydłem. Brrr.

\begin{dialogue}
\ds{} Och. To bardzo miło z twojej strony \dm{} odpowiedział głosem tak miękkim, jak poduszka Pyrroqa. \dm{} A teraz wybaczcie, muszę dopilnować ostatnich poprawek przy naszym bankiecie. \dm{}
Porwał butlę i kieliszki, poleciał na górę.
\end{dialogue}

Nadar był arogancki, jednak pomimo wad potrafił, jak nikt, walczyć z uniwersalnością.
Nie miał własnej mocy, jak niektórzy, lecz wcale jej nie potrzebował.

Mateusz wpatrywał się w niego, jak Kula w płótno namalowane przez białego cyrkowca.
Jego największe zainteresowanie wzbudzały dwa pistolety, zawieszone przy pasie, i laserowa pałka na plecach.

\begin{dialogue}
\ds{} To urządzenie pozwala zaszyfrować i odszyfrować dowolną osobę w splocie czasoprzestrzeni. \dm{} Nocny tłumaczył działanie swoich zabawek. \dm{}
Cel zachowuje się tak, jakby był zamrożony w czasie. Jakby ktoś nacisnął pauzę na globalnym odtwarzaczu filmu katastroficznego pod tytułem ,,życie.''
W pełni bezpieczny sposób na unieszkodliwianie wrogów bez zabijania.
Wadą jest tylko to, że naboje do niego są olbrzymie i jednorazowe, wyglądają jak akumulatory do wiertarek.
W środku takiego wkładu zapisuje się symetryczny obraz klucza, jedyny sposób na przywrócenie zaszyfrowanej osoby z powrotem do życia.
Strzelając można zamrozić, a potem odmrozić daną osobę.
Kosma, właśnie zgłosiłaś się na ochotnika, aby zaprezentować naszemu gościowi ten wynalazek. \dm{} Wycelował we mnie szyfrator. Co za świ...
\end{dialogue}

\divider{}

Antyrax podniósł lekko worek, Piotr Lektor spróbował złapać oddech, ale zaraz znowu światło zgasło mu przed oczyma, niczym elektryczność komputera w trakcie aktualizowania krytycznej części BIOSu.

\divider{}

\begin{dialogue}
...nia z niego. \dm{} Nie nazywaj mnie Kosma! Jestem Katarzyna.
\ds{} Jak widzisz, działa znakomicie. \dm{} Spostrzegłam, że w czasie gdy byłam zaszyfrowana, zdążył już się przebrać w galowy, pożyczony z garderoby strój. 
Właśnie nakładał puder na swojego irokeza.
\ds{} Nadar, coś ty? Od kiedy ubierasz się elegancko dla Kuli? \dm{} zapytałam z niedowierzaniem. \dm{} Przecież nie gustujesz w niczym innym niż dresy.
\ds{} Od kiedy wywalił mnie pośrodku kosmosu w tym skafandrze, za przypadkowe rozlanie barszczu na obrus. 
Lewitując w bezkresnej pustce, miałem sporo czasu na przemyślenie mojego zachowania i stanie się nowym człowiekiem. \dm{} odpowiedział.
\ds{} Naprawdę? \dm{} Wtedy coś mną tknęło. \dm{} Oczywiście, że nie naprawdę. Znowu się zgrywasz tak? \dm{} Tylko się wrednie zaśmiał.
\ds{} To drugie to pikler \dm{} kontynuował. \dm{} 
Potrafi zapeklować kogoś do umieszczonego tutaj słoika ze szkła wymiarowego, żeby się nigdy nie wydostał.
Wystarczy tylko odłożyć go na najniższą półkę w jakiejś głębokiej piwnicy na całą wieczność.
\dm{} Spostrzegł, że gość niekoniecznie rozumie. \dm{}
Szkło wymiarowe to takie coś, co przechodzi równo przez wszystkie wymiary, także w czasie. 
Wygląda jak szkło, ale istnieje od zawsze na zawsze. 
Ma nieskończoną długość, szerokość, głębokość i... wszystkie inne ości. \dm{}
Nadar nie przestawał wyjawiać sekretów naszej organizacji.
\ds{} A to jest miecz świetlny?
\ds{} To jest laserowa pałka, laserpała, taki przecinak. W przeciwieństwie do miecza świetlnego, nadal można nią zdzielić w łeb, jak się zepsuje.
Po uruchomieniu zaczyna wirować, wzdłuż pojawiają się promienie dasera. Pozwalają łatwo ciąć materię. Czyli trochę jak miecz świetlny, tak.
Daser (\differentlan{Death Amplification by Stimulated Emission of Radiation}) przecina prawie wszystko jak masło, a pozostałe rzeczy jak ser. 
No, mózg przeciąłby jak powietrze.
Fajna zabawka.
\ds{} I szkło wymiarowe też przetnie? \dm{} Mateusz zapytał, widać że uważnie słuchał.
\ds{} Nadar, on nie przeszedł jeszcze wszystkich testów \dm{} wtrąciłam. \dm{} Nie zdradzaj mu tylu sekretów, bo nie wiadomo, czy na pewno z nami zostanie.
\ds{} No popatrz na niego, Kosma. \dm{} Nocny obchodził i studiował Mateusza ze wszystkich stron, zupełnie jak geniusz oglądający
kostkę Rubika i dochodzący, dlaczego nadal mu nie wychodzi. \dm{} Myślisz, że sobie nie poradzi? Poza tym, już pierwsze części testu zdał poprawnie. 
Odpowiedział na abstrakcyjny list Profesora, ubrał się w najprawdziwszy strój francuski, a potem odważył się wsiąść do dziwnej, wielkiej, latającej kuli z kosmosu,
jak poboczny bohater w każdym oklepanym horrorze dla mas.
\ds{} Niby racja, ale doskonale wiesz, jak chory test potwory mogą tym razem wymyślić.
Pamiętasz, jak Mikołaj przetestował Ziemowita? Kazał mu się przebrać za klauna, przyjść na zabawę urodzinową dla dzieci i robiąc magiczną sztuczkę, zamordować jednego z nich, bo był złym owocem klonu.
Coś nie wyszło, morderstwo nie było czyste, wszyscy skąpali się w zielonej krwi tego podrabiańca.
Drugim zadaniem było uciec z więzienia, do którego go wrzucili po aresztowaniu. Może gdybyśmy wiedzieli, gdzie jest prawdziwe dziecko, to inaczej by poszło?
\ds{} Przecież zdał.
\ds{} Albo tego, jak mu było, Błażeja, co Hdro zostawił w Capitalu i kazał jakimś sposobem wrócić na Ziemię.
Człowiek sam na planecie, w całości zamieszkanej przez wszystkie gatunki smoków.
\dm{} Kontynuowałam rozmowę, zupełnie ignorując osobę, na której temat ją toczyliśmy.
\ds{} Nie zaliczył, bo ukradł rakietę jakiejś rodzince błękitnych celebritów, będącej wakacjach w stolicy latających jaszczurów, zamiast rozegrać to w pokojowy sposób. 
Nawet nie przejął się, że w środku statku wciąż były jaja właścicieli!
A gdy rzucił się za nim opłacony pościg bordowych pasowców, on ich bezwzględnie pozabijał, strzelając do każdego z rakietowego działka.
Na szczęście nie miał klucza warstw, nie mógł uciec z ich kwadry, doleciał do jej końca i rozbił się o ścianę wszechświata. 
W dodatku to była ściana drugiej kwadry, nie czwartej! 
Idiota tylko się oddalił od Ziemi. Może w akcie zemsty postanowił zaatakować Potworan?
\dm{} Spostrzegłam, że Mateusz chyba przestał rozumieć, ale Nadar nadal nawijał. \dm{}
Jak się rozbił, to lewitował w smoczej przestrzeni, w chmurze resztek statku kosmicznego przez pół megapulsa, prawie umierając z głodu.
W końcu te małe smoczki, których jaja były w rakiecie, wykluły się i zjadły go żywcem. Śmierć, jak w wielkiej maszynie do szycia.
Dobrze mu tak.
\ds{} Przepraszam, że wam przerwę, ale co ze mną? To jakiś test zręczności, albo inteligencji? 
Zginę, pożarty przez coś? \dm{} Nowy bezwstydnie wszedł w słowo, jak wpychający się w kolejkę do przepełnionego samolotu bez numerowanych miejsc, 
odlatującego zaraz z wyspy, która wkrótce ma myć zmieciona przez wielką falę tsunami.
\dm{} Co mam zrobić, żeby go zdać?
\ds{} Masz być sobą. \dm{} Odpowiedziałam równocześnie z Nadarem.
\ds{} Ciebie chyba postanowił przetestować Plazma. \dm{} Zamyślił się. \dm{} On lubi militarne klimaty, pewnie trafisz na Planetę Wojny.
Albo będziesz wyżynał jakieś miasto, albo sam będziesz wyżynany. Musisz sam zdecydować.
To bardziej test charakteru, niż umiejętności. Ta kula chaosu to wstrętne i niebezpieczne miejsce.
Najlepszą śmiercią... rozerwanie na kawałki przez jakąś futurystyczną wunderwaffe, najgorszą, prawdopodobnie wcielenie do Czarnej Armii.
Obedrą cię tam ze skóry, wyłupią oczy i zęby, wcisną w cyber-zbroję i zaleją powodującym szaleństwo, uzależniającym smarem khaki, 
będziesz umierał powolną śmiercią, dobrze się bawiąc przy mordowaniu niewinnych pod komendą Kryształowej Królewny.
Ja tam wolę robić to samo, nie będąc rozpuszczanym przez czarny kwas.
\end{dialogue}

Tymczasem rozległ się dźwięk dzwonu, oznajmiającego posiłek.
Poszliśmy zgodnie na najwyższe piętro statku, Mateusz tym razem trzymał się z tyłu, jak samochód w grze wyścigowej prowadzony przez kawałek taśmy klejącej, trzymającej klawisz gazu.
Przechodząc przez muzeum, Nadar poklepał radośnie gablotę z demonicznym mózgiem, który w odpowiedzi kłapnął groźnie szczerbatą protezą zębów.

Na najwyższym piętrze znajdował się salon, biblioteka, scena teatralna, ogród i kapliczka.
Dach przyjmował tutaj miłą, półkulistą wypukłość, ze szczytu zwisał żyrandol na lampy oliwne.
Mechalyczny zapytał mnie, dlaczego w ogródku rosną tylko ziemskie kwiaty.

\begin{dialogue}
\ds{} Nie wiem czy wiesz, ale Ziemia jest uważana przez wielu za najpiękniejszą planetę wszechświata \dm{} odpowiedziałam. \dm{}
A przynajmniej na pewno przez naszego Profesora.
\ds{} Tos? \dm{} Mateusz zwrócił się w kierunku muzeum.
\ds{} Tos jest bardziej niezwykły, niż piękny. Poza tym, grzyby trzeba by hodować w amoniakowej szklarni.
No i nie powąchasz ich jak kwiatów \dm{} wyjaśniłam. \dm{} Aha, jeszcze Tosowe życie puszcza wszędzie zarodniki. 
Jeden wdech atmosfery tej planety i fungusy zaczną ci rozpuszczać żywcem nos. Dwa wdechy i spleśnieją ci płuca. Trzy wdechy i grzybnia wkręci się w mózg, jak narracja mediów głównego ścieku.
\ds{} A kaplica? \dm{} Popatrzył na mały budyneczek w rogu.
\ds{} Jak pewnie zauważyłeś, Profesor jest bardzo religijny. Poza tym, dobrze mieć miejsce, gdzie można pomodlić się o ratunek, będąc atakowanym przez kosmicznych piratów. 
Każdy większy statek morski ma przecież kaplicę, to dlaczego statek kosmiczny także nie miałby mieć? 
\ds{} Jeszcze zapytam, kto zbudował Kulę?
\ds{} Ona jest bardziej strukturą wszechświata, niż konstrukcją, jest zapisana w Matrycy. 
Profesor otrzymał ją w prezencie od Nieba, kiedyś zrobił coś bardzo wielkiego i dobrego dla świata, dzięki czemu zyskał nadzwyczajną przychylność aniołów. \dm{}
Rozejrzałam się, szukając naszego gospodarza. \dm{}
Ale on nie lubi, gdy się o nim rozmawia.
\ds{} Tylko, kim on jest? \dm{} Mateusz nagle zapytał.
\ds{} W sumie nikt nie wie dokładnie, kim, lub czym, jest nasz niezwykły właściciel tej sferycznej zabawki, pomimo że z zewnątrz wygląda jak człowiek. 
Niektórzy mówią, że aniołem, inni, że dziwnym człowiekiem,
jest też teoria jakoby był ostatnim z jakiejś umarłej cywilizacji. 
Posługuje się jedynym w swoim rodzaju pismem i językiem, którego nikt inny we wszechświecie nie rozumie.
Widzi szerszy zakres barw, słyszy więcej dźwięków, nie wiem jednak, czy jest supersilny...
\end{dialogue}

Przed kontynuowaniem nieprzyjemnej rozmowy uratował mnie dzwonek, zwiastujący rozpoczęcie bankietu.
Odruch Pawłowa nakazał mi odczuwać radość większą, niż przy ogłoszeniu wyników na ogólnowszechświatowy konkurs ciast, który swego czasu wygraliśmy. 
Powinnam być rzadziej zapraszana.

\divider{}

Antyrax podniósł worek, Piotr Lektor był sztywny, jego wyraz twarzy poskręcany był w dziwności. 
Tylko jedno oko lekko mu drgało, niczym przepalająca się świetlówka.

Tymczasem wszyscy inni uciekli. Nie było już ani demonów, ani Neofantasora. Światostwórca nie był aż taki głupi, żeby uwierzyć, że niebezpieczeństwo minęło. 
Na pewno czyhali na niego, pochowani w ruinach miasteczka.

Stąpał cicho. Pomimo to, jego kroki były doskonale słyszalne w grobowej ciszy, spowijającej dolinę, niczym dźwięk chrupania na obiedzie bezzębnych staruszków.
Zero wiatru, zero ptaków, zero mieszkańców. Wszystko umarło. 
Przypominało to spacer po zasypanym świeżym śniegiem lesie, 
zwiedzanie zgliszczy po pożarze planety zniszczonej flarą słoneczną, 
siedzenie w uszkodzonej rakiecie bez zasilania w czasie kosmicznego pochówku żywcem.
Za chwilę on sam umrze, śmierć przyjdzie po cichu i autor nawet nie spostrzeże się, kiedy.
Zaskoczy go, jak kontroler biletów zaskakuje pasażerów tramwaju.

Wtedy wysuneła się zza winkla Winkla. 

\begin{dialogue}
\ds{} Pisze się ,,wysunęła'' \dm{} powiedziała i cisnęła w niego błędem ortograficznym. Ostrze wbiło mu się w pierś do połowy, jak nóż osoby nieumiejącej kroić mango.
\end{dialogue}
Antyrax poczuł truciznę dyktanda, rozpływającą mu się po żyłach. To był koniec. Żadna ilość abstrakcji nie wygra ze zwyczajną poprawnością językową.
Choćby stworzył kompletny, co do atomu, świat, to i tak niewiele by to dało. Neofantasor zdmuchnąłby go jak piasek z dłoni.
Powinien wrócić, i pisać programy komputerowe, zamiast tworzyć epikę. Przynajmniej tam kompilator powie mu o brakujących średnikach.

\begin{dialogue}
\ds{} Daj mi jeden powód, dla którego miałabym cię oszczędzić. \dm{} Demonica zawiesiła nad pisarzem olbrzymie ostrze z poprawnie zastosowanych dialogowych myślników.
\ds{} Mateusz musi dolecieć na miejsce, prawda? \dm{} odpowiedział.
\ds{} Jakoś leci i leci, a nadal nie wiadomo, gdzie tak dokładnie jest. Rzucasz na prawo i lewo pojęciami, zupełnie ich nie tłumacząc.
Uniwersalność, światłografy, warstwy, potwory. O co chodzi?
\ds{} Chciałem opowiedzieć o nich później, akcja rozwija się powoli, ciekawiej jest najpierw rzucić hasło, a potem dopiero je opisać.
Poza tym, obowiązkową częścią fantastyki naukowej, jest nieopisywanie niektórych zagadnień.
\ds{} Od kilku stron ta opowieść to jeden wielki opis! I nie jest to żadna fantastyka naukowa, nie ma w tym nauki. Są kule z sacroterii.
\ds{} Nic byś nie zrozumiała, gdyby nie opisy. Jakbym napisał, że polecieli na Felicję, używając górnej warstwy, chociaż Kula nie posiadał do niej kluczy, to co byś sobie wyobraziła?
\ds{} Że to jakaś nadprzestrzeń, coś w stylu alternatywnego wymiaru. A klucze to pewnie wysokotechnologiczne urządzenia do wchodzenia w nią.
\ds{} Prawie. Górna i dolna warstwa odpowiadają za obieg energii we wszechświecie. Górna rozprowadza, a dolna zbiera.
Bez opisu nie wiedziałabyś, że nie wolno korzystać z górnej warstwy, gdyż wprowadza to zawirowania w przepływie w czasoprzestrzeni pod nią.
Zawirowania powodują niedomiary i nadmiary Boskiej Siły, co się objawia większą skłonnością ludzi do popełniania grzechów, lub dobrych uczynków.
Każda ziemska wojna była spowodowana podobną fluktuacją.
W ogóle wszechświat składa się z czterech kwadr...
\ds{} Znowu aniołowie i chrześcijaństwo, wszyscy już o tym piszą. Nudne się to robi.
\ds{} Ale gdybym napisał, że statek Kula jest zasilany przez samego Belzebuba, a Profesor ma na swojej lasce czaszkę dziecka, oraz potwory były by tylko od niszczenia światów, to nie było by w tym nic niezwykłego, prawda? Może dostałbym porównania do Warhammera 40K? Otóż nie. Statek Kula działa na energię Boską, potwory służą aniołom, a ALOPP składa się z białych katolików polaków. 
I tak, są rasistami. Bo widzieli setki cywilizacji i wiedzą które upadają, a które nie.
Nikt nie opisał wcześniej takiego świata, a ja będę pierwszy.
\ds{} No więc opowiadaj. Masz kilka minut, zanim trucizna dotrze do twojego mózgu, a wtedy już nigdy więcej nie popełnisz żadnego błędu ortograficznego.
\end{dialogue}
Demonica wyprostowała się, jakby właśnie czekała z reklamacją przeterminowanego serka, na przyjście managera w sklepie z produktami o krótkim okresie do spożycia.

\divider{}

Zasiedli do stołu. Srebrne sztućce z diamentowymi akcentami oraz ręcznie rzeźbione talerze, dobrze współgrały z tkanym obrusem ze złotych nici.
Nigdy nie widział tyle zastawy dla jednej osoby. 
Otrzymał po trzy noże i widelce, łyżkę, łyżeczkę, widelczyk, pałeczki, dziwny szeroki nóż, trzy kieliszki różnych kształtów, duży talerz, dwa małe i jeden głęboki.
Do kryształowych szklanic Profesor nalał wszystkim titanicowego wina.

Na przystawkę było sushi z kawiorem i truflami. Gdy przyszli, było już nałożone na talerzu. 
Kula odmówił krótką modlitwę, dziękując za dar egzystencji, w imieniu wszelkiego życia, czasu i przestrzeni.

Jedzenie, jak wszystko, było bardzo wystawne. A jakże. Lecz w stu procentach pochodzenia ziemskiego.
Mateusz spodziewał się jakichś nieziemskich przysmaków, dziwacznych owoców, grzybów z Tosa, czy steku ze smoka.
Czy to była prawda, że wszechświat jest całkowicie pusty, a Ziemia jest jedynym znośnym miejscem w zimnym niebycie?

Bardzo zaskoczyło naszego głównego bohatera to, jak kulturalnie zachowywał się Nadar. Był arogancki i nie leżały mu galowe ubrania, lecz z zachowania wychował się na dworze królewskim.
Z kolei Katarzyna przywiązywała olbrzymią wagę do ubioru, ale nie potrafiła poprawnie złapać pałeczek.
Profesor miał naturalnie i jedno i drugie.

Gospodarz wstał i przemówił, jakby właśnie oznajmiał obywatelom stowarzyszonego kraju, że jego wojsko wygrało dla nich wojnę i uratowało ich od najazdu nieprzyjaciela.

\begin{dialogue}
\ds{} Tradycją jest, że przy głównym daniu wybieramy się w jakieś malownicze miejsce, dezaktywujemy tarcze i otwieramy dach, spożywając posiłek na świeżym powietrzu \dm{}
rozpoczął tajemniczo. \dm{} Proponuję, aby tym razem, nasz nowy gość wybrał miłą okolicę, w której będziemy mogli najlepiej delektować się dzisiejszą pieczenią z bażanta.
\ds{} Tylko nie środek oceanu, ani bezludna plaża. \dm{} Nadar bezwstydnie się wtrącił. \dm{} Mam na jakiś czas dosyć wody. 
\ds{} To może pustynia? Piaskowe wydmy są bardzo ładne w promieniach zachodzącego słońca \dm{} odpowiedziała Kasia.
\ds{} Nie, piasek wpada do jedzenia i chrzęści w zębach.
\ds{} Szczyt Andów? Ładne widoki z jednej strony na puszczę, z drugiej na ocean.
\ds{} Zimno tam jest, będzie ci zamarzać zupa na talerzu. Potem skończysz cała mokra od topniejącego śniegu.
\ds{} Amazonia.
\ds{} Komary.
\ds{} Antarktyda.
\ds{} Pingwiny.
\ds{} To chyba zaleta.
\ds{} Nie, jeśli podkradają ci jedzenie z talerza.
\ds{} Nie wiem, może Paryż?
\ds{} Śmierdzi, jak szambo przywódcy komunistów na Kryonii.
\ds{} Gejzery na Islandii.
\ds{} Dla mnie okej.
\ds{} Przepraszam bardzo, ale wyjątkowo nie przepadam za zapachem siarkowodoru. \dm{} Kula nagle się przyłączył. \dm{} Za bardzo przypomina mi moją przeszłość.
\ds{} Sawanna?
\ds{} Za dużo turystów na safari, ciągle robią zdjęcia. Poza tym lwy łaszą się o kawałki ze stołu.
\ds{} A może Etna? \dm{} Mateusz niespodziewanie wypalił.
\ds{} Etna? \dm{} Kosma nie kojarzyła.
\ds{} Środek wulkanu, kula do połowy zanurzona w płynnej lawie, fontanny ognia oświetlające otoczenie. Subtelne pomruki z wnętrza Ziemi.
Jeśli dobrze zrozumiałem, tarcze powinny nas przed magmą ochronić.
\end{dialogue}

Nastała niezręczna cisza. Czy Profesor śmiał się w duchu, że Mechalyczny przecenił zdolności statku, czy może rozpatrywał ideę poważnie?

\begin{dialogue}
\ds{} To doskonały pomysł. Aktywujemy wszystkie warstwy na raz. Normalnie spowodowałoby to, że czulibyśmy się 
jak w szklanej kuli, bez wiatru, bez temperatury otoczenia. Lecz w tym przypadku byłoby to i tak wskazane, ze względu na toksyczne wyziewy.
Nie ma nic lepszego od podziwiania lawowych rozbryzgów z odległości wyciągniętej ręki. Będzie, jak na powierzchni gwiazdy Edestii, gdy zapadała się w czarną dziurę.
\end{dialogue}

Mateusz nie spodziewał się takiego obrotu spraw, ale cieszył się niemiłosiernie na myśl o zapatrzeniu się w przelewający się żywioł.
Uczestniczenie w narodzinach supernowej zostawi sobie na później.

Tymczasem statek wynurzył się z Morza Śródziemnego, jak piankowa deska do pływania wylatuje w powietrze z basenu. 
Profesor niepostrzeżenie przemknął przez Cieśninę Gibraltarską i skierował się prosto w kierunku włoskiego wulkanu.
Ponieważ jednak Kula nie posiadała żadnych okien, goście mogli jedynie uwierzyć mu na słowo. Przynajmniej do póki nie otworzył dachu salonu.

\begin{dialogue}
\ds{} Nie wierzę, że nie zapytałeś się jeszcze Kuli, jak działa ten niezwykły twór. \dm{} Nadar zrobił sobie przerwę od jedzenia 
tłustego żurku i przykrył chlebową miskę, chlebową przykrywką.
\ds{} Czyli to nie jest tajemnica na równi z gabinetem Profesora? \dm{} Mateusz zapytał. \dm{} Myślałem, że nie wolno mi było tego wiedzieć.
Albo powiem inaczej. Nie chciałem być wywalony przez naszego gospodarza w kosmos tylko dlatego, że zadałem niewłaściwe pytanie w niewłaściwym momencie.
\ds{} Panie Mateuszu Mechalyczny. \dm{} Profesor uśmiechnął się tajemniczo. \dm{} Co innego wyciąganie od innych zakazanych informacji, a co innego rozlewanie barszczu. 
Ciekawość jest wysoko ceniona w ALOPP, a także przeze mnie. Proszę śmiało pytać.
\ds{} Zatem skąd to jedzenie? \dm{} Zaczął od najbliższej mu rzeczy. \dm{} Nie widziałem tutaj żadnej kuchni, nie widziałem spiżarni.
Te dania po prostu się tutaj pojawiły, jak przyszliśmy. Po przystawce z sushi odwróciłem się na chwilę i znalazłem zaraz przed sobą bochen chleba z zupą.
A sacroteria? Co to za materiał? Co może być na tyle silne, aby wytrzymać napór gorącej lawy? 
Skąd w skafandrach bierze się nieskończona ilość powietrza? Jaki jest tutaj obieg wody? 
Dokąd ucieka dym z pieca w łaźni? Jak się tym czymś w ogóle steruje? Skąd czerpie energię? Gdzie ma silniki?
\ds{} Odpowiedź na twoje wszystkie pytania znajduje się za tobą na ścianie \dm{} Kula oznajmił, jakby to było oczywiste.
\end{dialogue}

We wskazanym miejscu wisiał elegancki zwój papieru, oprawiony w polaryzacyjnie mieniącą się szybę i ramkę niczym wzburzony ocean kiślu.
Podobnie wyglądało do gabloty z diabelskim mózgiem i słoika na piklerze. Musiało więc to być szkło wymiarowe.

%TODO ENDE
\divider{} \divider{} \divider{} 

\niceframe{
\begin{Fontlukas}
\begin{center}
ŚWIATŁOGRAF
\end{center}
Z Mocy Najwyższego, potwierdza się nadanie specjalnej właściwości powstałemu Profesorowi \weirdchar{profesor}.

Sacroteriowy twór, w formie uniwersalnego statku kosmicznego \weirdchar{kula}, jest niniejszym przekazany Profesorowi od zawsze na zawsze, dla dowolnych celów.
Dokładny projekt został nieodzownie zapisany w Matrycy.

Nie pobiera się żadnej opłaty od właściciela.

Z Bogiem.\\
Departament Światłografów.
\end{Fontlukas}}

Fragment papieru u dołu wyglądał na pozaplątywany w dziwaczne supełki.

\begin{dialogue}
\ds{} Światłograf to odwrotność cyrografu. Daje ci, jak to jest ładnie opisane, pewną właściwość. \dm{} Nadar zaczął opisywać, zamiast Kuli. \dm{}
Może dawać ci żyć wiecznie, strzelać promieniami z rąk, wygrywać w lotka, eksplodować innym mózgi za pomocą pstryknięcia palcami, czy właśnie posiadać taką oto okrągłą rzecz.
\ds{} A sacroteria? 
\ds{} Sacrum i materia. \dm{} Tym razem Katarzyna rozpoczęła wyjaśnienia. \dm{} 
Materia, która wygląda i reaguje tak samo, jak zwyczajna materia, lecz może zachowywać się w pewnych przypadkach całkowicie po swojemu. \dm{} 
Wzięła w palce końcówkę obrusu. \dm{} Z jakiego materiału jest to jest zrobione? 
Powiedzielibyśmy, że z nici i złota, może jakiś jedwab, albo czego się tam używa przy szyciu obrusów.
Ale to jest sacroteria. Może i się zachowywać i plamić jak obrus, ale może równie dobrze zrastać po przerwaniu, jak żywa skóra, albo automatycznie solić leżące na niej potrawy.
W Matrycy jest zapisany algorytm działania tego obrusu, jak i zasady działania całej sacroterii we wszechświecie. 
Cała kula, i to jedzenie, także jest z sacroterii.
\ds{} Prawie, w Matrycy zapisano, że wytworzone potrawy są całkowicie zwyczajną materią. \dm{} Gospodarz poprawił. \dm{}
Jednak to nieistotne, gdyż nie byłby waćpan w stanie w żadnym stopniu doświadczalnie tego stwierdzić, jedynie zaglądając do Matrycy ma się całkowitą pewność. 
Matryca to prawdziwe miejsce, ma kształt wielkiej płyty, położonej nad całym wszechświatem, powyżej górnej warstwy. Naturalnie, nikt śmiertelny nie ma do niej dostępu.
\ds{} Zwykle za światłograf pobierana jest opłata, ilość dobra do wytworzenia. \dm{} Znowu Nadar zaczął. \dm{} 
Może opierać na liczbę uratowanych dusz, jakiś wielki czyn, czy właśnie, jak tutaj, na nic. 
Ale myślę, że jednak nasz Profesor, kiedyś coś tak fajnego wykonał, że Niebo go polubiło.
\ds{} A ta plamka? \dm{} Była trochę jak miejsce gdzie ktoś próbował załatać wypaloną dziurę w naprężonym namiocie, ale brakło mu materiału.
\ds{} To odcisk duszy Kuli. Światłograf musi być podpisany. Nie jakąś przyziemną krwią, lecz czymś wiecznym i niezniszczalnym, twoją duszą.
\ds{} Pora na widowisko \dm{} właściciel statku przerwał rozmowę.
\end{dialogue}

Stuknął laską i wtedy cały dach począł się otwierać na osiem stron, niczym kwiat. 
Żyrandol poleciał na jednym płatku na bok i spoczął w specjalnym do tego miejscu, nad sceną.
Pierwsze, co zobaczyli to niebieskie niebo otoczone czarnym kominem skał.
Widok przywodził Nadarowi i Katarzynie wspomnienia, gdy w przepełnionej rakiecie kosmicznej uciekali z eksplodującej, mechanicznej planety przez wylot silnika.

Czerwona śmierć przelewała się nad otwartymi kawałkami dachu i obijała o niewidzialną barierę, spływając majestatycznie.
Płomienne światło tworzyło wspaniałą, bankietową atmosferę.
Rozsunięte fragmenty dostawały z pełną siłą żywiołu, lecz czerwone futro wcale się nie paliło, lawa po nim spływała, 
jak po odpychającym wodę za pomocą swojej mocy Hdrze, oblanym na śmigus-dyngus przez Ferra, wiadrem do śmigłowcowego gaszenia pożarów.

Tak, jak Kula zapowiedział, na główne danie pojawił się pieczony bażant w sosie kurkowym.
Ponownie eleganckie jedzenie i ponownie z Ziemi. 
Czy dałoby się wygenerować potrawę idealną? 
Coś perfekcyjnego, lecz nieistniejącego?

A potem zdał sobie sprawę, że równie dobrze mógłby delektować się pizzą, pijąc aromat pizzy z próbówki.
To właśnie nieidealność potraw nadaje idealny smak.

\divider{} 

\begin{dialogue}
\ds{} Akcja, Antyrax! Daj mi akcję \dm{} przerwała mu Winkla. \dm{} Mam nadzieję, że w czasie tego bankietu zdarzy się coś ciekawszego.
\end{dialogue}

\divider{}

Wtem poczuli silne uderzenie. Wstrząsnęło statkiem, porozlewało wino z kieliszków i spowodowało, że Kula nie trafił widelcem w grilowanego ziemniaka.
Zobaczyli na niebie całą armię latających helikopterów, jeden z nich wystrzelił drugi pocisk i zaraz druga eksplozja zawibrowała otoczeniem.
Było dokładnie tak, jak w amerykańskich filmach o UFO. Z tą różnicą, że role się odwróciły.

Profesor wstał, ukłonił się, i zbiegł po schodach. Chwilę potem wrócił, niosąc olbrzymią, ozdobną tubę wmontowaną w taboret. 
Albo był to rozmontowany gramofon, albo broń masowego rażenia. Końcówka wiła się, jak zniszczona niepoprawną teleportacją.
Tuba była podłączona wężykiem do lejka, który to przyłożył sobie do ust.
Skierował otwór w górę.

\begin{dialogue}
\ds{} Jakim prawem przerywacie nam uroczystą konsumpcję bażanta, ciskając w nas wybuchowymi pociskami? \dm{}
powiedział pretensjonalnym tonem, a tuba wzmocniła jego dźwięk do potęgi megafonu. \dm{}
Polecam natychmiast opuścić krater wulkanu, inaczej komuś może stać się krzywda! \dm{} Odpowiedziała mu trzecia rakieta, kolidująca z tarczą.
\ds{} Wojsko Stanów Zjednoczonych, tajny oddział do walki z kosmitami. \dm{} Nocny wyciągnął z kieszeni okrągłą komórkę i podawał informacje.
\ds{} Proszę was. Nie musimy uciekać się do używania prądowych urządzeń. Schowajcie je z powrotem. \dm{} Nikt się jednak nie przejął uwagą starszego pana.
\ds{} Nadar. Masz coś mocniejszego, niż laserpała? \dm{} Katarzyna grzebała sobie pod suknią, niemal odwracając się na lewą stronę.
\ds{} Mogę tak ustawić dasery, aby strzelały w górę. Lecz wtedy potnie ich na kawałki.
\ds{} To może bierz tego dużego szyfratorem, wystraszysz resztę.
\ds{} Nie, bo spadnie do lawy i zginą ludzie.
\ds{} To ostatnie ostrzeżenie! \dm{} Kula nie brzmiał przekonująco nic a nic.
\ds{} Mam znikarkę, zniknę komuś pół kadłuba.
\ds{} Oszalałaś? Jeszcze uderzy w tarczę statku i sami znikniemy.
\ds{} \differentlan{You are being arrested under UN law.} \dm{} Dało się słyszeć stanowczy głos z góry.
\ds{} \differentlan{You are being destroyed under ummm... our own law.} \dm{} Nadar wyrwał Kuli lejek.
\ds{} Cokolwiek zrobimy, to spadną do wulkanu i umrą.
\ds{} No to zostaje pikler, potem trzeba będzie ustalić z Niebem, żeby cudownie wyciągnęli ich ze szkła wymiarowego.
\ds{} A co jeśli trafimy po drodze na uniwersalność? Wsadzisz ich razem, żeby się pozabijali? Nie mam zapasowych słoików.
\ds{} \differentlan{Please leave your spaceship right now!}
\ds{} Sam jesteś \differentlan{spaceship}. \dm{} Kula nie dawał za wygraną.
\ds{} Profesorze, czy ma pan jakąś defensywną broń na pokładzie?
\ds{} Owszem. Dobre słowo i miłość do wrogów.
\ds{} \differentlan{You have no power against army of the United Stat...}
\ds{} \differentlan{Shut up!} \dm{} Katarzyna przekrzyczała ryk lawowych fontann i wirników.
\ds{} Patrzcie! \dm{} Jednej z maszyn, pod wpływem gorąca, eksplodował silnik. Pilot wykatapultował się tak niefortunnie, że spadł prosto na szczyt półkulistej wypukłości tarczy.
\ds{} Kula wpuść go! Helikopter spada! \dm{} Po słowach Nadara, Profesor wskazał laską na żołnierza, który przesiąkł przez tarczę, zaraz w to miejsce uderzył wrak śmigłowca. 
Pomarańczowe światło przyćmiło na chwilę poświatę wulkanu. Kasia wystrzeliła szyfratorem w nieproszonego gościa.
\ds{} Mamy zakładnika, odpuście, albo go zab... coś mu zrobimy. \dm{} Starszy pan krzyczał do tuby.
\ds{} Oni nie rozumieją po polsku, Profesorze.
\ds{} \differentlan{You out, or he die.} \dm{} Mateusz nie spodziewał się po Kuli tak słabej znajomości angielskiego. Jednak amerykanie zrozumieli i przerwali ogień.
\ds{} \differentlan{We will negotiate, please open your forcefield.} \dm{} Statki powietrzne zniknęły z pola widzenia, zjawił się za to jeden mały helikopterek, z przywiązaną białą flagą.
Zbliżał się powoli i stanął w powietrzu, oczekując aż zniknie ochrona. Było w nim coś nietypowego, przypominał plastikową zabawkę, jaką kilkadziesiąt lat temu kupowało się dzieciom pod choinki.
\ds{} To pułapka! \dm{} Nadar naskoczył Katarzynie na plecy i skulił razem ze sobą. Chwilę potem olbrzymia eksplozja wstrząsnęła światem. Wszyscy upadli tam, gdzie stali.
\ds{} Ja pierdolę. \dm{} Nocny pomógł wstać Kosmatej. \dm{} Drugiej nie przeżyjemy. Musimy uciekać.
\ds{} Więc to koniec rozmowy. \dm{} Kula odłożył lejek na widełki, dotknął kulki na lasce, i wielki kwiat począł zamykać swe płatki. Zaczęli się wznosić.
\ds{} Stop! Tam, to chyba daser! \dm{} Dziewczyna wskazała palcem zieloną linię nad kraterem.
\ds{} Skąd...?
\ds{} Nie możemy wylecieć, bo przekroją nas na pół!
\ds{} A może uciekać do dołu? \dm{} Mateusz zaproponował.
\ds{} W dół?
\ds{} Do wnętrza Ziemi.
\ds{} Dałoby się. \dm{} Kula popatrzył w podłogę.
\ds{} Nie polecą za nami w lawę.
\ds{} I też spali się, cokolwiek wystrzelą.
\ds{} W spokoju wylecimy potem innym wulkanem, nawet na dnie oceanu.
\ds{} A zatem w dół. \dm{} Profesor odwrócił laskę.
\end{dialogue}

Białe zwierciadło poczęło się zanurzać coraz głębiej w Etnie.
Po kilku sekundach, zniknęło pod fontannami płynnych skał, niczym śnieżka wrzucona do gotującej się wody.
Hałas kotłującego się wnętrza wulkanu wolno niknął i nastała absolutna cisza.
Panowała atmosfera przegranej walki, goście podziwiali tekstury dywanów i uciekali wzrokiem od siebie nawzajem, jak gdyby chcieli się przepraszać za wyrządzone szkody.
Zaszyfrowane, pokryte niebieskawą poświatą, ciało żołnierza, leżało w kwietniku.
Połamane tulipany mizernie wyłaziły spod nieproszonej osoby.
Na jego całkowicie przeciętnej twarzy malowało się jedynie zdziwienie, nie było tam ani śladu jakiegokolwiek strachu.

\begin{dialogue}
\ds{} Co powiecie wszyscy na deser? \dm{} Profesor przerwał milczenie.
\ds{} Jemu też? \dm{} Mateusz spojrzał na mocno nieelegancki mundur wroga.
\ds{} Oczywiście. Tylko trzeba go odpowiednio ubrać.
\end{dialogue}

Wylecieli, rozbijając jeden ze szczytów Islandii.
Lodowa chmura wzbiła się w powietrze, jak rozbryzg śniegu na twarzy dziecka, gdy przegrywa w wojnę na śnieżne kule.
Tęczowy poblask tysięcy lodowych kryształków był przez kilka sekund dobrze widoczny.
Zaskoczeni turyści i nic nie przewidujący wulkanolodzy cykali im zdjęcia, z których później i tak nic nie wyjdzie.

\divider{}

Winkla popatrzyła się krzywo.
\begin{dialogue}
\ds{} Eeee, nieee. To ma być akcja? 
\ds{} Tak na wymuszenie, ciężko coś dobrego stworzyć. \dm{} Antyrax kładł się powoli na ziemi. Trucizna robiła swoje.
\ds{} Daj mu spokój, mnie się podobało. \dm{} Z cienia wyszedł Everywhere Man. \dm{} Ale tylko trochę.
\ds{} Trochę za mało, musi być idealnie.
\ds{} Nigdy się nie da napisać nic idealnie.
\ds{} Przepraszam, ja tu umieram! \dm{} Pisarz ledwo siedział.
\ds{} Niech mu będzie. \dm{} Winkla westchnęła, wyciągając strzykawkę z antidotum. Wyciąg z prac pierwszoklasistów. \dm{} Ale czekamy na prawdziwą walkę, w której na prawdę mogą przegrać.
\end{dialogue}

\divider{}

Przed chwilą spadałem prosto w objęcia kulistego UFO. Zaraz potem obudziłem się, przywiązany do krzesła, ubrany w jakieś cyrkowe ciuchy.
Siedziałem przy eleganckim stole, przede mną, na srebrnym talerzyku, leżał kawałek czegoś, co przypominało ciasto. Jedną rękę miałem wolną.
Trzy osoby tajemniczo się mi przyglądały. Wolałbym standardowo trafić na stół operacyjny, z próbnikiem w dupie.
\begin{dialogue}
\ds{} \differentlan{Here, have a dessert} \dm{} powiedział gość z irokezem. Naszła mnie ochota, żeby rozsmarować mu to ciasto na twarzy, ale może rzeczywiście skończyłbym wtedy z próbnikiem.
\ds{} Nadar, co z nim zrobimy? \dm{} Jeden z nich, młodszy, zapytał go po polsku. Niesamowite, rozmawiają po polsku! Lepiej nie zdradzać, że rozumiem ten język.
\ds{} Damy mu zjeść, a następnie rozkroimy i wsadzimy próbnik w dupę, żeby przeprowadzać chore eksperymenty.
\ds{} A nie warto wcześniej trochę go przepytać? Zaraz, co?
\ds{} Sam nam wszystko wyśpiewa, gdy będzie mutował się w krowę.
\ds{} Co, ale... auć... aha, tak najpierw w krowę, a potem w osła. Będzie boleć, oj będzie. \dm{} Młody nagle zmienił biegun.
\ds{} Potem podrzucimy jakiemuś niewyżytemu seksualnie farmerowi w Afryce.
\ds{} Mam lepszy pomysł. Wytniemy mu mózg, umieścimy w słoiku, i podłączymy do sztucznego ciała. Stanie się prawdziwym cyborgiem.
\ds{} Sprawdzimy, ile taki wojak zniesie orgazmów na godzinę. Pewnie wytrzyma z dwa dni sadystycznej męki, a później wysiądzie mentalnie \dm{} dziewczyna się odezwała.
\ds{} Podobno wojska specjalne Stanów całkiem dobrze sobie radzą w dziczy. Ciekawe, jak długo przeżyje sam w dżungli czerwonych kartaczy \dm{} starszy pan zajął głos. \dm{}
Te smoki bardzo lubią ludzi, wpierw pieką żywcem w swoim ogniu, a potem zjadają kawałeczek po kawałku.
\ds{} Albo od razu w całości na surowo, żeby ofiara utopiła się w kwasie żołądkowym.
\ds{} Nie, to za szybka i za prosta śmierć. Proponuję zawieść go na Tos, żeby wgryzły się w niego pasożytnicze grzyby.
\ds{} Super, przejmą nad nim kontrolę wystarczająco mocno, aby sterować ruchami i jednocześnie na tyle słabo, by zachować pełną świadomość.
\ds{} A może po prostu uwolnimy go? Odwieziemy od razu do domu.
\ds{} Doskonały pomysł, wsadzą go do wariatkowa i będą męczyć, żeby coś im o nas powiedział. Wyręczą nas z roboty. 
Ciekawe, jak zareagują na jego opowieści o bankiecie w kuli? Powie, że poczęstowali go ciastem?
\ds{} Pewnie trafi na stół operacyjny podziemnego laboratorium, z próbnikiem w dupie.
\ds{} Dość, zrobię wszystko, co mi rozkażecie! \dm{} zawołałem. Nie wierzyłem w ich groźby, ale też nie miałem ochoty przekonywać się, czy rzeczywiście nie są prawdziwe.
\ds{} Mówiłem, że to Polak? Swój, swojego wszędzie pozna. \dm{} Ten, którego nazwali Nadar, wykonał triumfalny gest. \dm{} Więc na początek zjedz ten przepyszny jabłecznik.
\end{dialogue}

Nie za bardzo miałem wybór. 
Gdyby chcieli mnie otruć, już by dawno to zrobili.
Poza tym, faktycznie wyglądał smacznie.

Niepewnie wziąłem widelec w wolną rękę i spróbowałem trochę kosmicznego jedzenia.
Smakowało, jak ciasto które robiła moja babcia, gdy jeździłem z wizytą do Polski.
Miękkie, kruche, i lekko ciągliwe.
Zjadłem całe i czekałem, aż zacznę mutować w krowę.

\begin{dialogue}
\ds{} Teraz do rzeczy. Skąd do cholery macie daser? \dm{} Nadar wyjął długą pałkę i uruchomił. Zielone lasery wystrzeliły równolegle do trzonu, a całość zawirowała.
\dm{} Wnioskuję, że domyślasz się, co to robi?
\ds{} Nie. Nie mam pojęcia. Nic nam nie mówili, nasz oddział dostał ten laser całkiem niedawno, nie pozwalali nawet go przetestować \dm{} odpowiedziałem zgodnie z prawdą. 
Byłem pewien, że i tak nie uwierzą.
\ds{} Zademonstruję ci zatem, jak działa. \dm{} Wziął mój karabin i skrzyżował z pałką. 
Przeszła, jak przez masło, dzieląc moją broń na dwie części. Stalowe ścinki posypały się na stół. \dm{} Skąd macie naszą technologię?
\ds{} Przysięgam, nie mam żadnej wiedzy, co ona robiła! Nie mówią nam tam niczego, wszystko jest w tajemnicy wojskowej. 
Lecąc na misję, nawet nie wiedziałem, z czym będziemy dzisiaj walczyć!
\ds{} Nadar, on chyba mówi prawdę, na pewno nie wtajemniczaliby go w zdobyte bronie kosmitów \dm{} dziewczyna przerwała.
\ds{} Jak wyglądało pudełko dasera? Co w nim było? Jak je przewozili? Do czego podłączali? \dm{} kontynuował.
\ds{} Tylko przez chwilę mi mignęło. Było w specjalnym śmigłowcu. W asymetrycznym pudle z dziurą w środku. Dwóch naukowców je obsługiwało. Chyba nie prowadzili kabli do niczego zewnętrznego.
\ds{} Jedno pudło?
\ds{} Tylko jedno, mówili że bardzo cenne.
\ds{} Gdzie je przetrzymują?
\ds{} Ten śmigłowiec dołączył do nas później, od innej strony, nie leciał razem ze wszystkimi.
\end{dialogue}

Mój rozmówca się rozluźnił i nawet trochę uśmiechnął.
Zdziwiłem się, tak samo jak pozostali.

\begin{dialogue}
\ds{} Chyba mówisz prawdę. To by znaczyło, że musieli ukraść nam kiedyś jakieś daserowe urządzenie, ale wciąż nie znają jego zasady działania.
\end{dialogue}

Milczałem.

\begin{dialogue}
\ds{} Mateusz, to będzie twoja pierwsza misja. Dowiesz się, gdzie trzymają ukradziony daser i odbijesz go z powrotem. 
\end{dialogue}

Mateusz przełknął ślinę.

\begin{dialogue}
\ds{} To co z nim w takim razie zrobimy? Teraz już na serio. Nie możemy go przecież tak po prostu wypuścić \dm{} dziewczyna zapytała poważniejszym tonem.
\ds{} Trzeba pokazać go potworom na sąd. Pewnie wsadzą naszego mordercę do kubistycznego więzienia.
\ds{} Czy to nie za ostro? Trochę go do mordowania też amerykańska armia zmusiła. Jest poza tym ta nowa planeta koncentracyjna do zsyłek.
\ds{} SS-manni także byli zmuszani do mordowania, żadna ulga mu się nie należy. 
I nie może być umieszczony z innymi ludźmi, jest za dobrze wyszkolony, wymorduje wszystkich pozostałych. To ma być zsyłka, a nie raj dla psychopatów.
\ds{} To może, nie wiem. Zaszyfrować go aż do końca wszechświata. To będzie, jak podróż w przyszłość, nawet nie zauważy.
\ds{} Zastanów się, to równa się śmierci. Obudzi się po bilionach pulsów tylko po to, aby zobaczyć Apokalipsę.
\ds{} Ja spróbuję go naprawić. \dm{} Starszy pan odezwał się po dłuższym czasie. Zdziwiłem się.
\ds{} Panie Profesorze, ta osoba jest niebezpieczna! Zdradzi i zabije pana.
\ds{} Ja wierzę, że każdy może się zmienić. Zrobimy z niego porządnego obywatela Felicji.
\ds{} Felicja jest przepełniona, nikt się tam więcej nie zmieści. Chyba nie chce pan wolny domowej?
\ds{} To zrobimy drugą Felicję, większą, dzikszą, o ustalonym prawie i dowolnej liczbie mieszkańców. Będzie równość i tolerancja dla wszystkich istot, będą mogły żyć w spokoju przed prześladowaniem.
Miejsce bezpieczne od przemocy, opresyjnych rządów i odrzucenia. Różnorodne i wspaniałe. Każda kultura wszechświata, stanowiące cudowną jedność.
Prawnie ustalę system, który dla każdego będzie równy.
A to będzie jej pierwszy osiedleniec.
\end{dialogue}

Wszyscy, prócz Profesora parsknęli śmiechem.
Był w nim pogłos stereotypowego polaka, cieszącego się z nieszczęścia sąsiada.

\begin{dialogue}
\ds{} To już lepiej na zsyłkę. Przynajmniej będzie miał szansę na dożycie starości \dm{} Nadar zakończył.
\end{dialogue}

Starszy pan, czerwony ze wściekłości, rozwiązał mnie i poprowadził od stołu. 
Za plecami słyszałem kolejne wizje, co by się na tej ,,lewackiej'' planecie działo.
Chłopcy rzucali obleśnymi pomysłami co do nowych praw, dziewczyna rozpatrywała kto, i jak musiałby ich przestrzegać.
\begin{dialogue}
\ds{} Zakaz jedzenia mięsa, bo zabijamy biedne zwierzątka.
\ds{} Ale takie zesłane smoki, jak mogłyby to przeżyć? One jedzą tylko mięso.
\ds{} Myślę, że potajemnie pożywiałyby się innymi obywatelami.
\ds{} Na pewno dałoby się napisać ustawę, rozwiązującą ten głodowy problem.
\ds{} No co ty, to przecież element ich kultury. Nie możemy nikomu zabronić być sobą, ty pieprzony rasisto, ha ha...
\ds{} ...nie wolno ci zakazać mi być rasistą... musisz tolerować moją nietolerancję...
\ds{} ...codzienne ćwiczenia seksualności...
\ds{} ...płeć będzie ustalona jako prosta?... jako przestrzeń może?
\ds{} Dzisiaj czuję się w $49 + 12i$ procentach kobietą... nie... rośliną.
\ds{} ...kwaterniony lepsze, czterowymiarowa czasoprzestrzeń płciowości...
\ds{} ...identyfikuję się, jako kamień... nie możesz mnie zjeść...
\ds{} ...jestem zjadaczem kamieni... mam takie prawo...
\ds{} ...czekaj, czekaj... a jaki jest wyznacznik macierzy płci takiego kamienia?
\ds{} ...cebula... inna odpowiedź mnie obraża...
\end{dialogue}

Schodziliśmy w dół po schodach, śmiechy na górze stawały się coraz bardziej niewyraźne.
Ten statek był gigantyczny w środku. Profesor Kula, jak mi się przedstawił, zapytał o moje imię.
Zaprowadził mnie do własnego pokoju i zostawił, nawet nie zamykając drzwi. 
Powiedział, abym dowolnie korzystał z dobrodziejstw Kuli i nie bał się prosić o pomoc. 
Zaproponował nawet kąpiel w basenie, ale oczywiście odmówiłem.
Pokój był bardzo malutki i bardzo elegancki.
Wszytko w tym obrzydliwym, rokokowym stylu rozjechanego czołgiem kota.
Większość stanowiło podwójne łóżko z daszkiem i firanką.
Do tego kilka krzeseł, stoliczek, szafka, toaletka.
To miejsce nie służyło do długotrwałego przesiadywania.
Gdyby urwać podpórkę od łóżka, rozbić lustro, związać firaną, rozłożyć na części krzesło, to mógłbym sobie stworzyć jakąś włócznię i tarczę.
Poczułem się, jak bohater filmów. Najpierw walka z pozaziemskim atakującym, potem próba przetrwania w niesprzyjających warunkach.
Na koniec zrobienie sieczki z kosmitów i majestatyczne eksplodowanie ich statku.

Kogo ja oszukuję, jestem bezsilny wobec ich technologii.
Zamroziliby mnie ponownie w czasie, gdybym tylko próbował coś odwalić.
Albo przetną na pół tą laserową pałką.
Będę musiał jakoś uciec. Na każdym statku kosmicznym są kapsuły ewakuacyjne.
Trzeba się rozejrzeć pod pretekstem zwiedzania.

Uchyliłem drzwi. Po drugiej stronie był podobny pokój do mojego.
Ujrzałem tam, tą dziewczynę w sukni.
Malowała się przez lustrem, spojrzała na mnie, uśmiechnęła się i pokiwała palcem, jak małemu dziecku.
Nigdzie nie pójdę.

Czyli to jest UFO, a oni są kosmitami.
Ale nie byli kosmitami. Tego jednego byłem pewien, no może poza Profesorem Kulą.
Bardziej przypominali gości z Ziemi, tak inni od właściciela. Pewnie ich zaprosił do siebie.
Gdzie w takim razie lecą?
Na wakacje, na inną planetę, czy do swojego kosmicznego domu?
Na pewno nie mieszkają na błękitnym globie, bo używają abstrakcyjnych technologii.
To oznacza, że istnieje jakaś pozaziemska cywilizacja ludzi, a może nawet polaków, poza naszym światem.
Skomplikowane to wszystko, nie mniej jednak, z pewnością nie zabiją mnie tak od razu.

Dopiero teraz zwróciłem dokładniejszą uwagę na swój ubiór.
Bardzo kosztowny i elegancki. Nie wiedziałem, gdzie jest mój oryginalny mundur. 
Nie miałem ze sobą nic innego, więc postanowiłem go zachować, może się przydać.
Usiadłem na łóżku. Było miękkie i wygodne, dawno nie leżałem na czymś takim, na chwilę położyłem się na plecach.
Nawet nie zauważyłem, kiedy zasnąłem.

Obudziło mnie głośne chrobotanie. Metaliczny dźwięk rozchodził się po całym statku.
Szklane ozdoby lekko dygotały, coś atakowało kulę.
Stwierdziłem, że skorzystam z zamieszania i wymknę się niepostrzeżenie.
Uchyliłem delikatnie drzwi, lecz na korytarzu nikogo nie było.

Moje eleganckie trzewiki hałasowały, jak na występie steperów.
Posuwałem się, zagłuszając swój ruch zewnętrznym hałasem.
Jeśli schodziliśmy w dół, a nie widziałem po drodze żadnego wyjścia, to znaczy że musiało być ono na najniższym piętrze, na którym jeszcze nie byłem.
Udało mi się dojść do schodów w dół, ostrożnie wychyliłem głowę zza sufitu dolnego piętra i zobaczyłem właz w ścianie kuli.
Tego szukałem.

Dziwny dźwięk wyraźnie był tu głośniejszy. Ktoś próbował przewiercić się przez drzwi.
Byłem pewien, że reszta mojej grupy przybyła mnie odbić.
Któż inny mógłby zaatakować latającą kulę?

Złapałem urządzenie do otwierania włazu i począłem kręcić, jakby od tego zależało moje życie.
W tym samym czasie usłyszałem za sobą zbieganie po schodach.
\begin{dialogue}
\ds{} Puść tę korbę! \dm{} Mateusz przybiegł, miał w ręce nóż do masła.
\ds{} Wypchaj się, wrócili tu po mnie \dm{} odpowiedziałem, odwracając głowę.
\ds{} Nikt po ciebie nie wrócił, jesteśmy pośrodku... \dm{} nagle zamilkł i otworzył szeroko oczy, jakby właśnie zobaczył ducha.
\end{dialogue}
Ostrożnie się odwróciłem, podejrzewając, że to wcale nie moja grupa przyszła mi z odsieczą.
To, co ujrzałem, zmroziło mi krew.

W otwartym na oścież włazie, na tle rozgwieżdżonego nieba, lewitował dziadek w wannie.
Patrzył się na nas tajemniczo, uśmiechając się. Na głowie miał czepek kąpielowy, w ręce trzymał słuchawkę prysznica, wszędzie były góry piany.
\begin{dialogue}
\ds{} Witam panów. Czy nie macie może pożyczyć trochę szamponu? Lecę już tak milion lat i wciąż nie mogę dokończyć kąpieli.
\end{dialogue}
Pokręciłem lekko głową.
\begin{dialogue}
\ds{} Nie szkodzi \dm{} zaśmiał się. \dm{} Umyję się tobą.
\end{dialogue}
Zamachnął się słuchawką, jak lassem, rzucił do środka i owinął ją wokół mojej nogi.
Począł ciągnąć z nadludzką siłą, przewrócił mnie. Złapałem się korby w ostatnim momencie.
Owinięta wokół stopy końcówka prysznica była jak macka, próbowałem ją strząsnąć, lecz zahaczyła się o sznurówki butów.
Szarpnął mocniej i obrotowa rękojeść korby zaraz wyślizgnęła mi się z objęcia, poleciałem dalej, w kierunku próżni.
Czepiając się palcami puszystego dywanu, zobaczyłem jak Mateusz nadal stoi, zahipnotyzowany.
Włosia materiału były za słabe, aby mnie utrzymać. Wciąż porywał mnie do siebie. Chwyciłem się ostatniej rzeczy przed śmiercią, framugi drzwi.
Wtedy też, połową mojego ciała poczułem zimną pustkę kosmosu.
Ktoś delikatnie objął mnie dłonią za kostkę, szarpnął, i plusnąłem w ciepłą wodę.

Wanna lekko się zachybotała po moim wejściu.
Zobaczyłem obok siebie wyszczerzoną twarz staruszka.
Adrenalina nie pozwoliła mi poczuć, że duszę się w próżni kosmicznej.
Złapałem się boku wanny, aby wyskoczyć, lecz moja ręka ześlizgnęła się, jakby była z mydła.
Popatrzyłem na swoje dłonie, które topiły się jak wosk.
Dziadek przejechał gąbką po mojej twarzy, to było jakby zabrał mi cały policzek.

Nagle różowy promień wystrzelił ze środka kuli.
Znalazłem się w nurcie rwącej rzeki, która ciągnęła wszystko z powrotem.
\begin{dialogue}
\ds{} Ojojojoj! Nieszczęście \dm{} zawołał dziadziuś.
\end{dialogue}
Zobaczyłem Nadara z wyciągniętym pistoletem, jego dziwny laser wciągał nasz obu. Wpadłem do lufy i zaraz uderzyłem we wklęsłą, szklaną, ścianę, rozpłaszczając się.
Wielka twarz obserwowała mnie czujnym wzrokiem.
Sekundę później od tyłu dobiła mnie lecąca wanna, a potem wpadający do niej kosmiczny podróżnik.
Woda przyszła ostatnia, zalewając wszystko, rozpuszczając mnie jeszcze bardziej.
Czułem, że umarłem, wciąż mając świadomość.

Po schodach zszedł Profesor Kula.
\begin{dialogue}
\ds{} Panie Mateuszu, specjalnie dla pana, wyszliśmy na chwilę z podróży górną warstwą, z powrotem do czasoprzestrzeni. \dm{} Nawet nie zauważył, co się tutaj przed chwilą stało.
\dm{} Znajdujemy się pomiędzy galaktykami,  w wielkiej pustce kosmosu. Najbliższa materia jest się trzy miliony lat świetlnych od nas. Proszę spojrzeć, 
o tam, to Droga Mleczna, nie jest niesamowita? Tak wygląda z zewnątrz, niczym biała spirala... \dm{} Przerwał, rozejrzał się. Popatrzył na Mateusza, Nadara i na mnie, zamkniętego w szklanej bańce.
\end{dialogue}

\divider{}

Winkla opuściła swój miecz i schowała. 
Nikt nic nie mówił, nikt nie reagował. 
Antyrax poczuł się obserwowany ze wszystkich, ciemnych zakamarków zaułka.
\begin{dialogue}
\ds{} Za bardzo skomplikowane i chaotyczne \dm{} odpowiedziała.
\end{dialogue}

\divider{}

Miałem dość. Omal nie zamieniliśmy się wszyscy w szampon.
Chciałem spać, chciałem wreszcie uciec od kosmosu, latającej po nim uniwersalności i kręcącego w nosie pudru.
Obudzić na miejscu, na Felicji.
Jednak nie byliśmy nawet w połowie drogi.

Zastałem Profesora i Kosmę w salonie, zasnęli przy kawie.
Z biblioteczki dobiegał głos wertowania książek.
Na stole leżała zabezpieczona kulka wymiarowa. Mały dziadziuś w środku mył się w najlepsze, niesamowitą ilością piany.
Od czasu do czasu kupka bąbelków poruszała się samoczynnie, jakby chciała zwiać z wanny, lecz nic jej to nie pomagało.
Zasłużył na to.

Obudziłem śpiochów i poinformowałem, że mam dosyć, idę się myć i spać.
Kula coś jeszcze mówił o jakimś występie kabaretowym na scenie, cieście biszkoptowym i chińskiej herbacie. 
Katarzyna podniosła się i wolno poszła w kierunku schodów.
Znowu spadła.

Miałem wrażenie, że wszyscy mieli dość, wszyscy z wyjątkiem Mecha.
Po przejrzeniu chyba całej biblioteki męczył mnie, żebym mu wyjaśnił jak złapałem uniwersalność do nieprzekraczalnego szkła wymiarowego,
co się stanie z porwanym żołnierzem, gdzie żyją smoki, kto to są potwory itp.
Siedziałem akurat w jacuzzi, uważając żeby nie zasnąć i się nie utopić i wybitnie nie miałem ochoty na rozmowy o infrastrukturze.
\begin{dialogue}
\ds{} Wasza praca zawsze tak wygląda? \dm{} torturował mnie.
\ds{} Tak.
\ds{} I często natrafiacie na takie dziwne zjawiska?
\ds{} Tak.
\ds{} I za każdym razem udaje wam się wygrać?
\ds{} Tak.
\ds{} Nie wierzę w to, kłamiesz.
\ds{} Tak.
\ds{} No dobra, a na górze w bibliotece jest Atlas Wszystkich Istot Wszechświata, ale jest trochę cienki. Większość zajmuje Ziemia.
\ds{} Tak.
\ds{} Czy to prawda, że istnieje tak mało żywych organizmów? Ledwo kilka planet? Kilkadziesiąt gatunków smoków?
\ds{} Tak.
\ds{} Zawsze będziesz odpowiadał mi ,,tak?''
\ds{} Tak.
\ds{} Jesteś chamski.
\ds{} O...tak.
\end{dialogue}
Poszedł sobie w końcu.

Wtem fala zimnej wody wylała mi się na głowę. 
Mech trzymał puste wiadro z wodą do polewania pieca.
\begin{dialogue}
\ds{} Zabiję cię \dm{} wysyczałem przez zęby.
\end{dialogue}
Nie uciekł, a stał tylko z wyrazem twarzy, który ja sam bym przyjął w takiej sytuacji.
Nienawidziłem go za to, że potrafił być tak podobny do mnie.
\begin{dialogue}
\ds{} Dobra Mech, właź. Pogadamy. \dm{} Już mi się nie chciało spać. więc równie dobrze mogłem obudzić się rozmową jeszcze bardziej.
\ds{} Mech?
\ds{} A jak mam cię nazywać? Mateusz to przecież nudne imię, a twoje nazwisko brzmi jak ,,Mechaniczny'', to prawie jak Mech. 
Ja bym był dumny z takiego przezwiska, znaczy gdybym nie lubił swojego własnego, słowiańskiego imienia.
\ds{} Niech będzie Mech, co właściwie tam się stało? \dm{} zapytał, zdejmując ubranie.
\ds{} A jak myślisz? Powiedz, co wywnioskowałeś z tego zdarzenia?
\ds{} No więc, uniwersalność to... taka jakby magia. \dm{} Zanurzył się w bąbelkach.
\ds{} Poprawnie. Zachowuje się jak stereotypowa magia, jaką znasz z książek i filmów. Jednak z tą różnicą, że nie można jej w żaden sposób kontrolować.
\ds{} Ale... \dm{} Myślał przez chwilę. \dm{} Jak ją więc przechwyciłeś?
\ds{} Sacroteria jest silniejsza. Sacroteria może być wszystkim, w szczególności może sterować uniwersalnością. Jest określana bezpośrednio przez zasady Matrycy.
\ds{} To nie wyjaśnia, jak dziadzio znalazł się w zamkniętej kuli.
\ds{} Pikler ma mały moduł tunelowy. Działa trochę, jak kwantowy efekt teleportacji cząstek. Z jednej strony wchodzi dowolna materia, zaraz pojawia się kilka centymetrów dalej, w słoiku...
\ds{} ...a ściana jest ze szkła wymiarowego, przez którą nic nie przejdzie.
\ds{} Poprawnie.
\ds{} A czy ten uniwersalny dziwak nie może po swojej stronie sam stworzyć czegoś podobnego do modułu tunelowego? I przeteleportować się z powrotem?
\ds{} Nawet jakby stworzył, nie będzie on działał, gdyż wewnątrz kuli nie działają zasady Matrycy, jego twór nie będzie miał swego rodzaju zasilania.
To trochę jak klucz bez zamka, albo silnik bez prądu. 
To Matryca mówi, że w miejscu, gdzie jest nasz moduł, ma powstać tunel. A nie, że moduł sam z siebie tworzy tunel. Moduł jest jedynie wskaźnikiem dla Matrycy. Rozumiesz?
\ds{} Chyba tak, czytałem coś o tym przed chwilą w bibliotece.
\ds{} Co innego czytać, co innego zobaczyć. Dopiero jak ujrzysz kwadratowe komórki przez sufit górnej warstwy i dotkniesz ręką ściany wszechświata, zrozumiesz.
\end{dialogue}

Obudził mnie dźwięk opuszczanego włazu.
Wygramoliłem się z eleganckiego, acz niewygodnego łóżka, przebrałem w normalne ubrania i zszedłem na dół.
Przy wyjściu spotkałem naszego Profesora.
Trochę nafukał na mnie, że jak śmiem chodzić po Kuli bez należytego ubioru, lecz miałem to gdzieś.
Wrota były otwarte na oścież, a czerwony dywan rozłożony w całej swej długości.

Rześka atmosfera porannej Felicji dobudziła mnie. Było już jasno, lecz słońce chowało się jeszcze za horyzontem.
Wylądowaliśmy na głównym placu w centrum, uliczny bruk mienił się od rosy, w gazowych lampach tańczyły małe płomyczki, choć nie rozjaśniały już nocy.
Nikt nie kwapił się do wyjścia na ulicę. Albo Profesor Kula tak często odwiedza ten świat, że wszyscy przywykli, albo nikt nie wyszedł, ponieważ akurat była niedziela.
W oddali z mgły wyłaniały się wieże lądownicze dla statków kosmicznych. Na jednej z takich, sądząc po kształcie, stał prawdopodobnie śmieciolot.
Na ławeczce siedział Mech i popijał pożeczkową herbatę.
Wszystko było w należytym porządku.
\begin{dialogue}
\ds{} I jak ci się podoba Felicja? \dm{} zapytałem.
\ds{} Wygląda całkiem przyjemnie, mógłbym tu mieszkać. Ale nie wiem, czy by mi pozwolili. Czytałem, że ta planetka została stworzona sztucznie, jako oaza szczęścia dla ludzi. 
Segregacja kandydatów na mieszkańców jest bardzo duża.
\ds{} Oaza szczęścia w teorii. W praktyce, oaza niestabilności, permanentnie leżąca na skraju wojny domowej. 
Albo dokładniej mówiąc, biorąc pod uwagę ilość obywateli nie przekraczającej setki, na skraju sprzeczki rodowej.
\ds{} Wiem, początkowo był tutaj perfekcyjny komunizm, jako poprawnie działający w bardzo małych społecznościach, na przykład w rodzinach, 
ale niedoszacowano ilości osób, dla jakich będzie jeszcze lepszy, niż gorszy i teraz wszyscy szukają alternatywnej ideologii do zastosowania.
\ds{} Ideologia i system społeczny zmienia się tu regularnie. 
Mieliśmy już megapuls monarchii absolutnej, megapuls czystej demokracji, megapuls tyranii rasowej, megapuls jakiegoś czegoś gdzie każdy miał chodzić w anonimowej masce, już nie zliczę.
\ds{} Jak myślisz, co teraz jest?
\end{dialogue}
Wzruszyłem ramionami.

Poprosiłem Mateusza ze sobą i poszliśmy w kierunku ratusza głównego.
Wszędzie było dziwnie pusto, jak gdyby nikt tutaj nie mieszkał. 
Chyba to jednak nie było z powodu niedzielnego poranku.
Na wszelki wypadek trzymałem rękę na szyfratorze.

\begin{dialogue}
\ds{} To logo ALOPP, co masz na piersi, czy ma jakąś symbolikę? \dm{} zapytał.
\ds{} Potworan z księżycami. Nasza główna planeta.
Na środku jest mały okrąg z pionową średnicą, oznacza bazową planetę i drzwi do lepszego świata, czy jakoś tak.
Drugi, większy okrąg z dwoma połączeniami wewnętrznego to cywilizacja potworów, jeśli dziewiątkę istot można w ogóle nazwać cywilizacją.
Nadbudówka nad powierzchnię, wielkie mosty, jakieś wieże, laboratoria, elektrownie, fabryki. Wszystko automatyczne. Ciągnie się od horyzontu po horyzont oznaczony poziomymi kreskami.
I jeszcze osiem kółek wokół to niezliczone księżyce.
\ds{} Nieźle.
\ds{} Ja wiem, mało ludzi, wielka pustka, nieistniejąca prawie przyroda, brak gwiazd na niebie, nie ma z kim pogadać. Wolę chyba tutejsze światy, ciekawsze.
\ds{} Zazwyczaj dom nie jest tym, co jest najbardziej niezwykłe na świecie, prawda? Musi być jedynie przytulny i bezpieczny.
\end{dialogue}

Wtem zza rogu wyskoczyło na nas potężne monstrum.
Było wysokie na kilka metrów, ważyło pięć ton i miało pomarańczowe łuski.
Wylądowało metr od Mechalycznego i ryknęło mu z całej siły prosto w twarz, strosząc kolce.
On jednak nawet nie zareagował.
\begin{dialogue}
\ds{} Plazma, nie wygłupiaj się \dm{} powiedziałem.
\ds{} Dziwne, zupełnie jak zamrożony przez Mikołaja \dm{} ryknął w odpowiedzi.
\ds{} Może umarł ze strachu tak szybko, że wręcz nie zdążył podskoczyć? \dm{} zaproponowałem.
\ds{} Ja cię znam \dm{} Mateusz się odezwał, oglądał potwora, jak eksponat muzealny. \dm{} Byłeś swojego czasu na targach fantastyki w Poznaniu, prawda?
\ds{} I to nie raz. \dm{} Pomarańczony obrońca wyszczerzył trójkątne zębiska. \dm{} Widziałeś mnie tam? Miło od czasu do czasu pochodzić wśród obcych, nie martwiąc się że zaczną uciekać z krzykiem.
\ds{} To ty zawsze wygrywasz konkurs na najlepszy cosplay, tak? Od początku byłem pewien, że to jednak nie był strój. Zbyt realistyczny.
Do tego czasami zapominałeś się i nie chodziłeś nienaturalnie sztywno, udając robota z kartonu. 
Wyginałeś pod sobą parkiet, żadna osoba nie utrzymałaby na sobie tak ciężkiego kostiumu \dm{} Mech się rozgadał. \dm{} 
Ale oczywiście, jak mówiłem wszystkim że to nie jest żaden sztuczny twór plastyka, tylko prawdziwa istota z kosmosu, to mnie wyśmiewali. 
\ds{} Widzę, żeś szczegółowy.
\ds{} Ja żyłem twoją tajemnicą przez kilka miesięcy! \dm{} opowiadał. \dm{} Ty nawet nie wiesz, jaką histerię wśród innych przebierających się spowodowałeś. 
Wszyscy chcieli wiedzieć kim jesteś i z jakiej gry była twoja postać. 
Właściwie to wręcz stworzyli twój trójwymiarowy model ze zdjęć, chcąc zbudować sobie twoją kopię.
\ds{} To by wyjaśniało, dlaczego ostatnio tak często mnie fotografują.
\ds{} Obejrzyj YouTuba czasami. Teraz wiem, że od początku miałem rację!
\ds{} To może następnym razem zrobię tak: \dm{} Plazma podniósł łapy, z których wystrzeliły ogniste płomienie i uformowały się w kształt Mateusza. 
Ognisty Mateusz spojrzał na prawdziwego Mateusza, prawdziwy Mateusz popatrzył z powrotem. Potem pomarańczowy duch rozmył się na wietrze. \dm{} 
Ciekawe, ile wystawcy zapłaciliby mi za zrobienie tak z ich logami?
\ds{} Zrób to, nie mogę się doczekać, jak YouTuberzy zaczną studiować książki od fizyki, próbując wyjaśnić to zjawisko. Może któryś spali swój dom, chcąc cię potem naśladować.
\ds{} Znamy się od niecałego kilopulsa, a ja już cię lubię. \dm{} Zamachał ogonem. \dm{} Nie myśl jednak, że dam ci fory. Lecimy na Planetę Wojny, dołączysz do Komodowej armii.
\ds{} Ta armia z pustyni, która ubiera się w pancerze wspomagane? \dm{} zapytał. \dm{} Kula miał u siebie książkę na temat tego świata. Tylko nie napisali, skąd nazwa.
\ds{} Sam się przekonasz, dlaczego Komodowa. Będziesz tam nowym rekrutem, przejdziesz wyczerpujące szkolenie na ich żołnierza. 
Komody szykują się na ofensywę Hirten, po tym jak grupa imigrantów z północy wprowadziła się do wyludnionego miasta, uciekając przed wirusem rozpylonym nad ich krajem przez Czarnych.
Czarni rozproszyli go, jako odwet za zajęcie złóż ropy na biegunie i sprzedaż jej po konkurencyjnej cenie do Kraju Adezeli, przez co spadł im eksport i... 
no sytuacja polityczna jest tam na tyle skomplikowana, że nawet politycy się w niej nie ogarniają.
\ds{} Ciekawie się tam dzieje.
\ds{} Niektórzy mówią też na nią Planeta Chaosu. Będę cię potajemnie obserwował i wydobędę z opresji, gdyby coś się działo...
\ds{} Przypomniało mi się. \dm{} Wręczyłem potworowi kulkę z dziadziem i półpłynnym szampo-żołnierzem. \dm{} O mało nas także nie zmydlił.
\ds{} Cholera, to już drugi, mnożą się, czy coś?
\ds{} Ten ma różową słuchawkę od prysznica, tamten miał zieloną. 
\ds{} Czyli pewnie wylał się z głównego słoja jakiś wszechświat wanno-dziadziusiów? Ilu ich jeszcze może być?
\ds{} Gorzej, że jego wąż jest ze zwyczajnej materii, więc ignoruje tarczę przeciw uniwersalności wokół Kuli.
Przypadkowo uratowaliśmy takiego żołnierza, ten tu wyciągnął go ze statku jak lassem, wrzucił do wanny i zmienił w pianę.
\ds{} Wąż nie jest jego częścią? Skąd oni wzięli te węże w pustce kosmicznej?
\ds{} Dziadziuś nie był rozmowny, ale wydusiłem z niego, że kupił je w pewnym sklepie Kanadzie. 
Zatelefonowałem do sprzedawczyni i potwierdziła, że kilka razy zjawił się u nich nagi mężczyzna, odziany jedynie w czepek.
\ds{} I nie zadzwoniła po policję?
\ds{} ,,To byłoby rasistowskie zachowanie.'' \dm{} Imitowałem wysoki głos.

%TODO Derepeater
\divider{} \divider{} \divider{} 

\end{dialogue}
Przerwały nam wołania dwójki osób z daleka. Przymoczarscy. 
\begin{dialogue}
\ds{} Hej, to zakazane pokazywać swoją postać! \dm{} Magda Przymoczarska dopadła naszą grupkę. \dm{} Wedle obecnego prawa, każdy obywatel ma obowiązek być niewidzialny.
\ds{} Co wy znowu odpierdalacie. \dm{} Przewróciłem oczyma. \dm{} Gdzie jest twój małżonek Michał?
\ds{} Michał przestrzega prawa. Wytłumaczę wam zaraz tą wspaniałą ideę. \dm{} Wciągnęła dużo powietrza. \dm{} Zatem. Nosząc pelerynę niewidkę, obywatel nie ma poczucia, że musi żyć tak, jak większość. 
Nie sugeruje się czynami innych, w związku z tym popełnia lepsze decyzje dla siebie.
\ds{} Ale także nie zostanie nigdy sprowadzony do bezpiecznej normalności, gdyby odleciał w swoje własne widzimisię, gdyż normalność nie jest w tym momencie zdefiniowana przez nieistniejącą społeczność. \dm{} Mateusz zauważył.
\ds{} Tak, ale skąd wiadomo, że powszechnie przyjęta normalność, zdefiniowana przez społeczność, jest najlepszym wyznacznikiem dla wszystkich? 
A może to właśnie ty masz najlepszy pomysł w rozwiązaniu problemu, a nie większość?
\ds{} Wtedy każdy będzie robił wszystko po swojemu i powstanie niekończąca się wojna o tak nieznaczące zagadnienia, jak sens noszenie ubrań, czy sposób używania sztućców, gdyż każdy będzie chciał to robić na swój własny, w jego mniemaniu najlepszy, sposób.
\ds{} A czy konieczne jest, aby każdy nosił ubrania? Abstrahując od tego, że są niewidzialni. Czy sensem życia jest sposób jedzenia, czy samo jedzenie?
\ds{} To co będzie bronić człowieka przez sraniem na ulicy, jeśli sobie wymyśli, że tak powinno się robić? \dm{} Mateusz się już denerwował.
\ds{} Wykształcenie. Będzie wiedział, że zarazki z kupy dosięgną jego samego, więc nie będzie popełniał autodestrukcyjnych czynów.
\ds{} A odpowiedzialność? Jeśli zrobię społeczeństwu coś źle, no nie wiem, zepsuję przypadkiem latarnię, to nikt nie dowie się, że to ja to zrobiłem.
W efekcie nikt nie będzie czuł się odpowiedzialny za dobro ogólne.
\ds{} Będzie, ponieważ sam korzysta z tego dobra ogólnego. I jeśli on popsuje lampę, to inni mogą zrobić to samo, z na przykład ławką, na której co poranek siada.
Zatem zakładając dobre intencje pozostałych osób, postara się naprawić wspomnianą lampę dla dobra ogółu, gdyż inni także naprawiliby rzeczy z których on sam korzysta.
\ds{} Nie można zakładać, że ludzie są tacy dobrzy. Jedna zła osoba rozwaliłaby cały system.
\ds{} Felicjanie są wybrani spośród wąskiej grupy osób. Wszyscy są wykształceni, mówią tym samym językiem, mają wspólną wiarę. 
Możemy zakładać rzeczy, niezakładalne w prawdziwej mieszance społeczeństwa Ziemi. Nie ma tu złych.
\end{dialogue}
Mateusz chyba odpuścił. Plazma jednak nie.
\begin{dialogue}
\ds{} Jebcie się wszyscy. Mateusz ma rację. Wystarczy jeden zły, aby całe to społeczeństwo przewróciło się na kolana. 
Co, jeśli tym złym nie będzie obywatel, ale na przykład podrzucony owoc klonu \dm{} zapytał rykliwym głosem. \dm{} Ktoś, kto wygląda jak jeden z was, ale zachowuje się zupełnie inaczej.
\ds{} Zniszczy go pierwsza, lepsza osoba, ponieważ jest wyedukowana w detekcji różnych zagrożeń wszechświata.
\ds{} Ale co jeśli będzie za słaba na zniszczenie tej postaci? \dm{} wtrąciłem. \dm{} Wiesz, może się zdarzyć nawet zdalnie sterowany, czerwony kartacz.
\ds{} Poruszamy już kwestię zewnętrznego bezpieczeństwa społeczności, a nie działania samej ideologii. 
Taki kartacz zostanie zabity przez zbiorową społeczność dla dobra ogółu, rozmawialiśmy przed chwilą o tym.
\ds{} Nie prawda, jeśli teraz wyleciałby na nas jakiś smok, którego pokonanie wymagałoby udziału kilku osób z bronią, to skąd miałbym wiedzieć, czy mam uciekać, bo jestem sam, czy walczyć, gdyż pozostałe osoby są obok mnie, ale niewidzialne. \dm{} Mech agresywnie gestykulował. \dm{} 
Już wiem. W waszej społeczności nie da się rozwiązać problemu, do którego rozwiązania potrzeba więcej jak jedną osobę.
\ds{} Racja... \dm{} Magda zapisała coś w swoim notesiku. \dm{} Więc niech każdy nosi przy sobie szyfrator, aby samodzielnie pokonać każde zło. A wszystkie inne problemy rozwiąże automatyzacja.
\ds{} Głupiaś. Ja, ognisty potwór, nie potrafię czasami się skutecznie obronić, myślisz że wystarczy dać człowiekowi broń, a będzie wszechmogący? \dm{} 
Plazma skończył rozmowę. \dm{} Chodźcie, trzeba zdjąć z Mateusza te rokokowe błyskotki.
\ds{} Stać, musisz byś niewidzialny, inaczej zostaniesz usunięty z Felicji.
\ds{} Według demokratycznego głosowania?
\ds{} Według mojej własnej decyzji, podjętej dla dobra ogólnego, popartej moim wykształceniem w kierunku...
\ds{} To ja prędzej was usunę. \dm{} Zamyślił się. \dm{} Niniejszym podbijam zbrojnie Felicję i ustanawiam się absolutnym królem do końca dnia. Cośtam, cośtam. 
Możecie zbrojnie wystąpić, ale nie widzę waszej wygranej z tym: \dm{} Wskazał kleszczami głowę, na której pojawiła się świecąca korona. \dm{} 
A to są moi niewolnicy. Zakazuję wam rozmowy z nimi.
\ds{} Ale...
\ds{} Wprowadzam prawo, że Magda Przymoczarska nie może się odzywać do końca dnia.
\end{dialogue}

Mechaniczny się chichrał, Magda wymachiwała rękoma, Michał wyłączył niewidzialność na pasie, jego wyraz twarzy mówił ,,a nie mówiłem?''

Poszliśmy do automatycznego krawca.
Na Felicji maszyny robią za ludzi większość roboty, lecz to ludzie muszą je obsługiwać. 
Inaczej oszaleliby z dobrobytu i nie szanowaliby swoich własności.
Jeśli chcą, na przykład, zrobić sobie ubranie, muszą najpierw posadzić bawełnę, zebrać ją, przygotować i na koniec wymyślić projekt do wydrukowania na maszynie.
W ten sposób sami pracują na swoje dobro.
Na Felicji nie używa się pieniędzy, ale im ktoś pracowitszy, tym bogatszy, gdyż sam sobie swoje dobro wyprodukował.
Pieniędzmi jest niejako wymiana usług. Ktoś zbierze dwa razy bawełnę, dla siebie i dla kogoś, kto zaprojektuje ciuch także jemu.

Krawiec znajdował się na głównym placu.
Za szybą stała maszyna, stylizowana na XIX wiek. 
Ekran komputera obsługującego miał piksele jako bolce wyskakujące z metalowej matrycy. 
Po oświetleniu ostrym światłem z góry, za pomocą cieni tworzył czarno-biały, lecz w pełni funkcjonalny, interfejs graficzny.
Doskonale to współgrało z ogólnym stylem planety.

\begin{dialogue}
\ds{} Nie mamy bawełny. \dm{} Zauważył Mateusz.
\ds{} Normalnie musiałbyś ją najpierw zebrać, żeby użyć w maszynie \dm{} Plazma chrząknął \dm{} ale trochę sobie oszukamy system.
Nadar, masz swój komunikator?
\end{dialogue}

Włączyłem na komunikatorze tryb ściągania materii, kolorowe cząsteczki zaczęły tryskać z obramowania, żeby utworzyć przed ekranem kawałek nici.
Ostrożnie pociągnąłem nowoformowany sznurek i wsadziłem do maszyny. 

Tymczasem Mateusz na analogowym ekranie projektował swoje ubranie. 
Jak na kogoś, kto miał pierwszy raz styczność z naszą technologią, szło mu całkiem dobrze.
Tworzył sobie bluzę i dresy, podobne do moich. Wygodne ubranie przyda mu się na Planecie Wojny.

Gdy skończył, wyciągnął z maszyny bardzo ciekawy zestaw. Miał potworowy charakter, lekki zarys kolców na plecach i rękach, oraz zwiększoną ilość materiału na rękawach i piersi.
Dzięki temu Mech wyglądał na nieco większego i groźniejszego.
Do tego wyprodukował sobie sporo zapinanych kieszeni. 
Spodnie były szerokie i nieuciskające, w sam raz do biegania i skakania nad przepaściami.
Kaptur miał wbudowaną kominiarkę, żeby chronić przez zimnem, lub wzrokiem innych.
Powiedział, że zawsze o czymś takim marzył.
Zaczynałem się go bać.

Poszliśmy następnie na dworzec tramwajowy. Pojedyncze, zasilane elektrycznie platformy na szynach, automatycznie wychodziły z tunelu.
Wsiedliśmy do jednej, Plazma poszedł na tył, rozwinął skrzydła, i swoimi silnikami odrzutowymi popchnął pojazd, biorąc nas na szybką przejażdżkę po planecie.
Było jak na kolejce górskiej, ale szybciej i mniej wyboisto.
Próbując przekrzyczeć hałas wiatru i silników Plazmy, pokazywałem mu mijane miejsca.

Felicja jest mniejsza, niż ziemski Księżyc, ale posiada kawałek gwiazdy neutronowej w jądrze, dzięki czemu jej grawitacja jest podobna do ziemskiej.
Minęliśmy pola uprawne, na których rosły wszystkie rośliny potrzebne do wyżywienia planety.
Dalej hodowle różnych, nie tylko ziemskich zwierząt.
Były automatyczne fabryki, na których można było dla siebie stworzyć dowolne przedmioty.
Istniały także elektrownie termojądrowe do zasilania całego systemu.

Ponadto małe morze z plażą i bezludną wyspą, ośnieżona góra do wspinaczki i jazdy na nartach.
Tajemnicza wielka jaskinia z mnóstwem zakamarków.
Był lasek z prawdziwymi dzikimi zwierzętami, bagnem, polanami i jeziorkami.
Wszystko małe i symboliczne, idealne do szybkiego wypadu.
Na większe wakacje należało zazwyczaj lecieć na inną planetę.

Zatrzymaliśmy się w felicjańskiej katedrze, na obowiązkową niedzielną mszę.
To zabawne, z tym nowym prawem niewidzialności byliśmy jedynymi w kościele. 
Otaczały nasz śpiewy znikąd, a nad ołtarzem lewitowały przedmioty.

Skończyliśmy wyprawę przy wieżach do lądowania statków kosmicznych.
Rozsiane były luźno na dużym terenie, aby nie przeszkadzały sobie nawzajem.
Zatrzymaliśmy się pod tą właśnie konstrukcją, którą widziałem wcześniej nad ranem.
Obły kształt śmieciolotu majaczył na szczycie.

Wjechaliśmy windą na korbę, Plazma polecił Mechowi kręcić.
Dzielny, wniósł mnie, siebie i Plazmę na sam szczyt, łącznie prawie siedem ton. 
Zmachał się tym całkiem nieźle, musiał przerywać kilka razy na odpoczynek, ale udało mu się.
Potwór był pod wrażeniem. Powiedział, że chciał go zahartować przed testem. 
To był oficjalny koniec bankietów, zwiedzań i przejażdżek. Teraz będzie ciężka praca.

\divider{}

Spostrzegł, że nie ma już ani Winkli, ani Everywhere Mana. 
Był sam w całym miasteczku.
Albo też demony chciały, żeby myślał, że jest sam.

Począł przechadzać się po wyludnionej, jak morderczy psychopata szukający swojej następnej ofiary. Zaglądał do dziur, wołał do studni. Nic.
Postanowił zwabić pozostałe demony tym, co lubiły najbardziej.
Usiadł na środku, zaczął opowiadać. Tylko co lubiły?
Musi pisać w odpowiednim stylu.

\divider{} %TODO Naprawa

\begin{dialogue}
\ds{} To jak tam u ciebie? Jak minęła podróż? \dm{} Zapytałem się Mateusza, puszczając stery śmieciolotu. W kosmosie i tak w nic nie uderzy.
\ds{} Jakoś.
\ds{} Słyszałem, że podziwiałeś mnie na targach w Poznaniu. Podobałem ci się?
\ds{} Taaak... \dm{} Mateusz zrobił dziwną minę.
\ds{} Przede mną nie musisz się ukrywać. Powiedz, co czujesz.
\ds{} Co ja czuję. \dm{} Skulił się w kąt rakiety.
\ds{} No to w takim razie powiem ci, co ja czuję. \dm{} Złapałem i zdjąłem mu majtki. \dm{} Pokażę, jak robią to kosmici.
\ds{} Puść, zostaw!
\ds{} W kosmosie nikt nie usłyszy twojego krzyku.
\end{dialogue}

\divider{}

\begin{dialogue}
\ds{} Dość, wystarczy! \dm{} Antyrax usłyszał wołanie ze studni. \dm{} Nikt nie chce pedalskiego, międzyrasowego kosmicznego gwałtu półjaszczura na człowieku!
\ds{} W nieważkości \dm{} dodał autor.
\ds{} Aaaahhh!
\end{dialogue}
No to wygląda na to, że to nie jest jedna z tych rzeczy, które demony lubią.
Spróbujmy jeszcze raz.

\divider{}

Śmieciolot był olbrzymim statkiem. 
Nazywały się tak, ponieważ były w pełni mechanicznie tworzone ze słabej jakości materiałów i na prostej mechanice.
Idealne na misje w niebezpieczne miejsca, gdzie ich ewentualna strata nie będzie dużym problemem. 
A co najważniejsze, nikt nie wyciągnie z nich żadnej cennej technologii.

W śmieciolocie numer 877 wszystko trzeszczało. 
Ten statek dawno powinien się rozlecieć, lecz nadal się jakoś trzymał.
Połowa urządzeń nie działała. Od systemu wykrywania uniwersalności, po żarówkę na korytarzu.

Lewitowałem w ciemnościach, niesiony bezwładnością w nieważkości.
Kluczące korytarze ciągnęły się po całym kadłubie.
Dwa razy przebijałem się przez zasłonę z kabli, raz trafiłem na ślepy zaułek.

W końcu doleciałem do kajuty Mateusza. 
Zapukałem, echo rozniosło się po ścianach.
Nikt nie odpowiedział, więc wszedłem do środka.
Kajuta oświetlona była przepalającą się żarówką.
Poduszki, rzeczy osobiste i papiery unosiły się bezwładnie po przestrzeni.
Nigdzie nie było człowieka, wyglądało na to, że opuścił to miejsce w pośpiechu.

Moją uwagę przykuła świeża ściana, stojąca na środku pokoju.
Wykonana była z dziwnego, metalicznego materiału.
Jej obecność nie miała konstrukcyjnego sensu.
Podpłynąłem, aby się jej bliżej przyjrzeć.

Wtem usłyszałem daleki chrobot i padła elektryczność.
Ciemność pożarła mnie razem z całym statkiem.
Zapaliłem ognik w ręce. Pomarańczowe, migotliwe światło wydłużało cienie i poruszało ścianami.
Spostrzegłem, że tajemnicza ściana zniknęła, nie było po niej żadnego śladu.
Jak to możliwe?

Wypłynąłem na korytarz.
Ogień mógł oświetlić jedynie kilka metrów w jedną i w drugą stronę.
Skierowałem się w prawo, bo stamtąd przyszedłem.
Leciałem środkiem przez kilka kilopulsów, aż trafiłem na zamknięty właz. 
Nie przypominałem sobie, żeby tutaj był. 
Złapałem go pazurami, zaparłem się ogonem o ścianę i spróbowałem przekręcić koło. 
Piszczący dźwięk rozniósł się po korytarzach, echo odpowiedziało straszliwym jękiem. 
Aż mi się łuski zjeżyły na ogonie.

Pokręciłem jeszcze, odpowiedział mi syk. 
Za drzwiami musiała być próżnia.
Mateusza na pewno tam nie było.

Poleciałem więc w drugą stronę, minąłem jeszcze raz kajutę w której byłem i zamarłem.
Przed biurkiem siedział wyschnięty szkielet. Nie Mateusza.
Jednak skąd się tu wziął? Nie pamiętam aby ktoś na ostatnich misjach zaginął. 
W ręce miał długopis, w kartce wybita była dziura od nadmiernego pisania w jednym punkcie. 
Nie dało się rozczytać, litery były zapisane jedna na drugiej po kilka razy.

Poleciałem dalej, trafiłem na ścianę. Jak to możliwe?
Obmacałem ją, jakbym nie był pewien, czy jest prawdziwa.
Składała się z tego samego materiału, co ta w kajucie.

Wściekły uderzyłem pięścią w środek, odpowiedział głuchy dźwięk, pazury weszły mi do połowy.
Ze środka zaczęła kapać gęsta ciecz i spływać po ścianie.
Przecież panowała nieważkość!
Odbiłem się w drugą stronę. Poleciałem z powrotem.
Lecąc, trzeci raz zajrzałem do kajuty.
Była zamknięta, chociaż pamiętam, że nie dotykałem wtedy drzwi.

Doleciałem do śluzy. Była otwarta na oścież.
Kto? Kiedy? Jak?
W śluzie latały papiery, pośrodku stała nienaturalna ściana. Znowu kajuta.

Wtedy wróciła elektryczność.
W żółtym świetle żarówki zobaczyłem, że ktoś leży w łóżku.
Mateusz wstał i przetarł oczy.
\begin{dialogue}
\ds{} Plazma, wyglądasz jakbyś zobaczył ducha \dm{} powiedział.
\ds{} Coś się dzieje. Mamy uniwersalność na pokładzie!
\ds{} Nie opowiada...
\end{dialogue}

Światło ponownie zgasło.
Na miejscu Mateusza wisiał w powietrzu szkielet.
Ściana także zniknęła.

\divider{} 

Nikt nie wyszedł. Może horrory to nie jest ich domena. 
Szkoda jednak psuć taki dobry horror. 
Może trzeba go przerobić.

\divider{}

Dość!
Skupiłem w ręce promień standardowy i strzeliłem w lewitujące kości.
Laser przeciął czaszkę na pół.

Potem posłałem falę ognia, wszystkie papiery zapaliły się jasnym światłem.
Otoczony tańczącymi iskrami odwróciłem się do korytarza.
Wycelowałem w ciemność i skierowałem w nią ogniki.
Ściany zajęły się ogniem, zmusiłem niepalne żelazo do palenia się.

Rozgrzałem pięść do białości, rozciągnąłem skrzydła i poleciałem w kierunku naprzeciwległej ściany.
Za mną rozciągało się ogniste piekło.
Rozciągnąłem na sobie pole siłowe.
Z całej siły uderzyłem w to samo miejsce, gdzie wcześniej.
Sztuczna ściana rozpadła się, jak szkło. Czarna ciecz ochlapała mnie, widziałem od wewnątrz kulistej bańki, jak spływa po ściankach, jak spala się w żywym ogniu.
Ostatkiem sił próbowała przebić się przez moją tarczę. Rozgrzałem ją wtedy do tysiąca stopni. Ciecz w mig wyparowała, jak woda wylana na patelnię.

Okazało się, że przebiłem się do maszynowni.
Języki ognia lizały mnie przez wybitą dziurę.
Zobaczyłem Mateusza. 
W jednej ręce miał karabin, w drugiej daser.
Daserem ciął wszystko, co się ruszało, karabinem strzelał naokoło, aby poruszać się siłą odrzutu.
Czarna maź formowała co chwila sztuczne ściany, pluła losowymi przedmiotami. 

Standardowym promieniem ciąłem ją na kawałki, nic to nie dawało.
Przynajmniej zwróciła na mnie uwagę i przestała atakować Mateusza.
Ogniste podmuchy widać trochę ją spowalniały, iskry przebijały się przez sztuczne ściany.

Zaczęło się robić gorąco. Pomyślałem o człowieku. 
Widać, że ocierał co chwila pot z czoła.
Ledwo dychał.

Na ścianie zobaczyłem skafander kosmiczny i butle z tlenem.
Ryknąłem i rzuciłem je człowiekowi, złapał.
Wtedy otoczyłem go kulistą tarczą, jak przed chwilą siebie i rycząc, wypuściłem z wnętrza cały mój wewnętrzny ogień.
Ściany poczęły się topić.
Uniwersalność zapaliła się ogniem. 
Guma z kabli wyparowała. 

Ostatkiem sił uderzyłem standardowym promieniem w dolną ścianę.
Przebiłem się przez zmiękczony ogniem kadłub i otworzyłem dziurę na próżnię.
Wielki lej ognia uformował się w kierunku wyłomu i porwał nas.
Wystrzeliliśmy w kosmos, jak korek z butelki.

Popatrzyłem się za siebie. Tył rakiety jaśniał pomarańczowym światłem, gejzer ognia wylatywał przez wybitą przeze mnie dziurę.
Mateusz, zamknięty w kuli, także podziwiał.

Gdy ubrał się w skafander, opuściłem pole siłowe i usnąłem z wyczerpania.

\divider{}

Nie?
Nikomu się nie podoba?
To może... nie wiem.

\divider{}

Lecę przez miliardy gwiazd. \\
Niczym martwy pingwin patrzący na spadający śnieg. \\
Nieobecnymi oczyma patrzy się w niebo. \\
Martwy. \\

Obca galaktyka. \\
Obce gwiazdy. \\
Obcy człowiek przy boku. \\
Wkrótce martwy. \\

Minie milion pulsów. \\
Minie miliard pulsów. \\
Nie będę martwy. \\
Lecz czym jest nieśmiertelność w nieistnieniu? \\

Rozżarzony statek. \\
Niczym iskra pośród gwiazd. \\
Niczym gwiazda pośród iskier. \\
Także zgaśnie. \\

Uniwersalność była. \\
Uniwersalność się skończyła. \\
Uniwersalność będzie na wieki. \\
Z wieczności uniwersalność powstała. \\

\begin{dialogue}
\ds{} Daj spokój. Zaczynasz przynudzać \dm{} Mateusz jęczał. 
\ds{} Trochę nudy potrzeba po tym, jak prawie zuniwersalizowało nas żywcem.
\ds{} Może. W każdym razie nie ma to znaczenia. Żyjemy, prawda? Poza tym jesteśmy już w układzie Planety Wojny. Za kila godzin trafimy do celu.
\ds{} Nieźle sobie poradziłeś, tak w ogóle.
\ds{} Ja? Przecież to ty mnie uratowałeś. 
\ds{} No, ja to ja. Ale tym miałeś jedynie laser i przeżyłeś! Mało kto przeżywa atak aktywnej formy uniwersalności. \dm{} Chciałem mu teraz wytłumaczyć jej podział na typy, ale zmieniłem zdanie. \dm{}
Skąd ty w ogóle masz daser? To zbyt cenna technologia, aby instalować ją w śmieciach. Pomyśl, co by się stało, gdyby uniwersalność ją przejęła.
\ds{} Wiem. Przeżyliśmy u Kuli atak wojsk USA, mieli ukradziony daser i prawie nas przecięli na pół.
\ds{} No, no.
\ds{} Katarzyna podarowała mi swój własny. Powiedziała, że ma być na czarną godzinę.
\ds{} Kto? Kosma? Pogadam z nią trochę. To nieodpowiedzialne.
\ds{} Przecież gdyby nie ona, mógłbym już nie żyć.
\ds{} Ale... \dm{} Miał rację, nie mogłem mu tego odmówić. \dm{} Sam widziałeś, co się dzieje z daserami w rękach wroga. Nie ma mowy, nie dostaniesz go na wyprawę.
Poza tym, i tak by ci go zabrali.
\ds{} Czy muszę na prawdę przechodzić ten test? To by nie wystarczyło? \dm{} Wskazał na oddalającą się iskrę.
\ds{} Już mówiłem. To test charakteru, masz się zachować, a nie walczyć.
\end{dialogue}

\divider{}

Wygląda na to, że nawet jego własne postaci nie chciały słuchać jego poetyckiej twórczości.
Antyrax postanowił podejść postapokaliptycznie.

\divider{}

Już z daleka zobaczyłem ogniste pociski, lecące w naszym kierunku.
Wyciągnąłem łapę i pozbawiłem je w jednej chwili ognia.
Bez odrzutu, spadły z powrotem na ziemię.
Zaraz potem pojawiły się tam wielkie grzyby atomowe, co za idiota strzela pociskami atomowymi w przelatujące samoloty?
Chyba, że wiedzieli, że nie jestem żadnym samolotem.

W pustynnej, zakurzonej atmosferze, zobaczyliśmy jezioro Hirten i przybrzeżne miasto o tej samej nazwie.
Teraz to nie nazywało się oczywiście Hirten, ale kto by tam ogarnął obecną sytuację polityczną.
Automatyczne systemy obronne skierowały na nas lasery. 
Wsadziłem Mateusza na grzbiet i przyjąłem pełen strumień na klatę. Gilgotało.

Zlecieliśmy, aby sunąć tuż przy ziemi.
Z piasku wyskakiwały na nas co chwila technogony, ale ja byłem za szybki, żeby dać się złapać.

Wlecieliśmy nad zatrute jezioro, śmiercionośne opary przyjemnie drapały w gardle.
Jednak ze względu na człowieka, którego niosłem w rękach, trzymaliśmy się daleko od powierzchni.
Hirten mieniło się w świetle słońca. 
To niesamowite, że nigdy jeszcze nie zostało zbombardowane. 

Wylądowałem na szczycie najwyższego wieżowca. 
Szklany taras był kiedyś apartamentem jakiegoś bogacza. W wyschniętym basenie leżały resztki damskich strojów kąpielowych.
Palmopodobne rośliny, posadzone w donicach, straciły całkowicie liście.
Wyblakłe kafelki pokryte były niezmywalną warstwą kurzu.
Tak kiedyś może wyglądać Ziemia.

Wybiłem szybę, żeby wejść do środka.
Było ciemno i nieprzyjemnie, przypomniało mi się wnętrze naszej rakiety.
Zapaliłem ognik i zobaczyliśmy stosy wyschniętych ciał, pośrodku parkietu do tańca.
Widać było, że umarli nagle, dusząc się. Atak chemiczny.

Musiał złapać ich w trakcie przyjęcia.
Uciekli z balkonu, zamknęli drzwi, i mieli nadzieję przeczekać.
Muzyka musiała grać aż do końca, a kolorowe płyty podłogowe pewnie migały jeszcze jakiś czas po ich śmierci.

Przyjrzałem się bliżej, była to rasa północnych.
Wyglądali jak brzydcy, chudzi ludzie.
Mieli na sobie i w sobie całe stosy elektroniki.
Transhumanizm... transextraterrestrianizm pełną gębą.
To mi podsunęło myśl, że niektórzy mogli przeżyć, ci którzy wcześniej zainstalowali sobie filtry w nosach.
Ale Komodowa Armia zapewne za kilka dni wkroczy do miasta, aby ich wykończyć.
Nie taki był plan, to miało się zdarzyć dopiero za rok.

Wyszliśmy na ulicę. Opadnięte samochody latające, wyłączone hologramy, uduszeni ludzie.
Ani żywej duszy. Usłyszałem w oddali tupot, podskoczyłem do najbliższego i schowaliśmy się w jakimś mieszkaniu.
Zaraz zza rogu wyszła trójka Czarnych. To oni zabili ludność?
Szkoda, że nie rozumiałem ich języka.

Poleciłem Mateuszowi zostać, sam podskoczyłem do nich.
Naturalnie zaczęli strzelać, ale ja nie byłem bezbronny.
Jednemu urwałem nogę, czarny smar khaki wymieszany z niebieską krwią, rozlał się po asfalcie.
Drugiego wyrzuciłem wysoko w powietrze, upadł kilkaset metrów dalej. 
Trzeciego unieruchomiłem, spajając mu ognistym promieniem łączenia w zbroi.

Głupio to musiało wyglądać, jak odpytywałem go, rysując obrazy na chodniku, pozostałościami jego kolegi.
Rozumiałem tylko jak przytakiwał i zaprzeczał.
Z tego się dowiedziałem, że rzeczywiście Komodowa Armia miała wykonać szturm na miasto, do którego przeprowadzili się imigranci z północy.
Jednak Kryształowa Królewna postanowiła potajemnie zatruć miasto, a potem ukryć w nim swoich podwładnych.
W ten sposób urządziła zasadzkę na Komodowych. Sprytne.

Podziękowałem mojemu informatorowi, a potem wróciłem do Mateusza i polecieliśmy pod osłoną nocy do najbliższej wioski pod władaniem Komodowców na pustyni.
Mateusz zapytał mnie, dlaczego zabiłem niewinne osoby.
Opowiedziałem mu, jacy ludzie są wcielani do Czarnej Armii, jak smar khaki zmienia charakter osoby do której wnika.
Najgorszy sort obywateli jest zakuwany z mechaniczne zbroje i zalewany odżywczym i bakteriobójczym środkiem, zwanym smarem khaki.
Stają się morderczymi maszynami, uzależnionymi od czarnej mazi. Maź jest uzupełniana częściej tym żołnierzom, którzy wykazali się większym okrucieństwem.
Kryształowa Królewna mądrze to wymyśliła.

\divider{}

Co się dzieje? 
Ciągle nikt nie wychodzi z żadnych dziur.
Znowu nie trafił z typem opowieści?
Nie było rady, jak spróbować ponownie.

\divider{}

Pociąg lewitował na strunie, tuż nad powierzchnią ziemi.
Święcący promień przechodził przez środek urządzenia, wychodził w silniku.
Prowadził maszynę, od podpory z pierścieniem, do podpory.
Silnik jądrowy dawał bardzo ładne, wiśniowe światło, trzy ogniki zostawiały za sobą strugę przyjemnego, mocno rozgrzanego powietrza.
Po bokach widniały laserowe działka, lecz nie reagowały na naszą obecność.

Podleciałem od spodu, aby ukradkiem zajrzeć przez okno i upewnić się, że to poprawny pociąg.
Wnętrze spowite było niebieskim światłem.
Rzędy smutnych wieśniaków siedziały, patrząc nieobecnymi oczami w pustkę.
To musieli być poborowi.
Odebrano im całą elektronikę, bez gogli rozszerzonej rzeczywistości nie umieli się zająć niczym innym, a zwłaszcza rozmową.

Hologramowy wskaźnik na ścianie wskazywał następną stację.
Nie znałem tutejszych języków, ale byłem pewien, że nikt na pokładzie pociągu nie potrafił czytać i pisać.
Z obrazka dowiedziałem się także, że końcową stacją jest duże zagłębie militarne Komodowej Armii.
O, to nasz cel.

Poleciałem przodem, wyprzedzając pociąg, aż trafiłem na małą wioskę przy wyschniętym korycie rzeki.
Bardzo biedna i nawet bez elektryczności.
Jedynym oświetlonym miejscem był przystanek kolejowy.
Wyglądał na wmuszony w slumsową, lepiankową architekturę, jakby spadł z nieba. Może tak właśnie było.
Czerwone hologramy ostrzegały przed nadejściem składu za kilka chwil.

Grupę młodych mieszkańców pustyni ustawiono na placu.
Płaczący rodzice żegnali się z nimi, nie oczekując że zobaczą ich jeszcze kiedykolwiek.
Dawali im na drogę ubrania, jedzenie, a nawet zabawki.

Wylądowałem po cichu w ciemnym zaułku i wypchnąłem Mateusza, aby dołączył do grupy.
Zapewniłem, że będę go ubezpieczał i że wezmę go, zanim pójdą wyżynać Hirten, a raczej, zanim wpadną w pułapkę.
Na tle mieszkańców Planety Wojny będzie wyglądał, jak upośledzony grubasek. Musi tylko ograniczać pokazywanie się na golasa.
Nie będzie także bariery językowej, każda wioska mówi tutaj różnym akcentem, więc nauczą ich od podstaw wspólnego języka.
Idąc, miał mord w oczach.

Wszyscy mówią, że przesadzam z tymi testami, ale uważam że jak ktoś potrafi przeżyć ideologiczne piekło, to przeżyje wszystko.
Nie można dopuścić, aby agent ALOPP podjął złą decyzję przy ratowaniu innych istot.
Czy Mateusz będzie próbował ochronić ich przed wpadnięciem w pułapkę?
Czy wywoła powstanie?
A może zaatakuje przełożonych?
Do czego jest skłonny, tego wszystkiego się dowiem.

\divider{}

Ze studni wygramolił się Myestro.
Podszedł kilka kroków, niepewnie, jak dzikie zwierzę.
Chyba był zainteresowany. Może należy kontynuować ten cyberpunkowy styl?

\divider{}

Drzwi otwarły się i smród wioski wdarł się do alemona.
Cała masa obrzydliwych wieśniaków zajęła ostatnie miejsca w pojeździe.
W tym jeden usiadł obok mnie, był jakiś dziwny.
Był ciemniejszy niż ja, bardziej pomarańczowy.
Widać teraz biorą do armii kogo popadnie.

Drzwi się zasunęły i klimatyzator począł czyścić powietrze ze stęchlizny. Niewiele to oczywiście dało.

Większość osób harała, pelne kisy spływały im z hateriów, jak małym dzieciom.
Nie wierzę, że jestem tutaj z nimi. Zamiast cieszyć się, że będą bronić swojego kraju przed Mroźnikami, oni wylewają masę hariów po nic.

Wzięli mi mój geter, musiałem siedzieć i patrzeć się przez okno na nudny krajobraz.
Światło duzji oświetlało otoczenie harbestowym, miękkim światłem.
Piasek, piasek i trochę skał.
W oddali, na horyzoncie widziałem łunę Torten, to miasto zostało bezpodstawnie zabrane nam przez Mroźników.
Czułem się dumny, że będę brał udział w jego odbiciu.

Mój sąsiad wyraźnie był na coś wściekły. 
Jak można być złym w takim momencie?
A może był upośledzony?
Spróbowałem nawiązać z nim kontakt, ale oczywiście nie rozumiał, co do niego mówiłem.
Standard.

Zza horyzontu wynurzyła się wielka forteca naszej armii.
Alemon zwolnił przed wrotami i zjechał na boczną duzję. 
Wielkie kleszcze złapały kadłub alemona i unieruchomiły go.
Sześciokątna rura przyssała się do drzwi, dał się słyszeć dźwięk spuszczanego powietrza.
Szkło się otwarło i czysta, pachnąca technologią atmosfera, zastąpiła wioskowy smród zgromadzonych.

Wychodzili pojedynczo, widać każdy chciał być ostatni.
Skorzystałem z okazji, przepchnąłem się przez śmierdzieli, żeby iść przodem, z dala od tej hołoty.
Co ciekawe, ten dziwny grubasek także chciał być na przedzie. Może jednak nie był z wioski?

Ruchomy regel prowadził nas przez odpowiednie korytarze.
Półprzezroczyste nizjeny wskazywały wszystkim drogę pismem obrazkowym.
Gdzieniegdzie przewinął się napis po zurdowsku.
Trochę głupio mi było przyznać, ale ja także nie rozumiałem zurdowskiego. 
Pomimo, że świetnie pisałem, czytałem i mówiłem w halat, behalat, po culsku i polivinowsku.

Dojechaliśmy do rzędu jaborysów. System wskazywał wielkimi medwerami, kto i jakiego jaborysa ma użyć do wyhutonowania się.
Poszedłem pierwszy, żeby pokazać innym, co to znaczy prawdziwa technologia.

Kabina była dość przestronna, jak tylko wszedłem, szyby straciły przezroczystość.
Nizjen wskazywał, że mam ściągnąć ubranie i wrzucić do dziury, co też uczyniłem.
Zimna para ochlapała mnie, potem jakiś śmierdzący detergent, potem ponownie woda.
Trwało to kilka razy.
Na koniec, świecąca ściana przeskanowała całe pomieszczenie.
Czytałem o niej, to hutona, paliła wszystkie niepożądane substancje w ciele.
Ja nie miałem ich za dużo, ale tamtym na pewno hutonowanie się bardzo przyda.

Posłusznie dałem się przeskanować, zapiekła mnie susetria, hateria i joty.
Lelonie paliły żywym ogniem, zarówno wewnątrz, jak i na zewnątrz. Ale to było zrozumiałe.
Straciłem również wszystkie dulice, ale przecież odrosną.
Wykończony, otrzymałem nowe ubranie.
Było bardzo proste, szare, lekkie i wykonane z faflocji.
Na lelonii i susterii miałem wydrukowany jakiś symbol. 
Fala z przecięciem z lewej, wpisana w wejt.
Mógł być to mój numer porządkowy, albo zurdowska litera.

Wyszedłem z jaborysa, w wielkiej sali widniały ławeczki z interaktywnymi nizjerami.
Usadowiłem się przy jednej, wyświetlił się mój symbol.
Krótka animacja wskazała, że od teraz będzie to mój numer porządkowy.
Usłyszałem, jak się wymawia. 
W zurdowskim systemie liczbowym był pierwszy w kolejności, byłem numerem jeden, to prezent za odwagę?
Nizjer wprowadzał mnie w tajniki nowego języka, z chęcią przyswajałem przydatną wiedzę.
Był niezwykle prosty.

Kolejne osoby korzystały z jaborysów, jedne bardziej krzyczały w trakcie hutonowania, inne były całkowicie cicho.
Mój pomarańczowy ulubieniec darł się w niebogłosy, aż mi się szkoda go zrobiło.
Wyszedł, wyglądając prawie tak samo. Widać miał tak niedorozwinięte dulice, że ich strata nie zmieniła mu wyglądu.
Jednak jego symbol był bardzo fajny. Symetryczny, podwójny fezel.
Wiedziałem już, jak się wymawia. Brzmiał groźnie.
Był w kolejności bardzo daleko od mojego, nie poprawiło mi to humoru.
Widać numery przydzielano losowo.

\divider{}

Myestro usiadł bardzo blisko.
To wystarczyło.
Antyrax zamachnął się na niego linią poleceń i złapał w powłokę.
ZSH trzymało go mocno i nie puszczało.
To wystarczyło, żeby zadać cios ostateczny.

Przyszykował program w BASHu, wbił mu go w ciało i uruchomił.
Wykryje on częste błędy w Antyraxowych powieściach, rozładowując demona słownikową perfekcyjnością.
Jednak program był póki co bardzo prosty, znajdował tylko błędy, które Antyrax popełnił wcześniej.
Mogło to nie wystarczyć.

Dla pewności kontynuował opowiadanie w cyberpunkowym kierunku.

\divider{}

Gdy wszystkich oczyszczono, na wielkim nizjerze wyświetlił się mój symbol, a po sali rozległa się jego wymowa.
Hos.
Poszedłem triumfalnie w kierunku wskazanych drzwi, wypinając na innych susterię z moim numerem.

Przechodziłem wąskimi korytarzami, prowadzony przez nizjery. 
Tutaj już nie było regeli, ale przecież byliśmy w wojsku, trzeba ćwiczyć, to zrozumiałe.

Zaprowadziły mnie do sporej salki ze stołami.
Na talerzach dymiły przepyszne vinite.
Niesamowite, czym nas tutaj karmili.
Usiadłem przy swoim symbolu, złapałem yltona i zacząłem jeść, nie czekając na innych.

Jak się spodziewałem, zaraz zjawili się pozostali.
Nie była to duża grupa, dokładnie dziewiętnaście osób.
W tym ten dziwak, Rohost.
Przydzielono mu miejsce na końcu mojego stołu.
Nie żebym się cieszył, że znalazł się w naszej grupie, ale może będzie z nim trochę ciekawiej.
Byle tylko nie opóźniał ćwiczeń.

Przycisk obok mojego talerza pozwalał dobrać sobie dokładkę.
Vinite wylatywały z zakrzywionej rurki.
Nieskończona ilość vinite! Gdybym wiedział, od razu zgłosiłbym się na pobór.

Wszyscy inni podchodzili do jedzenia z dystansem, wiadomo było, że we wsi trudno o składniki na vinite, zapewne większość z nich pierwszy raz widziała te pyszności na oczy.
Rohostowi absolutnie nie smakowało, jednak zmuszał się do jedzenia.
Co za marnotrawstwo.
Pomimo to zjadł pięciokrotność mojej porcji, gdzie on to wszystko mieścił?

Na ścianie wisiał kolejny nizjer.
Tam wyświetlono animację, z której wynikała pozycja i nazwa naszego oddziału w całej fortecy. Bedur, widać grupom przypisywano litery.
Dostaniemy niedługo własne pancerze wspomagane, zwane hiperami.
Będziemy w nich ćwiczyć walkę, wcześniej odbędzie się trochę siłowych i zręcznościowych zajęć bez nich.
Ćwiczenia potrwają cały sept, a potem szturmujemy Torten i wyżynamy Mroźników.
Dla mnie spoko.
Pierwszy raz w animacji użyto zurdowskich liter, ale w małych ilościach.
Jakiś wysoki rangą oficer powiedział coś w ich języku, a potem wszystko poszło do nowa.

Gdy wszyscy się najedli, ściana się otwarła, odsłaniając sporą salę z rzędami concatesów.
Surowo urządzone pomieszczenie oświetlone było białym światłem podłogowych żył.
Oddzielało concatesy od siebie, aby każdy miał swoją małą, pseudoprywatną przestrzeń.
Każdy concates miał odpowiedni symbol na futusi, oraz przydzieloną bulię z zestawem wszystkich potrzebnych urządzeń.
Było też wejście do personalnego jaborysa.

Znalazłem przydzielony mi concates, przejrzałem zawartość bulji, gdzie znalazłem prosty geter.
Nie pozwalał na komunikację ze światem zewnętrznym, ale miał sporo książek do nauki zurdowskiego, a także szczegółowe instrukcje użytkowania hipera.
Nie mogłem się doczekać następnego dnia, nie sądzę, żebyśmy już juro otrzymali hipery, ale kto wie.
Wsunąłem się pod futusę i zafulałem, jak tylko zgasło światło.

\divider{}

Nagle coś uderzyło Antyraxa w głowę, tak że wypuścił z objęć Myestra.

Kolejny przyszedł się zabawić?

Antyrax wstał, lecz nie mógł nikogo zobaczyć.
Cisza spowijała wymarłą wioskę dokładnie tak samo, jak wcześniej.

\begin{dialogue}
\ds{} Pokażcie się, bo dokończę ten zboczony fragment! \dm{} Antyrax zawołał. Nikt mu jednak nie odpowiedział. \dm{} Dodam do tego transformacje płci, korki analne i masochistyczne tortury.
\ds{} Aaah... \dm{} Ledwo żywy Myestro jęknął z ziemi.
\ds{} No dobra, lepiej zatkajcie uszy...
\end{dialogue}

Kolejne uderzenie zwaliło go z nóg. 
WickedH stanął przed nim, wcale się nie kryjąc.
Chciał więcej abstrakcji, jeszcze więcej.

Antyrax posłusznie wziął się za opowiadanie.

\divider{}

Światło powoli się rozjaśniało, budząc nas delikatnie.
Zauważyłem, że mój ulubieniec od dawna nie śpi, tylko rysuje coś na swoim geterze, zakrywając tył, aby nie było widać.
Bardzo dziwne.

Od razu poszliśmy na śniadanie, nawet się nie przebierając. I tak nie było po co.
Dzisiaj w stołówce było gorzej, jafanele nadziewane duduzją.
Nikomu nie smakowało, nawet Rohostowi. Wybrzydza, jak Kryształowa Królewna.

W czasie jedzenia nie było komunikatu, za to później ściana po drugiej stronie naszej sypialnie otwarła się i silne światło słoneczne oślepiło nas.

Otwór wychodził na spore patio w kształcie trójkąta, oddzielone wysokimi ścianami.
Wszechobecny piasek pokrywał je grubą warstwą.
Znajdowało się wewnątrz militarnej fortecy, można było założyć, że pozostałe patia ustawione są promieniście po okręgu, oddzielone od świata głównym murem.
Od zewnętrznej strony widniały schody okalające wyjście, prowadziły do wielkich wrót.
Zapewne tędy będziemy opuszczać fortecę.

W piasku wysunął się kolejny nizjer, ledwo go było widać w tym słońcu.
Z ukazała nam się twarz znajomego generała.
Mówił, mocno gestykulując w akompaniamencie różnorakich animacji.
Kazał się nazywać dyrygentem, był przydzielony do naszej grupy.

Powiedział, a raczej pokazał nam, że pierwszy fodor zajmiemy się ogólnym treningiem siłowo-zręcznościowym zanim w ogóle mamy myśleć o hiperach.
Będą też zajęcia teoretyczne.

Z piasku wysunęły się pale zakończone poduszkami i schodki.
Naszym zadaniem było przeskoczyć na drugą stronę, trudne zadanie, jak na pierwszy raz.
\begin{dialogue}
\ds{} Zumi \dm{} powiedział, wyświetlając symbol jednego z nas. Zumi zaczął gramolić się po schodach.
\ds{} Ofte. \dm{} Wskazał drugiego.
\ds{} Sisto.
\ds{} Matart.
\ds{} Lukne.
\ds{} Hos. \dm{} O, to ja.
\end{dialogue}
Wszyscy wcześniejsi pospadali w trakcie skoków, musiałem teraz pokazać się z jak najlepszej strony.
I udało mi się, spadłem dopiero na piątej poduszce, z dwudziestu.
Najdalej ze wszystkich.

Na koniec poszedł Rohost, przeskoczył wszystko, jakby od dawna to robił.
Niesamowite, na pewno nie był upośledzony. Kim w takim razie był?

Powtarzaliśmy skoki, aż wszyscy zdadzą.
Komu się udało, miał do końca zadania wolne, mógł szyderczo obserwować pozostałych.
Przeskoczyłem jako piąty i korzystając z zamieszania, zakradłem się z powrotem do sypialni.

Nikogo nie było, porwałem więc ukradkiem geter Rohosta i począłem przeglądać.
Nie próżnował w nocy, zapisał całe mnóstwo kreskowych obrazków.
Jakieś mapy i matematyczne obliczenia.
Zaznaczył pozycję fortecy, wioski przy której wsiadł do alemona, przebieg duzji, kierunek łuny, jezioro i... miasto Torten? Ale to jest źle, Torten jest w innym miejscu.
Do tego były wykresy z naszą planetą i słońcem.
Ponadto sporo całkowicie niezrozumiałego pisma.

Skopiowałem sobie wszystkie jego dzieła na własne urządzenie i odłożyłem mu jego geter z powrotem tak, jak był.
Wyszedłem na zewnątrz, ostatnie osoby kończyły zadanie. Nikt się nie zorientował.

Następne ćwiczenie było siłowe. 
Poszło mi bardzo źle, ale to Rohost był na końcu.
Jak to się stało, że umiał tak dobrze skakać, a był słaby jak wejdel.

Było jeszcze kilka zadań testujących naszą prędkość, reakcję i suternę.
Rohost ponownie poległ, nie był suterny nic, a nic.

Zmęczeni, zjedliśmy kolację składającą się z pieczonych didisów.
Każdy z nas poszedł się wyhutonować w jaborysie.
Rohost zakrył wejście do jaborysa swoją futusą, rzeczywiście wstydliwy jak Królewna.

Potem była część teoretyczna, w stołówce stoły skryły się w podłodze, robiąc z niej salę wykładową.
Animacje pomieszane z zurdowskimi literami uczyły nas o naszych wrogach.
Było sporo o Czarnej Armii, o Mroźnikach, ich kraju i mieście Torten. Było wspomniane o Kraju Daa, xuxowcach i Zagłębiu Curr.
Przypomniały mi się notatki Rohosta.

Wieczorem był czas na naukę języka, co też wszyscy bez wyjątku posłusznie uczynili.

\divider{}

\begin{dialogue}
\ds{} Ale się guzdrzesz! Daj życia do tej opowieści! \dm{} WickedH pogonił Antyraxa.
\end{dialogue}

\divider{}

Następny dzień był bardzo podobny do poprzedniego.
Śniadanie, ćwiczenia, obiad, wykłady, nauka języka.
Zaczęło się pokazywać, kto był najsłabszy z nas wszystkich.
Nie był to Rohost, gdyż w jednych dyscyplinach przodował, a w innych był na szarym końcu.
Nie byłem to też ja, na szczęście.

Wieczorem, zamiast nauki, postanowiłem przejrzeć ukradzione notatki.
Dyskretnie przeglądałem obrazy getera, rozumiejąc z nich coraz więcej.
Były mocno niekompletne.

Dwa dni później ponownie udało mi się szybko wygrać i ponownie zakradłem się do sypialni.
Przybyło sporo notatek, widać że robił postępy.
Moją uwagę przykuł również nietypowy zapach jego concatesa, nigdy nie czułem wcześniej niczego podobnego.
Poszperałem pod futusą i na okinach został mi dziwny proszek.

W drugiej części dnia uczyliśmy się o technogonach, zamieszkujących pustynię.
Te mechaniczne, drążące tunele maszyny, przejmowały władzę nad każdą elektroniką i jako takie były szalenie niebezpieczne dla naszej armii.
Dlatego nigdy nie należało zapuszczać się na otwartą przestrzeń samemu.
Forteca miała skomplikowane zabezpieczenia, w fortecy byliśmy bezpieczni.

Wieczorem ponownie pominąłem lekcje zurdowskiego, aby przejrzeć notatki.
Była to matematyka w liczbach o podstawie dziesięciu. Może inaczej zapisana, lecz idea nadal oczywista.
Nasz dziwak obliczył pozycję fortecy w abstrakcyjnej siatce otaczającej planetę, korzystając z pozycji słońca i gwiazd na niebie.
Wyliczył kąt padania światła słonecznego w czasie.
Nanosząc te dane na mapy, wychodziły mu sprzeczności.
Chodziło o to, że łuna Torten była oddalona od samego miasta o dobre kilkaset... jakichś jednostek.
Przydałoby się to zweryfikować.

Znalazłem także kilka artystycznych szkiców, pokazywały opuszczone miasto z czarnymi ludzikami.
O co mogło chodzić? Przecież nie o Torten, bo było ono zamieszkane przez Mroźnych imigrantów, wysokie, białe osoby.
To może miała być wizja przyszłości, jak już wybijemy miasto?

Przedostatniego dnia przed rozdaniem hiperów poprzenosili wiele osób do innych grup.
Połowę najlepszych zostawili.
Odbył się także konkurs strzelecki, chociaż bardziej polegał na mordowaniu niewinnych.
Mieliśmy powyżywać się na kukłach Mroźników i Czarnych.
Przy czym były to całkiem dobrej jakości elektroniczne kopie prawdziwych istot.
Nawet krzyczały i fajtasiły, gdy zadawało im się ciosy.
Można było pomylić je z rzeczywistością, dopiero prawdziwie zmasakrowane ukazywały ukrytą elektronikę.

Ja rozumiem odbijać miasto z rąk oprawców, ale atakować niewinnych, tylko dlatego, że byli naszymi wrogami?
Nie sprawiało mi to przyjemności.
Właściwie tylko ja i Rohost mieliśmy moralne opory przed masakrowaniem robotów.
Wszyscy inni wyżywali się całą swoją energią.
Słyszałem, jak opowiadali że z chęcią pozbyliby się nie tylko Mroźników, ale każdego kto stanie im na drodze. 
Nieprawdopodobne, gdzie ja się wpakowałem? To normalne zachowanie ludzi z wiosek?

Co gorsza, tego dnia zobaczyli naszą słabość.
W jednym pukju staliśmy się obiektem drwin i wyzwisk.
Jeszcze przed chwilą to my wywoływaliśmy poklask umiejętnościami, musiało to zagotować w pozostałych niezwykłe pokłady złości, które mogli w końcu wypuścić.
Jeden z nich narysował na dwóch kukłach nasze symbole. Po kilku chwilach zostały z niej pojedyncze śrubki.
Na szczęście był czas obiadu, więc ostateczna konfrontacja odsunęła się w czasie.

Wieczorem postanowiłem nadgonić zurdowskiego.
Rohost wrócił do sali późno, gdy pogasły już wszystkie światła.
Był mocno poobijany, widać, że stoczył z kimś bójkę. Na rękach miał enezję, czyżby uderzył kogoś pięścią w sudurę? To mogło go zabić. Mam nadzieję, że zabiło.
A potem przestraszyłem się swoich myśli, zamieniałem się w jednego z nich.

Rohost poszedł od razu do jaborysa i długo nie wychodził.
To myła moja szansa, nikt nie patrzył, wszyscy już spali.
Prześlizgnąłem się po podłodze, omijając concatesy i dorwałem jego getera, jeszcze raz kopiując sobie całą zawartość.
Potem zajrzałem od dołu pod futusę, zakrywającą wejście do jaborysa tajemniczej postaci.
Zobaczyłem kosmitę, całkowicie obcą osobę, jedynie kształtem podobną do nas.
Nie miał żadnych otworów na ciele, nie miał lelonii, czy sutratów.
Gdzie susteria, gdzie joty, a hateria?
Kim on do melmej delnej był?

Uciekłem czym prędzej na swoje miejsce.
Powinienem to zgłosić dyrygentowi.
Ale wtedy będę sam przeciw wszystkim.
Jeśli jest szansa, że Rohost wie cokolwiek więcej, to powinienem z tego skorzystać.

Otworzyłem moje zdobycze. 
Na pierwszej stronie był list w łamanym zurdowskim.
Rohost powiedział, żebyśmy trzymali się razem i żebym na przyszłość lepiej chował swój geter.
Nie doceniłem go.

\divider{}

WickedH zaczął chrapać.

\divider{}

Nadszedł ten dzień.
Ustawiliśmy się grzecznie przed wyjściem, oczekując na otwarcie drzwi na dziedziniec.
Rohost stał przygnębiony, z tyłu.

Najpierw pokazała się wąska szczelina.
Każdy próbował złapać przez nią jakikolwiek widok dziedzińca.
Przepychali się, odrzucając mnie na sam koniec. Nie spieszyło mi się.
Wtem drzwi się otwarły na oścież i wybiegliśmy na zewnątrz, niemal się tratując.

Na środku placu stało dwadzieścia nowiusieńkich hiperów.
Każdy, naturalnie opisany swoim symbolem.
Hiper Hos stał z tyłu, zaraz obok hipera Rohost.

Maszyny miały delikatną fakturę drewna, podkreślając dostojność i elegancję.
Kanciaste kształty powodowały, że wyglądaliśmy w nich znacznie masywniej, niż w rzeczywistości.
Gdzieniegdzie wychodził ręcznie rzeźbiony ornament.
Toczone koturny przytwierdzone były do podeszw zewisów, aby zapewnić dobrą przyczepność przy chodzeniu.
Środek susterii zdobiły szklane, szlifowane wstawki. Mieniły się w słońcu wszystkimi kolorami spektrum.
Po bokach przytwierdzone były rzędy srebrnych kulek, aby dopełnić uroku.

To właśnie ta elegancja naszych egzoszkieletów nadawała im taki postrach u wrogów.
Rohost tarzał się ze śmiechu. Co go tak rozbawiło?
Próbował się powstrzymywać, ale ilekroć spojrzał na stojące maszyny, ponownie go brało.

Pora na przymiarki.
Wspiąłem się na swojego hipera i wszedłem do środka.
Był ciasny i niewygodny.
Od razu miałem w głowie lekcje, na których straszyli nas, jak to bez hipera nie przeżyjemy nawet kukusty na pustyni.
Spróbowałem się ruszyć, ale nie udało się.
Chyba hiper był wyłączony.

Gdy wszyscy weszli do swoich muszli, pojawił się mi przed oczyma nizjer.
Dyrygent mówił z niego po zurdowsku, zrozumiałem jedynie ogólny sens wypowiedzi, trzeba było bardziej przykładać się do języka.

Chodziło o to, że teraz dopiero zaczyna się prawdziwe szkolenie.
Od tej chwili mamy zakaz opuszczania hiperów, pod groźbą porażenia prądem.
Urządzenia miały wbudowane tasery.
Na pokaz, wszystkim nam to zaprezentował, to były jak tortury, nigdy wcześniej nie byłem rażony.

W hiperach będziemy jeść, spać i hutonować się, gdyż mają wbudowany jaborys.
Możemy zapomnieć o naszych wygodnych concatesach.
Na te słowa zamknął wrota do sypialni. Na zawsze.

A teraz była pora na ćwiczenia, wszystko jak dawniej, ale z hiperami.
Z piasku wysunęły się kolumny z poduszkami.
Hos pierwszy.

Chodzenie w hiperze było trudne, ale dałem radę wspiąć się na schodki.
Skoczyłem na pierwszą poduszkę i odbiłem się tak mocno, że przeskoczyłem dwie.
Oczywiście spadłem w piasek, haterią w dół, przynajmniej nie bolało.
Dyrygent się zaśmiał i poraził mnie taserem. Więc to tak teraz będzie.
Wszystkie vinite świata nie były tego warte.

Drugi był Rohost, słyszałem jego jęki jeszcze jak spadał.
Trzeci Wewwt, ten nowy.
Bezbłędnie przeskoczył tor przeszkód.
Pozostali także radzili sobie całkiem nieźle.
Od razu widać było, kto teraz będzie najsłabszy, jak to możliwe?
Dlaczego?

W nocy załączył się nam jaborys na najmocniejszym trybie, pozbawił mnie ubrania.
Co oni planowali? Zamknąć nas w tych maszynach na zawsze? Przecież nie byliśmy Czarną Armią.
Od tego czasu wypatrywałem, czy przypadkiem nie podpinają nas do węża z pupallą, żeby zalać tą trującą zawiesiną.

Treningi zmieniły charakter, więcej było strzelań i mordowań, a mniej historii świata.
Mieliśmy być zawsze posłuszni dyrygentowi. Codziennie pokazywali nam zdjęcia łuny Torten na horyzoncie, jako naszego jedynego sensu życiowego.
Ale ja wiedziałem, to nie mogło być Torten.

Pewnego dnia, niedaleko fortecy spadł meteoryt.
Naturalnie wysłali nas, aby to bliżej zbadać.
Rohost jednak powiedział, żebym nie szedł. Próbował przekonać także pozostałych, ale nikt nie chciał go słuchać, a prawie go znów pobili.
Opowiadał coś o kosmicznej śmierci, o nienaturalnym pochodzeniu tego materiału.
Uwierzyłem mu.

Otwarli główne wrota, ukryliśmy się w zagłębieniach muru tak, że nikt nas nie zauważył.
Osiemnastka hiperów ustawiła się w rzędzie, najwredniejsi z przodu.
Mam nadzieję, że nikt nie będzie ich liczyć.
Gdy pierwszy żołnierz, Matart, doszedł do zgliszczy, coś wielkiego i czarnego wyskoczyło na niego.
Pozostali naturalnie poczęli strzelać, lecz w mgnieniu plucka srebrne ściany stanęły między nimi.
Znad tajemniczej blokady wypadły kawałki hipera.
Wybuchła panika, wszyscy, prócz Matarta, uciekali z powrotem do fortecy.
Chwilę potem potężna eksplozja rozniosła tamto miejsce.
Zgliszcza paliły się jeszcze przez pięć dni.

W ogólnym zamieszaniu, nikt nie zauważył że nawet nie wyszliśmy na zewnątrz.
Wszyscy zostali przepytani, co wiedzą o dziwnym zjawisku i skąd pojawił się ogień.
Wyniki śledztwa nie były zadowalające, toteż przez kolejne trzy doby mieliśmy najcięższy trening dotychczas, ciągle razili nas prądem.
Bez jedzenia, bez hurysowania się, bez snu. Jeden z nas, Qrton nie przeżył ciągłego testowania, umarł z wycieńczenia. Żałosne, zabijają własnych żołnierzy jako kara za utratę żołnierza.
Nawet Czarna Armia nie ma takich praktyk.
Brakujące miejsca Matarta i Qrtona szybko zostały zapełnione kolejnymi osobami.

\divider{}

\begin{dialogue}
\ds{} Wiesz co? Mam tego dosyć \dm{} WickedH powiedział. \dm{} Dawaj mi to. \dm{} Wyrwał Antyraxowi z ręki worek z alternatywnym wszechświatem i wskoczył do środka.
\ds{} Co ty robisz? Zginiesz tam. \dm{} Antyrax próbował złapać go za znikający w ciemności ogon. \dm{} Zginiesz tam \dm{} powtórzył i uśmiechnął się.
\end{dialogue}

\divider{}

Noc przerwało wielkie uderzenie i wycie syreny.
Światła na głównym murze zapaliły się pełną mocą i skierowały się w górę.
Pośrodku fortecy stała wielka, czarna postać.

\begin{dialogue}
\ds{} ,,A teraz zabawimy się.'' \dm{} powiedziała donośnym głosem. Nic nie zrozumiałem.
\end{dialogue}

Poczęliśmy strzelać z naszych wbudowanych w stroje mekerów.
Gorące kukety wybuchały na ciele atakującego, jeden po drugim. Nic to jednak nie dawało.
Zacząłem się bać. Pierwszy prawdziwy wróg. Rohost był tak samo zaskoczony, jak ja.

\begin{dialogue}
\ds{} ,,Co powiecie na to?'' \dm{} Uderzył pięścią z całej siły w środek fortecy. Wewnętrzne ściany rozsypały się w proch. Zobaczyłem, że żadna inna grupa nie posiada jeszcze hiperów.
\ds{} ,,Wypierdalaj z mojego opowiadania!'' \dm{} Rohost odkrzyknął. Nie wiedziałem, że zna jego język. \dm{} ,,Zabijasz niewinnych, oni mają wyprane mózgi, dlatego cię atakują!''
\ds{} ,,Jaka szkoda.'' \dm{} Przejechał ręką po głównym murze, zwalając wszystkie światła. 
\end{dialogue}
Nastała ciemność, rozświetlana jedynie xedowymi eksplozjami kuketów.
Po raz drugi zobaczyłem łunę domniemanego Torten.
Popatrzyłem w gwiazdy, przypomniałem sobie rysunki Rohosta.
To nie było Torten.

Głowa dziwnego stworzenia zapaliła się ogniem.
On jednak jakby nawet tego nie zauważył.
Pomarańczowy laser wystrzelił zza muru, ale także nie poprawił naszej sytuacji.
Wielka, czarna stopa spadała na nas.
Nie uciekniemy.
Wtem czarny gigant został odrzucony z wielkim impetem w bok, zniknął za murem.

\divider{}

Antyrax z całej siły kopnął w worek.

\divider{}

Panika, panika jakiej nie było nawet przy meteorycie.
Nie było świateł, wszyscy strzelali jak popadnie.
Co za banda idiotów.

Coś trafiło mnie w okolice querta.
Odwróciłem się, to Rohost walił we mnie kawałkiem pukpula.
Krzyknąłem, że to ja, Hos.
On jednak uderzył mnie znowu, aż zabolało.
Zamachnąłem się, by mu oddać, ale on wywinął się i zdzielił mnie ponownie.
Prąd przeszył mi ciało, moduł do zadawania kar włączył sie niekontrolowanie. 
Co za geniusz, zabije mnie moim własnym hiperem.

Mój były kolega jednak nie przestawał bić, padłem na ziemię, a on walił dalej.
A niby tak bał się mordowania niewinnych, wiedziałem, że nie można mu było ufać.
Taki czyn wzbudzi uznanie wśród pozostałych.

Wtem prąd się wyłączył, a mój nizjer zasygnalizował uszkodzenie modułu hipera.
Popatrzyłem na Rohosta, stał, opierając się o pukpul.
Wsadził mi go w rękę i pokazał miejsce na swojej zbroi, gdzie przed chwilą mnie bił.
Zrozumiałem.

Nie było mi łatwo i trwało to znacznie dłużej, niż w jego przypadku, lecz udało mi się ,,wyłączyć'' i jego taser.
Co teraz?
Teraz uruchomił tryb głośnomówiący.
\begin{dialogue}
\ds{} Nasza armia zdrada. \dm{} Próbował łamanego zurdowskiego. \dm{} Dyrygent Mroźnik.
\ds{} Atakować, nie rozmawiać! \dm{} Odpowiedział mu głos dyrygenta.
\ds{} Hirten pułapka... \dm{} kontynuował.
\ds{} Torten \dm{} poprawiłem go. \dm{} Hirten było przed najazdem Mroźników.
\ds{} Kaporten. \dm{} Usłyszałem nieznajomy, rykliwy głos. \dm{} Kaporten Czarna Armia trucizna imigranci północ wy pułapka.
\ds{} A światła? \dm{} Wysłałem zapytanie w eter nie mając pojęcia, kto słucha, a kto mówi. \dm{} Światła są obok Torten... Hirten... Kaport... tego miasta.
\ds{} Światła pułapka Mroźnicy my. \dm{} Rohost odpowiadał. \dm{} Dyrygent wie podstęp. Dyrygent nas śmierć.
\ds{} Hos, Rohost. To wasze ostatnie słowa. Do widzenia. \dm{} Dyrygent wariował, nic się jednak nie stało. \dm{} Co zrobiliście z hiperami?
\ds{} To, co trzeba było zrobić już dawno.
\ds{} To naruszenie zasad! Wszyscy, zabić Hosa i Rohosta. Rozkaz. \dm{} Wycelowała w nas osiemnastka mekerów.
\ds{} Pamiętajcie, kto jest prawdziwym wrogiem. To niekończąca się wojna. Wszyscy zostaliście osz... \dm{} Eksplozja ogłuszyła mnie. Zobaczyłem tylko pomarańczowe światło.
\end{dialogue}

Gdy się obudziłem, był już dzień. Wszędzie wokół tliły się szczątki fortecy i mojej byłej armii.
Musiał wybuchnąć pożar, jednak mnie jakimś cudem ominął.
Przeszukałem zgliszcza, lecz nic nie znalazłszy, postanowiłem wrócić do cywilizacji wzdłuż duzji. 
Niektóre certy były poprzewracane, innym brakowało pierścieni, żaden alemon nigdy więcej tędy nie pojedzie.

Wtedy sobie przypomniałem wykłady, wychodziłem na pustynię. Technogony z pewnością zwęszą wkrótce mój hiper i wykopią się spod ziemi.
To koniec, chyba że zostawię hipera tutaj, ale wtedy idąc, pewnie i tak umrę z pragnienia.
Były jeszcze pozostałości tajemniczego meteorytu, ale na nie to nawet bałem się spojrzeć.
Postanowiłem zostać w ruinach, niech hiper będzie moim grobem.

Tej nocy śniły mi się dziwne wydarzenia z Culitrią, jakie przeżyłem będąc dzieckiem.
Rankiem przyleciało do mnie dziwne stworzenie.

\divider{}

Antyrax wyciągnął WickedH z worka.
\begin{dialogue}
\ds{} Żyjesz? \dm{} zapytał. 
\end{dialogue}
Nie żył.

Zostało jeszcze dwóch.
A jemu właśnie skończyły się pomysły.

Jak na zawołanie, stanęli przed nim Poriux i Mały Gołąb.
\begin{dialogue}
\ds{} Co? Właśnie skończyły ci się pomysły, nie? \dm{} Poriux zapytał.
\ds{} Tttak... \dm{} Antyrax się przyznał.
\ds{} Zapewne masz jeszcze wspaniałe zakończenie, ja z chęcią je przyjmę. Lubię zakończenia. \dm{} Mały Gołąb wyszczerzył demoniczne zęby.
\ds{} Ty? Zakończenie? To mi się należy zakończenie! \dm{} Poriux się oburzył.
\ds{} To ja je dostanę, mam brązowe piórka u Neofantasora, a ty masz ledwo dwa opowiadania w bibliotece! \dm{} Zaczęli się szarpać.
\ds{} Tylko dlatego, że poprawnie piszesz, a nie dlatego, że ciekawie.
\ds{} Nie ma znaczenia myśl, jeśli nie możesz jej odpowiednio przekazać!
\ds{} Nie ma znaczenia najładniejsze pismo, jeśli nie niesie w sobie żadnej myśli!
\end{dialogue}

Antyrax usiadł więc na ławeczce, pogrzebał zamaszyście w worku i wyciągnął tak oczekiwane przez wszystkich czytelników zakończenie.
Rzucił nim w sprzeczające się demony.

\divider{}

\begin{dialogue}
\ds{} No cóż, i tak całkiem nieźle ci poszło \dm{} Plazma powiedział, nawet się do mnie nie odwracając. 
\ds{} Zamknij się. \dm{} Odrzuciłem. \dm{} Mam dosyć tego waszego świata. Ten test był niewykonalny, nie dało się nikogo uratować.
\ds{} Dało się, lecz trzeba było do tego trochę więcej sprytu. Mogłeś chociaż uratować sam siebie. A tak wracasz do swojego bezużytecznego życia na Ziemi.
\ds{} Wolałem zginąć, niż nie próbować, niepotrzebnie mnie ratowałeś.
\ds{} Obiecałem, że nic ci się nie stanie.
\ds{} To mogłeś uratować i Hosa, co ci szkodziło?
\ds{} Nie mieszamy się w niezapoznane cywilizacje.
\ds{} Jedyne, co robicie, to się mieszacie. Najpierw trójka Czarnych, potem zrównanie z podłożem fortecy Komód.
\ds{} Oni i tak by zginęli, idąc do Trap-orten. \dm{} Zaśmiał się ze swojego żartu. \dm{} Nawet jeśli Hos by odmówił, to za niesubordynację by go zabili.
\end{dialogue}

Opuszczaliśmy atmosferę Planety Wojny w takim samym śmieciolocie, jak wcześniej. Śmieciolot numer 878.
Na szczęście ten był całkowicie nowy, wszystkie urządzenia i żarówki jeszcze działały poprawnie.

Gotowałem się ze złości. Byłem wściekły na siebie, na establishment Komodowej Armii, na Kryształową Królewnę i jej plan podbicia miasta, na imigrantów z północy, którzy zastawili pułapkę.
Na Nadara, który przekonywał mnie o prostocie i łatwości zdania testu, na Kosmę, która dała mi daser z którego nic nie wynikło, na Profesora, którego nie obchodziło nic poza swoją Kulą.
Ale najbardziej chyba na Plazmę, za jego hipokryzję. 
Potwory niby dobro robią, wierzą w Boga, a gdy co do czego przychodzi, to nie mają oporu mordować.
Racja, czasem nie ma wyboru jak nie zabijać, czasem nic to nie zmieni. Jednak zawsze da się polepszyć byt, gdy przypadkowo ma się okazję kogoś uratować.
Plazma się tego nie podjął. 

Nieuratowanie niewinnego jest tym samym, co zabicie go z premedytacją.
Odciągnięcie innej osoby od próby ratowania pozostałych, także jest morderstwem.
A ja miałem pokonywać morderców.

Złapałem więc stery śmieciolotu, wyrywając je Plaźmie.
Statek gwałtownie skręcił, zarzuciło mnie po łuku, lecz zachowałem uścisk.
Kilkukrotnie cięższego potwora odrzuciło w tył, że aż jakimś cudem wbił się kolcami w podłogę.

\begin{dialogue}
\ds{} Przypomnij sobie, co zrobiłem z uniwersalnością. Chcesz skończyć tak samo? \dm{} Plazma ryknął głośno, acz stanowczo. \dm{} Puść stery.
\ds{} Spiedalaj, wracam po Hosa, czy tego chcesz, czy nie.
\ds{} Zgadzam się, lecz zamiast na Ziemię, trafisz potem prosto do naszego więzienia. Resztę życia spędzisz w klitce metr na metr na metr.
\ds{} Prędzej wcielę się do Czarnej Armii, niż zgodzę się na to. \dm{} Niechcący zarzuciłem śmieciolotem, zrobiłem łuk w drugą stronę. Zauważyłem wystającą, znajomą rękojeść ze schowka.
\ds{} Ja katastrofę przeżyję...
\ds{} A czy to przeżyjesz? \dm{} Sięgnąłem do schowka i wyciągnąłem daser. 
\end{dialogue}
Popatrzyłem mu głęboko w oczy. W pionowych źrenicach zobaczyłem coś nowego, coś czego u potworów nigdy nie widziałem. Strach.
Widziałem, jak zaciska swoje trójkątne kleszcze na podłodze, aż żelazne wiórki poleciały.
Uderzał na boki końcówką ogona, jak wściekły kot.

\begin{dialogue}
\ds{} Nie baw się tym, bo się skaleczysz. Mi członki odrosną, a tobie?
\ds{} Z chęcią się przekonam, czy za członki rozumiesz także i głowę.
\end{dialogue}

Jak tylko przekroczyłem atmosferę, ponownie wystrzeliły w nas rakiety.
Chyba czegoś nie przewidziałem.
Musiały być w tej rakiecie mechanizmy obronne, przeskanowałem wzrokiem pulpit, ale nie znalazłem żadnych czerwonych guzików.
\begin{dialogue}
\ds{} Gdzie są rakiety? \dm{} krzyknąłem.
\ds{} W dupie. \dm{} Przesunął ogon.
\ds{} Nie masz dupy, czytałem o waszej biologii. Nie masz nawet mózgu!
\end{dialogue}
Spróbowałem zrobić manewr wymijający, jednak rakiety nie dały się oszukać.

Przypomniałem sobie, co mam w ręce.
Wycelowałem daserem i nacisnąłem spust.
Zielony promień przebił się przez szybę i rozciął pocisk na pół.
Suchy wiatr uderzył mi w twarz.
Nie widziałem drugiej rakiety, spanikowałem i zacząłem strzelać na oślep.
\begin{dialogue}
\ds{} Idiota. \dm{} skomentował ognisty obrońca.
\end{dialogue}

Coś szarpnęło śmieciolotem i rakieta wpadła do kokpitu przez wydaserowaną przeze mnie dziurę. Niewypał.
Brakowało jeszcze tylko pieczonego dzika.

Z daleka ujrzałem Hirten, znaczy Torten, znaczy Kaport... chuj z tym.
Obok jaśniejąca pułapka mroźnych imigrantów.
Wbrew zapewnieniom potwora, śmieciolot wcale nie był trudny do sterowania.
Bezpiecznie zwolniłem opadanie na wysokości kilkudziesięciu kilometrów.

Na południu znalazłem wypaloną plamę na pustyni, która niegdyś była fortecą Komód i moim domem przez dwa miesiące.
Teraz spalona i zwęglona kupa gruzów nie wynosiła się więcej, jak kilka metrów nad ziemię.
Tylko gdzie był Hos? Miałem nadzieję, że się nie spóźniłem.

\begin{dialogue}
\ds{} Technogony, pamiętasz? \dm{} Plazma wstał, wyrywając kolce z podłogi. \dm{} 
Przejmują władzę nad technologią i takie tam. Nie zdziwiłbym się, jakby Hos nie przetrwał kilopulsa po naszym odlocie.
\end{dialogue}
Coś szarpnęło sterami, a potem statek zaczął obniżać lot.
Zaraz wielki robal na nas wyskoczy z ziemi.
Skierowałem daser pod nogi i aktywowałem, wycinając dużą dziurę w podłodze.
Nie było jednak tam żadnego technogona.

Wtem przeleciał mi koło twarzy pocisk, wyleciał przez dziurę w dachu. 
Zobaczyłem w dole tylko różowy błysk.
Dopadłem konsoli, szukając trybu głośnomówiącego.
Tego też nie było.
Plazma ziewnął.

\begin{dialogue}
\ds{} ,,Hos, ja Rohost. Pokój.'' \dm{} Krzyczałem z całych sił po zurdowsku.
\ds{} Śmieciolot jest trochę podobny do dronów czarnej armii. Mieliście to na zajęciach? \dm{} ryknął obojętnie. \dm{} Pomóc ci w czymś?
\ds{} Poradzę sobię.
\end{dialogue}

Biegałem z lewa w prawo, z prawa w lewo, Hos nadal celował w nas swoim działkiem.
Może narysować mu mój symbol na prześcieradle? Ale nie mam czym narysować i nie mam prześcieradła.
A może jest jakieś w kajucie?
Statek coraz bardziej obniżał lot.

Zobaczyłem wtedy, jak z daleka nadciąga burza piaskowa.
Nie, to nie była burza, to było bardziej jak parowóz sunący metr pod piaskiem.
Technogon zwęszył naszą obecność i zaraz wyskoczy z ziemi.
Wycelowałem daserem jeszcze raz, trafiłem centralnie w niego, lecz niewiele to pomagało. Jedynie zwolnił nieco bieg.
Wyrysowałem swój symbol na piasku, lecz Hos nie zauważył i nadal w nas celował.

Parowóz był bliżej i bliżej.
W akcie desperacji, złapałem stery i szarpnąłem śmieciolotem.
Różowa eksplozja zniszczyła silniki, polecieliśmy w dół.
\begin{dialogue}
\ds{} No to hej. \dm{} Plazma wybił drugą dziurę w suficie, rozłożył skrzydła i został w miejscu.
\end{dialogue}

Tuż przed zderzeniem także wyskoczyłem przez plazmową dziurę.
Miękkie gruzy i masa kadłuba zamortyzowały upadek, naturalnie się przewróciłem, prosto na twarz.

\begin{dialogue}
\ds{} ,,Rohost? Co ty robisz w statku Czarnych? Nie wiem, po co wróciłeś, ale nie pozwolę ci skrzywdzić Profesora.'' \dm{} Głos Hosa rozległ się, jak przez megafon.
\ds{} ,,Kogo?'' \dm{} Chyba się przesłyszałem. Albo nie zrozumiałem zurdowskiego.
\end{dialogue}

Wtedy z piasku wyskoczyła nasza śmierć.
Nie nie śmierć. Nie elektryczny robal.
Była to wielka, biała kula.
Była to Kula.

Zawisła metr od gruzu i właz opadł na dół, rozwijając czerwony dywan.
\begin{dialogue}
\ds{} ,,Panie Dernasztapie, jak miło znów pana widzieć. Ziemskie przysmaki już na pana czekają.'' \dm{} Profesor Kula przywitał się z Hosem, w perfekcyjnym zurdowskim. \dm{} ,,O, i jest także Mateusz. Cóż za niespodzianka.''
\ds{} ,,Chwila, Profesor go zna?'' \dm{} Hos dopiero teraz obniżył działko hipera.
\ds{} ,,Oczywiście, to nowy kandydat na członka ALOPP. Mieliśmy przyjemność zjeść razem bankiet nie tak dawno temu'' \dm{} 
Profesor powiedział coś, co wydawało się dla niego tak codzienne, jakby regularnie zapraszał na bankiety losowe osoby z całego wszechświata. \dm{} 
I jak, udało się zdać test? \dm{} dokończył po polsku.
\ds{} Dalej, powiedz, co pięknego narobiłeś. \dm{} Usłyszał lądującego za sobą Plazmę. \dm{} ,,Rohost chciał ratunek Hos, ale atak Plazma.''
\end{dialogue}

Hos otworzył hipera, wyszedł do połowy i wtopił we mnie swój wzrok.
Plazma wylądował i wtopił we mnie swój wzrok.
Kula wtapiał się we mnie swoim wzrokiem już od pewnego czasu.

\begin{dialogue}
\ds{} Zaatakowałem potwora, bo chciałem ratować Hosa... tego... Dernasztapa. Ukradłem daser i groziłem mu, że go poćwiartuję. \dm{} Kula zdawał słuchać mnie uważnie.
\ds{} To są szczegóły. \dm{} Plazma drapał się za kolcem na ogonie. \dm{} Co tak na prawdę zrobiłeś?
\ds{} Nie uratowałem nikogo z fortecy, nawet siebie. Plazma musiał po mnie przyjść i spalić wszystkich.
\ds{} Panie Mateuszu Mechalyczny, nie jest pan dzieckiem, co tak na prawdę złego zrobił pan?
\ds{} Jeszcze... nie wiem. \dm{} Kopnąłem kamyczek. \dm{} Zniszczyłem drugi śmieciolot?
\ds{} On nie wie. \dm{} Potwór wydał dziwny dźwięk.
\ds{} ,,On nie wie.'' \dm{} Kula użył zurdowskiego.
\ds{} ,,Ja też nie wiem.'' \dm{} Tajemniczo rozłożył ręce Dernasztap.
\ds{} Zatem poczekamy, aż sobie przypomni \dm{} Plazma warknął. Pomarańczowe pole siłowe objęło mnie całego. \dm{} 
Profesorze, łaskawy był byś nas podrzucić do głównego więzienia? To będzie jego nowy dom.
Nie mamy co prawda strojów galowych, ale nie będziemy wychodzić z garderoby.
\ds{} Broń Boże. \dm{} Kula złapał się za pierś. \dm{} Strój musi być.
\end{dialogue}

Siedziałem, wciąż otoczony plazmowym polem siłowym. Tyle, że miałem też na sobie niewygodny, wypożyczony kontusz.
Z góry słychać było śmiechy i dźwięki uczty.
Hos gorączkowo coś tłumaczył, co jakiś czas przerywali mu potwór i gospodarz.

Pomarańczowa stróżka energii wychodziła z mojego więzienia i niknęła w suficie, w miejscu gdzie siedział kilka pięter wyżej Plazma.
Uderzyłem kilka razy pięścią w tarczę.
Musiał poczuć, bo zatrząsnął mną tak, że poobijałem się ze wszystkich stron.
Nie było szansy na ucieczkę.

Dernasztap przyszedł do mnie. Przeprosił za to, że chciał mnie zabić, wyjaśnił iż byłem w statku przypominającym pojazdy Czarnych, a
Kryształowa Królewna jest znana z robienia układów z kosmitami. Mogłem być ich tajnym agentem, to tak pięknie się układało.
Przyniósł mi naleśnika z cukrem, nazywał go vinite, jego ulubioną potrawą, nie było lepszej na całej planecie. 
Nie mogłem go jednak przełożyć przez pomarańczową ścianę. Tak musiał czuć się ten dziadziuś, odcięty od świata.
Na odchodne powiedziałem, żeby poprosił Profesora o bigos, to zobaczy czym jest prawdziwe jedzenie. 

Wszyscy zeszli do garderoby. Hos trzymał garnuszek bigosu i wciągał go wielkimi haustami. Ciekawe co będą dawać mi w kubistycznym więzieniu.
Patrzyłem się tępo we wrota kuli. Już było mi wszystko jedno. Bezużyteczność w więzieniu, bezużyteczność na Ziemi, co za bezużyteczna różnica.
Ktoś pokręcił korbą, nawet się nie odwróciłem.
Jak wyglądało ALOPPowe więzienie, nie wiem bo zamknąłem oczy.

Poczułem powiew świeżego wiatru.
Był dość suchy i niewybitnie gorący. Nie pachniał Ziemią.
Dlaczego Plazma otworzył swoją powłokę?
Ciepłe promienie otarły mi się o twarz, nie z jednej, lecz z dwóch stron.
Powietrze smakowało technologicznie.
Słyszałem niewyobrażalną pustkę.
Otworzyłem powieki.

Gradient nieba zmieniał się od jasnego błękitu, po fioletowy horyzont.
Na rozległym niebie brak było jakiejkolwiek chmurki.
Jednak nie było przez to monotonne, wielka ilość najróżniejszych półksiężyców zdobiła nieboskłon.
Miały różne wielości, różne kolory i wszystkie skierowane w stronę dwóch słońc.
Jeden z nich nie był kulą, a wielkim sześcianem.
Jeden miał iskrzące światełka po nocnej stronie.
Jeden poruszał się bardzo szybko.
Między dwoma było połączenie.
Słońca miały różne wielkości, jedno całkowicie białe, drugie mniejsze, fioletowe.
Wyciągnąłem rękę, która rzucała ledwo dostrzegalny szary i niebieskawy cień.

Staliśmy na małym placyku z białego kamienia. Kończył się przepaścią.
Wąska, nieskończenie prosta droga odchodziła w bok, podpierana wysokimi pylonami, niknącymi we mgle.
Z drugiej strony coś, co przypominało tor pociągu magnetycznego, schodziło ostro w dół do niedostrzegalnej powierzchni.
Na horyzoncie było jeszcze kila przeczących grawitacji mostów, oraz gigantyczny, nierozpoznawalny kształt.

Czy to było więzienie? Nie wyglądało na więzienie.
\begin{dialogue}
\ds{} Witaj na Potworanie. Zostaniesz tu na zawsze \dm{} Plazma złowrogo się zaśmiał \dm{} gdyż nigdy nie będziesz miał ochoty wrócić na Ziemię.
\end{dialogue}
Nie rozumiałem.

Ktoś włożył mi do kieszeni okrągły przedmiot.
Poczułem gładkie ekrany z dwóch stron i gumowy brzeg.
Komunikator. Taki sam, jaki miał Nadar.
Z drugiej strony poczułem kształt szyfratora.
Na moje kolana położono jakieś ubranie, wyglądało podobnie to tego, jakie zaprojektowałem, lecz z lepszych materiałów i na środku miało srebrny symbol Potworanu.

\begin{dialogue}
\ds{} I przy okazji \dm{} rykliwy głos odezwał się za moimi plecami \dm{} tak, odcięta głowa może mi odrosnąć.
\end{dialogue}

\divider{}

Z demonów została kupka prochów.
Nikt więcej nie został. Miasto było wolne, a Antyrax jedynym jego mieszkańcem.

Wtedy wielki cień spowił miasteczko.
Autor nie przypominał sobie o żadnym nadchodzącym zaćmieniu słońca.
Popatrzył się w górę.
Jednak to nie było zaćmienie, lecz tułów Neofantasora, który schylił się nad biednym Antyraxem.
Czy uderzy w niego? Zada ostateczną klęskę? Czy pogratuluje? Może wysypie kolejną dychę wrogów?

Neofantasor jednak zbliżył do niego wyciągniętą dłoń, jakby prosił o coś.
Drugą zwinął w pięść i zawiesił nad pisarzem.
Na co czekał? Czego oczekiwał?

Chciał Pucharu.
Neofantasor czekał na jego wygraną w konkursie. Lub na przegraną i opuszczenie pięści z impetem.
Jednak tej jednej rzeczy Antyrax wymyślić nie potrafił.
Usiadł centralnie pod pięścią i czekał. Już nic nie zależało od niego.


