\chapter{Bankiet w Kuli}

\info{Wioska jest atakowana przez literackie demony. Tylko dzielny, acz niedoświadczony wojownik jest wstanie je pokonać, używając do tego
swojego miecza kreatywności i worka z alternatywnym wszechświatem.}

Gdy tylko czubek głowy Wielkiego Neofantasora wyłonił się zza gór, odpalono armaty załadowane gorącym atramentem.
Katapulty wystrzeliły litery, a papierowe samoloty wzniosły się w powietrze.
Wszystko na darmo, wielkolud ani drgnął.

Wojownicy z wioski, uzbrojeni w pióra i kałamarze, ruszyli do boju.
Atakowali wroga, używając całego swojego doświadczenia. Opowiadania cyberpunkowe, fantasy i fantastyki naukowej cięły powietrze.
Horrory, wiersze i dzieła detektywistyczne rozbijały się o jego ciało.
Jednak niewzruszony Neofantasor pozostawał niewzruszony.
Stanął przed wioską i wysypał z rękawa, niczym wytrzepując piasek z buta po całodniowym siedzeniu na plaży, dziesiątkę straszliwych demonów.
Demony szybko i sprawnie rozprawiły się z całą obroną wioski, rzucając po jednej gwiazdce, w każdego z wojaków.
Nikt z obrońców nie przeżył, wieś została bezbronna.

Ale jak to w baśniach zwykle bywa, znalazł się młody syn doktora --- Antyrax.
Trochę ułomny językowo i bez jakiegokolwiek doświadczenia, postanowił dzisiaj zginąć w walce.
Nie miał, jak wszyscy, zbroi, pióra, czy zapasu atramentu.
Posiadał jedynie szklany miecz, wypełniony płynną kreatywnością, dmuchawkę z serum śmiechu, lateksowe buty i worek z alternatywnym wszechświatem.

Najbliższa demonka, Obudzona, została z zaskoczenia kopnięta z lateksowego buta prosto w pupę, i nawet nie poczuła.
Dopiero serum śmiechu, wstrzyknięte prosto w szyję, zwróciło jej uwagę.
Wtedy Antyrax sięgnął do swojego worka i wyciągnął... rzecz.
Obudzona popatrzyła na zagubionego wojownika, to na przedmiot, który trzymał, po znowu na wojownika. Zamrugała, nie rozpoznając zupełnie, co to jest i czy się tego bać.
Ostatni pisarz też obracał w dłoni i oglądał rzecz ze wszystkich stron, ale i on nie wiedział, co właściwie właśnie wyciągnął.
Wyrzucił za siebie i spróbował ponownie.

Po kilku minutach wyrzucania różnych, nieokreślonych obiektów z worka, spostrzegł że grupka demonów go otoczyła i z zaciekawieniem obserwuje jego zmagania z samym sobą.
Nie umiał używać własnego wszechświata. Co za matoł.
Postanowił więc wykonać swoją powinność w tradycyjny sposób, zamachnął się i wbił swój miecz w najbliższego demona, który odskoczył z bólu. Stare, sprawdzone metody zawsze działają.
Nie długo cieszył się z sukcesu, zaraz wszystkie straszydła razem wzięły i dmuchnęły w niego strumieniem przecinków --- jego największą słabością.

Pierwszy lecący znak interpunkcyjny ominął, podskakując na lateksowych butach, drugi skontrował mieczem, lecz tysiąc kolejnych odrzuciło go na kilkaset metrów w tył.
Upadł obolały w błoto, po drugiej stronie wioski nad którą chyba właśnie przeleciał. Jakimś cudem nic sobie nie połamał.
Lecz to nie był koniec kontrataku, lecący błąd ortograficzny prawie rozciął go na pół, Antyrax zdążył odskoczyć, lecz puścił miecz, który pękł na drobne kawałeczki, zmiażdżony przez ,,Ó''.
Kreatywność wsiąkła w błoto, tworząc nowe królestwo błotnych mrówko-androidów, zasilanych parą z wody basenowej.

Antyrax był teraz bezbronny, a bezbronnym człowiekiem żaden z demonów się już więcej nie interesował. Wstał i zobaczył, jak jego wioska właśnie jest równana z ziemią.
Cień Wielkiego Neofantasora, wychylającego się zza góry, wyglądał jak dłoń dziecka, które zagarnia klocki z ziemi, aby je zaraz obślinić i połknąć.
Czy to był już koniec dla niego i dla wszystkiego co znał?

\begin{dialogue}
\ds Nie, to nie koniec \dm pomyślał, sięgając po worek. \dm Jestem pisarzem i to ja ustalam zakończenia. 
\end{dialogue}

Wtedy jego lateksowym butom wyrosły śmigła, wspomagane skryptem w Bashu, i poniosły go prosto pod nos Obudzonej, zostawiając za sobą linuksowy ślad.
Zaciekawiona demonka rozpoznała niedojdę, którego wcześniej zmiażdżyła.
Antyrax nie miał już żadnej broni, tylko worek. Sięgnął do niego i długo nie wyciągał ręki.

\ds{} Mam coś specjalnie dla ciebie, Obudzono. \de{}

\divider{}

\ds{} Kolejny dzień w pracy, kolejny dzień robienia bezużytecznych czynności dla bezużytecznej korporacji w tym bezużytecznym świecie \dm{} pomyślałem. \de{}

Wziąłem komórkę i zacząłem przeglądać maile.
Spam z reklamą talerzy, zaproszenie do znajomych na Facebooku, oczywiście od obcej osoby, powiadomienie o komentarzu na YouTube,
powiadomienie o mailu na innej skrzynce pocztowej, przypomnienie o opłacie za subskrypcję tej bezużytecznej gry komputerowej i tysiąc innych bezużyteczności.
O, znowu rozpętałem gównoburzę na Mirko i zablokowali mi konto Twittera za niepoprawną politycznie myślozbrodnię.
Bezużyteczność do kwadratu.

Z nadmiaru bezużyteczności zdrzemnąłem się w ubraniu na godzinę. Obudziło mnie głośne gruchanie z okna.
Czyżby jakaś użyteczność w końcu mnie spotkała?
Na oknie siedział biały gołąbek, w dzióbku trzymał kopertę.
Nie będąc pewnym, czy nadal nie śnię, odebrałem przesyłkę i przyjrzałem się kopercie.
Gołąb zaraz zniknął, oddalając się bezszelestnie.

Koperta była z papieru czerpanego.
Adresowana była elegancką kursywą do mnie, do Mateusza Mechalycznego, zamieszkałego na ulicy Szerokiej w Gdańsku.
Województwo Pomorskie, Polska, Ziemia, Układ Słoneczny, Galaktyka Droga Mleczna, Gromada Lokalna, Czwarta Kwadra.
Z tyłu widniała pieczęć z wosku pszczelego, wypukły obraz kuli i jakiś napis z niezrozumiałych znaków.
Przecież był 2017 rok, kto dzisiaj jeszcze wysyła papierowe listy, zamiast maili?
Zatem to był kawał. To musiał być kawał. Pytanie jeszcze, po co ktoś mi go wyciął?

Ostrożnie otworzyłem kopertę, trzymając ją obcęgami, spodziewając się że coś strasznego zaraz z niej na mnie wyskoczy.
Nic takiego się jednak nie stało.
Pisany odręcznie list był tym, co zwykle znajduje się w kopertach.

\curlyframe{
\begin{Fontlukas}
Szanowny Panie Mateuszu Mechalyczny.

Mam zaszczyt zaprosić Pana na uroczysty bankiet z okazji wyboru na jednego z przyszłych członków ALOPP.
To wielki zaszczyt, móc gościć nowych agentów tej organizacji na pokładzie mojej \weirdchar{kula}.
Mam najskrytszą nadzieję, że zostanie Pan przyjęty i będzie mieć Pan we współpracy z \weirdchar{monster} udział w walce o wspólne dobro.

Zapraszam do siebie dnia 20 października, roku Pańskiego 2017.
Myślę, że marina w Głównym Mieście Gdańska będzie doskonałym miejscem na lądowanie mojej kuli i tam się spotkajmy w lokalne południe.
Po obiedzie wybierzemy się w podróż na Felicję, gdzie pozna Pan swoich przyszłych, mam nadzieję, członków przybranej rodziny.

Przypominam, że w \weirdchar{kula}, oprócz najwyższej kultury osobistej,
od zawsze obowiązywał rokokowy styl ubioru.
Wszyscy goście powinni przywiązać najwyższą dbałość o szczegóły swojego wyglądu.
Uprzejmie proszę także, aby nie posiadać na pokładzie żadnych urządzeń użytkujących elektryczność.

Z Bogiem. \\
--- Profesor \weirdchar{profesor}
\end{Fontlukas}}

Kilka dziwnych znaków zostało wtrąconych pomiędzy litery. Zaczynało się robić ciekawie. Autor tego dowcipu chciał, abym za trzy dni, w XVIII wiecznym stroju pałacowym,
udał się w sam środek miasta w dniach szczytu, nie zabierając ze sobą żadnej elektroniki.
Potem tajemniczo miałem przejść tajemniczy test na zostanie tajemniczym członkiem jakiejś tajemniczej organizacji.
Trzeba dokładniej przestudiować ten tajemniczy list.

Szybkie szukanie Felicji w internecie wskazało jedną stronę o teoriach spiskowych.
Felicja miała być planetką, stworzoną przez kosmitów, na której hodowano ludzi, aby przeprowadzać na nich straszliwe eksperymenty.
Jeśli to prawda, może zabrakło im tam królików doświadczalnych i porywają kolejne ofiary?
Ale wtedy przecież nie dawali by mi wolnej ręki do odmowy.

Na tej samej stronie podano: ALOPP jest pozaziemską organizacją terrorystyczną, zrzeszającą ludzi w celu mordowania mieszkańców własnej planety.
Ale od czego był to skrót, to nikt nie wiedział.

,,Czwarta Kwadra'' dawała za dużo losowych wyników, aby wywnioskować z nich, o co mogło Profesorowi chodzić.

Lokalne południe w Gdańsku, czyli dwunasta godzina czasu słonecznego, uwzględniając jeszcze czas letni, to trochę przed trzynastą czasu strefowego.
Przyjdę o 12:00, najwyżej trochę poczekam.

Wikipedia natomiast wskazała, że gołębie pocztowe w żadnym wypadku nie mogłyby doręczyć listu bezpośrednio do odbiorcy.
Ich mechanika polega na wracaniu do macierzystego gołębnika, z dowolnego miejsca na świecie, i tylko tyle.
Listów z pewnością nie wsadzano im do dzióbka, a przywiązywało się je do nóżek.
Rozejrzałem się po pokoju, czy przypadkiem nie miałem w nim gołębnika, aby hodować gołębie pocztowe, ale nie.
Moja teoria o dowcipie zaczęła się lekko sypać.

Kilka razy, w różnych częściach świata, widziano w tym samym momencie kuliste UFO i ludzi w strojach rodem z Wersalu.
Podobno zdjęcie zrobione kuli nigdy nie wychodziło poprawnie, a większość ludzi magicznie zapominała o zdarzeniu chwilę po odlocie tajemniczej struktury.
Nieliczni pamiętali i rozpowiadali to dziwo, ale nikt im oczywiście później nie wierzył.
Kulę widywano w różnych miejscach, nie ograniczała się, jak na filmach, tylko do USA, przelatywała przez centra miast, pływała pod wodą, cumowała do Międzynarodowej Stacji Kosmicznej, straszyła samoloty, ślizgała się po biegunowych lodach i toczyła bitwy z wojskami wszystkich krajów świata.

Następnego dnia zabrałem list do znajomego chemika.
Potwierdził on moje obawy, list wykonany był oryginalną techniką sprzed kilkuset lat.
Skład chemiczny papieru i atramentu, odpowiadał tym, używanym dawno temu.
W dodatku narzędzie pisania z pewnością było ptasim piórem.
Na myśl o bezużyteczności otaczającego mnie świata, postanowiłem pojutrze zrobić coś użytecznego.

\divider{}

Pod naporem nietypowości i kreatywności dzieła, Obudzona zaczęła krzyczeć, zwijać się w konwulsjach i palić czarnym ogniem.
Zmieniła się w mały księżyc i poleciała, jak frisbee, z powrotem w kierunku Wielkiego Neofantasora.

Triumf Antyraxa nie trwał długo, za chwilę, od tyłu, złapała go Dedirid.
Jej czarna ręka owinęła się wokół delikatnego światostwórcy, jak czarny worek na zwłoki.
Poczęła go zacieśniać, niczym lekarz mierzący ciśnienie.
Zaraz wyciśnie naszego bohatera, jak tubkę pasty do zębów.
Wywijając się rybio, Antyrax zanurkował do worka i długo nie wychodził.
Demonka przez godziny nachylała się nad otworem, aby capnąć go jak tylko wystawi głowę.
Gdy tylko coś się wysunęło, porwała to z ochotą.
Była to jednak kolejna opowieść, zabójczo eksperymentalna, niesamowicie abstrakcyjna.
Nietypowość poparzyła jej łapska.

\divider{}

Mateusz wypożyczył wymaganą górę z wypożyczalni kostiumów.
Zastanawiał się, czy nie podkraść jakiegoś szustokora z muzeum, ale to chyba nie było by zbyt poprawne zachowanie.
Bał się, czy mierna jakość kaftana, spowodowana nieoryginalnością, nie będzie zwracać nadmiernej uwagi w świecie najprawdziwszych atłasowych pasów i perłowych guzików.
Postanowił kupić więc kilka ozdób ze sztucznej biżuterii, które wyglądały dość kosztownie, a stworzone były z
byle-czego, i doszyć w losowych miejscach. Miał nadzieję, że Profesor i inni goście nie zauważą różnicy.

Z pończochami nie było żadnego problemu, znalazł je w damskim sklepie.
Tak samo coś, co można było podciągnąć pod starodawną koszulę.
Musiała być flanelowa z wystającymi rękawami.
Ogarnął także puder.

Peruka, cóż. Przynajmniej znalazł za szafą trójkątną czapkę piracką po poprzednich lokatorach.
Najgorzej, że zazwyczaj chodził na łyso, gdyż rodzice nie obdarzyli go mocnymi włosami.
Potrzebował więc na szybko przykleić coś sobie na łeb.
Liczył w głowie, ile lat może dostać za kradzież peruki sędziemu, gdy spostrzegł wyprzedaż starych futer.
Używając magii nożyczek, kleju i starego mopa, wygenerował coś, co po przykryciu trójkątną piracką czapką wyglądało dość znośnie.

U zegarmistrza kupił za grosze kopertę zegarka, pustą w środku, z brakującymi wskazówkami, całość zaczepioną na łańcuszku.
Zegarek nie musiał działać, ważne aby był.

O dziwo, to buty przysporzyły mu najwięcej problemu.
Niby lakierki z klamrą nie są niczym bardzo skomplikowanym, a jednak nikt ich nie produkuje.
Może właśnie dlatego, że były modne trzysta lat temu?
Wpadł na pomysł, aby kupić coś podobnego i przerobić.
Znalazł buty dla zakładów pogrzebowych, gdyż tylko te odpowiednio się błyszczały, i przyszył im klamry od spodni.
Z daleka nie było widać różnicy.

W domu ubrał się i przejrzał w lustrze.
Połączenie Napoleona, Ludwika XIV i informatyka z Gdańska.
Muszą zrozumieć.

O jedenastej godzinie, owego wielkiego dnia, wdział pełny strój.
Nie mógł się przecież tak pokazać w mieście.
Pończochy zatem przykrył spodniami dresowymi.
Na elegancki szustokor nałożył nieco za dużą bluzę z kapturem.
Lustrzane lakierki przykrył jakimiś szmatami, żeby nie zwracały zbytniej uwagi.
Tylko pseudo-perukę schował do plecaka.

To nie mogło pójść tak łatwo.
Z daleka zobaczył kordon policji i wojska, stojący w Zielonej Bramie, blokował wstęp każdemu wychodzącemu z Długiego Targu.
Ucieszył się i kamień spadł mu z serca. Oznaczało to, że jednak nie padł ofiarą żartu.
Znalazł w końcu promyk użyteczności w oceanie bezużyteczności. Każda normalna osoba, wiedząc że wielka latająca kula-zapominajka wylądowała w centrum miasta,
ewakuował by się z niego jak najdalej. 
Mateusz jednak nie był normalny, i może dlatego właśnie został zaproszony na najbardziej nienormalną ucztę w świecie.

Do mariny spróbował dostać się okrężną drogą, przebiegł przez Krowi Most na Wyspę Spichrzów.
Klucząc uliczkami zbliżył się do portu, jednak tutaj też była blokada.
Widział w każdym bądź razie kawałek wody w basenie jachtowym, nietypowe fale odbijały się od brzegów, coś się tam jednak działo.
Popatrzył smutno w kanał i pomyślał, że chyba zostanie mu wskoczyć do wody i popłynąć wpław, pod mostem omijając strażników.
Ale przecież pewnie nie zostałby wpuszczony mokry do rakiety.
No i co z pudrem, który już sobie wcześniej pieczołowicie nałożył?
Głupi. Podziurawią go zaraz jak ser, gdy tylko zobaczą kogoś płynącego kanałem wpław.
To był koniec.

Szustokor był bardzo gruby, rozpiął więc bluzę żeby się nie ugotować, teraz wszystko było mu jedno, czy ktoś go zauważy.
Był tak blisko, a jednocześnie tak daleko. Wszystko miało prysnąć, jak bańka.
Czuł się jak rybka w siatce, wrzucona do oceanu.
Zaraz będzie widział swoją życiową porażkę, jak na dłoni.
Co robić? Co robić?

Wybawienie przyszło nieoczekiwanie.
Oto bowiem mama z małą dziewczynką podpłynęły do niego skuterem wodnym, oferując szybką podwózkę.
Myśląc, że to pomyłka, zdjął bluzę, pokazując swój strój w połowie okazałości.
Kobieta jednak nie uciekła, nie przestraszyła się dziwaka, tylko się uśmiechnęła.

\begin{dialogue}
\ds{} Chyba się teraz nie poddasz? \dm{} zapytała.
\ds{} Skąd... kim...
\ds{} Kula miał wielu gości. \dm{} Położyła rękę na piersi. \dm{} Jak tylko zobaczyłam, że wrócił do miasta, wiedziałam. Ktoś będzie potrzebować pomocy.
\ds{} Ja... \dm{} Mateusz milczał przez chwilę. \dm{} ha, ha. Prawie się nabrałem.
To niezwykłe, jak wynajęliście żołnierzy, żeby zastawili miasto specjalnie dla mnie?
\end{dialogue}

Tajemnicza osoba przewróciła oczyma, zdjęła swoją córkę na chodnik i zeskoczyła, wrzuciła klucze od skutera głównemu bohaterowi do kieszeni szustokora i poszła, nie odzywając się więcej, trzymając dziecko za rękę.

Mateusz w rokokowym stroju wersalskim zasuwał na skuterze wodnym kanałem Nowej Motławy, ozdoby szustokora mieniły się w pełnym słońcu tak samo, jak latające wokół niego
krople wody i stalowe kule, wystrzelone z mostu przez żołnierzy zabezpieczających
lądowanie wielkiej białej kuli w centrum miasta.
Schował się na chwilę pod mostem, jak zagoniony przez wilki królik w norze, a gdy wypłynął z drugiej strony, wtedy ją zobaczył.

Była wielka, jak budynek, wypolerowana, biała i doskonale kulista.
Dołem dotykała lekko powierzchni wody, tworząc promieniste fale.
Odbijała w sobie cały Gdańsk.
Mateusz zobaczył w niej malutkiego siebie na łódeczce-zabawce, malutkie budynki, żołnierzyków, spichrze, basenik, niebo, helikopter jak muchę i blask pełnego słońca.
Już nikt nie strzelał, już tylko wszyscy patrzyli. I bali się.
On się nie bał. Przyszedł tu na bankiet.
Przyszedł we francuskim stroju.
Przyszedł tu, bo został zaproszony.

Zszedł ze skutera na pomost i poprawił perukę, wtedy też właz w dolnej części zaczął się otwierać.
Tak jak się spodziewał, był to dźwięk szczęku łańcuchów, a nie elektrycznego silnika.
Ze środka powiał zapach kurzu, wosku i lekkiej stęchlizny.
U dołu rozwinął się elegancki, czerwony dywan.
W przejściu stanął On.
Nosił strój wspanialszy, niż Mateusz mógł sobie kiedykolwiek wyobrazić, tak inny od jego własnego, a przecież z tego samego okresu historycznego.
Przy jego ozdobach, sztuczna biżuteria Mateusza rzeczywiście wyglądała na sztuczną.
Jego najprawdziwsza peruka przyćmiła wielkością cały futrzany twór z głowy gościa.
Lakierki błyszczały się tak samo, jak jego statek kosmiczny z którego wyszedł.
W ręku trzymał laskę z białą kulką, pomniejszoną wersją tego, co znajdowało się tuż za nim.

\begin{dialogue}
\ds{} Jestem Profesor Kula. Miło mi pana gościć na moim statku.
\end{dialogue}

\divider{}

Antyrax wyszedł z worka, gdy z Dedirid została już tylko kupka popiołu.
Jeszcze ośmiu.
Tym razem demony nie bardzo chciały go atakować.
Antyrax więc wskazał jednego z nich palcem, niczym sędzia nowoskazanego na śmierć.
Lenna. Pora na mikropomysły.
Sięgnął do worka i od razu złapał to, czego szukał.

\divider{}

\begin{dialogue}
\ds{} Waćpan Mateusz Mechalyczny, jak mniemam \dm{} powitałem gościa. \dm{} Waćpan Mateusz Mechalyczny niepewnie, acz żwawo podszedł, ukłonił się, i schował za framugą włazu,
znikając przed przeszywającym wzrokiem miasta.
\ds{} Proszę wybaczyć mi mój ubiór i maniery, Panie Profesorze Kula. \dm{} Ukłonił się ponownie, prawie do ziemi. \dm{}
Musiałem przedrzeć się przez kordon wojska i ominąć grad pocisków, aby przybyć do pańskiego statku.
\ds{} Nazywam się Kula, mości Mateuszu Mechalyczny, nie Kula \dm{} poprawiłem, zamykając korbą właz. \dm{} A ta kula nazywa się Kula. 
I nie jest jakimś statkiem kosmicznym, a Kulą.
\ds{} Kula... \dm{} niepewnie odpowiedział.
\ds{} Nie Kula, Kula. Moje nazwisko, nazwa tego miejsca, typ urządzenia, i bryła geometryczna. Kula, Kula, Kula i kula. To trzy różne słowa, zupełnie inaczej wymawiane.
Zupełnie inaczej zapisywane.
\end{dialogue}

Mateusz Mechalyczny podrapał się po głowie, ścierając puder.

\begin{dialogue}
\ds{} Ignoruj go, on mówi i słyszy na częstotliwościach poza zakresem naszych uszu. \dm{} Katarzyna Kosmata zjechała po falistej poręczy schodów i przysunęła do nas, nawet się nie witając.
\dm{} Jestem Kasia, cześć.
\ds{} Droga Katarzyno Kosmata! \dm{} skarciłem ją. \dm{} Maniery! Niech panna nie prezentuje złego przykładu naszemu gościowi. Panie Mateuszu, mam zaszczyt przedstawić panu...
\end{dialogue}

Gość jednak utopił wzrok w olbrzymiej sukni Katarzyny, nachalnie gapiąc się na każdy jej detal.
A już się radowałem, że chociaż on będzie potrafił tu zachować maniery. Nadaremnie.
Najgorsze w tej sytuacji było to, że ona sama wręcz go do tego zachęcała. 
Zamiast się przedstawić, zaczęła tłumaczyć skąd i jaka część ubioru pochodzi.

Najpierw zawiesił oczy na jej biuście, wodząc źrenicami to w lewo to w prawo.
Najprawdopodobniej podziwiał plecionkę z anielskich włosów, którą obszyta była góra.
Anielskie włosy są całkowicie przezroczyste, gdy odpadną od właściciela, więc
aby stworzyć ten element ubioru, trzeba było prawdopodobnie zbierać je z niebiańskich podłóg w całym raju.
\begin{dialogue}
\ds{} Trafiłam przez to na dywanik anielskiego ministra poprawnego zachowania, chciał to podciągnąć pod brak szacunku dla zarządu Nieba, ale wybroniłam się tym, że wszyta w suknię świętość
będzie dodatkowo ochraniać mnie przed demonami, czy jakoś tak.
\end{dialogue}

Następnie zszedł niżej, aby przyjrzeć się lepiej pasu.
Katarzyna gustowała się w nieprawdopodobnie kosztownych ubiorach, lecz jej pas był wykonany ze zwyczajnego, ziemskiego jedwabiu.
Może chciała tym pokazać, jakoby jej suknia była w równym stopniu wykonana ze składników z calutkiego wszechświata?
\begin{dialogue}
\ds{} Ten jedwab pochodzi od jedwabników karmionych nektarem jedynie z najrzadszych gatunków orchidei, podlewanych krystaliczną wodą źródlaną z Himalajów
\dm{} wyjaśniła cały sekret.
\end{dialogue}

Po pasie, przyszedł czas na szyję. Katarzyna założyła tym razem kolię ze zmutowanych pereł Khaliniskali...
czy to była Rezurma? Nie pamiętam, kto ostatnio przejmował stolicę i nazwę tej przeklętej... Planety Wojny, jak ją wszyscy nazywają.
Kulki mieniły się i pulsowały wszystkimi kolorami tęczy. Od podczerwieni, po nadfiolet.
Te perły można było znaleźć tylko w małżach, żyjących w skażonym jeziorze, na północy pustynnego kontynentu Terb. Chwila, teraz to już nie był już Terb... no na północy tego największego kontynentu planety.
Zdaje się że to albo Czarna Armia, albo Komodowa utopiła tam kiedyś zbiorniki z kancerogennym żelem, w celu wewnętrznego wyniszczenia przybrzeżnego miasta Hirten... wtedy to było Hirten.
Nie udało się, mieszkańcy Hirten wyczuli podstęp i zamiast umrzeć na nowotwory, od picia skażonej wody, poumierali z pragnienia.
W każdym razie flora i fauna w jeziorze przeszła nieprzyjemne zmiany fizyczne.

Buty. Był to wspólny wytwór czterech Khrnzrhkh.
Najpierw poprosiła Chrrkrhkrrkk o stworzenie lodowej podstawy.
Potem pewnie Iłiścirr obudował to swoją czarną rkkizniisi, Buffsirr dodał czerwone wkładki z buffzerda, a Fluszszrisss utwardził ogniem.
\begin{dialogue}
\ds{} Te buty zostały zrobione przez potworów, najpierw Mikołaj stworzył lodową podstawę, Psychit zalał ektoplazmą, Pyrroq dodał czerwone klejnoty-bomby, a Plazma utwardził ogniem. \dm{}
Katarzyna Kosmata właśnie spowodowała, że kolejna osoba będzie nazywać Khrnzaalk potworami, zamiast porządnie w ich własnym języku.
\end{dialogue}

Zahaczył o wachlarz.
Ten był stworzony z półprzezroczystych łusek białego cyrkowca.
Te smoki wyginęły doszczętnie po ataku czerwonych kartaczy na ich wyspę.
Właściwie, jedyne pozostałe cyrkowce można teraz znaleźć w zoo w Capitalu.
Szkoda ich, miały wspaniałą kulturę.
Cyrkowe baśnie do dziś opowiada się małym smoczkom do legowiska, a cyrkowi malarze są niedoścignionym przykładem talentu w wielu kulturach.
Kartacze to zwykłe zwierzęta. Żeby chociaż te barbarzyńskie pasowce ich rozbiły, ale nie. Największy i najbardziej bezmózgi gatunek smoków zaatakował, rozszarpał i pożarł najwspanialszych.

Ostatecznie Mateusz popatrzył się na jej twarz.
Nie, nie na twarz, a na makijaż.
Oczywiście, Katarzyna Kosmata nie mogła spocząć na wyrywaniu łusek prawie wymarłym smokom.
Jej puder był stworzony ze zmielonych ciosów mamuta.
Jak ona odkopała je z syberyjskiego błota i wybieliła, tego nie wiem.
\begin{dialogue}
\ds{} Przekonałam Chronosa, jednego z potworów, żeby odwrócił trochę czas i przywrócił im świeżość.
\end{dialogue}

Nikomu się nie udało przekonać kiedykolwiek Pfiishuss do jakiegokolwiek używania swojej mocy! Jak ona to zrobiła?

To jednak nie był koniec podziwiania, poszedł wzrokiem wyżej.
Fryzura Katarzyny była przeogromna. A wszystko to z naturalnych włosów. Wiem na pewno, że Floria... znaczy Hhurnna przywiązuje podobną uwagę do wyglądu, co Kosmata. Na pewno nie odmówiłaby Katarzynie podkręcenia jej cebulek włosowych w celu ich przyspieszenia. Ale w sumie Chronos także mógł to zrobić.
\begin{dialogue}
\ds{} W Rossmanie sprzedają taki super szampon do włosów. Nic więcej nie potrzeba. Może na twoje też pomoże. 
\end{dialogue}

Na jej fryzurze osiadły wielobarwne motyle. Co jakiś czas, któryś wzbijał się w powietrze, robił pętlę wokół jej głowy i lądował z powrotem.
Były to najprawdziwsze motyle, hodowane i tresowane w tajnej placówce pod motylarną w Burggarten.
Ciekawe, jak je zdobyła. Znając Katarzynę, pewnie jak gdyby nigdy nic weszła przez tajne wejście, w tej pełnej sukni, z naładowanym szyfratorem w ręce, i powiedziała:
,,Dajcie mnie tych tresowanych motyli na głowę, bo zaraz mam bankiet we wielkiej, latającej kuli.''
Być może tylko po to w ogóle przyjechała dzisiaj do Wiednia.
Przyjechała po motyle, i żeby przyprawić o zawał serca całą Austrię.
Więc tym razem postanowiła wsiąść do Riesenrad i pojechać wagonikiem na sam szczyt, gdzie wcześniej specjalnie umówiła się ze mną, abym podleciał po nią Kulą.
Oczywiście, jak tylko przyleciałem, wybuchła panika. Koło się zatrzymało, uwięzieni w wagonikach ludzie próbowali uciekać po konstrukcji koła, 
przed wielką białą kulą, cumującą właśnie do najwyższej budki. Katarzyna otwarła drzwiczki i robiąc krok nad przepaścią, weszła do pojazdu.
Pomachała wachlarzem pozostałym, ledwo żywym ze strachu osobom w budce, i odlecieliśmy.
Następnym razem pewnie stanie na szczycie Empire State Building, a ja będę robił za King Konga.
I też będę potem uciekał przez myśliwcami.
Czy można się uzależnić od amnezji, którą pokryty jest statek?
Uzależnić od siania paniki w ludziach, którzy i tak za chwilę o wszystkim zapomną?

Przeczyściłem gardło.
\begin{dialogue}
\ds{} Znaczy... witam... bardzo mi miło, dzień dobry... eee... Kasiu-ażyno. \dm{} Stał bez ruchu kilka pulsów, aż zdecydował się delikatnie ująć jej dłoń i pocałować.
Ważne, że się starał. \dm{} Ja Mateusz... jestem.
\end{dialogue}

Katarzyna zarumieniła się. Widać zostało w niej jeszcze trochę kultury osobistej. A może to był makijaż?

Mateusz dostał oczopląsu, jego wzrok skakał od ozdoby do ozdoby. Każdej falki, każdego wgłębienia musiał dotknąć, niczym sprawdzając czy rzeczywiście wykonane są z hebanu i masy perłowej. 
Poprowadziłem ich po schodach do salonu, w którym nakryty był już stół dla czterech osób.
Gość aż przysiadł z wrażenia.
\begin{dialogue}
\ds{} Na naszym bankiecie spodziewamy się w sumie trzech gości \dm{} oznajmiłem zgromadzonym. \dm{} Zatem dołączy do nas jeszcze jedna osoba.
Będzie to Nadar Nocny, który aktualnie bada, albo szabruje, wrak Titanica. Podróż potrwa około dwa i pół kilopulsa, to jest niecałe dwie godziny.
\dm{} Poprawiłem żabot. \dm{}
Pan Nocny jest dość... ekscentryczny. Dla jednych jest najlepszym przyjacielem, a inni go nienawidzą.
Szczerze powiedziawszy, nie popieram jego charakteru, ale obawiam się że może pan, panie Mateuszu, naleźć w nim bratnią duszę.
\ds{} No dobrze, gdzie są ukryte kamery? \dm{} Mateusz niespodziewanie wypalił.
\ds{} Proszę pana, zaręczam, że w całym tym miejscu nie znajduje się ani jedno obrzydliwe elektroniczne urządzenie. Ta strefa jest wolna od nieprzyjemnych pól magnetycznych i elektrycznych.
\ds{} On chyba nadal nie wierzy, Profesorze. \dm{} Katarzyna zaproponowała. 
\ds{} Nadal nie wierzy w Kulę? Pomimo, że sam w niej stoi? \dm{} Uśmiechnąłem się. \dm{} Nigdy nie widziałem takiego zaparcia przy obronie własnych idei. Panie Mateuszu, wierzę, że będzie pan wspaniałym agentem.
\end{dialogue}

Położyłem rękę na lasce. Lekko ścisnąłem małym i wskazującym palcem, aby obniżyć lot.
Następnie przycisnąłem w dół otwartą dłonią, aby pokonać siłę wyporu.
Zaczęliśmy się wtedy zanurzać coraz głębiej i głębiej w Atlantyku, zostawiając za sobą pióropusz tęczowych rozbryzgów.

Tymczasem zacząłem oprowadzać naszego gościa po Kuli.
Wycieczkę rozpoczęliśmy, wracając do głównego włazu na najniższym piętrze.
Ta otwierana w dół, mająca od wewnątrz kształt schodów, wykrzywiona płyta, była jedyną, niepokrytą czerwonym futrem, częścią pancerza.
Zamiast tego posiadała czerwony dywan i wysuwaną poręcz, automatycznie rozwijane przy kontakcie z podłożem.
Operowana za pomocą skomplikowanego systemu łańcuchowo-sprężynowego na korbę.

Nie zmieniając piętra, przeszliśmy do garderoby.
To właśnie tutaj trzymałem awaryjne suknie, habity, koszule i trzewiki, w razie gdyby któremuś z gości zdarzyło się nie posiadać wystarczająco odświętnego ubioru do uczestniczenia w uczcie.
Mateusz zwrócił mi uwagę na grube kombinezony, wiszące w kącie. Wedle jego wizji, były to stroje nurkowe.
Wyjaśniłem, że pomimo mylącego dla niektórych wyglądu, w rzeczywistości nadawały się zarówno do nurkowania w oceanie, jak i w próżni kosmicznej.
Są integralną częścią Kuli, wyjaśniałem, trochę jak ściany i meble. Czerpią z niej energię do podtrzymywania życia. 
Będąc w takim kombinezonie, nigdy nie zabraknie ci tlenu i pożywienia.
Na szczęście nie zauważył, iż jeden z haków był pusty. Nie chciałem się tłumaczyć, że zgubiłem kawałek wyposażenia swojej rakiety.

Zapytany o śluzę ciśnień, aby bezpiecznie wychodzić na zewnątrz, opowiedziałem mu o niewidocznej tarczy rozciągniętej na włazie, chroniła ona wnętrze przed różnorakimi hazardami zewnętrznymi, takimi jak próżnia, uniwersalność, demony, czy brak kultury osobistej.
Nie był przekonany, więc kręcąc jeszcze raz korbą, otworzyłem ponownie właz. 
Płynęliśmy aktualnie tuż przy samym dnie morskim, zostawiając za sobą chmurę wzburzonego piasku.
Falista, lekko wypukła powierzchnia wody, utworzyła się na głębokości framugi. 
Mateusz z niedowierzaniem zamoczył rękę w głębiach oceanu, wyciągając garść osadzającego się piasku.
I meduzę.

Na kolejnych piętrach znajdowały się pokoje gościnne. Gość uprzejmie podziękował za pokój, ale nalegał, abyśmy szli dalej.

W centralnej części Kuli znajdowała się łaźnia, muzeum i mój gabinet. Do tego ostatniego nikogo nie wpuszczam.
Ludzie snują różne domysły na temat tego, co znajduje się za dębowymi drzwiami. 
Zasilanie całej Kuli, mój zwyczajny pokój, jakieś kosmiczne artefakty, prawda o moim pochodzeniu?
Nikt z nich nigdy nie miał racji, a ja nikomu nigdy prawdy nie pokażę.

W wyłożonym terakotą pomieszczeniu panował standardowy zaduch. Gość zdziwił się niemiłosiernie, znajdując tutaj basen, jacuzzi, saunę fińską, masażery wodne, a także mały wodospad.
Pośrodku stał wielki piec na węgiel. Bez niego zimna pustka kosmosu szybko by nas dopadła.
Mateusz powiedział, że w XVIII wieku nie używano łaźni i że po stylu wnętrza spodziewał się co najwyżej wychodka w kącie. 
Zaśmiałem się na myśl, iż wziął Kulę za stuprocentowy wycinek pałacu w Wersalu.
Kultura idealna nie istnieje, zacząłem wykład, z każdej należy wyciągnąć najlepsze części. 
I tak, łącząc na przykład rzymskie starożytne łaźnie, francuski nowożytny wystrój, średniowieczne królewskie dania i słowiańską mowę przyszłości, 
stworzyłem tą właśnie latającą wyspę kultury idealnej.

Najciekawsza część statku teraz.
Moje muzeum zawiera artefakty z różnych zakątków wszechświata. Gość zapytał o wartość zebranych przedmiotów.
Nie wszystko da się sprowadzić do liczby pieniędzy, dałem mu wykład, nie wszystko ma tak zwaną cenę. 
Jeszcze się o tym nie raz przekonasz.

Wskazałem skałę przyczepioną widełkami do podstawy. 
To na przykład jest kawałek meteorytu, który uderzył w księżyc planety Tos. Wartość tego kamienia jest równoważna wartości losowego polnego kamienia z Ziemi, 
znajduje się tutaj ze względu na historię, jaką ze sobą niesie.
Otóż, uderzenie meteorytu było tak silne, że wybiło księżyc z orbity, popychając go w kierunku Tosa.
Po stu latach ciągłego zbliżania się do powierzchni, w końcu satelita zahaczył o atmosferę, gwałtownie zwolnił i zderzył się z planetą.
Każdy organizm, większy od jednokomórkowca, został zniszczony.

Co ciekawe, mieszkańcy tego świata byli na tyle rozwinięci naukowo, że doskonale widzieli i rozumieli zbliżającą się katastrofę.
Jednak nadal za mało rozwinięci technologicznie, aby móc jej uniknąć.
Przewidzieli dzień swojego końca co do dnia, a koniec rzeczywiście nastąpił.

Tak, wiem że to smutne, ale cóż począć? Gorsze rzeczy zdarzały się w zbiorowej historii życia. 
Tylko pierwotne grzyby przetrwały katastrofę.
Toksyczna atmosfera, brak słońca i wysoka temperatura post-apokaliptycznego świata wręcz przyspieszyły ich ewolucję.
Na przykład, tutaj masz dziób takiego latającego ptakochomora. To grzyb i ptak jednocześnie, ładnie świeci w ciemności.
Da się go spożywać, niestety nie jest bardzo wysublimowany w smaku.

Przeszliśmy dalej. Kamień z lodowej strony Kryonii, nic niezwykłego. 
No może poza tym, że musiał być wydobyty spod kilku kilometrów litego lądolodu.
Co w Kryonii jest takiego wspaniałego? Obraca się ona wokół swojego słońca jak Księżyc wokół Ziemi. 
Wiecznie zwrócona tą samą stroną.
Na Kryonii nie ma zatem dni oraz nocy, a gwiazda zawsze jest w tej samej części nieba.
Nocna część jest lodową pustynią, dzienna ma pośrodku wiecznie szalejące tornado.
Może kiedyś zobaczysz Pałac Nadiru, położony w centrum wiecznej zmarzliny, jest przepięknym dziełem sztuki lodowej.
Wielka iglica z kryształowych łuków, kopuł, balkonów i kolumn.
Podświetlona trytowym światłem na przeróżne kolory.
Freon, wielki lodowy król Kryonii rządzi swoim państwem dobrze i sprawiedliwie.
Szkoda, że część jego ludu tego całkowicie nie rozumie. 
Demokraci, socjaliści, libertarianie, i reszta niepoliczalnych ruchów społecznych chce go dosłownie zwalić z tronu i pogrążyć cywilizację w chaosie.

\begin{dialogue}
\ds{} Skąd wie pan, że byłoby gorzej, niż jest teraz? \dm{} zapytał.
\ds{} Może kiedyś zostanie waćpan zaproszony przez Freona i wtedy, na własne oczy zobaczy pan, że na pewno nie byłoby lepiej, niż jest teraz \dm{} odpowiedziałem. \dm{}
Zresztą, prędzej czy później to i tak się pewnie stanie. Freon się starzeje i nie znalazł jeszcze na swój tron godnego następcy. Nikt inny nie może go zastąpić.
Więc albo rozkaże wybrać kogoś głosem ludu, albo znajdzie kogoś godnego spoza planety. To mogłoby doprowadzić do wojny domowej, 
rozumiecie, nikt nie chciałby być rządzony przez obcego kosmitę z kosmosu, nie ważne jak dobrze by rządził.
\ds{} Jak spoza planety? 
\ds{} To jedna z tych cywilizacji, zwanych zapoznanymi. Na tyle rozwinięta technologicznie i kulturalnie, przede wszystkim kulturalnie, 
że ma dostęp do warstw wszechświata. Warstwy to takie jakby obszary ,,pod'', ,,nad'' i ,,z boku'' czasoprzestrzeni
Pozwalają na szybką i dowolną podróż w każde miejsce, do każdej galaktyki, do każdego układu, używając minimalnej ilości paliwa.
\ds{} Jak to? Czyli taka na przykład Kryonia może w każdej chwili przelecieć jakąś warstwą i zaatakować Ziemię? 
\dm{} Zląkł się. \dm{} I czy Ziemia także jest zapoznana?
\ds{} Powiedziałem, rozwinięta kulturalnie cywilizacja. Czy waćpan jest absolutnie pewien, że ludzkość nie zaatakowałaby obcej planety, gdyby lot do niej byłby tak trudny, jak do Księżyca?
No właśnie. Poza tym, Ziemi bronią jeszcze Khrnzrhki.
\ds{} Kto? 
\ds{} Potwory \dm{} westchnąłem. I ja też się w końcu poddałem. \dm{} Robią za policję wszechświata, dbają o pokój na wszystkich zapoznanych planetach i poza nimi. 
ALOPP, do którego waćpan został zaproszony, to skrót od Akademii Ludzkiej Otoczonej Protekcją Potworów. 
Jako agent Akademii, będziesz im waść pomagał, będziesz dbał o pokój we wszechświecie, zatrzymywał wojny, walczył ze złem w różnych postaciach. 
To niebezpieczna i bardzo ciężka praca, ale jakże ciekawa.
Raz będziesz uciekał na skuterze grawitacyjnym, przed stadem czerwonych kartaczy, a innym razem zasiądziesz w sali obradowej Pałacu Nadiru. 
Znaczy, oczywiście jeśli okażesz się godny.
\ds{} Godny?
\ds{} Chodzi o charakter. Do ALOPP należy odpowiedzialność przed przyszłością. 
Każdemu agentowi może zdarzyć się stanąć przed wyborem decydującym o milionach istot. 
Dlatego kandydaci są poddawani testowi osobowości, żeby mieć pewność że zawsze wybiorą większe dobro. 
Test ma kilka faz, sprawdza reakcję na zaistniałe sytuacje.
\ds{} Kilka faz? Jak mam je pozdawać? \dm{} Zaczął panikować.
\ds{} Spokojnie, pierwszą ma pan już za sobą. Przecież jest pan tutaj z nami. 
Pierwsza faza sprawdzała reakcję na abstrakcyjne sytuacje.
Można było wyrzucić ten list, można było zgłosić go władzom, można było pokazać w internecie, a można było, jak pan, potraktować go poważnie.
Jest test sprawdzający zaangażowanie, posłuszeństwo, wykonywanie rozkazów, siłę psychiczną itp.
\end{dialogue}

Kontynuowałem oprowadzanie.
Zapytałem go, czy potwierdzi, iż wskazany przeze mnie kawałek zegarkowatych mechanizmów wygląda intrygująco.
Poleciłem zgadnąć, co to było, podpowiedziałem jakoby to nie był żaden zegar.
Nie zgadł, jak mógłby zgadnąć?
\begin{dialogue}
\ds{} Otóż, jest to mózg reprezentanta pewnej wybitnie nieprzyjemnej nacji robotów.
I mówiąc roboty, nie mam tylko na myśli ludzików zasilanych na prąd, jak to się przyjęło w ziemskiej kulturze.
Mam na myśli wszelkie żywe istoty zbudowane z nieżywych składników. Pozornie zwykła materia, lecz natchniona myślą.
\ds{} Proszę mi wybaczyć, ale wygląda dla mnie trochę, jak kupka śmieci.
\ds{} Bo nią jest!
Te... struktury, powstały z ludzkiego złomu, jako sztuczne ciała dla głodnych demonów.
To jest tak, że niektóre demony są za słabe, aby pożywiać się ciałami prawdziwych istot, zatem muszą się zadowalać martwą materią, najlepiej tą, towarzyszącą ludziom przez jak najwięcej lat ich życia.
\dm{} Położyłem rękę na gablotce ze szkła wymiarowego. \dm{}
Ludzki złom. Wszystko, co ludziom w czasie życia było tak bliskie, jak własne części ciała, ale jednak nadal sztuczne i wymienne. 
Protezy kończyn, sztuczne szczęki, wózki do poruszania się, kule do chodzenia, rozruszniki serc, inhalatory, tego typu rzeczy.
Te roboty mogą więc także składać się z metalu i elektroniki, ale nie są zasilane energią elektryczną, lecz szatańską!
\end{dialogue}
Na te słowa Mateusz zrobił krok w tył.
\begin{dialogue}
\ds{} Szatańską \dm{} powtórzyłem z grodzą. \dm{} Szatańska opętana kupa śmieci.
Powstały, jako wcielenie najczystszego zła, zasilanie parą z palonych zwłok, stworzone z ludzkich odrzutów, zlepione na ślinę i cyrograf.
Może pan się przyjrzeć, ten element przykładowo, jest wykonany ze sztucznej szczęki.
\ds{} Dziwna ta szczęka.
\ds{} Nie, no. Nie ludzkiej sztucznej szczęki, czy ludzie mają po dziesięć półkolistych zębów, jak te tutaj?
Wiele istot we wszechświecie ma przecież zęby i większość z nich, tak jak ludzie, czasami potrzebuje sztucznych.
\end{dialogue}
Mechalyczny przysunął się z powrotem, chociaż nie krył obrzydzenia. Ciekawość brała górę.
\begin{dialogue}
\ds{} To rurka, od kuli od podpierania się, tym razem ludzkiej kuli. 
Właściciel był jakimś wielkim gangsterem, skazali go bodajże za morderstwo na własnych dzieciach, powiesił się w więzieniu, oczekując na śmierć.
Im większy grzesznik, tym dla takiego demona smaczniejszy.
A to jest wężyk, który był kiedyś w rozruszniku do serca... smoka.
\ds{} Mam wrażenie, że wciąż się porusza. To znaczy że nadal żyje?
\ds{} Absolutna racja, nadal żyje, lecz akurat nie pamiętam imienia demona, który go zasila.
\ds{} A... a to nie jest trochę niebezpieczne go tutaj trzymać?
\ds{} Tylko trochę. W najgorszym razie, w razie ucieczki, i tak pierwsze co by ten demon zrobił, to czmychnął jak najdalej od tego świątecznego miejsca. 
Poza tym, jest zamknięty w gablocie wymiarowego szkła, przez wymiarowe szkło nic się nie przebije.
\end{dialogue}

%NOTE Pogłębienie


Ani atomowa, ani termojądrowa, zwyczajna na proch. Kiedyś zadarliśmy trochę za bardzo z wojskiem Stanów Zjednoczonych. 
Mocno nadszarpnęli nam ochronne powłoki i w końcu ta mała bombka przebiła się przez pancerz i wpadła prosto do pieczonego dzika.
Gdyby wybuchła, to na pewno nie podróżowałby pan teraz z nami.
Piekielni amerykanie. Zbudowali swoje pociski z żelaza, wydobywanego przez niewolników w Afryce,
z prochem wyciągniętym z fajerwerków, co miały być wystrzelone na Halloween, na koniec pokropili zapalniki krwią z abordowanych dzieci.
W związku z tym, znacznie prościej udało im się przebić przez osłony Kuli.

Przy okazji wytłumaczyłem mu ochrony zastosowane w tym pojeździe. Były trzy powłoki, z tym że trzecia to już fizyczny pancerz. 
Pierwsza powłoka zatrzymuje szybko poruszające się obiekty.
Druga chroni przed naporem niepożądanych substancji, już ją widziałeś, jak blokowała wodę przed wdarciem się do środka.

Wycieczkę przerwał dźwięk otwieranego włazu i chlapanie wody.
Poszliśmy zatem przywitać trzeciego gościa. Mateusz był bardzo podekscytowany i pobiegł przodem.

\divider{}

Lenna popatrzyła się na Antyraxa i pokiwała w aprobacie głową.
Potem sama skierowała swoje kroki w kierunku Neofantasora.

Demony były już chyba przerażone, gdyż teraz poczęły wszystkie uciekać.
Jednak Antyrax był szybszy. Złapał jednego z nich za nogę (a właściwie to jego lateksowy but złapał nogą nogę), przyciągnął do siebie, i wcisnął mu swój worek na głowę.
Piotr Lekter zaczął się dusić, trująca abstrakcja wgryzała się w jego demonowe płuca, a lateksowy but, z siłą wolnego oprogramowania, ściskał mu szyję.

\ds{} Dość, wystarczy. Dam ci te dwie gwiazdki! \dm{} Z worka słychać było jedynie stłumione jęki. \dm{} Trzy! Niech będą trzy gwiazdki. I komentarz. \de{}

Antyrax jednak nie odpuszczał. Ruchy Piotra Lektra stawały się coraz wolniejsze i wolniejsze.

\divider{}

Nadar. Czemu to akurat jego musiał ten Kula zaprosić?
Planowaliśmy eleganckie przyjęcie, a ta niewychowana świnia pewnie pociągnie w swoje odmęty i Mateusza.

Jak tylko usłyszałam szczęk łańcuchów, przerwałam robienie makijażu i wystawiłam głowę z pokoju.
Zobaczyłam nowego gościa, zbiegającego po schodach do szatni, biegł tak szybko, że spłoszył mi motyle.
Nie spieszyło mi się powitać Nadara równie prędko, ale ciekawiło mnie zobaczyć reakcję Mateusza, gdy zobaczy tego szaleńca.
Z trudem przecisnęłam się w tej sukni przez drzwi i ostrożnie podeszłam do pierwszego schodka w dół.
Oczywiście rajstopy, od razu hyc i resztę schodów koziołkowałam, robiąc podwójne salto, lądując na głowie z nogami majtającymi się w powietrzu.

Wiedziałam, co teraz usłyszę i nie zawiodłam się.

\ds{} Ale dupa, co nie? \dm{} Nadar zamykał korbą właz, gapiąc się na moje machające w górze nogi. \de{}

\ds{} No, nawet... \dm{} Mateusz okazał się równie niewychowany. Nie wierzę, że się z nim zaprzyjaźniłam. \de{}

Zaraz przybiegł Kula i pomógł mi się postawić do pionu. Był czerwony ze złości.
Ale czy dlatego, że właśnie ze statku z sykiem uchodziła kultura, czy dlatego że świństwo uzyskało nowego członka?

\ds{} To ty! \dm{} Kula trzymał laskę w górze, niczym śmiercionośny laser krojący Nocnego na pół. \dm{} To ty wziąłeś czwarty kombinezon z mojej garderoby! Szukałem go po całym wszechświecie. To integralna część Kuli, generuje go Matryca, więc jest niereplikowalny. Nie wolno go zabierać! \de{}

\ds{} Przecież nie zabrałem, a pożyczyłem. Zresztą i tak zawsze się kurzył w tej twojej półkulistej szafie. \dm{}
Nadar uznał to za wystarczające wytłumaczenie, rozpiął strój. Pod spodem miał swoje standardowe dresy. \dm{} 
A na przeprosiny mam prezent. Wyłowiłem ci, Profesorze, zestaw kieliszków i butelkę najdoskonalszego wina prosto z kapitańskiego mostka.
Mieli ją wypić na ukończony rejs, ale wiadomo co się stało. Niech zamiast tego Kula ukończy swój własny i nie uderzy w żadną lodową kometę po drodze. \de{}

Profesor Kula w jednym pulsie zmienił się z czerwonego z powrotem w białego.

\ds{} Och. To bardzo miło z twojej strony. \dm{} Odpowiedział miękkim głosem. \dm{} A teraz wybaczcie, muszę dopilnować ostatnich poprawek przy naszym bankiecie. \dm{}
Porwał butlę i kieliszki, znikając w górnych piętrach.

Kto, jak kto, ale Nadar doskonale wyczuwał charakter innych osób.
Wiedział co zrobić, żeby zabolało i co żeby było dokładnie na odwrót.
Jednak pomimo wad, potrafił, jak nikt, walczyć z uniwersalnością.

Mateusz wpatrywał się w Nadara, jak Kula w obraz autora białego cyrkowca.
Jego największe zainteresowanie wzbudzały dwa pistolety gościa, zawieszone przy pasie, i laserowa pałka na plecach.
Pierwsze, to był standardowy szyfrator, jaki każdy w ALOPP posiadał.

\ds{} To urządzenie pozwala zaszyfrować i odszyfrować dowolną osobę w splocie czasoprzestrzeni.
Tak jakby zamrozić w czasie.
W pełni bezpieczny sposób na unieszkodliwianie wrogów bez zabijania.
Wadą jest tylko to, że naboje do niego są takie olbrzymie i jednorazowe.
W środku wkładu zapisuje się symetryczny obraz klucza, jedyny sposób na przywrócenie zaszyfrowanej osoby do życia.
Kasiu, właśnie zgłosiłaś się na ochotnika, aby zaprezentować naszemu gościowi ten wynalazek. \dm{} Nadar wycelował we mnie szyfrator. Co za świ...

\divider{}

Antyrax podniósł lekko worek, Piotr Lekter spróbował złapać oddech, ale zaraz znowu światło zgasło mu przed oczyma.

\divider{}

...nia z niego. \de{}

\ds{} Jak widzisz, działa znakomicie. \dm{} Spostrzegłam, że w czasie gdy byłam zaszyfrowana, zdążył już się przebrać w elegancki strój, pożyczony z garderoby. Właśnie nakładał puder na swojego irokeza. \de{}

\ds{} Nadar, coś ty? Od kiedy ubierasz się elegancko dla Kuli? \dm{} zapytałam z niedowierzaniem. \dm{} Przecież nie gustujesz w niczym innym niż dresy. \de{}

\ds{} Od kiedy wywalił mnie w skafandrze, pośrodku kosmosu, za przypadkowe rozlanie barszczu na obrus. 
Lewitując w bezkresnej pustce, miałem sporo czasu na przemyślenie swojego zachowania i stanie się nowym człowiekiem. \dm{} odpowiedział. \de{}

\ds{} Naprawdę? \dm{} Wtedy coś mną tknęło. \dm{} Oczywiście, że nie na prawdę. Znowu się zgrywasz tak? \dm{} Tylko się wrednie zaśmiał. \de{}

\ds{} To drugie to pikler. \dm{} kontynuował. \dm{} Potrafi zapeklować kogoś, do umieszczonego tutaj, słoika ze szkła wymiarowego, żeby nigdy się nie wydostał.
Tylko odłożyć na najniższą półkę w jakiejś głębokiej piwnicy na całą wieczność.
\dm{} Spostrzegł, że gość niekoniecznie rozumie. \dm{}
Szkło wymiarowe to takie coś, które przechodzi równo przez wszystkie wymiary, także w czasie. Wygląda jak szkło, ale istnieje od zawsze na zawsze. Ma nieskończoną długość, szerokość,
głębokość i... wszystkie inne ości. \dm{}
Nadar nie przestawał wyjawiać sekretów naszej organizacji. \dm{}
To jest laserowa pałka, taki przecinak, po uruchomieniu zaczyna wirować, wzdłuż pojawiają się promienie dasera. Daser przecina prawie wszystko jak masło,
a pozostałe rzeczy jak ser. No, twój mózg przeciąłby jak powietrze.
Fajna zabawka. \de{}

\ds{} I szkło wymiarowe przetnie? \ds{} Mateusz zapytał, widać że uważnie słuchał. \de{}

\ds{} Nadar, on nie przeszedł jeszcze wszystkich testów \dm{} wtrąciłam \dm{} nie zdradzaj mu tylu sekretów, bo nie wiadomo, czy na pewno z nami zostanie.

\ds{} No popatrz na niego, Magda \dm{} Nadar obchodził i studiował Mateusza ze wszystkich stron. \dm{} Myślisz, że sobie nie poradzi?
Poza tym, już pierwsze części przeszedł doskonale. Odpowiedział na abstrakcyjny list Profesora, ubrał się w najprawdziwszy strój francuski, a potem 
odważył się wsiąść do wielkiej latającej kuli z kosmosu. \de{}

\ds{} Niby racja, ale doskonale wiesz, jak chory test potwory mogą tym razem wymyślić.
Pamiętasz, jak Mikołaj przetestował Ziemowita? Kazał mu się przebrać za klauna, przyjść na zabawę dla dzieci i robiąc magiczną sztuczkę, zamordować jednego z nich, co był owocem klonu.
Coś nie wyszło i wszyscy skąpali się w zielonej krwi tego podrabiańca.
Drugim zadaniem było uciec z więzienia do którego go wrzucili. \de{}

\ds{} Przecież zdał. \de{}

\ds{} Albo tego, jak mu było, Błażeja, co Hdro zostawił w Capitalu i kazał jakimś sposobem wrócić na Ziemię.
Człowiek sam na planecie, w całości zamieszkanej przez wszystkie gatunki smoków. 
\dm{} Kontynuowałam rozmowę, zupełnie ignorując osobę na której temat ją toczyliśmy. \de{}

\ds{} Nie zaliczył, bo ukradł rakietę jakiejś rodzince błękitnych celebritów będącej wakacjach w Capitalu, zamiast rozegrać to w pokojowy sposób. Nawet się nie przejął że w środku wciąż były ich jaja!
A gdy rzucił się za nim pościg bordowych pasowców, on ich bezwzględnie pozabijał, strzelając z rakietowego działka.
Na szczęście nie miał kluczy warstw i rozbił się o ścianę kwadry. W dodatku drugiej kwadry, nie czwartej! Idiota tylko się oddalił od Ziemi.
Lewitował w smoczej przestrzeni przez dwa dni, prawie umierając z głodu. W końcu te małe smoczki z rozbitej rakiety się wykluły i zjadły go żywcem.
Dobrze mu tak. \de{}

\ds{} Przepraszam, że wam przerwę, ale co ze mną? To jakiś test zręczności, albo inteligencji? Zginę, pożarty przez coś? \dm{} Mateusz bezwstydnie przerwał. \dm{} Co mam zrobić, żeby go zdać? \de{}

\ds{} Masz być sobą. \dm{} Odpowiedziałam równocześnie z Nadarem. \de{}

\ds{} Ciebie chyba będzie testował Plazma. \dm{} Nadar się zamyślił. \dm{} On lubi militarne klimaty, pewnie trafisz na Planetę Wojny.
Albo będziesz wyżynał jakieś miasto, albo sam będziesz wyżynany. Musisz sobie poradzić. 
Najlepsza śmierć... to będzie rozerwanie na kawałki przez jakąś futurystyczną wunderwaffe, najgorsza, pewnie wcielenie do Czarnej Armii.
Obedrą cię ze skóry, wyłupią oczy i zęby, wcisną w cyber-zbroję i zaleją szaleniotwórczym smarem khaki, będziesz umierał powolną śmiercią, dobrze się przy tym bawiąc przy mordowaniu niewinnych. Ja tam wolę robić to samo, nie będąc rozpuszczanym przez czarny kwas.
\de{}

Tymczasem rozległ się dźwięk dzwonu, oznajmiającego posiłek.
Zgodnie poszliśmy na najwyższe piętro, Mateusz tym razem trzymał się z tyłu.
Przechodząc przez muzeum, Nadar poklepał radośnie gablotę demonicznego mózgu, który w odpowiedzi kłapnął groźnie szczerbatą protezą zębów.

Na najwyższym piętrze znajdował się salon, biblioteka, scena teatralna, ogród i kapliczka.
Dach przyjmował tutaj miłą wklęsłość, ze szczytu zwisał żyrandol na lampy oliwne.
Mateusz zapytał mnie, dlaczego w ogródku rosną tylko ziemskie kwiaty.

\ds{} Nie wiem czy wiesz, ale Ziemia jest uważana przez wielu za najpiękniejszą planetę wszechświata \dm{} odpowiedziałam. \dm{}
A przynajmniej na pewno przez naszego Profesora. \de{}

\ds{} Tos? \dm{} Mateusz zwrócił się w kierunku muzeum. \de{}

\ds{} Tos jest bardziej niezwykły niż piękny. Poza tym, grzyby trzeba by hodować w amoniakowej szklarni.
No i nie powąchasz ich jak kwiatów \ds{} wyjaśniłam. \dm{} Aha, jeszcze Tosowe życie puszcza wszędzie zarodniki. 
Jeden wdech atmosfery tej planety i zaczną ci rozpuszczać żywcem nos. Dwa wdechy i spleśnieją ci płuca. Trzy wdechy i grzybnia wkręci się w mózg. \de{}

\ds{} A kaplica? \dm{} Zwrócił uwagę na mały budyneczek w rogu. \de{}

\ds{} Jak pewnie zauważyłeś, Profesor jest bardzo religijny. Poza tym dobrze mieć miejsce, gdzie można modlić się o litość, będąc atakowanym przez kosmicznych bandytów. 
Każdy większy statek ma przecież kaplicę, to czemu kosmiczny nie miałby mieć? Kula zrobił kiedyś bardzo wiele dobrego, w prezencie otrzymał ten właśnie kulisty twór. \dm{}
Rozejrzałam się, gdzie jest nasz gospodarz. \dm{}
Ale on nie lubi, gdy się o nim rozmawia. \de{}

\ds{} Tylko, kim on jest? \dm{} Mateusz nagle zapytał. \de{}

\ds{} W sumie nikt nie wie dokładnie, kim, lub czym jest Profesor. Niektórzy mówią, że aniołem, inni że dziwnym człowiekiem,
jest też teoria jakoby był ostatnim z jakiejś umarłej cywilizacji. 
Posługuje się jedynym w swoim rodzaju pismem i językiem, którego nikt inny we wszechświecie nie używa.
Widzi szerszy zakres barw, słyszy więcej dźwięków, nie wiem czy jest supersilny... \de{}

Uratował mnie dzwonek, zwiastujący rozpoczęcie bankietu.

\divider{}

Antyrax podniósł worek, Piotr Lekter był sztywny, jego wyraz twarzy poskręcany był w dziwności. Tylko jedno oko lekko mu drgało.

Tymczasem wszyscy inni uciekli. Nie było już ani demonów, ani Neofantasora. Antyrax nie był aż taki głupi, żeby uwierzyć, że niebezpieczeństwo minęło. 
Na pewno czyhali na niego, pochowani w ruinach miasteczka.

Stąpał cicho. Pomimo to, jego kroki były dobrze słyszalne w grobowej ciszy spowijającej dolinę.
Zero wiatru, zero ptaków, zero mieszkańców. Wszystko umarło.
Za chwilę on sam umrze, śmierć przyjdzie po cichu i autor nawet nie spostrzeże się kiedy.

Wtedy wysuneła się zza winkla Winkla. 

\ds{} Pisze się ,,wysunęła'' \dm{} powiedziała i cisnęła w niego błędem ortograficznym. Ostrze wbiło mu się w pierś do połowy. \de{}

Antyrax poczuł, jak trucizna dyktanda rozpływa mu się po żyłach. To był koniec. Żadna ilość abstrakcji nie wygra ze zwyczajną poprawnością językową.
Choćby stworzył kompletny co do atomu świat, to i tak niewiele by dało. 
Powinien wrócić i pisać programy komputerowe, zamiast tworzyć epikę. Przynajmniej tam kompilator powie mu o brakujących średnikach.
Opatrzył się wyrwanymi ze słownika kartkami, pomogło, ale na krótko.

\ds{} Daj mi jeden powód, dla którego miałabym cię oszczędzić. \dm{} Demonka zawiesiła nad pisarzem olbrzymie ostrze z poprawnie zastosowanych w dialogach myślników. \de{}

\ds{} Mateusz musi dolecieć na miejsce, prawda? \dm{} odpowiedział. \de{}

\ds{} Jakoś leci i leci, a nadal nie wiadomo gdzie tak dokładnie jest. Rzucasz na prawo i lewo pojęciami, zupełnie ich nie tłumacząc.
Uniwersalność, światłografy, warstwy, potwory. O co chodzi? \de{}

\ds{} Chciałem opowiedzieć o nich później, akcja się rozwija powoli, ciekawiej jest najpierw rzucić hasło, a potem dopiero je opisać. \de{}

\ds{} Od kilku stron ta opowieść to jeden wielki opis! \de{}

\ds{} Nic byś nie zrozumiała, gdyby nie opisy. Jakbym napisał, że polecieli na Felicję, używając górnej warstwy, chociaż Kula nie posiadała do niej kluczy, to co byś sobie wyobraziła? \de{}

\ds{} Że to jakaś nadprzestrzeń, coś w stylu alternatywnego wymiaru. A klucze to pewnie wysokotechnologiczne urządzenia do wchodzenia w nią. \de{}

\ds{} Prawie. Górna i dolna warstwa odpowiadają za obieg energii we wszechświecie. Górna rozprowadza, a dolna zbiera.
Dolna warstwa czasami ma gejzery, niekontrolowane wybuchy energii, które tymczasowo uniemożliwiają korzystanie z niej w tym miejscu.
Bez opisu nie wiedziałabyś, że nie wolno korzystać z górnej warstwy, gdyż wprowadza to zawirowania w energii z różnymi dziwnymi skutkami w czasoprzestrzeni pod nią.
Te skutki to niedomiary i nadmiary Boskiej Energii, co się objawia większą skłonnością ludzi do popełniania grzechów, lub dobrych uczynków.
Zabronione jest to przez Niebo, znajdujące się w pierwszej kwadrze, w ogóle wszechświat składa się z czterech kwadr.
Tam siedzą aniołowie i mówią, co cywilizacjom wolno, a czego nie wolno oraz gdzie sprowadzić kataklizm, a gdzie zesłać cud. Do transportu używa się dolnej warstwy, gdyż przelot przez nią nie powoduje zawirowań energii. 
Dodatkowo klucze to takie artefakty rozdawane zapoznanym cywilizacjom przez odpowiedni instytut niebiański, mają kształt kwiatka z guzikiem pośrodku. Nie zużywają energii, wystarczy nacisnąć, aby otworzyć w tym miejscu okrągły portal. Tym samym wprowadzamy jeszcze więcej tajemniczości do postaci Profesora Kuli.
To ktoś, komu aniołowie pozwolili dowolnie korzystać z górnej warstwy, a jego statek może dowolnie przemieszczać się w dowolne miejsca.
Tyle właśnie informacji jest potrzebne, żeby poprawnie zrozumieć to zdanie. \de{}

\ds{} Znowu aniołowie i chrześcijaństwo, wszyscy już o tym piszą. Nudne się to robi. \de{}

\ds{} Ale gdybym napisał, że statek Kula jest zasilany przez samego Belzebuba, a Profesor ma na swojej lasce czaszkę dziecka, oraz potwory były tylko od niszczenia światów, to nie było by w tym nic niezwykłego, prawda? Wręcz pewnie dostałbym porównania do Warhammera 40K. Otóż nie. Statek Kula działa na energię Boską, potwory służą aniołom, a ALOPP składa się z białych katolików polaków. 
Nikt nie opisał wcześniej takiego świata, a ja będę pierwszy. \de{}

\ds{} No więc opowiadaj. Masz kilka minut, zanim trucizna dotrze do twojego mózgu, a wtedy już nigdy więcej nie popełnisz żadnego błędu ortograficznego. \de{}

\divider{}

Zasiedli do stołu. Srebrne sztućce z diamentowymi akcentami, oraz ręcznie rzeźbione talerze, dobrze współgrały z tkanym obrusem ze złotych nici.
Nigdy nie widział tylu zastawy dla jednej osoby. Otrzymał po trzy noże i widelce, łyżkę, łyżeczkę, widelczyk, pałeczki, dziwny szeroki nóż, trzy kieliszki, duży talerz, dwa małe talerzyki i głęboki talerz.
Do kryształowych kieliszków profesor nalał wszystkim. Titanicowego wina.

Na przystawkę było sushi z kawiorem i truflami. Gdy przyszli, było już nałożone na talerzu. 
Profesor odmówił krótką modlitwę, dziękując za dar egzystencji w imieniu wszelkiego życia, czasu i przestrzeni.

Jedzenie, jak wszystko, bardzo kosztowne, a jakże, lecz w stu procentach pochodzenia ziemskiego.
Mateusz spodziewał się jakichś nieziemskich przysmaków, dziwacznych owoców, grzybów z Tosa, czy steku ze smoka.
Czy to była prawda, że wszechświat jest całkowicie pusty, a Ziemia jest jedynym znośnym miejscem w kosmosie?

Bardzo zaskoczyło Mateusza to, jak kulturalnie zachowywał się Nadar. Jest arogancki i nie leżą mu galowe ubrania, lecz z zachowania wychował się na dworze królewskim.
Z kolei Katarzyna przywiązała olbrzymią wagę do ubioru, ale nie potrafiła poprawnie złapać pałeczek.
Profesor miał naturalnie i jedno i drugie.

Kula wstał i przemówił.

\ds{} Tradycją jest, że przy głównym daniu wybieramy się w jakieś malownicze miejsce, dezaktywujemy tarcze i otwieramy dach, spożywając posiłek na świeżym powietrzu. \dm{}
Rozpoczął tajemniczo. \dm{} Proponuję, aby tym razem nasz główny gość wybrał miłą okolicę, w której będziemy mogli najlepiej delektować się dzisiejszą pieczenią z bażanta. \de{}

\ds{} Tylko nie środek oceanu, ani malownicza plaża \dm{} Nadar bezwstydnie się wtrącił. \dm{} Mam na jakiś czas dość wody. 
Poza tym ostatnio jedliśmy w takim miejscu. \de{}

\ds{} To może pustynia? Piaskowe wydmy są bardzo ładne w promieniach zachodzącego słońca \dm{} odpowiedziała Kasia. \de{}

\ds{} Nie, piasek wpada do jedzenia i chrzęści w zębach. \de{}

\ds{} Szczyt Andów? Ładne widoki z jednej strony na puszczę, z drugiej na ocean. \de{}

\ds{} Zimno tam jest, będzie ci zamarzać zupa na talerzu. Potem będziesz cała mokra od topniejącego śniegu. \de{}

\ds{} Amazonia. \de{}

\ds{} Komary. \de{}

\ds{} Antarktyda. \de{}

\ds{} Pingwiny. \de{}

\ds{} To chyba zaleta. \de{}

\ds{} Nie, jeśli podkradają ci jedzenie z talerza. \de{}

\ds{} Nie wiem, Paryż. \de{}

\ds{} W miastach wojsko strzela i rozlewa wino z kieliszków. \de{}

\ds{} Gejzery na Islandii. \de{}

\ds{} Dla mnie okej. \de{}

\ds{} Przepraszam bardzo, ale wyjątkowo nie przepadam za zapachem siarkowodoru. \dm{} Kula nagle się przyłączył. \de{}

\ds{} Sawanna? \de{}

\ds{} Za dużo turystów na safari, robią ciągle zdjęcia. \de{}

\ds{} A może Etna? \dm{} Mateusz niespodziewanie wypalił. \de{}

\ds{} Etna? \de{}

\ds{} Środek wulkanu, kula do połowy zanurzona w płynnej lawie, fontanny ognia oświetlające otoczenie. Subtelne pomruki z wnętrza Ziemi.
Jeśli dobrze zrozumiałem, tarcze powinny nas przed nią ochronić. \de{}

Nastała niezręczna cisza. Czy Profesor śmiał się w duchu, że Mateusz przecenił zdolności statku, czy może rozpatrywał pomysł poważnie?

\ds{} To doskonały pomysł. Aktywujemy wszystkie warstwy na raz. Normalnie spowodowałoby to, że czulibyśmy się 
jak w szklanej kuli, bez wiatru, bez temperatury. Lecz w tym przypadku byłoby to i tak wskazane.
Nie ma niczego lepszego od podziwiania lawowych rozbryzgów z odległości metra. \de{}

Mateusz nie spodziewał się takiego obrotu spraw, ale cieszył się niemiłosiernie na myśl o zapatrzeniu się w przelewający się żywioł.

Tymczasem statek wynurzył się z Morza Śródziemnego. Profesor niepostrzeżenie przemknął przez Cieśninę Gibraltarską i skierował się prosto w kierunku włoskiego wulkanu.
Ponieważ jednak Kula nie posiadała żadnych okien, goście mogli jedynie uwierzyć mu na słowo. Przynajmniej do póki nie otworzy dachu salonu.

\ds{} Nie wierzę, że nie pytałeś się jeszcze nikogo, czym dokładnie jest to miejsce. \dm{} Nadar zrobił sobie przerwę od jedzenia 
tłustego żurku i przykrył chlebową miskę, chlebową przykrywką. \de{}

\ds{} Czyli to nie jest tajemnica na równi z gabinetem Profesora? \dm{} Mateusz zapytał. \dm{} Myślałem, że nie wolno mi było tego wiedzieć.
Albo powiem inaczej. Nie chciałem być wywalony przez Profesora Kulę w kosmos tylko dlatego, że zadałem niewłaściwe pytanie w niewłaściwym momencie. \de{}

\ds{} Panie Mateuszu Mechalyczny. \dm{} Profesor uśmiechnął się tajemniczo. \dm{} Co innego wyciąganie od innych zakazanych informacji, a co innego rozlewanie barszczu. 
Ciekawość jest wysoko ceniona w ALOPP, a także przeze mnie. Proszę śmiało pytać. \de{}

\ds{} Zatem skąd to jedzenie? \dm{} Zaczął od najbliższej mu rzeczy. \dm{} Nie widziałem tutaj żadnej kuchni, nie widziałem spiżarni.
To jedzenie po prostu się tutaj pojawiło, jak przyszliśmy. Po przystawce z sushi, odwróciłem się na chwilę i znalazłem zaraz przed sobą bochen chleba z zupą.
Co to za czary? Co to za materiał? Co może być na tyle silne aby wytrzymać napór gorącej lawy? 
Skąd w skafandrach bierze się nieskończona ilość powietrza? Jaki jest tutaj obieg wody? 
Gdzie ucieka dym z pieca w łaźni? Jak się tym czymś w ogóle steruje? Skąd czerpie energię? Gdzie ma silniki? \de{}

\ds{} Odpowiedź na twoje wszystkie pytania znajduje się za tobą na ścianie. \dm{} Kula oznajmił, jakby to było oczywiste. \de{}

Na ścianie wisiał elegancki zwój papieru, oprawiony w mieniące się polaryzacyjnie szkło.
Podobnie do gabloty z diabelskim mózgiem i słoika na piklerze. Musiało więc być to szkło wymiarowe.

\niceframe{
\begin{Fontlukas}
\begin{center}
ŚWIATŁOGRAF
\end{center}
Potwierdza z Mocy Najwyższego nadanie specjalnej właściwości Profesorowi \weirdchar{profesor}.

Sacroteriowy twór w formie uniwersalnego statku kosmicznego jest niniejszym przekazany Profesorowi od zawsze na zawsze dla dowolnych celów.
Dokładny projekt został nieodzownie zapisany w Matrycy.

Nie pobiera się żadnej opłaty od właściciela.

Z Bogiem.
\end{Fontlukas}}

U dołu fragment papieru wyglądał na pozaplątywany w dziwaczne supełki.

\ds{} Światłograf to odwrotność cyrografu. Daje ci, jak to jest ładnie opisane, pewną właściwość. \ds{} Nadar zaczął opisywać zamiast Kuli. \dm{}
Może umożliwiać ci żyć wiecznie, strzelać promieniami z rąk, wygrywać w lotka, eksplodować innym mózgi za pomocą pstryknięcia palców, czy właśnie posiadać taką oto okrągłą rzecz. \de{}

\ds{} A sacroteria? \de{}

\ds{} Sacrum i materia. \dm{} Tym razem Katarzyna rozpoczęła wyjaśnienia. \dm{} Materia, która wygląda i reaguje jak zwyczajna materia, lecz może zachowywać się w pewnych przypadkach całkowicie po swojemu. \dm{} Wzięła w palce końcówkę obrusu. \dm{} Z czego to jest zrobione? 
Powiedzielibyśmy że z nici i złota, może jakiś jedwab, albo z czego tam się robi obrusy.
Ale to jest sacroteria. Może się zachowywać i plamić jak obrus, ale może równie dobrze zrastać po przerwaniu jak żywa skóra. 
W Matrycy jest zapisane prawo tego obrusu, jak i prawa całej sacroterii we wszechświecie. Cała kula i to jedzenie także jest z sacroterii. \de{}

\ds{} Prawie, w Matrycy zapisano, że wytworzone jedzenie jest całkowicie zwyczajną materią. \dm{} Gospodarz poprawił. \dm{}
Jednak to nie istotne, gdyż nie byłby waćpan w stanie w żadnym stopniu doświadczalnie tego stwierdzić, jedynie zaglądając do Matrycy ma się całkowitą pewność. 
Matryca to prawdziwe miejsce, ma kształt wielkiej płyty położonej nad całym wszechświatem, powyżej górnej warstwy. Naturalnie nikt nie ma do niej
dostępu. \de{}

\ds{} Zwykle za światłograf pobierana jest opłata w ilości dobra do wytworzenia. \dm{} Znowu Nadar zaczął. \dm{} Może opierać na liczbę uratowanych dusz, 
jakiś wielki czyn, czy właśnie na nic. Ale myślę, że jednak nasz Profesor kiedyś coś fajnego wykonał, żeby Niebo go tak polubiło. \de{}

\ds{} A ta plamka? \de{}

\ds{} To odcisk duszy Profesora Kuli. Światłograf musi być podpisany. Nie jakąś przyziemną krwią, lecz czymś wiecznym i niezniszczalnym, twoją duszą. \de{}

\ds{} Pora na widowisko \dm{} właściciel kuli przerwał rozmowę. \de{}

Stuknął laską i wtedy cały dach począł się otwierać niczym kwiat. 
Żyrandol został w miejscu, trzymany przez niewidzialną siłę, unosił się niczym zaczepiony o dach, który właśnie rozszedł się na osiem stron. 

Czerwona śmierć przelewała się przez płatki i obijała o niewidzialną barierę, spływając majestatycznie.
Płomienne światło tworzyło wspaniałą atmosferę na ucztę.
Rozsunięte fragmenty dostawały z pełną siłą żywiołu, lecz czerwone futro wcale się nie paliło, lawa po nim spływała, jak po mokrej kaczce.

Tak, jak zapowiedziano, na główne danie podano pieczonego bażanta w sosie kurkowym. 
Ponownie eleganckie jedzenie i ponownie z Ziemi. A może tym razem było coś w nim niezwykłego? Mateusz pomyślał, że mogłaby istnieć planeta bażantów. Doskonałe warunki rozwoju, brak naturalnych wrogów i posmak niedalekiej supernowej.
\begin{dialogue}
\ds{} Przepraszam, skąd pochodzi ten bażant? \dm{} zapytał.
\ds{} Z Małopolski \dm{} Profesor odpowiedział.
\ds{} Cóż spodziewałem się, że może tym razem spróbujemy czegoś bardziej kosmicznego.
\ds{} Wyplujesz te słowa. \dm{} Nadar delektował się każdym kęsem. \dm{} Jeszcze będziesz tęsknił, żeby zjeść prawdziwe, Ziemskie jedzenie. 
\ds{} Nadar ma rację. \dm{} Katarzyna pochłaniała ptaka wielkimi kawałami. \dm{} Czasami zdarza się nam być zapraszanymi na różne pozaziemskie przyjęcia. Nie spodziewaj się tam niczego dobrego.
\end{dialogue}

Nie chciał wierzyć ich słowom. Ale chyba nie miał wyboru.

\divider{} 

\begin{dialogue}
\ds{} Akcja, Antyrax! Daj mi akcję \dm{} przerwała mu Winkla. \dm{} Mam nadzieję, że w czasie tego bankietu zdarzy się coś ciekawszego.
\end{dialogue}

\divider{}

Wtem poczuli silne uderzenie, które wstrząsnęło statkiem, porozlewało wino z kieliszków i spowodowało, że Kula nie trafił widelcem w obiad.
Dźwięk eksplozji dochodził z góry, tam też patrząc, zobaczyli na niebie całą armię latających helikopterów, właśnie jeden z nich wystrzelił drugi pocisk i zaraz druga eksplozja wstrząsnęła otoczeniem.

Profesor wstał, ukłonił się i zbiegł po schodach. Zaraz też wrócił, niosąc olbrzymią, ozdobną tubę od gramofonu wmontowaną w taboret. Tuba była podłączona wężykiem do lejka, który to przyłożył sobie do ust.

\begin{dialogue}
\ds{} Jakim prawem przerywacie nam uroczystą konsumpcję bażanta, ciskając w nas wybuchowymi pociskami? \dm{}
krzyknął, a tuba wzmocniła jego dźwięk do potęgi megafonu. \dm{}
Proszę natychmiast opuścić krater wulkanu, inaczej komuś może stać się krzywda! \dm{} Odpowiedziała mu trzecia rakieta, kolidująca z tarczą.
\ds{} Wojsko Stanów Zjednoczonych, tajny oddział do walki z kosmitami. \dm{} Nocny wyciągnął z kieszeni okrągłą komórkę i podawał informacje.
\ds{} Proszę was. Nie musimy uciekać się do elektroniki. Schowajcie to z powrotem. \dm{} Nikt się nie przejął uwagą starszego pana.
\ds{} Nadar. Masz coś mocniejszego, niż laserpała? \dm{} Katarzyna grzebała sobie pod suknią, niemal odwracając się na lewą stronę.
\ds{} Jest pikler, ale to na mocne. Potem ich nie wyciągniemy.
\ds{} Bierz tego dużego szyfratorem, wystraszysz resztę.
\ds{} Nie, bo spadnie do lawy.
\ds{} To ostatnie ostrzeżenie! \dm{} Kula nie brzmiał przekonująco nic a nic.
\ds{} Mam znikarkę.
\ds{} Oszalałaś? Jeszcze uderzy w tarczę statku. Przywołam guziki.
\ds{} Ile to może potrwać? Poza tym ukradną je i skopiują.
\ds{} ,,You are being arrested under UN law.'' \dm{} Dało się słyszeć głośnik z góry.
\ds{} ,,You are being destroyed under ummm... The Law.'' \dm{} Nadar wyrwał Kuli lejek.
\ds{} Cokolwiek zrobimy, to spadną do wulkanu.
\ds{} No to zostaje pikler, potem to ustalimy z Niebem.
\ds{} A co jeśli trafisz po drodze na uniwersalność? Wsadzisz ich razem? Nie mam zapasowych słoików.
\ds{} ,,Please leave your spaceship right now!''
\ds{} Sam jesteś ,,spaceship''. \dm{} Profesor nie dawał za wygraną.
\ds{} A mini-ferro?
\ds{} Ja nie wiem, jak on działa.
\ds{} ,,You have no power against army of the United Stat...''
\ds{} ,,Shut up!'' \dm{} Katarzyna przekrzyczała hałas lawowych fontann. 
\ds{} Patrzcie! \dm{} Jednej z maszyn eksplodował silnik, pilot wykatapultował się tak niefortunnie, że spadł prosto na szczyt statku.
\ds{} Kula wpuść go! Helikopter spada! \dm{} Nieprzytomny żołnierz przesiąkł przez tarczę, zaraz w to miejsce uderzył wrak śmigłowca. 
Pomarańczowe światło przyćmiło na chwilę poświatę wulkanu. Kasia wystrzeliła szyfratorem w nieproszonego gościa.
\ds{} Mamy zakładnika, odpuście, albo go zab... coś mu zrobimy.
\ds{} Oni nie rozumieją po polsku, Profesorze.
\ds{} ,,You out, or he die.'' \dm{} Mateusz nie spodziewał się po Kuli tak słabej znajomości angielskiego. Jednak zrozumieli i przerwali ogień.
\ds{} ,,We will negotiate, please open your forcefield.'' \dm{} Statki powietrzne zniknęły z pola widzenia, przyleciał za to jeden mały helikopterek z przywiązaną białą flagą.
Zbliżał się powoli i stanął w powietrzu, oczekując aż zniknie ochrona.
\ds{} To pułapka! \dm{} Nadar złapał Katarzynę za plecy i skulił razem ze sobą. Chwilę potem olbrzymia eksplozja wstrząsnęła światem. Wszyscy upadli tam, gdzie stali.
\ds{} Ja pierdolę. \dm{} Nocny pomógł wstać Kosmatej. \dm{} Drugiej nie przeżyjemy.
\ds{} Musimy uciekać, zasuwam dach! \dm{} Wielki kwiat począł zamykać swe płatki, wypływając z jeziora ognia.
\ds{} Stop! Tam, to chyba daser!
\ds{} Skąd...?
\ds{} Nie zmieścimy się obok! Przekroją nas na pół!
\ds{} A może do dołu? \dm{} Mateusz zaproponował.
\ds{} W dół?
\ds{} Do wnętrza Ziemi.
\ds{} Dałoby się.
\ds{} Nie polecą za nami.
\ds{} I też nic nie wystrzelą.
\ds{} W spokoju otworzymy przejście do górnej warstwy.
\ds{} A zatem w dół.
\end{dialogue}

Białe zwierciadło poczęło się zanurzać coraz głębiej w Etnie.
Po kilku sekundach zniknęło pod fontannami płynnych skał.
Hałas kotłującego się wnętrza wulkanu wolno niknął i nastała absolutna cisza.
Panowała atmosfera przegranej walki, goście patrzyli w podłogę i uciekali wzrokiem, jakby chcieli się nawzajem przepraszać za wyrządzone szkody.
Zaszyfrowane, pokryte niebieskawą poświatą ciało żołnierza leżało w kwietniku.

\begin{dialogue}
\ds{} Co powiecie wszyscy na deser? \dm{} Profesor przerwał milczenie.
\ds{} Jego też? \dm{} Mateusz spojrzał na połamane kwiaty pod pilotem.
\ds{} Oczywiście. Tylko trzeba go odpowiednio ubrać.
\end{dialogue}

\divider{}

Winkla popatrzyła się krzywo.
\begin{dialogue}
\ds{} Eeee, nieee. To ma być akcja? 
\ds{} Tak nagle nic się nie da stworzyć. \dm{} Antyrax kładł się powoli na ziemi. Trucizna robiła swoje.
\ds{} Daj mu spokój, mnie się podobało. \dm{} Z cienia wyszedł Everywhere Man. \dm{} Ale tylko trochę.
\ds{} Trochę to za mało, musi być idealnie.
\ds{} Nigdy się nie da nic idealnie.
\ds{} Przepraszam, ja tu umieram! \dm{} Pisarz ledwo siedział.
\ds{} Niech mu będzie \dm{} Winkla westchnęła, wyciągając strzykawkę z antidotum, wyciąg z prac pierwszoklasistów. \dm{} Ale czekamy na nienudzący opis uniwersalności.
\ds{} I trochę akcji. Użyj tego twojego żołnierzyka.
\end{dialogue}

\divider{}

Przed chwilą spadałem prosto w objęcia UFO, zaraz potem znalazłem się przywiązany do krzesła, ubrany w jakieś cyrkowe ubrania.
Siedziałem przy eleganckim stole, przede mną leżał kawałek czegoś, co przypominało ciasto. Jedną rękę miałem wolną.
Trzy osoby tajemniczo się mi przyglądały. Wolałbym standardowo trafić na stół operacyjny z próbnikiem w dupie.
\begin{dialogue}
\ds{} ,,Here, have a dessert'' \dm{} powiedział ten z irokezem. Miałem ochotę rozsmarować mu to ciasto w twarzy, ale może rzeczywiście skończyłbym wtedy z próbnikiem.
\ds{} Nadar, co z nim zrobimy? \dm{} Jeden z nich, młodszy, zapytał po polsku. Niesamowite, rozmawiają po polsku! Lepiej nie zdradzać się, że rozumiem.
\ds{} Damy mu zjeść, a potem rozkroimy i wsadzimy próbnik w dupę, żeby przeprowadzić chore eksperymenty.
\ds{} A nie warto wcześniej trochę go przepytać? Zaraz, co?
\ds{} Sam nam wszystko wyśpiewa, gdy będzie mutował w krowę.
\ds{} Co, ale... auć... aha, tak najpierw w krowę, a potem w osła. Będzie boleć, oj będzie. \dm{} Młody nagle zmienił biegun.
\ds{} Potem podrzucimy do jakiegoś niewyżytego farmera w Afryce.
\ds{} Zostawimy mu mózg, wsadzimy do słoika i podłączymy do sztucznego ciała.
\ds{} Sprawdzimy, ile taki wojak zniesie orgazmów na godzinę. Pewnie wytrzyma z dwa dni seksualnej męki, a potem wysiądzie mentalnie. \dm{} Dziewczyna się odezwała.
\ds{} Podobno całkiem dobrze sobie radzą w dziczy. Ciekawe ile czasu przeżyje sam w świecie czerwonych kartaczy. \dm{} Starszy pan się odezwał.
\ds{} Te smoki lubią ludzi, najpierw palą swoim ogniem, a potem zjadają kawałeczek po kawałku.
\ds{} Albo od razu w całości, żeby ofiara utopiła się w kwasie żołądkowym.
\ds{} Nie, to za szybka śmierć. Proponuję zawieść go na Tos, żeby wgryzły się w niego pasożytnicze grzyby, które nigdy nie pozwolą mu umrzeć.
\ds{} Super, przejmą nad nim kontrolę wystarczająco mocno, aby sterować ruchami i jednocześnie na tyle słabo zachować pełną świadomość.
\ds{} A może uwolnimy go? Odwieziemy prosto do domu.
\ds{} Doskonały pomysł, wsadzą go do wariatkowa i będą męczyć żeby im coś powiedział. Wyręczą nas z roboty. Ciekawe, jak zareagują na opowieści o bankiecie w kuli?
\ds{} Pewnie trafi na stół operacyjny z próbnikiem w dupie.
\ds{} Dość, zrobię wszystko, co mi rozkażecie! \dm{} Nie wierzyłem w ich opowieści, ale też nie miałem ochoty przekonywać się o swoich racjach.
\ds{} Mówiłem, że to Polak? Swój swojego wszędzie pozna. \dm{} Ten, którego nazwali Nadar, wykonał triumfalny gest. \dm{} Więc na początek zjedz ten przepyszny jabłecznik.
\end{dialogue}

Nie za bardzo miałem wybór. 
Gdyby chcieli mnie otruć, już dawno by to zrobili.
Poza tym, rzeczywiście wyglądał przepysznie.

Niepewnie wziąłem widelec i zjadłem trochę kosmicznego jedzenia.
Smakował, jak ciasto które robiła moja babcia, gdy jeździłem z wizytą do Polski.
Miękkie, kruche i lekko ciągliwe.
Zjadłem całe i czekałem aż zacznę mutować w krowę.

\begin{dialogue}
\ds{} Teraz do rzeczy. Skąd do cholery macie daser? \dm{} Nadar wyjął długą pałkę i uruchomił. Zielone lasery wystrzeliły z końca w dół, a całość zawirowała.
\dm{} Wnioskuję, że domyślasz się, co to robi?
\ds{} Nie. Nie mam pojęcia. Nic nam nie mówili, dostaliśmy ten laser niedawno, nie pozwalali nawet go przetestować \dm{} odpowiedziałem zgodnie z prawdą. Byłem pewien, że i tak nie uwierzą.
\ds{} Zademonstruję ci zatem. \dm{} Wziął mój karabin i uderzył w niego pałką. Przeszła, jak masło, dzieląc moją broń na dwoje, stalowe ścinki posypały się na stół. \dm{} Skąd macie naszą technologię?
\ds{} Przysięgam, nie mam pojęcia nawet co ona robiła! Nie mówią nam niczego, wszystko jest w tajemnicy. Nawet nie wiedziałem, z czym będziemy dzisiaj walczyć!
\ds{} Nadar, on chyba mówi prawdę, na pewno nie wtajemniczaliby go w zdobycze kosmitów \dm{} dziewczyna powiedziała.
\ds{} Jak wyglądało to pudełko? Co w nim było? Jak je przewozili? Do czego podłączali? \dm{} kontynuował.
\ds{} Tylko przez chwilę mi mignęło. Było w specjalnym śmigłowcu. W asymetrycznym pudle z dziurą w środku. Dwóch naukowców go obsługiwało. Chyba nie podłączali do niczego.
\ds{} Jedno?
\ds{} Tylko jedno, mówili że bardzo cenne.
\ds{} Gdzie je przetrzymują?
\ds{} Ten śmigłowiec dołączył do nas później, nie leciał ze wszystkimi.
\end{dialogue}

Nadar się rozluźnił i nawet trochę uśmiechnął.
Zdziwiłem się, pozostali także.

\begin{dialogue}
\ds{} Chyba mówisz prawdę. To by znaczyło, że musieli ukraść nam kiedyś jakieś daserowe urządzenie, ale wciąż nie wiedzą, na jakiej zasadzie działa.
\end{dialogue}

Milczałem.

\begin{dialogue}
\ds{} Mateusz, to będzie twoja pierwsza misja. Dowiesz się, gdzie trzymają ukradziony daser i weźmiesz go z powrotem. 
\end{dialogue}

Mateusz przełknął ślinę.

\begin{dialogue}
\ds{} To co z nim w takim razie zrobimy? Teraz już na serio. Nie możemy go przecież wypuścić \dm{} dziewczyna nie wyglądała, jakby tym razem żartowała.
\ds{} Trzeba by pokazać go potworom. Pewnie wsadzą go do kubistycznego więzienia.
\ds{} Czy to nie za ostro? Przecież trochę go do tego zmusili. Jest też ta nowa planeta koncentracyjna do zsyłek.
\ds{} Za dobrze wyszkolony, wymorduje wszystkich innych. To ma być zsyłka, a nie krwawa jatka.
\ds{} To może, nie wiem. Zaszyfrować aż do końca wszechświata. To dla niego będzie jak podróż w przyszłość.
\ds{} Przecież to równa się śmierci. Obudzi się tylko po to, żeby zobaczyć Apokalipsę.
\ds{} Ja spróbuję go naprawić. \dm{} Starszy pan odezwał się po dłuższym czasie. Wszyscy się zdziwili.
\ds{} Panie Profesorze, ta osoba jest niebezpieczna! Zdradzi i zabije pana.
\ds{} Ja wierzę, że każdy może się zmienić. Zrobimy z niego porządnego obywatela Felicji.
\ds{} Felicja jest przepełniona, nikt się tam więcej nie zmieści. Chyba nie chcesz wolny domowej?
\ds{} To zrobimy drugą Felicję, dzikszą, o ustalonym prawie i większej liczbie obywateli. Będzie równość i tolerancja dla wszystkich istot, będą mogły żyć w spokoju przed prześladowaniem.
Miejsce bezpieczne od przemocy, opresyjnych rządów i odrzucenia. Różnorodne i wspaniałe. Wszystkie kultury wszechświata stanowiące wspaniałą jedność.
Prawnie ustalę system który dla każdego będzie równy.
A to będzie jej pierwszy obywatel.
\end{dialogue}

Wszyscy, prócz Profesora parsknęli śmiechem, próbując go zdusić.

\begin{dialogue}
\ds{} To już lepiej na zsyłkę. Przynajmniej będzie miał szansę na dożycie starości \dm{} Nadar zakończył.
\end{dialogue}

Starszy pan poprowadził mnie od stołu i rozwiązał. Za plecami słyszałem tylko kolejne wizje, co by się na tej ,,lewackiej'' planecie działo.
Chłopacy rzucali obleśnymi pomysłami co do ustanowionych praw, dziewczyna rozpatrywała kto i dlaczego nie mógłby ich przestrzegać.
\begin{dialogue}
\ds{} Zakaz jedzenia mięsa, bo zabijamy biedne zwierzątka.
\ds{} Ale takie zesłane smoki, mogłyby to przeżyć?
\ds{} Myślę, że pożywiałyby się innymi obywatelami.
\ds{} Na pewno dałoby się napisać ustawę rozwiązującą ten problem.
\ds{} No co ty, to przecież element ich kultury...
\ds{} Nie możemy zabronić innym być sobą, ty pieprzony rasisto, ha ha...
\ds{} ...nie wolno ci zabronić mi być rasistą... musisz tolerować moją nietolerancję...
\ds{} Codzienne ćwiczenia seksualności...
\ds{} ...płeć będzie jako prosta?... jako przestrzeń...
\ds{} ...identyfikuję się... jako kamień... nie możesz mnie zjeść...
\ds{} ...jestem zjadaczem kamieni... mam takie prawo...
\ds{} ...czekaj, czekaj... korek w dupie dla każdego...
\ds{} ...ustawą...
\end{dialogue}

Schodziliśmy w dół po schodach, śmiechy na górze stawały się coraz bardziej niewyraźne.
Ten statek był gigantyczny w środku. Profesor Kula, jak się przedstawił, zapytał mnie o imię.
Zaprowadził mnie do własnego pokoju i zostawił, nawet nie zamykając drzwi. Powiedział, żebym
dowolnie korzystał z dobrodziejstw Kuli i nie bał się prosić o pomoc. Zaproponował nawet kąpiel w basenie, ale oczywiście odmówiłem.
Pokój był bardzo malutki i bardzo elegancki.
Wszytko w tym obrzydliwym, rokokowym stylu.
Większość stanowiło podwójne łóżko z daszkiem i firanką.
Do tego kilka krzeseł, stoliczek szafka, toaletka.
To miejsce nie służyło do długotrwałego przesiadywania.
Gdyby urwać podpórkę od łóżka, rozbić lustro, związać firaną, rozłożyć krzesło, to bym mógł sobie stworzyć jakąś włócznię i tarczę.

Kogo ja oszukuję, jestem bezsilny wobec ich technologii.
Mogli mnie zamrozić w czasie ponownie, gdybym tylko próbował coś odwalić.
Albo przeciąć na pół tą laserową pałką.
Będę musiał jakoś uciec. Może mają tutaj kapsuły awaryjne?
Trzeba się rozejrzeć pod pretekstem zwiedzania.

Czyli to jest UFO, a oni są kosmitami. 
Ale nie byli. Tego jednego byłem pewien, no może poza Profesorem Kulą.
Bardziej przypominali gości, tak inni od właściciela. Pewnie ich zaprosił do siebie na wyprawę.
Czy lecą w takim razie gdzieś na obcą planetę?
Na wakacje, czy do domu.
Do domu, nie mieszkają na Ziemi, bo używają abstrakcyjnych technologii.
To znaczy że jest jakaś cywilizacja ziemian, a może nawet polaków, poza naszą planetą.
Skomplikowane to wszystko, nie mniej jednak z pewnością nie zabiją mnie tak od razu.

Dopiero teraz zwróciłem dokładniejszą uwagę na swój ubiór.
Bardzo kosztowny i elegancki. Nie wiedziałem, gdzie jest mój oryginalny mundur. Nie miałem nic innego, więc go zachowam, może się przydać.
Usiadłem na łóżku. Było miękkie i wygodne, na chwilę położyłem się na plecach.
Nawet nie wiedziałem, kiedy zasnąłem.

Obudziło mnie głośne walenie. Metaliczny dźwięk rozchodził się po całym statku.
Szklane ozdoby dzwoniły z każdym uderzeniem, coś atakowało Kulę.
Stwierdziłem, że skorzystam z zamieszania i wymknę się niepostrzeżenie.
Uchyliłem delikatnie drzwi, lecz na korytarzu nikogo nie było.

Moje eleganckie trzewiki hałasowały, jak na występie steperów. 
Podbiegałem więc za każdym uderzeniem o kilka kroków, korzystając z zagłuszenia moich butów.
Jeśli schodziliśmy w dół, a nie widziałem po drodze żadnego wyjścia, to znaczy że musiało być na najniższym piętrze.
Udało mi się dojść do schodów w dół, ostrożnie wychyliłem głowę zza sufitu niższego piętra i zobaczyłem właz w ścianie kuli.
Tego szukałem.

Hałas wyraźnie dochodził zza nich. Ktoś próbował je wyważyć.
Byłem pewien, że są to moi koledzy, którzy przybyli mnie odbić.
Któż inny mógłby śmieć zaatakować latającą kulę?

Złapałem korbę do otwierania włazu i począłem kręcić jakby od tego zależało moje życie.
W tym samym czasie usłyszałem za sobą kroki zbiegania po schodach.
\begin{dialogue}
\ds{} Puść tą korbę! \dm{} Ten młodszy przybiegł.
\ds{} Wypchaj się, wrócili po mnie \dm{} odpowiedziałem, odwracając głowę.
\ds{} Nikt po ciebie nie wrócił, jesteśmy pośrodku... \dm{} nagle zamilkł i otworzył szeroko oczy, jakby właśnie zobaczył ducha.
\end{dialogue}
Ostrożnie się odwróciłem, podejrzewając, że to wcale nie moja grupa przyszła mi z odsieczą.
To, co zobaczyłem nie przerażało.

W otwartym na oścież włazie, na tle rozgwieżdżonego nieba, lewitował dziadek w wannie.
Patrzył się na nas tajemniczo i się uśmiechał. Na głowie miał czepek kąpielowy, w ręce trzymał słuchawkę prysznica, wszędzie były góry piany.
\begin{dialogue}
\ds{} Witam panów. Czy nie macie może pożyczyć trochę szamponu? Lecę już tak milion lat i wciąż nie mogę dokończyć kąpieli.
\end{dialogue}
Pokręciłem lekko głową.
\begin{dialogue}
\ds{} Nie szkodzi \dm{} zaśmiał się. \dm{} Użyję więc ciebie.
\end{dialogue}
Zamachnął się słuchawką, jak lassem, rzucił do środka i owinął ją wokół mojej nogi.
Począł ciągnąć z nadludzką siłą, przewracając mnie. Złapałem się korby w ostatnim momencie.
Owinięta końcówka prysznica była jak macka, próbowałem ją strząsnąć, lecz zahaczyła się o sznurówki butów.
Szarpnął mocniej i obrotowa rękojeść korby zaraz wyślizgnęła mi się z objęcia, poleciałem dalej w kierunku drzwi.
Czepiając się palcami puszystego dywanu zobaczyłem, jak tamten nadal stoi, jak zahipnotyzowany.
Włosia dywanu były za słabe, żeby mnie utrzymać. Złapałem się ostatniej rzeczy, framugi drzwi.
Wtedy też poczułem połową mojego ciała zimną pustkę kosmosu.
Ktoś delikatnie objął mnie dłonią za kostkę, szarpnął i znalazłem się w ciepłej wodzie.

Wanna lekko się zachybotała po moim wejściu.
Zobaczyłem obok siebie wyszczerzoną twarz staruszka.
Adrenalina nie pozwoliła mi poczuć, że duszę się w próżni kosmicznej.
Złapałem się boku wanny, aby wyskoczyć i moja ręka ześlizgnęła się, jakby była z mydła.
Popatrzyłem na swoje dłonie, które stawały się coraz bardziej mydlane.
Dziadek przejechał gąbką po mojej twarzy, poczułem jak zabiera mi cały policzek.

Nagle różowy promień wystrzelił ze środka kuli.
Znalazłem się w nurcie rwącej rzeki, która ciągnęła wszystko z powrotem.
\begin{dialogue}
\ds{} Ojojojoj! Nieszczęście \dm{} zawołał dziadziuś.
\end{dialogue}
Zobaczyłem Nadara z wyciągniętym pistoletem, wpadłem do lufy i uderzyłem we wklęsłą ścianę, rozpłaszczając się.
Wielka twarz Nadara obserwowała mnie czujnym wzrokiem.
Zaraz od tyłu dobiła mnie lecąca wanna, a potem wpadający do niej dziadek.
Woda przyszła ostatnia, zalewając wszystko.
Czułem, że umarłem.

Po schodach zszedł Profesor Kula.
\begin{dialogue}
\ds{} Panie Mateuszu, specjalnie dla pana wyszliśmy na chwilę z podróży górną warstwą, z powrotem do czasoprzestrzeni. \dm{} Nawet nie zauważył, co się tutaj przed chwilą stało.
\dm{} Znajdujemy się pomiędzy galaktykami,  w wielkiej pustce kosmosu. Najbliższa materia znajduje się trzy miliony lat świetlnych od nas. Proszę spojrzeć, 
o tam, to Droga Mleczna. Tak wygląda z zewnątrz, niczym biała spirala... \dm{} Rozejrzał się. Popatrzył na Mateusza, Nadara i na mnie, zamkniętego w szklanej bańce.
\end{dialogue}
Nie mogłem się odwrócić, aby zobaczyć cokolwiek, nic nie mogłem. Byłem zupą.

% 
% \divider{}
% 
% Winkla opuściła swój miecz i schowała. 
% Nikt nic nie mówił, nikt nie reagował. 
% Antyrax poczuł się obserwowany ze wszystkich ciemnych zakamarków zaułka.
% 
% \begin{dialogue}
% \ds{} Czy wiecie, po co Bóg stworzył wszechświat? Po co istocie wszechmogącej jakikolwiek niewszechmogący byt? \dm{} zapytał.
% \end{dialogue}
% Odpowiedziała mu cisza.
% \begin{dialogue}
% \ds{} Ja też nie wiem. Ale podejrzewam, że z tego samego powodu, dla którego ludzie zakładają sobie ogródki i akwaria.
% Chcą widzieć, jak się rozwijają, jak żyją własnym życiem, jak ich akcje wpływają na rzeczywistość.
% \ds{} Wszechświat jest ogródkiem Boga? \dm{} Everywhere Man nie dowierzał. 
% \ds{} Dokładnie, a co musi właściciel robić, żeby ogródek nie umarł?
% \ds{} Pielić chwasty, nawozić, podlewać \dm{} Winkla wspomniała.
% \ds{} Albo może wynająć do tego służbę, kogoś kto będzie dbał o ogród za niego. Służbie da się narzędzia i pozwoli na dostęp do każdego zakamarku ogrodu.
% \ds{} Potwory, tak? Potwory policją, albo raczej ogrodnikami wszechświata \dm{} Pan Wszystko złapał pomysł. \dm{} Ale jak to wyjaśnia, czym jest uniwersalność? 
% Dlaczego dziadzio w wannie?
% \ds{} Sami ogrodnicy nie pociągną ogrodu. Ktoś musi nimi kierować, musi decydować gdzie i jakie kwiaty posadzić. Musi mieć wizję.
% \ds{} Więc Bóg mówi potworom, jakie chwasty wyrwać i gdzie co posadzić, tak? \dm{} Winkla rozejrzała się po zdruzgotamym miasteczku, zerwała połamanego tulipana.
% \ds{} No co ty. Jeśli byłby Bogiem, to wynająłby jeszcze kogoś z dobrym gustem, aby decydował za niego, prawda? \dm{} Wszystko miał częściową rację.
% \ds{} Oraz kazałby stworzyć drugi ogród, aby sprawdzić, czy jego decyzje, które zamierza podjąć, będą rzeczywiście dobre.
% \ds{} Może stworzyć nieskończenie wiele ogrodów, nieskończenie wiele zespołów ogrodników i nieskończenie wiele artystów do decydowania o tym jednym tylko głównym ogrodzie.
% Wtedy tylko ogród będzie doskonały. Doskonały i pasywny. Rośnie sam i sam się rządzi.
% \ds{} Nie rozumiem.
% \ds{} Skąd wziąć nieskończoną ilosć kwiatów i nieskończoną ilość ogrodników i ogrodów? Potrzebny jest Ogród Uniwersalny. Byt będący wszystkim jednocześnie, duplikujący się 
% \end{dialogue}










% 
% Dlaczego ludzie zapominają... to dobre pytanie.
% Otóż Kula pokryta jest amnezją. Każdy, kto na nią spojrzy, nawet pośrednio --- zapomina.
% Pamiętają tylko ci, którzy wierzą. Wierzą w Profesora Kulę, wierzą w bankiety w niebiosach, wierzą w złocone wnętrze.
% Na pewno nie będzie to dla pana zaskoczeniem, że większość uważa Kulę zwykle za balon meteorologiczny, fatamorganę, dowcip, sztuczkę magiczną, nowoczesny samolot wojska, itp.
% Pan uwierzył, dlatego pan tutaj jest.






%TODO Przysłowie "Gabinet profesora kuli"
