\chapter{Sam przeciw wszystkim}

\info{Wioska jest atakowana przez literackie demony. Tylko dzielny, acz niedoświadczony wojownik jest wstanie je pokonać, używając do tego
swojego miecza kreatywności i worka z alternatywnym wszechświatem.}

Gdy tylko czubek głowy Wielkiego Neofantasora wyłonił się zza gór, odpalono armaty załadowane gorącym atramentem.
Katapulty wystrzeliły litery, a papierowe samoloty wzniosły się w powietrze.
Wszystko na darmo, wielkolud ani drgnął.

Wojownicy z wioski, uzbrojeni w pióra i kałamarze, ruszyli do boju.
Atakowali wroga, używając całego swojego doświadczenia. Opowiadania cyberpunkowe, fantasy i fantastyki naukowej cięły powietrze.
Horrory, wiersze i dzieła detektywistyczne rozbijały się o jego ciało.
Jednak niewzruszony Neofantasor pozostawał niewzruszony.
Stanął przed wioską i wysypał z rękawa, niczym wytrzepując piasek z buta po całodniowym siedzeniu na plaży, dziesiątkę straszliwych demonów.
Demony szybko i sprawnie rozprawiły się z całą obroną wioski, rzucając po jednej gwiazdce, w każdego z wojaków.
Nikt z obrońców nie przeżył, wieś została bezbronna.

Ale jak to w baśniach zwykle bywa, znalazł się młody syn doktora --- Antyrax.
Trochę ułomny językowo i bez jakiegokolwiek doświadczenia, postanowił dzisiaj zginąć w walce.
Nie miał, jak wszyscy, zbroi, pióra, czy zapasu atramentu.
Posiadał jedynie szklany miecz, wypełniony płynną kreatywnością, dmuchawkę z serum śmiechu, lateksowe buty i worek z alternatywnym wszechświatem.

Najbliższa demonka, Obudzona, została z zaskoczenia kopnięta z lateksowego buta prosto w pupę, i nawet nie poczuła.
Dopiero serum śmiechu, wstrzyknięte prosto w szyję, zwróciło jej uwagę.
Wtedy Antyrax sięgnął do swojego worka i wyciągnął... rzecz.
Obudzona popatrzyła na zagubionego wojownika, to na przedmiot, który trzymał, po znowu na wojownika. Zamrugała, nie rozpoznając zupełnie, co to jest i czy się tego bać.
Ostatni pisarz też obracał w dłoni i oglądał rzecz ze wszystkich stron, ale i on nie wiedział, co właściwie właśnie wyciągnął.
Wyrzucił za siebie i spróbował ponownie.

Po kilku minutach wyrzucania różnych, nieokreślonych obiektów z worka, spostrzegł że grupka demonów go otoczyła i z zaciekawieniem obserwuje jego zmagania z samym sobą.
Nie umiał używać własnego wszechświata. Co za matoł.
Postanowił więc wykonać swoją powinność w tradycyjny sposób, zamachnął się i wbił swój miecz w najbliższego demona, który odskoczył z bólu. Stare, sprawdzone metody zawsze działają.
Nie długo cieszył się z sukcesu, zaraz wszystkie straszydła razem wzięły i dmuchnęły w niego strumieniem przecinków --- jego największą słabością.

Pierwszy lecący znak interpunkcyjny ominął, podskakując na lateksowych butach, drugi skontrował mieczem, lecz tysiąc kolejnych odrzuciło go na kilkaset metrów w tył.
Upadł obolały w błoto, po drugiej stronie wioski nad którą chyba właśnie przeleciał. Jakimś cudem nic sobie nie połamał.
Lecz to nie był koniec kontrataku, lecący błąd ortograficzny prawie rozciął go na pół, Antyrax zdążył odskoczyć, lecz puścił miecz, który pękł na drobne kawałeczki, zmiażdżony przez ,,Ó''.
Kreatywność wsiąkła w błoto, tworząc nowe królestwo błotnych mrówko-androidów, zasilanych parą z wody basenowej.

Antyrax był teraz bezbronny, a bezbronnym człowiekiem żaden z demonów się już więcej nie interesował. Wstał i zobaczył, jak jego wioska właśnie jest równana z ziemią.
Cień Wielkiego Neofantasora, wychylającego się zza góry, wyglądał jak dłoń dziecka, które zagarnia klocki z ziemi, aby je zaraz obślinić i połknąć.
Czy to był już koniec dla niego i dla wszystkiego co znał?

\ds{} Nie, to nie koniec \dm{} pomyślał, sięgając po worek. \dm Jestem pisarzem i to ja ustalam zakończenia. \de{}

Wtedy jego lateksowym butom wyrosły śmigła, wspomagane skryptem w Bashu, i poniosły go prosto pod nos Obudzonej, zostawiając za sobą linuksowy ślad.
Zaciekawiona demonka rozpoznała niedojdę, którego wcześniej zmiażdżyła.
Antyrax nie miał już żadnej broni, tylko worek. Sięgnął do niego i długo nie wyciągał ręki.

\ds{} Mam coś specjalnie dla ciebie, Obudzono. \de{}

\divider{}

\ds{} Kolejny dzień w pracy, kolejny dzień robienia bezużytecznych czynności dla bezużytecznej korporacji w tym bezużytecznym świecie \dm{} pomyślałem. \de{}

Wziąłem komórkę i zacząłem przeglądać maile.
Spam z reklamą talerzy, zaproszenie do znajomych na Facebooku, oczywiście od obcej osoby, powiadomienie o komentarzu na YouTube,
powiadomienie o mailu na innej skrzynce pocztowej, przypomnienie o opłacie za subskrypcję tej bezużytecznej gry komputerowej i tysiąc innych bezużyteczności.
O, znowu rozpętałem gównoburzę na Mirko i zablokowali mi konto Twittera za niepoprawną politycznie myślozbrodnię.
Bezużyteczność do kwadratu.

Z nadmiaru bezużyteczności zdrzemnąłem się w ubraniu na godzinę. Obudziło mnie głośne gruchanie z okna.
Czyżby jakaś użyteczność w końcu mnie spotkała?
Na oknie siedział biały gołąbek, w dzióbku trzymał kopertę.
Nie będąc pewnym, czy nadal nie śnię, odebrałem przesyłkę i przyjrzałem się kopercie.
Gołąb zaraz zniknął, oddalając się bezszelestnie.

Koperta była z papieru czerpanego.
Adresowana była elegancką kursywą do mnie, do Mateusza Mechalycznego, zamieszkałego na ulicy Szerokiej w Gdańsku.
Województwo Pomorskie, Polska, Ziemia, Układ Słoneczny, Galaktyka Droga Mleczna, Gromada Lokalna, Czwarta Kwadra.
Z tyłu widniała pieczęć z wosku pszczelego.
Wypukły obraz kuli i jakiś napis z niezrozumiałych znaków.
Przecież był 2017 rok, kto dzisiaj jeszcze wysyła papierowe listy zamiast maili?
To był kawał. To musiał być kawał. Pytanie jeszcze, po co ktoś mi go wyciął?

Ostrożnie otworzyłem kopertę, trzymając ją obcęgami, spodziewając się że coś strasznego zaraz z niej na mnie wyskoczy.
Nic takiego się jednak nie stało.
Pisany odręcznie list był tym, co zwykle znajduje się w kopertach.

\curlyframe{
\begin{Fontlukas}
Szanowny Panie Mateuszu Mechalyczny.

Mam zaszczyt zaprosić Pana na uroczysty bankiet z okazji wyboru na jednego z przyszłych członków ALOPP.
To wielki zaszczyt, móc gościć nowych agentów tej organizacji na pokładzie mojej \weirdchar{kula}.
Mam najskrytszą nadzieję, że zostanie Pan przyjęty i będzie mieć Pan we współpracy z \weirdchar{monster} udział w walce o wspólne dobro.

Zapraszam do siebie dnia 20 października, roku Pańskiego 2017.
Myślę, że marina w Głównym Mieście Gdańska będzie doskonałym miejscem na lądowanie mojej kuli i tam się spotkajmy w lokalne południe.
Po obiedzie wybierzemy się w podróż na Felicję, gdzie pozna Pan swoich przyszłych, mam nadzieję, członków przybranej rodziny.

Przypominam, że w \weirdchar{kula} oprócz najwyższej kultury osobistej,
od zawsze obowiązywał elegancji ubiór francuski.
Wszyscy goście powinni przywiązać najwyższą dbałość o szczegóły swojego wyglądu.
Uprzejmie proszę także, aby nie posiadać na pokładzie żadnych urządzeń użytkujących elektryczność.

Z Bogiem. \\
--- Profesor \weirdchar{profesor}
\end{Fontlukas}}

Kilka dziwnych znaków zostało wtrąconych pomiędzy litery. Zaczynało się robić ciekawie. Autor tego dowcipu chciał, abym za trzy dni, w XVIII wiecznym stroju pałacowym,
udał się w sam środek miasta w dniach szczytu, nie zabierając ze sobą żadnej elektroniki.
Potem tajemniczo miałem przejść tajemniczy test na zostanie tajemniczym członkiem jakiejś tajemniczej organizacji.
Trzeba dokładniej przestudiować ten tajemniczy list.

Szybkie szukanie Felicji w internecie wskazało jedną stronę o teoriach spiskowych.
Felicja miała być planetką, stworzoną przez kosmitów, na której hodowano ludzi, aby przeprowadzać na nich straszliwe eksperymenty.
Jeśli to prawda, może zabrakło im tam królików doświadczalnych i porywają kolejne ofiary?
Ale wtedy przecież nie dawali by mi wolnej ręki do odmowy.

Na tej samej stronie podano: ALOPP jest pozaziemską organizacją terrorystyczną, zrzeszającą ludzi w celu mordowania mieszkańców własnej planety.
Ale od czego był to skrót, to nikt nie wiedział.

,,Czwarta Kwadra'' dawała za dużo losowych wyników, aby wywnioskować z nich, o co mogło Profesorowi chodzić.

Lokalne południe w Gdańsku, czyli dwunasta godzina czasu słonecznego, uwzględniając jeszcze czas letni, to trochę przed trzynastą czasu strefowego.
Przyjdę o 12:00, najwyżej trochę poczekam.

Wikipedia natomiast wskazała, że gołębie pocztowe w żadnym wypadku nie mogłyby doręczyć listu bezpośrednio do odbiorcy.
Ich mechanika polega na wracaniu do macierzystego gołębnika, z dowolnego miejsca na świecie, i tylko tyle.
Listów z pewnością nie wsadzano im do dzióbka, a przywiązywało się je do nóżek.
Rozejrzałem się po pokoju, czy przypadkiem nie miałem w nim gołębnika, aby hodować gołębie pocztowe, ale nie.
Moja teoria o dowcipie zaczęła się lekko sypać.

Kilka razy, w różnych częściach świata, widziano w tym samym momencie kuliste UFO i ludzi w strojach rodem z Wersalu.
Podobno zdjęcie zrobione kuli nigdy nie wychodziło poprawnie, a większość ludzi magicznie zapominała o zdarzeniu chwilę po odlocie tajemniczej struktury.
Nieliczni pamiętali i rozpowiadali to dziwo, ale nikt im oczywiście później nie wierzył.
Kulę widywano w różnych miejscach, nie ograniczała się, jak na filmach, tylko do USA, przelatywała przez centra miast, pływała pod wodą, cumowała do stacji kosmicznych, straszyła samoloty, ślizgała się po biegunowych lodach i toczyła bitwy z wojskami wszystkich krajów świata.

Następnego dnia zabrałem list do znajomego chemika.
Potwierdził on moje obawy, list wykonany był oryginalną techniką sprzed kilkuset lat.
Skład chemiczny papieru i atramentu, odpowiadał tym, używanym dawno temu.
W dodatku narzędzie pisania z pewnością było ptasim piórem.
Na myśl o bezużyteczności otaczającego mnie świata, postanowiłem pojutrze zrobić coś użytecznego.

\divider{}

Pod naporem nietypowości i kreatywności dzieła, Obudzona zaczęła krzyczeć, zwijać się w konwulsjach i palić czarnym ogniem.
Zmieniła się w mały księżyc i poleciała, jak frisbee, z powrotem w kierunku Wielkiego Neofantasora.

Triumf Antyraxa nie trwał długo, za chwilę, od tyłu, złapała go Dedirid.
Jej czarna ręka owinęła się wokół delikatnego światostwórcy, jak czarny worek na zwłoki.
Poczęła go zacieśniać jak lekarz mierzący ciśnienie.
Zaraz wyciśnie naszego bohatera, jak tubkę pasty do zębów.
Wywijając się rybio, Antyrax zanurkował do worka i długo nie wychodził.
Demonka przez godziny nachylała się nad otworem, aby capnąć go jak tylko wystawi głowę.
Gdy tylko coś się wysunęło, porwała to z ochotą.
Była to jednak kolejna opowieść, zabójczo eksperymentalna, niesamowicie abstrakcyjna.
Nietypowość poparzyła jej łapska.

\divider{}

Mateusz wypożyczył wymaganą górę z wypożyczalni kostiumów.
Zastanawiał się, czy nie podkraść jakiegoś szustokora z muzeum, ale to chyba nie było by zbyt poprawne zachowanie.
Bał się, czy mierna jakość kaftana, spowodowana nieoryginalnością, nie będzie zwracać nadmiernej uwagi w świecie najprawdziwszych atłasowych pasów i perłowych guzików.
Postanowił kupić więc kilka ozdób ze sztucznej biżuterii, które wyglądały dość kosztownie, a stworzone były z
byle-czego, i doszyć w losowych miejscach. Miał nadzieję, że Profesor i inni goście nie zauważą różnicy.

Z pończochami nie było żadnego problemu, znalazł je w damskim sklepie.
Tak samo coś, co można było podciągnąć pod starodawną koszulę.
Musiała być flanelowa z wystającymi rękawami.
Ogarnął także puder.

Peruka, cóż. Przynajmniej znalazł za szafą trójkątną czapkę piracką po poprzednich lokatorach.
Najgorzej, że zazwyczaj chodził na łyso, gdyż rodzice nie obdarzyli go mocnymi włosami.
Potrzebował więc na szybko przykleić coś sobie na łeb.
Liczył w głowie, ile lat może dostać za kradzież peruki sędziemu, gdy spostrzegł wyprzedaż starych futer.
Używając magii nożyczek, kleju i moli wytrzepanych z płaszcza, wygenerował coś, co po przykryciu trójkątną piracką czapką wyglądało dość znośnie.

U zegarmistrza kupił za grosze kopertę zegarka, z brakującymi wskazówkami, całość zaczepioną na łańcuszku.
Zegarek nie musiał działać, ważne aby był.

O dziwo, to buty przysporzyły mu najwięcej problemu.
Niby lakierki z klamrą nie są niczym bardzo skomplikowanym, a jednak nikt ich nie produkuje.
Może właśnie dlatego, że były modne trzysta lat temu?
Wpadł na pomysł, aby kupić coś podobnego i przerobić.
Znalazł buty dla zakładów pogrzebowych, gdyż tylko te odpowiednio się błyszczały, i przyszył im klamry od spodni.
Z daleka nie było widać różnicy.

W domu ubrał się i przejrzał w lustrze.
Połączenie Napoleona, Ludwika XIV i informatyka z Gdańska.
Muszą zrozumieć.

O jedenastej godzinie, owego wielkiego dnia, ubrał się w pełny strój.
Nie mógł się przecież tak pokazać w mieście.
Pończochy zatem przykrył spodniami dresowymi.
Na elegancki szustokor nałożył nieco za dużą bluzę z kapturem.
Lustrzane lakierki przykrył jakimiś szmatami, żeby nie zwracały zbytniej uwagi.
Tylko pseudo-perukę schował do plecaka.

To nie mogło pójść tak łatwo.
Z daleka zobaczył kordon policji i wojska, stojący w Zielonej Bramie, blokował wstęp każdemu wychodzącemu z Długiego Targu.
Ucieszył się i kamień spadł mu z serca. Oznaczało to, że jednak nie padł ofiarą żartu.
Znalazł w końcu promyk użyteczności w oceanie bezużyteczności. Każda normalna osoba, wiedząc że wielka latająca kula-zapominajka wylądowała w centrum miasta,
ewakuował by się z niego jak najdalej. 
Mateusz jednak nie był normalny, i może dlatego właśnie został zaproszony na najbardziej nienormalną ucztę w świecie.

Do mariny spróbował dostać się okrężną drogą, przebiegł przez Krowi Most na Wyspę Spichrzów.
Klucząc uliczkami zbliżył się do portu, jednak tutaj też była blokada.
Widział w każdym bądź razie kawałek wody w basenie jachtowym, nietypowe fale odbijały się od brzegów, coś się tam jednak działo.
Popatrzył smutno w kanał i pomyślał, że chyba zostanie mu wskoczyć do wody i popłynąć wpław, pod mostem omijając strażników.
Ale przecież nie zostałby wpuszczony mokry do rakiety.
No i co z pudrem, który już sobie wcześniej pieczołowicie nałożył?

Szustokor był bardzo gruby, rozpiął więc bluzę żeby się nie ugotować, teraz wszystko było mu jedno, czy ktoś go zauważy.
Był tak blisko, a jednocześnie tak daleko. Wszystko miało prysnąć, jak bańka.
Czuł się jak rybka w siatce, wrzucona do oceanu.
Zaraz będzie widział swoją życiową porażkę, jak na dłoni.
Co robić? Co robić?

Wybawienie przyszło nieoczekiwanie.
Oto bowiem mama z małą dziewczynką podpłynęły do niego skuterem wodnym, oferując szybką podwózkę.
Myśląc, że to pomyłka, zdjął bluzę, pokazując swój strój w połowie okazałości.
Kobieta jednak nie uciekła, nie przestraszyła się dziwaka, tylko się uśmiechnęła.

\ds{} Profesor Kula nie lubi spóźnialskich. Pospiesz się \dm{} powiedziała. \de{}

\ds{} Skąd... kim... \de{}

\ds{} Świat jest wielki, a zasięg potworów... znaczy tych właśnie... to jest taka jakby policja wszechświata... jest większy.
Jesteś nowy, wnioskuję że zostałeś zaproszony na test. Pospiesz się.\de{}

\ds{} Test. Ale co mam robić, żeby go zdać? \de{}

\ds{} Być sobą. \de{}

\ds{} Jak mam ci się... \de{}

\ds{} Uratuj w przyszłości też komuś życie. \dm{} Pogłaskała swoją córeczkę po głowie. \de{}

Mateusz w XVIII wiecznym stroju wersalskim zasuwał na skuterze wodnym kanałem Nowej Motławy, ozdoby szustokora mieniły się w pełnym słońcu tak samo, jak latające wokół niego
krople wody i stalowe kule, wystrzelone z mostu przez żołnierzy zabezpieczających
lądowanie wielkiej białej kuli w centrum miasta.
Schował się na chwilę pod mostem, jak zagoniony przez wilki królik w norze, a gdy wypłynął z drugiej strony, wtedy ją zobaczył.

Była wielka, jak budynek, wypolerowana, biała i doskonale kulista.
Dołem dotykała lekko powierzchni wody, tworząc promieniste fale.
Odbijała w sobie cały Gdańsk.
Mateusz zobaczył w niej malutkiego siebie na łódeczce-zabawce, malutkie budynki, żołnierzyków, spichrze, basenik, niebo, helikopter jak muchę i blask pełnego słońca.
Już nikt nie strzelał, już tylko wszyscy patrzyli. I bali się.
On się nie bał. Przyszedł tu na bankiet.
Przyszedł we francuskim stroju.
Przyszedł tu, bo został zaproszony.

Zszedł ze skutera na pomost i poprawił perukę, wtedy też właz w dolnej części zaczął się otwierać.
Tak jak się spodziewał, był to dźwięk szczęku łańcuchów, a nie elektrycznego silnika.
Ze środka powiał zapach kurzu, wosku i lekkiej stęchlizny.
U dołu rozwinął się elegancki, czerwony dywan.
W przejściu stanął On.
Nosił strój wspanialszy, niż Mateusz mógł sobie kiedykolwiek wyobrazić, tak inny od jego własnego, a przecież z tego samego okresu historycznego.
Przy jego ozdobach, sztuczna biżuteria Mateusza rzeczywiście wyglądała na sztuczną.
Jego najprawdziwsza peruka przyćmiła wielkością cały futrzany twór z głowy gościa.
Lakierki błyszczały się tak samo, jak jego statek kosmiczny z którego wyszedł.
W ręku trzymał laskę z białą kulką, pomniejszoną wersją tego, co znajdowało się tuż za nim.

\ds{} Jestem Profesor Kula. Miło mi pana gościć na moim statku. \de{}

\divider{}

Antyrax wyszedł z worka, gdy z Dedirid została już tylko kupka popiołu.
Jeszcze ośmiu.
Tym razem demony nie bardzo chciały go atakować.
Antyrax więc wskazał jednego z nich palcem, niczym sędzia nowoskazanego na śmierć.
Lenna. Pora na mikropomysły.
Sięgnął do worka i od razu złapał to, czego szukał.

\divider{}

\ds{} Waćpan Mateusz Mechalyczny, jak mniemam \dm{} powitałem gościa. \dm{} Waćpan Mateusz Mechalyczny niepewnie, acz żwawo podszedł, ukłonił się, i schował za framugą włazu
znikając przed przeszywającym wzrokiem miasta. \de{}

\ds{} Proszę wybaczyć mi mój ubiór i maniery, Panie Profesorze Kula. \dm{} Ukłonił się ponownie, prawie do ziemi. \dm{}
Musiałem przedrzeć się przez kordon wojska i ominąć grad pocisków, aby przybyć do pańskiego statku. \de{}

\ds{} Nazywam się Kula, mości Mateuszu Mechalyczny, nie Kula \dm{} poprawiłem, zamykając korbą właz. \dm{} A ta kula nazywa się Kula. 
I nie jest jakimś statkiem kosmicznym, a Kulą. \de{}

\ds{} Kula... \dm{} niepewnie odpowiedział. \de{}

\ds{} Nie Kula, Kula. Moje nazwisko, nazwa tego miejsca, typ urządzenia, i bryła geometryczna. Kula, Kula, Kula i kula. To trzy różne słowa, zupełnie inaczej wymawiane.
Zupełnie inaczej zapisywane. \de{}

Mateusz Mechalyczny podrapał się po głowie, ścierając puder.

\ds{} Ignoruj go, on mówi i słyszy na częstotliwościach poza naszym zakresem. \dm{} Katarzyna Kosmata zjechała po schodach i przysunęła do nas, nawet się nie witając.
\dm{} Jestem Kasia, cześć. \de{}

\ds{} Droga Katarzyno Kosmata! \dm{} skarciłem ją. \dm{} Maniery! Niech panna nie prezentuje złego przykładu naszemu gościowi. Panie Mateuszu, mam zaszczyt przedstawić panu...\de{}

Gość jednak utopił wzrok w olbrzymiej sukni Katarzyny, nachalnie gapiąc się na każdy jej detal.
A już się radowałem, że chociaż on będzie potrafił tu zachować maniery. Nadaremnie.
Najgorsze w tej sytuacji było to, że ona sama wręcz go do tego zachęcała. 
Zamiast się przedstawić, zaczęła tłumaczyć skąd, jaka część ubioru pochodzi.

Najpierw zawiesił oczy na jej biuście, wodząc źrenicami to w lewo to w prawo.
Najprawdopodobniej podziwiał plecionkę z anielskich włosów, którą obszyta była góra.
Anielskie włosy są całkowicie przezroczyste, gdy odpadną od właściciela, więc
aby stworzyć ten element ubioru, trzeba było prawdopodobnie zbierać je z niebiańskich podłóg w całym raju.

\ds{} Trafiłam przez to na dywanik anielskiego ministra poprawnego zachowania, chciał to podciągnąć pod brak szacunku dla zarządu Nieba, ale wybroniłam się tym że wszyta świętość
będzie dodatkowo ochraniać mnie przed demonami, czy jakoś tak. \de{}

Następnie zszedł niżej, aby przyjrzeć się lepiej pasu.
Katarzyna gustowała się w nieprawdopodobnie kosztownych ubiorach, lecz jej pas był wykonany ze zwyczajnego, ziemskiego jedwabiu.
Może chciała tym pokazać, jakoby jej suknia była w równym stopniu wykonana ze składników z calutkiego wszechświata?

\ds{} Ten jedwab pochodzi od jedwabników karmionych nektarem jedynie z najrzadszych gatunków orchidei, podlewanych krystaliczną wodą źródlaną z Himalajów. 
\dm{} Powiedziała, wyjaśniając cały sekret. \de{}

Po pasie, przyszedł czas na szyję. Katarzyna założyła tym razem kolię ze zmutowanych pereł Khaliniskali...
czy to była Rezurma? Nie pamiętam, kto ostatnio przejmował stolicę tej przeklętej... Planety Wojny, jak ją wszyscy nazywają.
Kulki mieniły się i pulsowały wszystkimi kolorami tęczy. Od podczerwieni, po nadfiolet.
Te perły można było znaleźć tylko w małżach żyjących w skażonym jeziorze, na północy pustynnego kontynentu Terb. Chwila, teraz to nie był już Terb... no na północy tego największego kontynentu planety.
Zdaje się że to albo Czarna Armia, albo Komodowa utopiła tam kiedyś zbiorniki z kancerogennym żelem, w celu wewnętrznego wyniszczenia przybrzeżnego miasta Hirten... wtedy to było Hirten.
Nie udało się, mieszkańcy Hirten wyczuli podstęp i zamiast umrzeć na nowotwory od picia skażonej wody, poumierali z pragnienia.
W każdym razie flora i fauna w jeziorze przeszła nieprzyjemne zmiany fizyczne.

Buty. Był to wspólny wytwór czterech Khrnzrhkh.
Najpierw poprosiła Chrrkrhkrrkk o stworzenie lodowej podstawy.
Potem pewnie Iłiścirr obudował to swoją czarną rkkizniisi, Buffsirr dodał czerwone wkładki ze swoich buffzerda, a Fluszszrisss utwardził swoim ogniem.

\ds{} Te buty zostały zrobione przez potworów, najpierw Mikołaj stworzył lodową podstawę, Psychit zalał ektoplazmą, Pyrroq dodał czerwone klejnoty-bomby, a Plazma utwardził ogniem. \dm{}
Katarzyna Kosmata właśnie spowodowała, że kolejna osoba będzie nazywać Khrnzaalk potworami, zamiast porządnie w ich własnym języku. \de{}

Zahaczył o wachlarz.
Ten był stworzony z półprzezroczystych łusek białego cyrkowca.
Te smoki wyginęły doszczętnie po ataku czerwonych kartaczy na ich planetę.
Właściwie jedyne pozostałe cyrkowce można teraz znaleźć w zoo w Capitalu.
Szkoda ich, miały wspaniałą kulturę.
Cyrkowe baśnie do dziś opowiada się małym smoczkom do legowiska, a cyrkowi malarze są niedoścignionym przykładem talentu w wielu kulturach.
Kartacze to zwierzęta. Żeby chociaż te barbarzyńskie pasowce ich rozbiły, ale nie. Największy i najbardziej bezmózgi gatunek smoków zaatakował, rozszarpał i pożarł najwspanialszych.

Ostatecznie Mateusz popatrzył się na jej twarz.
Nie, nie na twarz, a na makijaż.
Oczywiście, Katarzyna Kosmata nie mogła spocząć na wyrywaniu łusek prawie wymarłym smokom.
Jej puder był stworzony ze zmielonych ciosów mamuta.
Jak ona odkopała je z syberyjskiego błota i wybieliła, tego nie wiem.

\ds{} Przekonałam Chronosa, kolejnego z potworów, żeby odwrócił trochę czas i przywrócił im świeżość. \de{}

Nikomu się nie udało przekonać kiedykolwiek Pfiishuss do jakiegokolwiek używania swojej mocy! Jak ona to zrobiła?

To jednak nie był koniec podziwiania, poszedł wzrokiem wyżej.
Fryzura Katarzyny była przeogromna. A wszystko to z naturalnych włosów, ich wzrost był stymulowany jakąś obrzydliwą, prądową technologią.
Można by na pewno osiągnąć ten sam efekt, używając naturalnych ziół, zamiast diabelskiej elektryczności.

Na jej włosach osiadły wielobarwne motyle. Co jakiś czas któryś wzbijał się w powietrze, robił pętlę wokół jej głowy i lądował z powrotem.
Były to najprawdziwsze motyle, hodowane i tresowane w tajnej placówce pod motylarną w Burggarten.
Ciekawe jak je zdobyła. Znając Katarzynę, pewnie jak gdyby nigdy nic przeszła przez tajne przejście, w tej pełnej sukni, z naładowanym szyfratorem w ręce, i powiedziała:
,,Dajcie mnie tych tresowanych motyli na głowę, bo zaraz mam bankiet we wielkiej, latającej kuli.''
Być może tylko po to w ogóle przyjechała dzisiaj do Wiednia.
Przyjechała po motyle, i po to ażeby przyprawić o zawał serca kilka osób.
Tym razem postanowiła wsiąść do Riesenrad i pojechać wagonikiem na sam szczyt, tam specjalnie umówiła się ze mną, abym podleciał po nią Kulą.
Oczywiście jak tylko przyleciałem, wybuchła panika. Koło się zatrzymało, uwięzieni w wagonikach ludzie próbowali uciekać po konstrukcji koła, 
przed wielką, białą kulą, cumującą właśnie do najwyższej budki. Katarzyna otwarła drzwiczki i robiąc krok nad przepaścią, weszła do pojazdu.
Pomachała wachlarzem pozostałym, ledwo żywym ze strachu osobom w budce, na do widzenia i odlecieliśmy.
Następnym razem pewnie stanie na szczycie Empire State Building, a ja będę robił za King Konga.
I też będę potem uciekał przez myśliwcami.
Czy można się uzależnić od amnezji, którą pokryty jest statek?
Uzależnić od siania paniki w ludziach, którzy i tak za chwilę o wszystkim zapomną?

Przeczyściłem gardło.

\ds{} Znaczy... witam... bardzo mi miło, dzień dobry... eee... Kasiu-ażyno. \dm{} Stał bez ruchu kilka pulsów, aż zdecydował się delikatnie ująć jej dłoń i pocałować.
Ważne, że się starał. \dm{} Ja Mateusz... jestem. \de{}

Katarzyna zarumieniła się. Widać zostało w niej jeszcze trochę kultury osobistej. A może to był makijaż?

Poprowadziłem ich do salonu, w którym nakryty był już stół dla czterech osób.

\ds{} Na naszym bankiecie spodziewamy się w sumie trzech gości \dm{} oznajmiłem zgromadzonym. \dm{} Zatem dołączy do nas jeszcze jedna osoba.
Będzie to Nadar Nocny, który aktualnie bada, albo szabruje, wrak Titanica. Podróż potrwa około dwa i pół kilopulsa. 
Gwoli wyjaśnienia, panie Mateuszu, puls to uniwersalna jednostka czasu, równa dwie trzecie ziemskiej sekundy.
Kilopuls to tysiąc pulsów, czyli do wraku Titanica dotrzemy za nieco ponad pół godziny. 
Pan Nocny jest dość... ekscentryczny. Dla jednych jest najlepszym przyjacielem, a inni go nienawidzą.
Szczerze powiedziawszy, nie popieram jego charakteru, ale obawiam się że może pan, panie Mateuszu, naleźć w nim bratnią duszę.\dm{}
Mateusz Mechalyczny uśmiechnął się lekko. \de{}

Położyłem rękę na lasce.
Zaczęliśmy się wtedy zanurzać coraz głębiej w Atlantyku, zostawiając za sobą pióropusz tęczowych rozbryzgów.

Tymczasem zacząłem oprowadzać naszego gościa po Kuli.
Wycieczkę rozpoczęliśmy, wracając do głównego włazu na najniższym piętrze.
Ta otwierana w dół, mająca od wewnątrz kształt schodów, wykrzywiona płyta, była jedyną, niepokrytą czerwonym futrem, częścią pancerza.
Zamiast tego posiadała czerwony dywan, automatycznie rozwijany przy kontakcie z podłożem.
Operowana za pomocą skomplikowanego systemu łańcuchowo-sprężynowego na korbę.

Nie zmieniając piętra, przeszliśmy do garderoby.
To właśnie tutaj trzymam awaryjne suknie, habity, koszule i trzewiki, w razie gdyby któremuś z gości zdarzyło się nie posiadać wystarczająco odświętnego ubioru.
Mateusz zwrócił mi uwagę na grube kombinezony, wiszące w kącie. Wedle jego wizji, były to stroje nurkowe.
Wyjaśniłem, że pomimo mylącego dla niektórych wyglądu, w rzeczywistości nadają się zarówno do nurkowania w oceanie, jak i w próżni kosmicznej.
Są integralną częścią Kuli, trochę jak ściany i meble. Czerpią z niej energię do podtrzymywania życia. Będąc w takim kombinezonie, nigdy nie zabraknie ci tlenu.
Na szczęście nie zauważył, iż jeden z haków był pusty. Jak ja bym mu wyjaśnił iż ja, Profesor Kula, zgubiłem kawałek wyposażenia swojej rakiety?

Zapytany o śluzę ciśnień, aby bezpiecznie wychodzić na zewnątrz, opowiedziałem mu o niewidocznej tarczy rozciągniętej na włazie, chroniła ona wnętrze przed różnorakimi hazardami zewnętrznymi, takimi jak próżnia, uniwersalność, demony, czy brak kultury osobistej.
Nie był przekonany, więc kręcąc jeszcze raz korbą, otworzyłem ponownie właz. 
Płynęliśmy aktualnie tuż przy samym dnie morskim, zostawiając za sobą chmurę wzburzonego piasku.
Falista, lekko wypukła powierzchnia wody, utworzyła się na głębokości framugi. 
Mateusz z niedowierzaniem zamoczył rękę w głębiach oceanu, wyciągając garść osadzającego się piasku.
I meduzę.

Na kolejnych piętrach znajdowały się pokoje gościnne. Gość uprzejmie podziękował za pokój, ale nalegał, abyśmy szli dalej.

W centralnej części Kuli znajdowała się łaźnia, muzeum i mój gabinet. Do tego ostatniego nikogo nie wpuszczam.

W wyłożonym terakotą pomieszczeniu, zdziwił się niemiłosiernie. Znalazł tam basen, jacuzzi, saunę fińską, masażery wodne, a także mały wodospad.
Pośrodku stał wielki piec na węgiel. Bez niego zimna pustka kosmosu szybko by nas dopadła.
Mateusz powiedział, że w XVIII wieku nie używano łaźni i że po stylu wnętrza spodziewał się co najwyżej wychodka w kącie. 
Zaśmiałem się na myśl, iż wziął Kulę za stuprocentowy wycinek pałacu w Wersalu.
Kultura idealna nie istnieje, z każdej należy wyciągnąć najlepsze części, zatem łącząc na przykład rzymskie starożytne łaźnie, francuski nowożytny wystrój i słowiańską mowę przyszłości, 
stworzyłem tą właśnie latającą wyspę kultury idealnej.

Pora była na najciekawszą część statku.
Moje muzeum zawiera artefakty z różnych zakątków wszechświata. Gość zapytał o wartość zebranych przedmiotów.
Nie wszystko da się sprowadzić do liczby pieniędzy, mości Mateuszu, nie wszystko ma tak zwaną cenę. 

To na przykład jest kawałek meteorytu, który uderzył w księżyc planety Tos. Wartość tego kamienia jest równoważna wartości losowego polnego kamienia z Ziemi, 
znajduje się tutaj ze względu na historię, jaką ze sobą niesie.
Otóż, uderzenie meteorytu było tak silne, że wybiło księżyc z orbity, popychając go w kierunku Tosa.
Po stu latach ciągłego zbliżania się do powierzchni, w końcu zahaczył o atmosferę i zderzył się z planetą.
Każdy organizm, większy od jednokomórkowca, został zabity.

Co ciekawe, mieszkańcy Tos byli na tyle rozwinięci naukowo, że doskonale widzieli i rozumieli zbliżającą się apokalipsę.
Jednak nadal za mało rozwinięci technologicznie, aby móc jej uniknąć.
Przewidzieli dzień swojego końca co do dnia, a koniec rzeczywiście nastąpił.

Tak, wiem że to smutne, ale cóż począć? Gorsze rzeczy zdarzały się w zbiorowej historii życia. 
Tylko pierwotne grzyby przetrwały katastrofę.
Toksyczna atmosfera, brak słońca i wysoka temperatura post-apokaliptycznego świata wręcz przyspieszyły ich ewolucję.
Na przykład tutaj masz dziób takiego latającego ptakochomora. To grzyb i ptak jednocześnie, ładnie świeci w ciemności.
Da się go spożywać, niestety nie jest bardzo wysublimowany w smaku.

Przeszliśmy dalej. Kamień z lodowej strony Kryonii, nic niezwykłego. 
No może poza tym, że musiał być wydobyty spod kilku kilometrów litego lądolodu.
Co w Kryonii jest takiego wspaniałego? Obraca się ona wokół swojego słońca jak Księżyc wokół Ziemi. 
Wiecznie zwrócona tą samą stroną.
Na Kryonii nie ma zatem dni oraz nocy, a gwiazda zawsze jest w tej samej części nieba.
Nocna część jest lodową pustynią, dzienna ma pośrodku wiecznie szalejące tornado.
Może kiedyś zobaczysz Pałac Nadiru, położony w centrum wiecznej zmarzliny, jest przepięknym dziełem sztuki lodowej.
Wielka iglica z kryształowych łuków, kopuł, balkonów i kolumn.
Podświetlona trytowym światłem na przeróżne kolory.
Freon, wielki lodowy król Kryonii rządzi swoim państwem dobrze i sprawiedliwie.
Szkoda, że część jego ludu tego całkowicie nie rozumie. 
Demokraci, komuniści, libertarianie, i reszta niepoliczalnych ruchów społecznych chce go dosłownie zwalić z tronu i pogrążyć cywilizację w chaosie.

\ds{} Skąd wiesz, że byłoby gorzej, niż jest teraz? \dm{} zapytał. \de{}

\ds{} Może kiedyś odwiedzisz Kryonię i wtedy, na własne oczy zobaczysz, że na pewno nie byłoby lepiej \dm{} odpowiedziałem. \dm{}
Ale prędzej czy później to i tak się pewnie stanie. Freon się starzeje i nie znalazł na swój tron godnego następcy. 
Więc albo rozkaże wybrać kogoś głosem ludu, albo znajdzie kogoś godnego spoza planety. To mogłoby doprowadzić do wojny domowej, 
rozumiesz, nikt nie chciałby być rządzony przez obcego kosmitę z kosmosu, nie ważne jak dobrze by rządził. \de{}

\ds{} Jak spoza planety? \de{}

\ds{} To jedna z tych cywilizacji, zwanych ,,zapoznanymi.'' Na tyle rozwinięta technologicznie i kulturalnie, przede wszystkim kulturalnie, 
że ma dostęp do warstw wszechświata. Warstwy to takie jakby obszary ,,pod'', ,,nad'' i ,,z boku'' czasoprzestrzeni.
Pozwalają na szybką i dowolną podróż w każde miejsce, do każdej galaktyki, do każdego układu, używając minimalnej ilości paliwa. \de{}

\ds{} Jak to? Czyli taka na przykład Kryonia może w każdej chwili zaatakować Ziemię? \dm{} Zląkł się. \dm{} I czy Ziemia także jest zapoznana? \de{}

\ds{} Powiedziałem, rozwinięta kulturalnie. Czy masz pewność, że ludzkość nie zaatakowałaby obcej planety, gdyby była położona w zasięgu Księżyca? 
No właśnie. Poza tym, są jeszcze Khrnzrhki. \de{}

\ds{} Kto? \de{}

\ds{} Potwory \dm{} westchnąłem. I ja też się w końcu poddałem. \dm{} Robią za policję wszechświata, dbają o pokój na wszystkich zapoznanych planetach i poza nimi. 
ALOPP, do którego waćpan został zaproszony, to skrót od Akademii Ludzkiej Otoczonej Protekcją Potworów. 
Jako agent Akademii, będziesz im waść pomagał, będziesz dbał o pokój we wszechświecie, zatrzymywał wojny, walczył ze złem w różnych postaciach. 
To niebezpieczna i bardzo ciężka praca, ale jakże ciekawa.
Raz będziesz uciekał na skuterze grawitacyjnym przed stadem czerwonych kartaczy, a innym razem zasiądziesz w sali obradowej Pałacu Nadiru. 
Znaczy, oczywiście jeśli okażesz się godny. \de{}

\ds{} Godny? \de{}

\ds{} Chodzi o charakter. Do potworów należy odpowiedzialność przed przyszłością. 
Każdemu agentowi może zdarzyć się stanąć przed wyborem decydującym o milionach istot. 
Dlatego kandydaci są poddawani testowi osobowości. Test ma kilka faz, sprawdza reakcję na zaistniałe sytuacje. \de{}

\ds{} Kilka faz? Jak mam je pozdawać? \dm{} Zaczął panikować. \de{}

\ds{} Spokojnie, pierwszą ma pan już za sobą. Przecież jest pan tutaj z nami. 
Pierwsza faza sprawdzała reakcję na abstrakcyjne sytuacje.
Można było wyrzucić ten list, można było zgłosić go władzom, można było pokazać w internecie, a można było potraktować go poważnie. \de{}

Kontynuowałem oprowadzanie.
Prawda, że ten kawałek zegarkowatych mechanizmów wygląda intrygująco? 
Proszę zgadnąć, co to jest. Podpowiem, to nie jest żaden zegar.
Ojej, niestety, nie zgadł pan.
Otóż jest to mózg reprezentanta pewnej wybitnie nieprzyjemnej nacji robotów.
I mówiąc roboty, nie mam na myśli prądowych ludzików, jak to się przyjęło w ziemskiej kulturze.
Te powstały z ludzkiego złomu jako sztuczne ciała dla głodnych demonów, które były za słabe, aby pożywiać się ciałami prawdziwych istot.
Ludzki złom, wszystko co ludziom było tak bliskie, niczym własne części ciała, ale nadal sztuczne i wymienne. Protezy, wózki, kule i podobne.
Te roboty więc składają się ze złomu, ale nie są zasilane energią elektryczną, a szatańską!
Powstały jako wcielenie zła, zasilanie parą wrzącego oleju z diabelskich kotłów, stworzone z ludzkich odrzutów, zlepione na ślinę i cyrograf.
Może pan się przyjrzeć, ten element przykładowo jest wykonany ze sztucznej szczęki.
Nie, no nie ludzkiej sztucznej szczęki, czy ludzie mają po dziesięć takich półkolistych zębów?
Wiele istot ma przecież szczęki i większość z nich, tak jak ludzie, na starość potrzebuje sztucznych.
To rurka od kuli do podpierania, tym razem ludzkiej kuli. A ten wężyk był kiedyś w rozruszniku do serca... smoka.
Tak, całość nadal się porusza, ma pan rację, ten sztuczny mózg nadal żyje, ale akurat nie pamiętam imienia demona, który go zasila.
Potwierdzam, może to trochę niebezpieczne go tutaj trzymać, ale w najgorszym razie, w razie jego ucieczki, i tak pierwsze co by ten demon zrobił, to czmychnął jak najdalej od tego świątecznego miejsca. Poza tym, jest zamknięty w kuli wymiarowego szkła, nic się przez to nie przebije.

Ani atomowa, ani termojądrowa, zwyczajna na proch. Kiedyś zadarliśmy trochę za bardzo z wojskiem Stanów Zjednoczonych. 
Mocno nadszarpnęli nam ochronne powłoki i w końcu ta mała bombka przebiła się przez pancerz i wpadła prosto do pieczonego dzika.
Gdyby wybuchła, to na pewno nie podróżowałby pan teraz z nami.
Piekielni amerykanie. Zbudowali swoje pociski z żelaza, wydobywanego przez niewolników w Afryce,
z prochem wyciągniętym z fajerwerków, co miały być wystrzelone na Halloween, na koniec pokropili zapalniki krwią z abordowanych dzieci.
W związku z tym, znacznie prościej udało im się przebić przez osłony Kuli.

Przy okazji wytłumaczyłem mu ochrony zastosowane w tym pojeździe. Były trzy powłoki, z tym że trzecia to już fizyczny pancerz. 
Pierwsza powłoka zatrzymuje szybko poruszające się obiekty.
Druga chroni przed naporem niepożądanych substancji, już ją widziałeś, jak blokowała wodę przed wdarciem się do środka.

Wycieczkę przerwał dźwięk otwieranego włazu i chlapanie wody.
Poszliśmy zatem przywitać trzeciego gościa. Mateusz był bardzo podekscytowany i pobiegł przodem.

\divider{}

Lenna popatrzyła się na Antyraxa i pokiwała w aprobacie głową.
Potem sama skierowała swoje kroki w kierunku Neofantasora.

Demony były już chyba przerażone, gdyż teraz poczęły wszystkie uciekać.
Jednak Antyrax był szybszy. Złapał jednego z nich za nogę (a właściwie to jego lateksowy but złapał nogą nogę), przyciągnął do siebie, i wcisnął mu swój worek na głowę.
Piotr Lekter zaczął się dusić, trująca abstrakcja wgryzała się w jego demonowe płuca, a lateksowy but, z siłą wolnego oprogramowania, ściskał mu szyję.

\ds{} Dość, wystarczy. Dam ci te dwie gwiazdki! \dm{} Z worka słychać było jedynie stłumione jęki. \dm{} Trzy! Niech będą trzy gwiazdki. I komentarz. \de{}

Antyrax jednak nie odpuszczał. Ruchy Piotra Lektra stawały się coraz wolniejsze i wolniejsze.

\divider{}

Nadar. Czemu to akurat jego musiał ten Kula zaprosić?
Planowaliśmy eleganckie przyjęcie, a ta niewychowana świnia pewnie pociągnie w swoje odmęty i Mateusza.

Jak tylko usłyszałam szczęk łańcuchów, przerwałam robienie makijażu i wystawiłam głowę z pokoju.
Zobaczyłam nowego gościa, zbiegającego po schodach do szatni, biegł tak szybko, że spłoszył mi motyle.
Nie spieszyło mi się powitać Nadara równie prędko, ale ciekawiło mnie zobaczyć reakcję Mateusza, gdy zobaczy tego szaleńca.
Z trudem przecisnęłam się w tej sukni przez drzwi i ostrożnie podeszłam do pierwszego schodka w dół.
Oczywiście rajstopy, od razu hyc i resztę schodów koziołkowałam, robiąc podwójne salto, lądując na głowie z nogami majtającymi się w powietrzu.

Wiedziałam, co teraz usłyszę i nie zawiodłam się.

\ds{} Ale dupa, co nie? \dm{} Nadar zamykał korbą właz, gapiąc się na moje machające w górze nogi. \de{}

\ds{} No, nawet... \dm{} Mateusz okazał się równie niewychowany. Nie wierzę, że się z nim zaprzyjaźniłam. \de{}

Zaraz przybiegł Kula i pomógł mi się postawić do pionu. Był czerwony ze złości.
Ale czy dlatego, że właśnie ze statku z sykiem uchodziła kultura, czy dlatego że świństwo uzyskało nowego członka?

\ds{} To ty! \dm{} Kula trzymał laskę w górze, niczym śmiercionośny laser krojący Nocnego na pół. \dm{} To ty wziąłeś czwarty kombinezon z mojej garderoby! Szukałem go po całym wszechświecie. To integralna część Kuli, generuje go światłograf, więc jest niereplikowalny. Nie wolno go zabierać! \de{}

\ds{} Przecież nie zabrałem, a pożyczyłem. Zresztą i tak zawsze się kurzył w tej twojej półkulistej szafie. \dm{}
Nadar uznał to za wystarczające wytłumaczenie, rozpiął strój. Pod spodem miał swoje standardowe dresy. \dm{} 
A na przeprosiny mam prezent. Wyłowiłem ci, Profesorze, zestaw kieliszków i butelkę najdoskonalszego wina prosto z kapitańskiego mostka.
Mieli ją wypić na ukończony rejs, ale wiadomo co się stało. Niech zamiast tego Kula ukończy swój własny i nie uderzy w żadną lodową kometę po drodze. \de{}

Profesor Kula w jednym pulsie zmienił się z czerwonego z powrotem. 

\ds{} Och. To bardzo miło z twojej strony. \dm{} Odpowiedział miękkim głosem. \dm{} A teraz wybaczcie, muszę dopilnować ostatnich poprawek przy naszym bankiecie. \dm{}
Porwał butlę i kieliszki, znikając w górnych piętrach.

Kto, jak kto, ale Nadar doskonale wyczuwał charakter innych osób.
Wiedział co zrobić, żeby zabolało i co żeby było dokładnie na odwrót.
Jednak pomimo wad, potrafił, jak nikt, walczyć z uniwersalnością.

Mateusz wpatrywał się w Nadara, jak Kula w obraz autora białego cyrkowca.
Jego największe zainteresowanie wzbudzały dwa pistolety gościa, zawieszone przy pasie, i laserowa pałka na plecach.
Pierwsze, to był standardowy szyfrator, jaki każdy w ALOPP posiadał.

\ds{} To urządzenie pozwala spauzować i odpauzować dowolną osobę w splocie czasoprzestrzeni.
W pełni bezpieczny sposób na unieszkodliwianie wrogów bez zabijania.
Wadą jest tylko to, że naboje do niego są takie olbrzymie i jednorazowe.
W środku wkładu zapisuje się symetryczny obraz klucza, jedyny sposób na odpauzowanie spauzowanej osoby z powrotem.
Kasiu, właśnie zgłosiłaś się na ochotnika, aby zaprezentować naszemu gościowi ten wynalazek. \dm{} Nadar wycelował we mnie szyfrator. Co za świ...

\divider{}

Antyrax podniósł lekko worek, Piotr Lekter spróbował złapać oddech, ale zaraz znowu światło zgasło mu przed oczyma.

\divider{}

...nia z niego. \de{}

\ds{} Jak widzisz, działa znakomicie. \dm{} Spostrzegłam, że w czasie gdy byłam spauzowana, zdążył już się przebrać w elegancki strój, pożyczony z garderoby. Właśnie nakładał puder na swojego irokeza. \de{}

\ds{} Nadar, coś ty? Od kiedy ubierasz się elegancko dla Kuli? \dm{} zapytałam z niedowierzaniem. \dm{} Przecież nie gustujesz w niczym innym niż dresy. \de{}

\ds{} Od kiedy wywalił mnie w skafandrze, pośrodku kosmosu, za przypadkowe rozlanie barszczu na obrus. 
Lewitując w bezkresnej pustce, miałem sporo czasu na przemyślenie swojego zachowania i stanie się nowym człowiekiem. \dm{} odpowiedział. \de{}

\ds{} Naprawdę? \dm{} Wtedy coś mną tknęło. \dm{} Oczywiście, że nie na prawdę. Znowu się zgrywasz tak? \dm{} Tylko się wrednie zaśmiał. \de{}

\ds{} To drugie to pikler. \dm{} kontynuował. \dm{} Potrafi zapeklować kogoś, do umieszczonego tutaj, słoika ze szkła wymiarowego, żeby nigdy się nie wydostał.
Tylko odłożyć na najniższą półkę w jakiejś głębokiej piwnicy na całą wieczność.
\dm{} Spostrzegł, że gość niekoniecznie rozumie. \dm{}
Szkło wymiarowe to takie coś, które przechodzi równo przez wszystkie wymiary, także w czasie. Wygląda jak szkło, ale istnieje od zawsze na zawsze. Ma nieskończoną długość, szerokość,
głębokość i... wszystkie inne ości. \dm{}
Nadar nie przestawał wyjawiać sekretów naszej organizacji. \dm{}
To jest laserowa pałka, taki przecinak, po uruchomieniu zaczyna wirować, wzdłuż pojawiają się promienie dasera. Daser przecina prawie wszystko jak masło,
a pozostałe rzeczy jak ser. No, twój mózg przeciąłby jak powietrze.
Fajna zabawka. \de{}

\ds{} I szkło wymiarowe przetnie? \ds{} Mateusz zapytał, widać że uważnie słuchał. \de{}

\ds{} Nadar, on nie przeszedł jeszcze wszystkich testów \dm{} wtrąciłam \dm{} nie zdradzaj mu tylu sekretów, bo nie wiadomo, czy na pewno z nami zostanie.

\ds{} No popatrz na niego, Magda \dm{} Nadar obchodził i studiował Mateusza ze wszystkich stron. \dm{} Myślisz, że sobie nie poradzi?
Poza tym, już pierwsze części przeszedł doskonale. Odpowiedział na abstrakcyjny list Profesora, ubrał się w najprawdziwszy strój francuski, a potem 
odważył się wsiąść do wielkiej latającej kuli z kosmosu. \de{}

\ds{} Niby racja, ale doskonale wiesz, jak chory test potwory mogą tym razem wymyślić.
Pamiętasz, jak Mikołaj przetestował Ziemowita? Kazał mu się przebrać za klauna, przyjść na zabawę dla dzieci i robiąc magiczną sztuczkę, zamordować jednego z nich, co był owocem klonu.
Coś nie wyszło i wszyscy skąpali się w zielonej krwi tego podrabiańca.
Drugim zadaniem było uciec z więzienia do którego go wrzucili. \de{}

\ds{} Przecież zdał. \de{}

\ds{} Albo tego, jak mu było, Błażeja, co Hdro zostawił w Capitalu i kazał jakimś sposobem wrócić na Ziemię.
Człowiek sam na planecie, w całości zamieszkanej przez wszystkie gatunki smoków. 
\dm{} Kontynuowałam rozmowę, zupełnie ignorując osobę na której temat ją toczyliśmy. \de{}

\ds{} Nie zaliczył, bo ukradł rakietę jakiejś rodzince błękitnych celebritów będącej wakacjach w Capitalu, zamiast rozegrać to w pokojowy sposób. Nawet się nie przejął że w środku wciąż były ich jaja!
A gdy rzucił się za nim pościg bordowych pasowców, on ich bezwzględnie pozabijał, strzelając z rakietowego działka.
Na szczęście nie miał kluczy warstw i rozbił się o ścianę kwadry. W dodatku drugiej kwadry, nie czwartej! Idiota tylko się oddalił od Ziemi.
Lewitował w smoczej przestrzeni przez dwa dni, prawie umierając z głodu. W końcu te małe smoczki z rozbitej rakiety się wykluły i zjadły go żywcem.
Dobrze mu tak. \de{}

\ds{} Przepraszam, że wam przerwę, ale co ze mną? To jakiś test zręczności, albo inteligencji? Zginę, pożarty przez coś? \dm{} Mateusz bezwstydnie przerwał. \dm{} Co mam zrobić, żeby go zdać? \de{}

\ds{} Masz być sobą. \dm{} Odpowiedziałam równocześnie z Nadarem. \de{}

\ds{} Ciebie chyba będzie testował Plazma. \dm{} Nadar się zamyślił. \dm{} On lubi militarne klimaty, pewnie trafisz na Planetę Wojny.
Albo będziesz wyżynał jakieś miasto, albo sam będziesz wyżynany. Musisz sobie poradzić. 
Najlepsza śmierć... to będzie rozerwanie na kawałki przez jakąś futurystyczną wunderwaffe, najgorsza, pewnie wcielenie do Czarnej Armii.
Obedrą cię ze skóry, wyłupią oczy i zęby, wcisną w cyber-zbroję i zaleją szaleniotwórczym smarem khaki, będziesz umierał powolną śmiercią, dobrze się przy tym bawiąc przy mordowaniu niewinnych. Ja tam wolę robić to samo, nie będąc rozpuszczanym przez czarny kwas.
\de{}

Tymczasem rozległ się dźwięk dzwonu, oznajmiającego posiłek.
Zgodnie poszliśmy na najwyższe piętro, Mateusz tym razem trzymał się z tyłu.
Przechodząc przez muzeum, Nadar poklepał radośnie gablotę demonicznego mózgu, który w odpowiedzi kłapnął groźnie szczerbatą protezą zębów.

Na najwyższym piętrze znajdował się salon, biblioteka, scena teatralna, ogród i kapliczka.
Dach przyjmował tutaj miłą wklęsłość, ze szczytu zwisał żyrandol na lampy oliwne.
Mateusz zapytał mnie, dlaczego w ogródku rosną tylko ziemskie kwiaty.

\ds{} Nie wiem czy wiesz, ale Ziemia jest uważana przez wielu za najpiękniejszą planetę wszechświata \dm{} odpowiedziałam. \dm{}
A przynajmniej na pewno przez naszego Profesora. \de{}

\ds{} Tos? \dm{} Mateusz zwrócił się w kierunku muzeum. \de{}

\ds{} Tos jest bardziej niezwykły niż piękny. Poza tym, grzyby trzeba by hodować w amoniakowej szklarni.
No i nie powąchasz ich jak kwiatów \ds{} wyjaśniłam. \dm{} Aha, jeszcze Tosowe życie puszcza wszędzie zarodniki. 
Jeden wdech atmosfery tej planety i zaczną ci rozpuszczać żywcem nos. Dwa wdechy i spleśnieją ci płuca. Trzy wdechy i grzybnia wkręci się w mózg. \de{}

\ds{} A kaplica? \dm{} Zwrócił uwagę na mały budyneczek w rogu. \de{}

\ds{} Jak pewnie zauważyłeś, Profesor jest bardzo religijny. Poza tym dobrze mieć miejsce, gdzie można modlić się o litość, będąc atakowanym przez kosmicznych bandytów. 
Każdy większy statek ma przecież kaplicę, to czemu kosmiczny nie miałby mieć? Kula zrobił kiedyś bardzo wiele dobrego, w prezencie otrzymał ten właśnie kulisty twór. \dm{}
Rozejrzałam się, gdzie jest nasz gospodarz. \dm{}
Ale on nie lubi, gdy się o nim rozmawia. \de{}

\ds{} Tylko, kim on jest? \dm{} Mateusz nagle zapytał. \de{}

\ds{} W sumie nikt nie wie dokładnie, kim, lub czym jest Profesor. Niektórzy mówią, że aniołem, inni że dziwnym człowiekiem,
jest też teoria jakoby był ostatnim z jakiejś umarłej cywilizacji. 
Posługuje się jedynym w swoim rodzaju pismem i językiem, którego nikt inny we wszechświecie nie używa.
Widzi szerszy zakres barw, słyszy więcej dźwięków, nie wiem czy jest supersilny... \de{}

Uratował mnie dzwonek, zwiastujący rozpoczęcie bankietu.

%TODO Wybierają miejsce obiadu
%TODO Przystawka

%Prerobienie fragmentu

Jak kula się porusza? Nie, żadna nowoczesna technologia na antymaterię. Pokazałem światłograf wiszący na ścianie. To taki jakby cyrograf, tyle że w drugą stronę.
Zobowiązujesz się do wypełnienia określonej ilości dobra za pomocą danej ci umiejętności.
A co dobrego jest w zapraszaniu nieznajomych na bankiety i wożeniu ich po całym wszechświecie?
Ja już swoją powinność spełniłem w całości, po wypełnieniu umowy dostajesz ową rzecz na własność w dowolnym celu, również i złym.


Dlaczego ludzie zapominają... to dobre pytanie.
Otóż Kula pokryta jest amnezją. Każdy, kto na nią spojrzy, nawet pośrednio --- zapomina.
Pamiętają tylko ci, którzy wierzą. Wierzą w Profesora Kulę, wierzą w bankiety w niebiosach, wierzą w złocone wnętrze.
Na pewno nie będzie to dla pana zaskoczeniem, że większość uważa Kulę zwykle za balon meteorologiczny, fatamorganę, dowcip, sztuczkę magiczną, nowoczesny samolot wojska, itp.
Pan uwierzył, dlatego pan tutaj jest.






%TODO Przysłowie "Gabinet profesora kuli"
