\chapter{Sam przeciw wszystkim}

\info{Wioska jest atakowana przez literackie demony. Tylko dzielny, acz niedoświadczony wojownik jest wstanie je pokonać, używając do tego
swojego miecza kreatywności i worka z alternatywnym wszechświatem.}

Gdy tylko czubek głowy Wielkiego Neofantasora wyłonił się zza gór, odpalono armaty załadowane gorącym atramentem.
Katapulty wystrzeliły litery, a papierowe samoloty wzniosły się w powietrze.
Wszystko na darmo, wielkolud ani drgnął.

Wojownicy z wioski, uzbrojeni w pióra i kałamarze, ruszyli do ataku.
Atakowali wroga, używając całego swojego doświadczenia. Opowiadania cyberpunkowe, fantasy i fantastyki naukowej cięły powietrze.
Horrory, wiersze i dzieła detektywistyczne rozbijały się o jego ciało.
Jednak niewzruszony Neofantasor pozostawał niewzruszony.
Stanął przed wioską i wysypał z rękawa, niczym wytrzepując piasek z buta po całodniowym siedzeniu na plaży, dziesiątkę straszliwych demonów.
Demony szybko rozprawiły się z całą obroną, rzucając po jednej gwiazdce w każdego z wojaków.
Nikt z nich nie przeżył, wieś była bezbronna.

Ale jak to w baśniach zwykle bywa, znalazł się młody syn doktora --- Antyrax.
Trochę ułomny językowo i bez jakiegokolwiek doświadczenia, postanowił dzisiaj zginąć w walce.
Nie miał, jak wszyscy, zbroi, pióra, czy zapasu atramentu.
Posiadał jedynie szklany miecz, wypełniony płynną kreatywnością, dmuchawkę z serum śmiechu, lateksowe buty i worek z alternatywnym wszechświatem.

Najbliższa demonka, Obudzona, została kopnięta z lateksowego buta i nawet nie poczuła.
Dopiero serum śmiechu wstrzyknięte w szyję zwróciło jej uwagę.
Wtedy Antyrax sięgnął do swojego worka i wyciągnął pewną rzecz.
Obudzona zamrugała, nie rozpoznając zupełnie, co to jest i czy się tego bać.
Wojownik oglądał rzecz ze wszystkich stron, ale także nie rozpoznał, co właściwie wyciągnął.

Po kilku minutach wyrzucania różnych, nieokreślonych przedmiotów z worka, spostrzegł że grupka demonów go otoczyła i z zaciekawieniem
obserwuje jego zmagania z samym sobą.
Nie umiał używać własnego wszechświata.
Postanowił więc zrobić to w tradycyjny sposób i wbił swój miecz w najbliższego demona, który odskoczył z bólu. Przynajmniej to działało.
Nie długo cieszył się z sukcesu, zaraz wszystkie demony razem wzięły i dmuchnęły w niego strumieniem przecinków --- jego największą słabością.

Pierwszy, lecący znak interpunkcyjny, ominął, podskakując na lateksowych butach, drugi, skontrował mieczem, lecz tysiąc kolejnych odrzuciło go na kilkaset metrów w bok.
Wpadł obolały w błoto po drugiej stronie wioski, ale nic sobie nie połamał.
Na koniec lecący błąd ortograficzny prawie rozciął go na pół, Antyrax zdążył odskoczyć, lecz puścił miecz, który pękł na drobne kawałeczki, zmiażdżony przez ,,Ó''.
Kreatywność wsiąkła w błoto, tworząc nowe królestwo błotnych mrówko-androidów zasilanych parą z wody basenowej.

Antyrax wstał i zobaczył, jak jago wioska jest równana z ziemią.
Cień Wielkiego Neofantasora wyglądał, jak dłoń dziecka, które zagarnia klocki z ziemi, aby je zaraz obślinić i połknąć.
Czy to był już koniec dla niego i dla wszystkich?

\ds{} Nie, to nie koniec \dm{} pomyślał, sięgając po worek. \de{}

Jego butom wyrosły śmigła i poniosły go prosto pod nos Obudzonej, zostawiając linuksowy ślad.
Zaciekawiona demonka rozpoznała niedojdę, którego wcześniej zmiażdżyła, jak mrówkę.
Antyrax nie miał już nic, tylko worek. Sięgnął do niego i długo nie wyciągał ręki.

\ds{} Mam coś specjalnie dla ciebie, Obudzona. \de{}

\divider{}

\ds{} Kolejny dzień w pracy, kolejny dzień robienia bezużytecznych czynności dla bezużytecznej korporacji w tym bezużytecznym świecie \dm{} pomyślałem. \de{}

Wziąłem komórkę i zacząłem przeglądać maile.
Spam z reklamą talerzy, zaproszenie do znajomych na Facebooku, od obcej osoby. Powiadomienie o komentarzu na YouTube, 
powiadomienie o mailu na innej skrzynce pocztowej, przypomnienie o opłacie za subskrypcję tej beznadziejnej gry komputerowej i tysiąc innych bezużyteczności.
O, znowu rozpętałem gównoburzę na Mirko i zablokowali mi konto Twittera za niepoprawne politycznie myślenie.
Bezużyteczność.

Z nadmiaru bezużyteczności zdrzemnąłem się w ubraniu i obudziłem godzinę później, na głośne gruchanie z okna.
Czyżby jakaś użyteczność w końcu mnie spotkała?
Na oknie siedział biały gołąbek, w dzióbku trzymał kopertę.
Nie będąc pewnym, czy nadal nie śnię, odebrałem przesyłkę i przyjrzałem się kopercie.
Gołąb zaraz zniknął, oddalając się bezszelestnie.

Koperta była z papieru czerpanego.
Adresowana była elegancką kursywą do mnie, do Mateusza Mechalycznego, zamieszkałego na ulicy Szerokiej w Gdańsku. 
Województwo Pomorskie, Polska, Ziemia, Układ Słoneczny, Galaktyka Droga Mleczna, Gromada Lokalna, Pierwsza Kwadra.
Z tyłu widniała pieczęć z wosku pszczelego.
Wypukły obraz kuli i jakiś napis z niezrozumiałych znaków.
To był kawał. To musiał być kawał.
Przecież był 2017 rok, kto jeszcze wysyła papierowe listy zamiast maili?

Ostrożnie otworzyłem kopertę, trzymając ją obcęgami, spodziewając się że coś strasznego zaraz na mnie wyskoczy.
Nic się jednak nie stało.
Pisany odręcznie atramentem list był tym, co zwykle znajduje się w kopertach zaklejonych pieczęcią.

\begin{em}
Szanowny Panie Mateuszu Mechalyczny.

Mam zaszczyt zaprosić Pana na uroczysty bankiet, z okazji wyboru na jednego z przyszłych członków ALOPP.
To wielki zaszczyt móc gościć nowych agentów tej organizacji na pokładzie mojego \emph{[słowo z dziwnych znaków]}.
Mam najskrytszą nadzieję, że zostanie Pan przyjęty i będzie mieć Pan we współpracy z \emph{[znak niczym wydrapany na ścianie]} udział w walce o wspólne dobro.

Zapraszam do siebie dnia 20 października, roku Pańskiego 2017.
Myślę, że marina w Głównym Mieście będzie doskonałym miejscem na lądowanie dla nas i tam się spotkajmy w lokalne południe.
Po obiedzie wybierzemy się w podróż na Felicję, gdzie pozna Pan swoich przyszłych, mam nadzieję, członków przybranej rodziny.

Przypominam, że w \emph{[to samo słowo z dziwnych znaków]} oprócz najwyższej kultury osobistej,
od zawsze obowiązywał elegancji ubiór wersalski.
Wszyscy goście powinni przywiązać najwyższą dbałość o szczegóły swojego wyglądu.
Uprzejmie proszę także, aby nie posiadać na pokładzie żadnych urządzeń działających na elektryczność.

Z Bogiem.
--- Profesor \emph{[kolejne słowo z dziwnych znaków]}
\end{em}

Zaczynało się robić ciekawie. Autor tego dowcipu chciał, abym za trzy dni, w XVIII wiecznym stroju pałacowym,
udał się w środek miasta w dniach szczytu, nie zabierając ze sobą żadnej elektroniki.
Potem miałem przejść test na zostanie członkiem jakiejś organizacji.
Trzeba dokładniej przestudiować ten list.

Szybkie szukanie Felicji w internecie wskazało jedną stronę o teoriach spiskowych.
Felicja miała być planetką, stworzoną przez kosmitów, na której hodowano ludzi, aby przeprowadzać na nich straszliwe eksperymenty.
Jeśli to prawda, może zabrakło im tam królików doświadczalnych i porywają kolejnego? 
Ale wtedy przecież nie dawali by mi wolnej ręki.

Na tej samej stronie podano: ALOPP miał być organizacją terrorystyczną zrzeszającą ludzi w celu mordowania obywateli własnej planety.
Ale od czego był to skrót, to nikt nie wiedział.

Pierwsza Kwadra dawała za dużo losowych wyników, aby wywnioskować o co może chodzić.

Lokalne południe, czyli dwunasta godzina czasu słonecznego, uwzględniając jeszcze czas letni, to trochę przed trzynastą czasu strefowego.

Wikipedia natomiast wskazała, że gołębie pocztowe, w żadnym wypadku nie mogłyby doręczyć listu bezpośrednio do odbiorcy.
Ich mechanika polega na wracaniu do macierzystego gołębnika z dowolnego miejsca na świecie, i tylko tyle.
Listów z pewnością nie wsadzano im do dzióbka.
Rozejrzałem się po pokoju, czy przypadkiem nie miałem w nim gołębnika, żeby hodować gołębie pocztowe, ale nie.

Kilka razy widziano kuliste UFO i ludzi w wersalskich strojach jednocześnie.
Podobno zdjęcie nigdy nie wychodziło poprawnie, a większość ludzi
magicznie zapominała o zdarzeniu chwilę po odlocie tajemniczej struktury.
Nieliczni pamiętali, ale nikt im oczywiście później nie wierzył.
Kulę widywano w różnych miejscach, nie ograniczała się do USA, przelatywała przez centra miast, pływała pod wodą
i toczyła bitwy z wojskami wszystkich krajów. Należy do tych danych podchodzić z dystansem.

Następnego dnia zabrałem list do znajomego chemika. 
Potwierdził on moje obawy, list wykonany był oryginalną techniką sprzed kilkuset lat.
Skład chemiczny papieru i atramentu, odpowiadał tym używanym dawno temu.
W dodatku narzędzie pisania z pewnością było ptasim piórem.
Na myśl o bezużyteczności otaczającego mnie świata, postanowiłem pojutrze zrobić coś użytecznego.

\divider{}

Obudzona pod naporem nietypowości i kreatywności dzieła, zaczęła krzyczeć, zwijać się w konwulsjach i palić czarnym ogniem.
Zmieniła się w mały księżyc i poleciała, jak frisbee, z powrotem w kierunku Wielkiego Neofantasora.

Triumf Antyraxa nie trwał długo, za chwilę od tyłu złapała go Dedirid.
Jej czarna ręka owinęła się wokół delikatnego światostwórcy, jak czarny worek na zwłoki.
Zaraz go wyciśnie, jak tubkę pasty do zębów.
Wywijając się rybio, Antyrax zanurkował do worka i długo nie wychodził.
Demonka przez godziny nachylała się nad otworem, aby capnąć go jak tylko wystawi głowę.
Gdy tylko coś się wysunęło, porwała to z ochotą.
Była to jednak kolejna opowieść, zabójczo eksperymentalna, niesamowicie abstrakcyjna.
Nietypowość poparzyła jej łapska.

\divider{}

Mateusz wypożyczył wymaganą górę z wypożyczalni kostiumów.
Zastanawiał się, czy nie podkraść jakiegoś szustokora z muzeum, ale to chyba nie było by zbyt poprawne.
Bał się, czy mierna jakość kaftana, spowodowana nieoryginalnością, nie będzie zwracać nadmiernej uwagi w świecie najprawdziwszych atłasowych pasów i perłowych guzików.
Postanowił kupić więc kilka ozdób ze sztucznej biżuterii, które wyglądały dość kosztownie, a stworzone były z 
byle-czego, i doszyć w losowych miejscach. Miał nadzieję, że Profesor i inni goście nie zauważą.

Z pończochami nie było żadnego problemu, znalazł je w damskim sklepie.
Tak samo coś, co można było podciągnąć pod dawną koszulę.
Musiała być flanelowa, z wystającymi rękawami.
Ogarnął także puder.

Peruka, cóż. Przynajmniej miał już trójkątną czapkę piracką po poprzednich lokatorach.
Najgorzej, że zazwyczaj chodził na łyso, gdyż rodzice nie obdarzyli go mocnymi włosami.
Potrzebował na szybko przykleić coś sobie na łeb.
Liczył w głowie, ile lat może dostać za kradzież peruki sędziemu, gdy spostrzegł wyprzedaż starych futer.
Używając magii nożyczek, kleju i zjedzonego przez mole płaszcza, wygenerował coś, co po przykryciu trójkątną piracką czapką wyglądało dość znośnie.

U zegarmistrza kupił za grosze kopertę zegarka z brakującymi wskazówkami, całość zaczepioną na łańcuszku.
Zegarek nie musiał działać, ważne, żeby był.

O dziwo, to buty przysporzyły mu najwięcej problemu.
Niby lakierki z klamrą nie są niczym bardzo skomplikowanym, a jednak nikt ich nie sprzedaje.
Może właśnie dlatego, że były modne trzysta lat temu.
Wpadł na pomysł, aby kupić coś podobnego i przerobić. 
Znalazł buty dla zakładów pogrzebowych, gdyż tylko te odpowiednio się błyszczały i przyszył im klamry od spodni.
Z daleka nie było widać.

W domu ubrał się i przejrzał w lustrze.
Połączenie Napoleona, Ludwika XIV i informatyka z Gdańska.
Muszą zrozumieć.

O jedenastej godzinie, owego wielkiego dnia, ubrał się w pełny strój.
Nie mógł się przecież tak pokazać w mieście.
Pończochy przykrył spodniami sztruksowymi.
Na elegancki szustokor nałożył nieco za dużą bluzę z kapturem.
Lustrzane lakierki przykrył jakimiś szmatami, żeby nie zwracały zbytniej uwagi.
Tylko pseudo-perukę schował do plecaka.

To nie mogło pójść tak łatwo.
Z daleka zobaczył kordon policji i wojska, które blokowało wstęp każdemu wychodzącemu z Długiego Targu.
Ucieszył się i kamień spadł mu z serca. Oznaczało to, że jednak nie padł ofiarą żartu.
Znalazł w końcu promyk użyteczności w oceanie bezużyteczności.

Do mariny spróbował dostać się okrężną drogą, przebiegł przez Krowi Most na Wyspę Spichrzów.
Klucząc uliczkami, zbliżył się do portu, jednak tutaj też była blokada.
Widział w każdym bądź razie kawałek wody, która była mocno niespokojna, coś tam było.
Popatrzył w kanał i pomyślał, że zostanie mu wskoczyć do wody i popłynąć.
Ale przecież nie zostałby wpuszczony mokry do rakiety.
No i co z pudrem.

Szustokor był bardzo gorący, rozpiął więc bluzę żeby się nie ugotować, teraz wszystko było mu jedno, czy ktoś to zauważy.
Był tak blisko, a jednocześnie tak daleko.
Jak rybka w siatce wrzucona do oceanu.
Zaraz będzie widział swoją porażkę, jak na dłoni.
Co robić? Co robić?

Wybawienie przyszło nieoczekiwanie.
Oto bowiem mama z małą dziewczynką podpłynęły do niego skuterem wodnym, oferując szybką podwózkę.
Myśląc, że to pomyłka, zdjął bluzę, pokazując swój strój w połowie okazałości.
Kobieta jednak nie uciekła, przestraszona dziwaka, tylko się uśmiechnęła.

\ds{} Profesor Kula nie lubi spóźnialskich. Pospiesz się \dm{} powiedziała. \de{}

\ds{} Skąd... kim... \de{}

\ds{} Świat jest wielki, a zasięg potworów... znaczy tych właśnie... to jest taka jakby policja wszechświata... jest większy.
Jesteś nowy, wnioskuję, że zostałeś zaproszony na test. Pospiesz się.\de{}

\ds{} Ale co mam robić, żeby go zdać? \de{}

\ds{} Być sobą. \de{}

\ds{} Jak mam ci się... \de{}

\ds{} Uratuj w przyszłości też komuś życie. \dm{} Pogłaskała swoją córeczkę po głowie. \de{}

Mateusz w XVIII wiecznym stroju wersalskim zasuwał kanałem Nowej Motławy na skuterze wodnym, ozdoby szustokora mieniły się tak samo, jak latające wokół 
krople wody i kule wystrzelone z mostu przez żołnierzy zabezpieczających
lądowanie wielkiej, białej kuli w centrum miasta. 
Schował się pod mostem, jak przestraszony królik w norze, a gdy wypłynął z drugiej strony, wtedy ją zobaczył.

Była wielka, jak budynek, wypolerowana, biała i doskonale kulista.
Dołem dotykała lekko powierzchni wody, tworząc promieniste fale.
Odbijała w sobie cały Gdańsk.
Mateusz zobaczył w niej malutkiego siebie na łódeczce-zabawce, malutkie budynki, żołnierzyków, spichrze, basenik, niebo, helikopter jak muchę i blask pełnego słońca.
Już nikt nie strzelał, już tylko patrzyli. I bali się.
On się nie bał. Przyszedł tu na bankiet.
Przyszedł w wersalskim stroju.
Przyszedł tu, bo został zaproszony.

Zszedł ze skutera na pomost i poprawił perukę, wtedy też właz w dolnej części zaczął się otwierać.
Tak, jak się spodziewał, był to dźwięk ulatującej pary wodnej i szczęku łańcuchów, a nie elektrycznego silnika.
Ze środka powiał zapach kurzu, wosku i lekkiej stęchlizny.
W przejściu stanął On.
Nosił najwspanialszy strój, jaki Mateusz kiedykolwiek widział, tak inny od jego własnego, a przecież z tego samego okresu.
Przy jego ozdobach, sztuczna biżuteria Mateusza rzeczywiście wyglądała na sztuczną.
Jego najprawdziwsza peruka przyćmiła wielkością cały futrzany twór.
Lakierki błyszczały się tak samo, jak jego statek kosmiczny.
W ręku trzymał laskę z białą kulką, pomniejszoną wersję tego, co znajdowało się tuż za nim.

\ds{} Jestem Profesor Kula. Miło mi pana gościć na moim statku. \de{}

\divider{}

Antyrax wyszedł z worka, gdy z Dedirid została już tylko kupka popiołu.
Jeszcze ośmiu.
Tym razem demony nie bardzo chciały go atakować.
Antyrax więc wskazał jednego z nich palcem, niczym sędzia młotkiem skazanego na śmierć.
Lenna. Pora na mikropomysły.
Sięgnął do worka i od razu złapał to, czego szukał.

\divider{}


















