\chapter{Bankiet w Kuli}

\info{Wioska jest atakowana przez literackie demony. Tylko dzielny, acz niedoświadczony wojownik jest wstanie je pokonać, używając do tego
swojego miecza kreatywności i worka z alternatywnym wszechświatem.}

Gdy tylko czubek głowy Wielkiego Neofantasora wyłonił się zza gór, odpalono armaty załadowane gorącym atramentem.
Katapulty wystrzeliły litery, a papierowe samoloty wzniosły się w powietrze.
Wszystko na darmo, wielkolud ani drgnął.

Wojownicy z wioski, uzbrojeni w pióra i kałamarze, ruszyli do boju.
Atakowali wroga, używając całego swojego doświadczenia. Opowiadania cyberpunkowe, fantasy i fantastyki naukowej cięły powietrze.
Horrory, wiersze i dzieła detektywistyczne rozbijały się o jego ciało.
Jednak niewzruszony Neofantasor pozostawał niewzruszony.
Stanął przed wioską i wysypał z rękawa, niczym wytrzepując piasek z buta po całodniowym siedzeniu na plaży, dziesiątkę straszliwych demonów.
Demony szybko i sprawnie rozprawiły się z całą obroną wioski, rzucając po jednej gwiazdce, w każdego z wojaków.
Nikt z obrońców nie przeżył, wieś została bezbronna.

Ale jak to w baśniach zwykle bywa, znalazł się młody syn doktora --- Antyrax.
Trochę ułomny językowo i bez jakiegokolwiek doświadczenia, postanowił dzisiaj zginąć w walce.
Nie miał, jak wszyscy, zbroi, pióra, czy zapasu atramentu.
Posiadał jedynie szklany miecz, wypełniony płynną kreatywnością, dmuchawkę z serum śmiechu, lateksowe buty i worek z alternatywnym wszechświatem.

Najbliższa demonica, Obudzona, została z zaskoczenia kopnięta z lateksowego buta prosto w pupę, i nawet nie poczuła.
Dopiero serum śmiechu, wstrzyknięte prosto w szyję, zwróciło jej uwagę.
Wtedy Antyrax sięgnął do swojego worka i wyciągnął... rzecz.
Obudzona popatrzyła na zagubionego wojownika, to na przedmiot, który trzymał, po znowu na wojownika. Zamrugała, nie rozpoznając zupełnie, co to jest i czy się tego bać.
Ostatni pisarz też obracał w dłoni i oglądał rzecz ze wszystkich stron, ale i on nie wiedział, co właściwie właśnie wyciągnął.
Wyrzucił za siebie i spróbował ponownie.

Po kilku minutach wyrzucania różnych, nieokreślonych obiektów z worka, spostrzegł że grupka demonów go otoczyła i z zaciekawieniem obserwuje jego zmagania z samym sobą.
Nie umiał używać własnego wszechświata. Co za matoł.
Postanowił więc wykonać swoją powinność w tradycyjny sposób, zamachnął się i wbił swój miecz w najbliższego demona, który odskoczył z bólu. Stare, sprawdzone metody zawsze działają.
Nie długo cieszył się z sukcesu, zaraz wszystkie straszydła razem wzięły i dmuchnęły w niego strumieniem przecinków --- jego największą słabością.

Pierwszy lecący znak interpunkcyjny ominął, podskakując na lateksowych butach, drugi skontrował mieczem, lecz tysiąc kolejnych odrzuciło go na kilkaset metrów w tył.
Upadł obolały w błoto, po drugiej stronie wioski nad którą chyba właśnie przeleciał. Jakimś cudem nic sobie nie połamał.
Lecz to nie był koniec kontrataku, lecący błąd ortograficzny prawie rozciął go na pół, Antyrax zdążył odskoczyć, lecz puścił miecz, który pękł na drobne kawałeczki, zmiażdżony przez ,,Ó''.
Kreatywność wsiąkła w błoto, tworząc nowe królestwo błotnych mrówko-androidów, zasilanych parą z wody basenowej.

Antyrax był teraz bezbronny, a bezbronnym człowiekiem żaden z demonów się już więcej nie interesował. Wstał i zobaczył, jak jego wioska właśnie jest równana z ziemią.
Cień Wielkiego Neofantasora, wychylającego się zza góry, wyglądał jak dłoń dziecka, które zagarnia klocki z ziemi, aby je zaraz obślinić i połknąć.
Czy to był już koniec dla niego i dla wszystkiego co znał?

\begin{dialogue}
\ds Nie, to nie koniec \dm pomyślał, sięgając po worek. \dm Jestem pisarzem i to ja ustalam zakończenia. 
\end{dialogue}

Wtedy jego lateksowym butom wyrosły śmigła, wspomagane programem w C, i poniosły go prosto pod nos Obudzonej, zostawiając za sobą linuksowy ślad.
Zaciekawiona demonica rozpoznała niedojdę, którego wcześniej zmiażdżyła.
Antyrax nie miał już żadnej broni, tylko worek. Sięgnął do niego i długo nie wyciągał ręki.
\begin{dialogue}
\ds{} Mam coś specjalnie dla ciebie, Obudzono.
\end{dialogue}

\divider{}

\begin{dialogue}
\ds{} Kolejny dzień w pracy, kolejny dzień robienia bezużytecznych czynności dla bezużytecznej korporacji w tym bezużytecznym świecie \dm{} pomyślałem.
\end{dialogue}

Wziąłem komórkę i zacząłem przeglądać maile.
Spam z reklamą talerzy, zaproszenie do znajomych na Facebooku, oczywiście od obcej osoby, powiadomienie o komentarzu na YouTube,
powiadomienie o mailu na innej skrzynce pocztowej, przypomnienie o opłacie za subskrypcję tej bezużytecznej gry komputerowej i tysiąc innych bezużyteczności.
O, znowu rozpętałem gównoburzę na Mirko i zablokowali mi konto Twittera za niepoprawną politycznie myślozbrodnię.
Bezużyteczność do kwadratu.

Z nadmiaru bezużyteczności zdrzemnąłem się w ubraniu na godzinę. Obudziło mnie głośne gruchanie z okna.
Czyżby jakaś użyteczność w końcu mnie spotkała?
Na oknie siedział biały gołąbek, w dzióbku trzymał kopertę.
Nie będąc pewnym, czy nadal nie śnię, odebrałem przesyłkę i przyjrzałem się kopercie.
Gołąb zaraz zniknął, oddalając się bezszelestnie.

Koperta była z papieru czerpanego.
Adresowana była elegancką kursywą do mnie, do Mateusza Mechalycznego, zamieszkałego na ulicy Szerokiej w Gdańsku.
Województwo Pomorskie, Polska, Ziemia, Układ Słoneczny, Galaktyka Droga Mleczna, Gromada Lokalna, Czwarta Kwadra.
Z tyłu widniała pieczęć z wosku pszczelego, wypukły obraz kuli i jakiś napis z niezrozumiałych znaków.
Przecież był 2017 rok, kto dzisiaj jeszcze wysyła papierowe listy, zamiast maili?
Zatem to był kawał. To musiał być kawał. Pytanie jeszcze, po co ktoś mi go wyciął?

Ostrożnie otworzyłem kopertę, trzymając ją obcęgami, spodziewając się że coś strasznego zaraz z niej na mnie wyskoczy.
Nic takiego się jednak nie stało.
Pisany odręcznie list był tym, co zwykle znajduje się w kopertach.

\curlyframe{
\begin{Fontlukas}
Szanowny Panie Mateuszu Mechalyczny.

Mam zaszczyt zaprosić Pana na uroczysty bankiet z okazji wyboru na jednego z przyszłych członków ALOPP.
To wielki zaszczyt, móc gościć nowych agentów tej organizacji na pokładzie mojej \weirdchar{kula}.
Mam najskrytszą nadzieję, że zostanie Pan przyjęty i będzie mieć Pan we współpracy z \weirdchar{monster} udział w walce o wspólne dobro.

Zapraszam do siebie dnia 20 października, roku Pańskiego 2017.
Myślę, że marina w Głównym Mieście Gdańska będzie doskonałym miejscem na lądowanie mojej kuli i tam się spotkajmy w lokalne południe.
Po obiedzie wybierzemy się w podróż na Felicję, gdzie pozna Pan swoich przyszłych, mam nadzieję, członków przybranej rodziny.

Przypominam, że w \weirdchar{kula}, oprócz najwyższej kultury osobistej,
od zawsze obowiązywał rokokowy styl ubioru.
Wszyscy goście powinni przywiązać najwyższą dbałość o szczegóły swojego wyglądu.
Uprzejmie proszę także, aby nie posiadać na pokładzie żadnych urządzeń użytkujących elektryczność.

Z Bogiem. \\
--- Profesor \weirdchar{profesor}
\end{Fontlukas}}

Kilka dziwnych znaków zostało wtrąconych pomiędzy litery. Zaczynało się robić ciekawie. Autor tego dowcipu chciał, abym za trzy dni, w XVIII wiecznym stroju pałacowym,
udał się w sam środek miasta w dniach szczytu, nie zabierając ze sobą żadnej elektroniki.
Potem tajemniczo miałem przejść tajemniczy test na zostanie tajemniczym członkiem jakiejś tajemniczej organizacji.
Trzeba dokładniej przestudiować ten tajemniczy list.

Szybkie szukanie Felicji w internecie wskazało jedną stronę o teoriach spiskowych.
Felicja miała być planetką, stworzoną przez kosmitów, na której hodowano ludzi, aby przeprowadzać na nich straszliwe eksperymenty.
Jeśli to prawda, może zabrakło im tam królików doświadczalnych i porywają kolejne ofiary?
Ale wtedy przecież nie dawali by mi wolnej ręki do odmowy.

Na tej samej stronie podano: ALOPP jest pozaziemską organizacją terrorystyczną, zrzeszającą ludzi w celu mordowania mieszkańców własnej planety.
Ale od czego był to skrót, to nikt nie wiedział.

,,Czwarta Kwadra'' dawała za dużo losowych wyników, aby wywnioskować z nich, o co mogło Profesorowi chodzić.

Lokalne południe w Gdańsku, czyli dwunasta godzina czasu słonecznego, uwzględniając jeszcze czas letni, to trochę przed trzynastą czasu strefowego.
Przyjdę o 12:00, najwyżej trochę poczekam.

Wikipedia natomiast wskazała, że gołębie pocztowe w żadnym wypadku nie mogłyby doręczyć listu bezpośrednio do odbiorcy.
Ich mechanika polega na wracaniu do macierzystego gołębnika, z dowolnego miejsca na świecie, i tylko tyle.
Listów z pewnością nie wsadzano im do dzióbka, a przywiązywało się je do nóżek.
Rozejrzałem się po pokoju, czy przypadkiem nie miałem w nim gołębnika, aby hodować gołębie pocztowe, ale nie.
Moja teoria o dowcipie zaczęła się lekko sypać.

Kilka razy, w różnych częściach świata, widziano w tym samym momencie kuliste UFO i ludzi w strojach rodem z Wersalu.
Podobno zdjęcie zrobione kuli nigdy nie wychodziło poprawnie, a większość ludzi magicznie zapominała o zdarzeniu chwilę po odlocie tajemniczej struktury.
Nieliczni pamiętali i rozpowiadali to dziwo, ale nikt im oczywiście później nie wierzył.
Kulę widywano w różnych miejscach, nie ograniczała się, jak na filmach, tylko do USA, przelatywała przez centra miast, pływała pod wodą, cumowała do Międzynarodowej Stacji Kosmicznej, straszyła samoloty, ślizgała się po biegunowych lodach i toczyła bitwy z wojskami wszystkich krajów świata.

Następnego dnia zabrałem list do znajomego chemika.
Potwierdził on moje obawy, list wykonany był oryginalną techniką sprzed kilkuset lat.
Skład chemiczny papieru i atramentu, odpowiadał tym, używanym dawno temu.
W dodatku narzędzie pisania z pewnością było ptasim piórem.
Na myśl o bezużyteczności otaczającego mnie świata, postanowiłem pojutrze zrobić coś użytecznego.

\divider{}

Pod naporem nietypowości i kreatywności dzieła, Obudzona zaczęła krzyczeć, zwijać się w konwulsjach i palić czarnym ogniem.
Zmieniła się w mały księżyc i poleciała, jak frisbee, z powrotem w kierunku Wielkiego Neofantasora.

Triumf Antyraxa nie trwał długo, za chwilę, od tyłu, złapała go Dedirid.
Jej czarna ręka owinęła się wokół delikatnego światostwórcy, jak czarny worek na zwłoki.
Poczęła go zacieśniać, niczym lekarz mierzący ciśnienie.
Zaraz wyciśnie naszego bohatera, jak tubkę pasty do zębów.
Wywijając się rybio, Antyrax zanurkował do worka i długo nie wychodził.
Demonica przez godziny nachylała się nad otworem, aby capnąć go jak tylko wystawi głowę.
Gdy tylko coś się wysunęło, porwała to z ochotą.
Była to jednak kolejna opowieść, zabójczo eksperymentalna, niesamowicie abstrakcyjna.
Nietypowość poparzyła jej łapska.

\divider{}

Mateusz wypożyczył wymaganą górę z wypożyczalni kostiumów.
Zastanawiał się, czy nie podkraść jakiegoś szustokora z muzeum, ale to chyba nie było by zbyt poprawne zachowanie.
Bał się, czy mierna jakość kaftana, spowodowana nieoryginalnością, nie będzie zwracać nadmiernej uwagi w świecie najprawdziwszych atłasowych pasów i perłowych guzików.
Postanowił kupić więc kilka ozdób ze sztucznej biżuterii, które wyglądały dość kosztownie, a stworzone były z
byle-czego, i doszyć w losowych miejscach. Miał nadzieję, że Profesor i inni goście nie zauważą różnicy.

Z pończochami nie było żadnego problemu, znalazł je w damskim sklepie.
Tak samo coś, co można było podciągnąć pod starodawną koszulę.
Musiała być flanelowa z wystającymi rękawami.
Ogarnął także puder.

Peruka, cóż. Przynajmniej znalazł za szafą trójkątną czapkę piracką po poprzednich lokatorach.
Najgorzej, że zazwyczaj chodził na łyso, gdyż rodzice nie obdarzyli go mocnymi włosami.
Potrzebował więc na szybko przykleić coś sobie na łeb.
Liczył w głowie, ile lat może dostać za kradzież peruki sędziemu, gdy spostrzegł wyprzedaż starych futer.
Używając magii nożyczek, kleju i starego mopa, wygenerował coś, co po przykryciu trójkątną piracką czapką wyglądało dość znośnie.

U zegarmistrza kupił za grosze kopertę zegarka, pustą w środku, z brakującymi wskazówkami, całość zaczepioną na łańcuszku.
Zegarek nie musiał działać, ważne aby był.

O dziwo, to buty przysporzyły mu najwięcej problemu.
Niby lakierki z klamrą nie są niczym bardzo skomplikowanym, a jednak nikt ich nie produkuje.
Może właśnie dlatego, że były modne trzysta lat temu?
Wpadł na pomysł, aby kupić coś podobnego i przerobić.
Znalazł buty dla zakładów pogrzebowych, gdyż tylko te odpowiednio się błyszczały, i przyszył im klamry od spodni.
Z daleka nie było widać różnicy.

W domu ubrał się i przejrzał w lustrze.
Połączenie Napoleona, Ludwika XIV i informatyka z Gdańska.
Muszą zrozumieć.

O jedenastej godzinie, owego wielkiego dnia, wdział pełny strój.
Nie mógł się przecież tak pokazać w mieście.
Pończochy zatem przykrył spodniami dresowymi.
Na elegancki szustokor nałożył nieco za dużą bluzę z kapturem.
Lustrzane lakierki przykrył jakimiś szmatami, żeby nie zwracały zbytniej uwagi.
Tylko pseudo-perukę schował do plecaka.

To nie mogło pójść tak łatwo.
Z daleka zobaczył kordon policji i wojska, stojący w Zielonej Bramie, blokował wstęp każdemu wychodzącemu z Długiego Targu.
Ucieszył się i kamień spadł mu z serca. Oznaczało to, że jednak nie padł ofiarą żartu.
Znalazł w końcu promyk użyteczności w oceanie bezużyteczności. Każda normalna osoba, wiedząc że wielka latająca kula-zapominajka wylądowała w centrum miasta,
ewakuował by się z niego jak najdalej. 
Mateusz jednak nie był normalny, i może dlatego właśnie został zaproszony na najbardziej nienormalną ucztę w świecie.

Do mariny spróbował dostać się okrężną drogą, przebiegł przez Krowi Most na Wyspę Spichrzów.
Klucząc uliczkami zbliżył się do portu, jednak tutaj też była blokada.
Widział w każdym bądź razie kawałek wody w basenie jachtowym, nietypowe fale odbijały się od brzegów, coś się tam jednak działo.
Popatrzył smutno w kanał i pomyślał, że chyba zostanie mu wskoczyć do wody i popłynąć wpław, pod mostem omijając strażników.
Ale przecież pewnie nie zostałby wpuszczony mokry do rakiety.
No i co z pudrem, który już sobie wcześniej pieczołowicie nałożył?
Głupi. Podziurawią go zaraz jak ser, gdy tylko zobaczą kogoś płynącego kanałem wpław.
To był koniec.

Szustokor był bardzo gruby, rozpiął więc bluzę żeby się nie ugotować, teraz wszystko było mu jedno, czy ktoś go zauważy.
Był tak blisko, a jednocześnie tak daleko. Wszystko miało prysnąć, jak bańka.
Czuł się jak rybka w siatce, wrzucona do oceanu.
Zaraz będzie widział swoją życiową porażkę, jak na dłoni.
Co robić? Co robić?

Wybawienie przyszło nieoczekiwanie.
Oto bowiem mama z małą dziewczynką podpłynęły do niego skuterem wodnym, oferując szybką podwózkę.
Myśląc, że to pomyłka, zdjął bluzę, pokazując swój strój w połowie okazałości.
Kobieta jednak nie uciekła, nie przestraszyła się dziwaka, tylko się uśmiechnęła.

\begin{dialogue}
\ds{} Chyba się teraz nie poddasz? \dm{} zapytała.
\ds{} Skąd... kim...
\ds{} Kula miał wielu gości. \dm{} Położyła rękę na piersi. \dm{} Jak tylko zobaczyłam, że wrócił do miasta, wiedziałam. Ktoś będzie potrzebować pomocy.
\ds{} Ja... \dm{} Mateusz milczał przez chwilę. \dm{} ha, ha. Prawie się nabrałem.
To niezwykłe, jak wynajęliście żołnierzy, żeby zastawili miasto specjalnie dla mnie?
\end{dialogue}

Tajemnicza osoba przewróciła oczyma, zdjęła swoją córkę na chodnik i zeskoczyła, wrzuciła klucze od skutera głównemu bohaterowi do kieszeni szustokora i poszła, nie odzywając się więcej, trzymając dziecko za rękę.

Mateusz w rokokowym stroju wersalskim zasuwał na skuterze wodnym kanałem Nowej Motławy, ozdoby szustokora mieniły się w pełnym słońcu tak samo, jak latające wokół niego
krople wody i stalowe kule, wystrzelone z mostu przez żołnierzy zabezpieczających
lądowanie wielkiej białej kuli w centrum miasta.
Schował się na chwilę pod mostem, jak zagoniony przez wilki królik w norze, a gdy wypłynął z drugiej strony, wtedy ją zobaczył.

Była wielka, jak budynek, wypolerowana, biała i doskonale kulista.
Dołem dotykała lekko powierzchni wody, tworząc promieniste fale.
Odbijała w sobie cały Gdańsk.
Mateusz zobaczył w niej malutkiego siebie na łódeczce-zabawce, malutkie budynki, żołnierzyków, spichrze, basenik, niebo, helikopter jak muchę i blask pełnego słońca.
Już nikt nie strzelał, już tylko wszyscy patrzyli. I bali się.
On się nie bał. Przyszedł tu na bankiet.
Przyszedł we francuskim stroju.
Przyszedł tu, bo został zaproszony.

Zszedł ze skutera na pomost i poprawił perukę, wtedy też właz w dolnej części zaczął się otwierać.
Tak jak się spodziewał, był to dźwięk szczęku łańcuchów, a nie elektrycznego silnika.
Ze środka powiał zapach kurzu, wosku i lekkiej stęchlizny.
U dołu rozwinął się elegancki, czerwony dywan.
W przejściu stanął On.
Nosił strój wspanialszy, niż Mateusz mógł sobie kiedykolwiek wyobrazić, tak inny od jego własnego, a przecież z tego samego okresu historycznego.
Przy jego ozdobach, sztuczna biżuteria Mateusza rzeczywiście wyglądała na sztuczną.
Jego najprawdziwsza peruka przyćmiła wielkością cały futrzany twór z głowy gościa.
Lakierki błyszczały się tak samo, jak jego statek kosmiczny z którego wyszedł.
W ręku trzymał laskę z białą kulką, pomniejszoną wersją tego, co znajdowało się tuż za nim.

\begin{dialogue}
\ds{} Jestem Profesor Kula. Miło mi pana gościć na moim statku.
\end{dialogue}

\divider{}

Antyrax wyszedł z worka, gdy z Dedirid została już tylko kupka popiołu.
Jeszcze ośmiu.
Tym razem demony nie bardzo chciały go atakować.
Antyrax więc wskazał jednego z nich palcem, niczym sędzia nowoskazanego na śmierć.
Lenna. Pora na mikropomysły.
Sięgnął do worka i od razu złapał to, czego szukał.

\divider{}

\begin{dialogue}
\ds{} Waćpan Mateusz Mechalyczny, jak mniemam \dm{} powitałem gościa. \dm{} Waćpan Mateusz Mechalyczny niepewnie, acz żwawo podszedł, ukłonił się, i schował za framugą włazu,
znikając przed przeszywającym wzrokiem miasta.
\ds{} Proszę wybaczyć mi mój ubiór i maniery, Panie Profesorze Kula. \dm{} Ukłonił się ponownie, prawie do ziemi. \dm{}
Musiałem przedrzeć się przez kordon wojska i ominąć grad pocisków, aby przybyć do pańskiego statku.
\ds{} Nazywam się Kula, mości Mateuszu Mechalyczny, nie Kula \dm{} poprawiłem, zamykając korbą właz. \dm{} A ta kula nazywa się Kula. 
I nie jest jakimś statkiem kosmicznym, a Kulą.
\ds{} Kula... \dm{} niepewnie odpowiedział.
\ds{} Nie Kula, Kula. Moje nazwisko, nazwa tego miejsca, typ urządzenia, i bryła geometryczna. Kula, Kula, Kula i kula. To trzy różne słowa, zupełnie inaczej wymawiane.
Zupełnie inaczej zapisywane.
\end{dialogue}

Mateusz Mechalyczny podrapał się po głowie, ścierając puder.

\begin{dialogue}
\ds{} Ignoruj go, on mówi i słyszy na częstotliwościach poza zakresem naszych uszu. \dm{} Katarzyna Kosmata zjechała po falistej poręczy schodów i przysunęła do nas, nawet się nie witając.
\dm{} Jestem Kasia, cześć.
\ds{} Droga Katarzyno Kosmata! \dm{} skarciłem ją. \dm{} Maniery! Niech panna nie prezentuje złego przykładu naszemu gościowi. Panie Mateuszu, mam zaszczyt przedstawić panu...
\end{dialogue}

Gość jednak utopił wzrok w olbrzymiej sukni Katarzyny, nachalnie gapiąc się na każdy jej detal.
A już się radowałem, że chociaż on będzie potrafił tu zachować maniery. Nadaremnie.
Najgorsze w tej sytuacji było to, że ona sama wręcz go do tego zachęcała. 
Zamiast się przedstawić, zaczęła tłumaczyć skąd i jaka część ubioru pochodzi.

Najpierw zawiesił oczy na jej biuście, wodząc źrenicami to w lewo to w prawo.
Najprawdopodobniej podziwiał plecionkę z anielskich włosów, którą obszyta była góra.
Anielskie włosy są całkowicie przezroczyste, gdy odpadną od właściciela, więc
aby stworzyć ten element ubioru, trzeba było prawdopodobnie zbierać je z niebiańskich podłóg w całym raju.
\begin{dialogue}
\ds{} Trafiłam przez to na dywanik anielskiego ministra poprawnego zachowania, chciał to podciągnąć pod brak szacunku dla zarządu Nieba, ale wybroniłam się tym, że wszyta w suknię świętość
będzie dodatkowo ochraniać mnie przed demonami, czy jakoś tak.
\end{dialogue}

Następnie zszedł niżej, aby przyjrzeć się lepiej pasu.
Katarzyna gustowała się w nieprawdopodobnie kosztownych ubiorach, lecz jej pas był wykonany ze zwyczajnego, ziemskiego jedwabiu.
Może chciała tym pokazać, jakoby jej suknia była w równym stopniu wykonana ze składników z calutkiego wszechświata?
\begin{dialogue}
\ds{} Ten jedwab pochodzi od jedwabników karmionych nektarem jedynie z najrzadszych gatunków orchidei, podlewanych krystaliczną wodą źródlaną z Himalajów
\dm{} wyjaśniła cały sekret.
\end{dialogue}

Po pasie, przyszedł czas na szyję. Katarzyna założyła tym razem kolię ze zmutowanych pereł Khaliniskali...
czy to była Rezurma? Nie pamiętam, kto ostatnio przejmował stolicę i nazwę tej przeklętej... Planety Wojny, jak ją wszyscy nazywają.
Kulki mieniły się i pulsowały wszystkimi kolorami tęczy. Od podczerwieni, po nadfiolet.
Te perły można było znaleźć tylko w małżach, żyjących w skażonym jeziorze, na północy pustynnego kontynentu Terb. Chwila, teraz to już nie był już Terb... no na północy tego największego kontynentu planety.
Zdaje się że to albo Czarna Armia, albo Komodowa utopiła tam kiedyś zbiorniki z kancerogennym żelem, w celu wewnętrznego wyniszczenia przybrzeżnego miasta Hirten... wtedy to było Hirten.
Nie udało się, mieszkańcy Hirten wyczuli podstęp i zamiast umrzeć na nowotwory, od picia skażonej wody, poumierali z pragnienia.
W każdym razie flora i fauna w jeziorze przeszła nieprzyjemne zmiany fizyczne.

Buty. Był to wspólny wytwór czterech Khrnzrhkh.
Najpierw poprosiła Chrrkrhkrrkk o stworzenie lodowej podstawy.
Potem pewnie Iłiścirr obudował to swoją czarną rkkizniisi, Buffsirr dodał czerwone wkładki z buffzerda, a Fluszszrisss utwardził ogniem.
\begin{dialogue}
\ds{} Te buty zostały zrobione przez potworów, najpierw Mikołaj stworzył lodową podstawę, Psychit zalał ektoplazmą, Pyrroq dodał czerwone klejnoty-bomby, a Plazma utwardził ogniem. \dm{}
Katarzyna Kosmata właśnie spowodowała, że kolejna osoba będzie nazywać Khrnzaalk potworami, zamiast porządnie w ich własnym języku.
\end{dialogue}

Zahaczył o wachlarz.
Ten był stworzony z półprzezroczystych łusek białego cyrkowca.
Te smoki wyginęły doszczętnie po ataku czerwonych kartaczy na ich wyspę.
Właściwie, jedyne pozostałe cyrkowce można teraz znaleźć w zoo w Capitalu.
Szkoda ich, miały wspaniałą kulturę.
Cyrkowe baśnie do dziś opowiada się małym smoczkom do legowiska, a cyrkowi malarze są niedoścignionym przykładem talentu w wielu kulturach.
Kartacze to zwykłe zwierzęta. Żeby chociaż te barbarzyńskie pasowce ich rozbiły, ale nie. Największy i najbardziej bezmózgi gatunek smoków zaatakował, rozszarpał i pożarł najwspanialszych.

Ostatecznie Mateusz popatrzył się na jej twarz.
Nie, nie na twarz, a na makijaż.
Oczywiście, Katarzyna Kosmata nie mogła spocząć na wyrywaniu łusek prawie wymarłym smokom.
Jej puder był stworzony ze zmielonych ciosów mamuta.
Jak ona odkopała je z syberyjskiego błota i wybieliła, tego nie wiem.
\begin{dialogue}
\ds{} Przekonałam Chronosa, jednego z potworów, żeby odwrócił trochę czas i przywrócił im świeżość.
\end{dialogue}

Nikomu się nie udało przekonać kiedykolwiek Pfiishuss do jakiegokolwiek używania swojej mocy! Jak ona to zrobiła?

To jednak nie był koniec podziwiania, poszedł wzrokiem wyżej.
Fryzura Katarzyny była przeogromna. A wszystko to z naturalnych włosów. Wiem na pewno, że Floria... znaczy Hhurnna przywiązuje podobną uwagę do wyglądu, co Kosmata. Na pewno nie odmówiłaby Katarzynie podkręcenia jej cebulek włosowych w celu ich przyspieszenia. Ale w sumie Chronos także mógł to zrobić.
\begin{dialogue}
\ds{} W Rossmanie sprzedają taki super szampon do włosów. Nic więcej nie potrzeba. Może na twoje też pomoże. 
\end{dialogue}

Na jej fryzurze osiadły wielobarwne motyle. Co jakiś czas, któryś wzbijał się w powietrze, robił pętlę wokół jej głowy i lądował z powrotem.
Były to najprawdziwsze motyle, hodowane i tresowane w tajnej placówce pod motylarną w Burggarten.
Ciekawe, jak je zdobyła. Znając Katarzynę, pewnie jak gdyby nigdy nic weszła przez tajne wejście, w tej pełnej sukni, z naładowanym szyfratorem w ręce, i powiedziała:
,,Dajcie mnie tych tresowanych motyli na głowę, bo zaraz mam bankiet we wielkiej, latającej kuli.''
Być może tylko po to w ogóle przyjechała dzisiaj do Wiednia.
Przyjechała po motyle, i żeby przyprawić o zawał serca całą Austrię.
Więc tym razem postanowiła wsiąść do Riesenrad i pojechać wagonikiem na sam szczyt, gdzie wcześniej specjalnie umówiła się ze mną, abym podleciał po nią Kulą.
Oczywiście, jak tylko przyleciałem, wybuchła panika. Koło się zatrzymało, uwięzieni w wagonikach ludzie próbowali uciekać po konstrukcji koła, 
przed wielką białą kulą, cumującą właśnie do najwyższej budki. Katarzyna otwarła drzwiczki i robiąc krok nad przepaścią, weszła do pojazdu.
Pomachała wachlarzem pozostałym, ledwo żywym ze strachu osobom w budce, i odlecieliśmy.
Następnym razem pewnie stanie na szczycie Empire State Building, a ja będę robił za King Konga.
I też będę potem uciekał przez myśliwcami.
Czy można się uzależnić od amnezji, którą pokryty jest statek?
Uzależnić od siania paniki w ludziach, którzy i tak za chwilę o wszystkim zapomną?

Przeczyściłem gardło.
\begin{dialogue}
\ds{} Znaczy... witam... bardzo mi miło, dzień dobry... eee... Kasiu-ażyno. \dm{} Stał bez ruchu kilka pulsów, aż zdecydował się delikatnie ująć jej dłoń i pocałować.
Ważne, że się starał. \dm{} Ja Mateusz... jestem.
\end{dialogue}

Katarzyna zarumieniła się. Widać zostało w niej jeszcze trochę kultury osobistej. A może to był makijaż?

Mateusz dostał oczopląsu, jego wzrok skakał od ozdoby do ozdoby. Każdej falki, każdego wgłębienia musiał dotknąć, niczym sprawdzając czy rzeczywiście wykonane są z hebanu i masy perłowej. 
Poprowadziłem ich po schodach do salonu, w którym nakryty był już stół dla czterech osób.
Gość aż przysiadł z wrażenia.
\begin{dialogue}
\ds{} Na naszym bankiecie spodziewamy się w sumie trzech gości \dm{} oznajmiłem zgromadzonym. \dm{} Zatem dołączy do nas jeszcze jedna osoba.
Będzie to Nadar Nocny, który aktualnie bada, albo szabruje, wrak Titanica. Podróż potrwa około dwa i pół kilopulsa, to jest niecałe dwie godziny.
\dm{} Poprawiłem żabot. \dm{}
Pan Nocny jest dość... ekscentryczny. Dla jednych jest najlepszym przyjacielem, a inni go nienawidzą.
Szczerze powiedziawszy, nie popieram jego charakteru, ale obawiam się że może pan, panie Mateuszu, naleźć w nim bratnią duszę.
\ds{} No dobrze, gdzie są ukryte kamery? \dm{} Mateusz niespodziewanie wypalił.
\ds{} Proszę pana, zaręczam, że w całym tym miejscu nie znajduje się ani jedno obrzydliwe elektroniczne urządzenie. Ta strefa jest wolna od nieprzyjemnych pól magnetycznych i elektrycznych.
\ds{} On chyba nadal nie wierzy, Profesorze. \dm{} Katarzyna zaproponowała. 
\ds{} Nadal nie wierzy w Kulę? Pomimo, że sam w niej stoi? \dm{} Uśmiechnąłem się. \dm{} Nigdy nie widziałem takiego zaparcia przy obronie własnych idei. Panie Mateuszu, wierzę, że będzie pan wspaniałym agentem.
\end{dialogue}

Położyłem rękę na lasce. Lekko ścisnąłem małym i wskazującym palcem, aby obniżyć lot.
Następnie przycisnąłem w dół otwartą dłonią, aby pokonać siłę wyporu.
Zaczęliśmy się wtedy zanurzać coraz głębiej i głębiej w Atlantyku, zostawiając za sobą pióropusz tęczowych rozbryzgów.

Tymczasem zacząłem oprowadzać naszego gościa po Kuli.
Wycieczkę rozpoczęliśmy, wracając do głównego włazu na najniższym piętrze.
Ta otwierana w dół, mająca od wewnątrz kształt schodów, wykrzywiona płyta, była jedyną, niepokrytą czerwonym futrem, częścią pancerza.
Zamiast tego posiadała czerwony dywan i wysuwaną poręcz, automatycznie rozwijane przy kontakcie z podłożem.
Operowana za pomocą skomplikowanego systemu łańcuchowo-sprężynowego na korbę.

Nie zmieniając piętra, przeszliśmy do garderoby.
To właśnie tutaj trzymałem awaryjne suknie, habity, koszule i trzewiki, w razie gdyby któremuś z gości zdarzyło się nie posiadać wystarczająco odświętnego ubioru do uczestniczenia w uczcie.
Mateusz zwrócił mi uwagę na grube kombinezony, wiszące w kącie. Wedle jego wizji, były to stroje nurkowe.
Wyjaśniłem, że pomimo mylącego dla niektórych wyglądu, w rzeczywistości nadawały się zarówno do nurkowania w oceanie, jak i w próżni kosmicznej.
Są integralną częścią Kuli, wyjaśniałem, trochę jak ściany i meble. Czerpią z niej energię do podtrzymywania życia. 
Będąc w takim kombinezonie, nigdy nie zabraknie ci tlenu i pożywienia.
Na szczęście nie zauważył, iż jeden z haków był pusty. Nie chciałem się tłumaczyć, że zgubiłem kawałek wyposażenia swojej rakiety.

Zapytany o śluzę ciśnień, aby bezpiecznie wychodzić na zewnątrz, opowiedziałem mu o niewidocznej tarczy rozciągniętej na włazie, chroniła ona wnętrze przed różnorakimi hazardami zewnętrznymi, takimi jak próżnia, uniwersalność, demony, czy brak kultury osobistej.
Nie był przekonany, więc kręcąc jeszcze raz korbą, otworzyłem ponownie właz. 
Płynęliśmy aktualnie tuż przy samym dnie morskim, zostawiając za sobą chmurę wzburzonego piasku.
Falista, lekko wypukła powierzchnia wody, utworzyła się na głębokości framugi. 
Mateusz z niedowierzaniem zamoczył rękę w głębiach oceanu, wyciągając garść osadzającego się piasku.
I meduzę.

Na kolejnych piętrach znajdowały się pokoje gościnne. Gość uprzejmie podziękował za pokój, ale nalegał, abyśmy szli dalej.

W centralnej części Kuli znajdowała się łaźnia, muzeum i mój gabinet. Do tego ostatniego nikogo nie wpuszczam.
Ludzie snują różne domysły na temat tego, co znajduje się za dębowymi drzwiami. 
Zasilanie całej Kuli, mój zwyczajny pokój, jakieś kosmiczne artefakty, prawda o moim pochodzeniu?
Nikt z nich nigdy nie miał racji, a ja nikomu nigdy prawdy nie pokażę.

W wyłożonym terakotą pomieszczeniu panował standardowy zaduch. Gość zdziwił się niemiłosiernie, znajdując tutaj basen, jacuzzi, saunę fińską, masażery wodne, a także mały wodospad.
Pośrodku stał wielki piec na węgiel. Bez niego zimna pustka kosmosu szybko by nas dopadła.
Mateusz powiedział, że w XVIII wieku nie używano łaźni i że po stylu wnętrza spodziewał się co najwyżej wychodka w kącie. 
Zaśmiałem się na myśl, iż wziął Kulę za stuprocentowy wycinek pałacu w Wersalu.
Kultura idealna nie istnieje, zacząłem wykład, z każdej należy wyciągnąć najlepsze części. 
I tak, łącząc na przykład rzymskie starożytne łaźnie, francuski nowożytny wystrój, średniowieczne królewskie dania i słowiańską mowę przyszłości, 
stworzyłem tą właśnie latającą wyspę kultury idealnej.

Najciekawsza część statku teraz.
Moje muzeum zawiera artefakty z różnych zakątków wszechświata. Gość zapytał o wartość zebranych przedmiotów.
Nie wszystko da się sprowadzić do liczby pieniędzy, dałem mu wykład, nie wszystko ma tak zwaną cenę. 
Jeszcze się o tym nie raz przekonasz.

Wskazałem skałę przyczepioną widełkami do podstawy. 
To na przykład jest kawałek meteorytu, który uderzył w księżyc planety Tos. Wartość tego kamienia jest równoważna wartości losowego polnego kamienia z Ziemi, 
znajduje się tutaj ze względu na historię, jaką ze sobą niesie.
Otóż, uderzenie meteorytu było tak silne, że wybiło księżyc z orbity, popychając go w kierunku Tosa.
Po stu latach ciągłego zbliżania się do powierzchni, w końcu satelita zahaczył o atmosferę, gwałtownie zwolnił i zderzył się z planetą.
Każdy organizm, większy od jednokomórkowca, został zniszczony.

Co ciekawe, mieszkańcy tego świata byli na tyle rozwinięci naukowo, że doskonale widzieli i rozumieli zbliżającą się katastrofę.
Jednak nadal za mało rozwinięci technologicznie, aby móc jej uniknąć.
Przewidzieli dzień swojego końca co do dnia, a koniec rzeczywiście nastąpił.

Tak, wiem że to smutne, ale cóż począć? Gorsze rzeczy zdarzały się w zbiorowej historii życia. 
Tylko pierwotne grzyby przetrwały katastrofę.
Toksyczna atmosfera, brak słońca i wysoka temperatura post-apokaliptycznego świata wręcz przyspieszyły ich ewolucję.
Na przykład, tutaj masz dziób takiego latającego ptakochomora. To grzyb i ptak jednocześnie, ładnie świeci w ciemności.
Da się go spożywać, niestety nie jest bardzo wysublimowany w smaku.

Przeszliśmy dalej. Kamień z lodowej strony Kryonii, nic niezwykłego. 
No może poza tym, że musiał być wydobyty spod kilku kilometrów litego lądolodu.
Co w Kryonii jest takiego wspaniałego? Obraca się ona wokół swojego słońca jak Księżyc wokół Ziemi. 
Wiecznie zwrócona tą samą stroną.
Na Kryonii nie ma zatem dni oraz nocy, a gwiazda zawsze jest w tej samej części nieba.
Nocna część jest lodową pustynią, dzienna ma pośrodku wiecznie szalejące tornado.
Może kiedyś zobaczysz Pałac Nadiru, położony w centrum wiecznej zmarzliny, jest przepięknym dziełem sztuki lodowej.
Wielka iglica z kryształowych łuków, kopuł, balkonów i kolumn.
Podświetlona trytowym światłem na przeróżne kolory.
Freon, wielki lodowy król Kryonii rządzi swoim państwem dobrze i sprawiedliwie.
Szkoda, że część jego ludu tego całkowicie nie rozumie. 
Demokraci, socjaliści, libertarianie, i reszta niepoliczalnych ruchów społecznych chce go dosłownie zwalić z tronu i pogrążyć cywilizację w chaosie.

\begin{dialogue}
\ds{} Skąd wie pan, że byłoby gorzej, niż jest teraz? \dm{} zapytał.
\ds{} Może kiedyś zostanie waćpan zaproszony przez Freona i wtedy, na własne oczy zobaczy pan, że na pewno nie byłoby lepiej, niż jest teraz \dm{} odpowiedziałem. \dm{}
Zresztą, prędzej czy później to i tak się pewnie stanie. Freon się starzeje i nie znalazł jeszcze na swój tron godnego następcy. Nikt inny nie może go zastąpić.
Więc albo rozkaże wybrać kogoś głosem ludu, albo znajdzie kogoś godnego spoza planety. To mogłoby doprowadzić do wojny domowej, 
rozumiecie, nikt nie chciałby być rządzony przez obcego kosmitę z kosmosu, nie ważne jak dobrze by rządził.
\ds{} Jak spoza planety? 
\ds{} To jedna z tych cywilizacji, zwanych zapoznanymi. Na tyle rozwinięta technologicznie i kulturalnie, przede wszystkim kulturalnie, 
że ma dostęp do warstw wszechświata. Warstwy to takie jakby obszary ,,pod'', ,,nad'' i ,,z boku'' czasoprzestrzeni
Pozwalają na szybką i dowolną podróż w każde miejsce, do każdej galaktyki, do każdego układu, używając minimalnej ilości paliwa.
\ds{} Jak to? Czyli taka na przykład Kryonia może w każdej chwili przelecieć jakąś warstwą i zaatakować Ziemię? 
\dm{} Zląkł się. \dm{} I czy Ziemia także jest zapoznana?
\ds{} Powiedziałem, rozwinięta kulturalnie cywilizacja. Czy waćpan jest absolutnie pewien, że ludzkość nie zaatakowałaby obcej planety, gdyby lot do niej byłby tak trudny, jak do Księżyca?
No właśnie. Poza tym, Ziemi bronią jeszcze Khrnzrhki.
\ds{} Kto? 
\ds{} Potwory \dm{} westchnąłem. I ja też się w końcu poddałem. \dm{} Robią za policję wszechświata, dbają o pokój na wszystkich zapoznanych planetach i poza nimi. 
ALOPP, do którego waćpan został zaproszony, to skrót od Akademii Ludzkiej Otoczonej Protekcją Potworów. 
Jako agent Akademii, będziesz im waść pomagał, będziesz dbał o pokój we wszechświecie, zatrzymywał wojny, walczył ze złem w różnych postaciach. 
To niebezpieczna i bardzo ciężka praca, ale jakże ciekawa.
Raz będziesz uciekał na skuterze grawitacyjnym, przed stadem czerwonych kartaczy, a innym razem zasiądziesz w sali obradowej Pałacu Nadiru. 
Znaczy, oczywiście jeśli okażesz się godny.
\ds{} Godny?
\ds{} Chodzi o charakter. Do ALOPP należy odpowiedzialność przed przyszłością. 
Każdemu agentowi może zdarzyć się stanąć przed wyborem decydującym o milionach istot. 
Dlatego kandydaci są poddawani testowi osobowości, żeby mieć pewność że zawsze wybiorą większe dobro. 
Test ma kilka faz, sprawdza reakcję na zaistniałe sytuacje.
\ds{} Kilka faz? Jak mam je pozdawać? \dm{} Zaczął panikować.
\ds{} Spokojnie, pierwszą ma pan już za sobą. Przecież jest pan tutaj z nami. 
Pierwsza faza sprawdzała reakcję na abstrakcyjne sytuacje.
Można było wyrzucić ten list, można było zgłosić go władzom, można było pokazać w internecie, a można było, jak pan, potraktować go poważnie.
Jest test sprawdzający zaangażowanie, posłuszeństwo, wykonywanie rozkazów, siłę psychiczną itp.
\end{dialogue}

Kontynuowałem oprowadzanie.
Zapytałem go, czy potwierdzi, iż wskazany przeze mnie kawałek zegarkowatych mechanizmów wygląda intrygująco.
Poleciłem zgadnąć, co to było, podpowiedziałem jakoby to nie był żaden zegar.
Nie zgadł, jak mógłby zgadnąć?
\begin{dialogue}
\ds{} Otóż, jest to mózg reprezentanta pewnej wybitnie nieprzyjemnej nacji robotów.
I mówiąc roboty, nie mam tylko na myśli ludzików zasilanych na prąd, jak to się przyjęło w ziemskiej kulturze.
Mam na myśli wszelkie żywe istoty zbudowane z nieżywych składników. Pozornie zwykła materia, lecz natchniona myślą.
\ds{} Proszę mi wybaczyć, ale wygląda dla mnie trochę, jak kupka śmieci.
\ds{} Bo nią jest!
Te... struktury, powstały z ludzkiego złomu, jako sztuczne ciała dla głodnych demonów.
To jest tak, że niektóre demony są za słabe, aby pożywiać się ciałami prawdziwych istot, zatem muszą się zadowalać martwą materią, najlepiej tą, towarzyszącą ludziom przez jak najwięcej lat ich życia.
\dm{} Położyłem rękę na gablotce ze szkła wymiarowego. \dm{}
Ludzki złom. Wszystko, co ludziom w czasie życia było tak bliskie, jak własne części ciała, ale jednak nadal sztuczne i wymienne. 
Protezy kończyn, sztuczne szczęki, wózki do poruszania się, kule do chodzenia, rozruszniki serc, inhalatory, tego typu rzeczy.
Te roboty mogą więc także składać się z metalu i elektroniki, ale nie są zasilane energią elektryczną, lecz szatańską!
\end{dialogue}
Na te słowa Mateusz zrobił krok w tył.
\begin{dialogue}
\ds{} Szatańską \dm{} powtórzyłem z grodzą. \dm{} Szatańska opętana kupa śmieci.
Powstały, jako wcielenie najczystszego zła, zasilanie parą z palonych zwłok, stworzone z ludzkich odrzutów, zlepione na ślinę i cyrograf.
Może pan się przyjrzeć, ten element przykładowo, jest wykonany ze sztucznej szczęki.
\ds{} Dziwna ta szczęka.
\ds{} Nie, no. Nie ludzkiej sztucznej szczęki, czy ludzie mają po dziesięć półkolistych zębów, jak te tutaj?
Wiele istot we wszechświecie ma przecież zęby i większość z nich, tak jak ludzie, czasami potrzebuje sztucznych.
\end{dialogue}
Mechalyczny przysunął się z powrotem, chociaż nie krył obrzydzenia. Ciekawość brała górę.
\begin{dialogue}
\ds{} To rurka, od kuli od podpierania się, tym razem ludzkiej kuli. 
Właściciel był jakimś wielkim gangsterem, skazali go bodajże za morderstwo na własnych dzieciach, powiesił się w więzieniu, oczekując na śmierć.
Im większy grzesznik, tym dla takiego demona smaczniejszy.
A to jest wężyk, który był kiedyś w rozruszniku do serca... smoka.
\ds{} Mam wrażenie, że wciąż się porusza. To znaczy że nadal żyje?
\ds{} Absolutna racja, nadal żyje, lecz akurat nie pamiętam imienia demona, który go zasila.
\ds{} A... a to nie jest trochę niebezpieczne go tutaj trzymać?
\ds{} Tylko trochę. W najgorszym razie, w razie ucieczki, i tak pierwsze co by ten demon zrobił, to czmychnął jak najdalej od tego świątecznego miejsca. 
Poza tym, jest zamknięty w gablocie wymiarowego szkła, przez wymiarowe szkło nic się nie przebije.
\end{dialogue}

%NOTE Pogłębienie

Powiodłem wzrokiem tam, gdzie wskazywał Mateusz.
\begin{dialogue}
\ds{} Ta nie jest ani atomowa, ani termojądrowa, zwyczajna na proch. \dm{} Otrzepałem kurz ze starego, pokrytego wyschniętym smarem pocisku. \dm{} 
Kiedyś zadarliśmy trochę za bardzo z wojskiem Stanów Zjednoczonych. 
Mocno nadszarpnęli nam ochronne powłoki i w końcu ta mała bombka przebiła się przez pancerz i wpadła prosto do pieczonego dzika.
Było blisko, gdyby wybuchła, długo bym musiał czyścić ściany z resztek jedzenia.
\ds{} Czyli goście przeżyliby wybuch?
\ds{} W Matrycy jest zapisana gwarancja na odbudowanie ciała, w razie gdyby coś mu się stało na pokładzie. 
Innymi słowy, nie można tutaj umrzeć, gdyż twoje ciało zaraz zostałoby automatycznie odbudowane.
Również dusza nie ucieknie, jest zamknięta tutaj jak w śnieżnej kuli.
Z definicji śmierć nie jest utratą ciała, lecz duszy. Dusza ucieka, gdy nie ma ciała do zamieszkania. W związku z tym szybkie odbudowanie ciała przyjmie duszę z powrotem.
\ds{} Po moim doświadczeniu z wojskiem w Gdańsku, widzę, że Kula często jest atakowana.
\ds{} Prawie zawsze, gdy składam wizytę na Ziemi \dm{} powiedziałem oczywistość. \dm{} Piekielni amerykanie. 
Zbudowali swoje pociski z żelaza, wydobywanego przez niewolników w Afryce.
Wypełnili je prochem wyciągniętym z fajerwerków, które miały być wystrzelone na szatańskie święto Halloween. 
Na koniec pokropili zapalniki krwią z abordowanych dzieci.
Takie coś znacznie prościej przebija kadłub stworzony z sacroterii.
\end{dialogue}

Przy okazji, wytłumaczyłem mu ochrony zastosowane w tym latającym pałacyku. 
Były trzy powłoki, z tym że trzecia to już fizyczny pancerz z sacroterii pokrytej amnezją.
Pierwsza powłoka zatrzymuje wszystkie szybko poruszające się obiekty.
Druga chroni przed naporem niepożądanych substancji, już ją widziałeś jak blokowała oceaniczną głębię przed wdarciem się do środka.

Mateusz zapytał się, jak to możliwe, że nie ma żadnych informacji o Kuli, pomimo że ląduje jak gdyby nigdy nic w centrach miast.
\begin{dialogue}
\ds{} Dlaczego ludzie zapominają... to dobre pytanie.
Otóż Kula pokryta jest amnezją. Każdy, kto na nią spojrzy, nawet pośrednio, zapomina.
Pamiętają tylko ci, którzy wierzą. 
Wierzą w Profesora Kulę, wierzą w bankiety w niebiosach, wierzą w złocone wnętrze.
\dm{} Objąłem rękami całe otoczenie. \dm{} 
Na pewno nie będzie to dla pana zaskoczeniem, że większość osób uważa Kulę zwykle za balon meteorologiczny, fatamorganę, dowcip, sztuczkę magiczną, nowoczesny samolot wojska, itp.
Pan uwierzył w prawdę, dlatego pan tutaj jest.
\end{dialogue}

Wycieczkę przerwał dźwięk otwieranego włazu i chlapanie wody.
Poszliśmy zatem przywitać trzeciego gościa. 
Mateusz był bardzo podekscytowany i pobiegł przodem.

\divider{}

Lenna popatrzyła się na Antyraxa i pokiwała w aprobacie głową.
Potem sama skierowała swoje kroki w kierunku Neofantasora.

Demony były już chyba przerażone, gdyż teraz poczęły wszystkie uciekać.
Jednak Antyrax był szybszy. Złapał jednego z nich za nogę (a właściwie to jego lateksowy but złapał nogą nogę), przyciągnął do siebie, i włożył mu swój worek na głowę.
Piotr Lekter zaczął się dusić, trująca abstrakcja wgryzała się w jego demonowe płuca, a lateksowy but, z siłą wolnego oprogramowania, ściskał mu szyję.

\begin{dialogue}
\ds{} Dość, wystarczy. Dam ci te dwie gwiazdki! \dm{} Z worka słychać było jedynie stłumione jęki. \dm{} Trzy! Niech będą trzy gwiazdki. I komentarz.
\end{dialogue}

Antyrax jednak nie odpuszczał. Ruchy Piotra Lektra stawały się coraz wolniejsze i wolniejsze.

\divider{}

Nadar. Czemu to akurat jego musiał ten Kula zaprosić?
Planowaliśmy eleganckie przyjęcie, a ta niewychowana świnia pewnie pociągnie w swoje odmęty i Mateusza.

Jak tylko usłyszałam szczęk łańcuchów, przerwałam robienie makijażu, i wystawiłam głowę z pokoju.
Zobaczyłam nowego gościa, zbiegającego po schodach do szatni, biegł tak szybko, że spłoszył mi motyle z głowy.
Nie spieszyło mi się powitać Nadara równie prędko, ale ciekawiło mnie zobaczyć reakcję Mateusza, gdy po raz pierwszy zobaczy tego szaleńca.
Z trudem przecisnęłam się w tej sukni przez drzwi na korytarz i ostrożnie podeszłam do pierwszego schodka w dół.
Oczywiście suknia zahaczyła o balustradę, od razu zrobiłam hyc i resztę drogi koziołkowałam, wywijając podwójne salto, lądując na głowie, z nogami majtającymi się w powietrzu.

Wiedziałam, co zaraz usłyszę i nie zawiodłam się.

\begin{dialogue}
\ds{} Ale dupa, co nie? \dm{} Nadar zamykał korbą właz, gapiąc się na moje machające w górze nogi.
\ds{} No, nawet... \dm{} Mateusz okazał się równie niewychowany. Nie wierzę, że się z nim zaprzyjaźniłam.
\end{dialogue}

Zaraz przybiegł Kula i pomógł mi się postawić do pionu. Był czerwony ze złości.
Ale czy dlatego, że właśnie ze statku z sykiem uchodziła kultura, czy dlatego że świństwo uzyskało nowego członka?

\begin{dialogue}
\ds{} To ty! \dm{} Kula trzymał laskę w górze, niczym śmiercionośny laser krojący Nocnego na pół. \dm{} To ty wziąłeś czwarty kombinezon z mojej garderoby! 
Szukałem go po całym wszechświecie. Kombinezon jest integralną częścią Kuli, generuje go Matryca tak samo, jak meble, dywany i ozdoby.
Jest niereplikowalny. Nie wolno go zabierać!
\ds{} Przecież nie zabrałem, a pożyczyłem. Zresztą i tak zawsze się kurzył w tej twojej półkulistej szafie. \dm{}
Nadar uznał to za wystarczające wytłumaczenie, rozpiął strój. Pod spodem miał swoje standardowe dresy. \dm{} 
A na przeprosiny mam prezent. Wyłowiłem ci, Profesorze, zestaw kieliszków i butelkę najdoskonalszego wina, prosto z kapitańskiego mostka.
Mieli ją wypić na ukończony rejs, ale wiadomo co się stało. Niech więc Kula ukończy swoją własną wyprawę i nie uderzy w żadną lodową kometę po drodze. \de{}
\end{dialogue}

Profesor Kula w jednym pulsie zmienił się z czerwonego z powrotem w białego.

\begin{dialogue}
\ds{} Och. To bardzo miło z twojej strony \dm{} odpowiedział miękkim głosem. \dm{} A teraz wybaczcie, muszę dopilnować ostatnich poprawek przy naszym bankiecie. \dm{}
Porwał butlę i kieliszki, poleciał na górę.
\end{dialogue}

Nadar był arogancki, jednak pomimo wad, potrafił jak nikt, walczyć z uniwersalnością.
Nie miał własnej mocy, jak niektórzy, lecz wcale jej nie potrzebował.

Mateusz wpatrywał się w Nadara, jak Kula w obraz, namalowany przez białego cyrkowca.
Jego największe zainteresowanie wzbudzały dwa pistolety, zawieszone przy pasie, i laserowa pałka na plecach.

\begin{dialogue}
\ds{} To urządzenie pozwala zaszyfrować i odszyfrować dowolną osobę w splocie czasoprzestrzeni. \dm{} Nadar tłumaczył działanie swoich zabawek. \dm{}
Zachowuje się tak, jakby była zamrożona w czasie.
W pełni bezpieczny sposób na unieszkodliwianie wrogów bez zabijania.
Wadą jest tylko to, że naboje do niego są takie olbrzymie i jednorazowe.
W środku takiego wkładu zapisuje się symetryczny obraz klucza, jedyny sposób na przywrócenie zaszyfrowanej osoby z powrotem do życia.
Strzelając można zamrozić, a potem odmrozić daną osobę.
Kosma, właśnie zgłosiłaś się na ochotnika, aby zaprezentować naszemu gościowi ten wynalazek. \dm{} Nadar wycelował we mnie szyfrator. Co za świ...
\end{dialogue}

\divider{}

Antyrax podniósł lekko worek, Piotr Lekter spróbował złapać oddech, ale zaraz znowu światło zgasło mu przed oczyma.

\divider{}

\begin{dialogue}
...nia z niego. \dm{} Nie nazywaj mnie Kosma! Jestem Katarzyna.
\ds{} Jak widzisz, działa znakomicie. \dm{} Spostrzegłam, że w czasie gdy byłam zaszyfrowana, zdążył już się przebrać w elegancki, pożyczony z garderoby strój. 
Właśnie nakładał puder na swojego irokeza.
\ds{} Nadar, coś ty? Od kiedy ubierasz się elegancko dla Kuli? \dm{} zapytałam z niedowierzaniem. \dm{} Przecież nie gustujesz w niczym innym niż dresy.
\ds{} Od kiedy wywalił mnie w pośrodku kosmosu w tym skadfandrze, za przypadkowe rozlanie barszczu na obrus. 
Lewitując w bezkresnej pustce, miałem sporo czasu na przemyślenie swojego zachowania i stanie się nowym człowiekiem. \dm{} odpowiedział.
\ds{} Naprawdę? \dm{} Wtedy coś mną tknęło. \dm{} Oczywiście, że nie na prawdę. Znowu się zgrywasz tak? \dm{} Tylko się wrednie zaśmiał.
\ds{} To drugie to pikler \dm{} kontynuował. \dm{} Potrafi zapeklować kogoś do umieszczonego tutaj słoika ze szkła wymiarowego, żeby nigdy się nie wydostał.
Wystarczy tylko odłożyć go na najniższą półkę w jakiejś głębokiej piwnicy na całą wieczność.
\dm{} Spostrzegł, że gość niekoniecznie rozumie. \dm{}
Szkło wymiarowe to takie coś, które przechodzi równo przez wszystkie wymiary, także w czasie. 
Wygląda jak szkło, ale istnieje od zawsze na zawsze. 
Ma nieskończoną długość, szerokość, głębokość i... wszystkie inne ości. \dm{}
Nadar nie przestawał wyjawiać sekretów naszej organizacji.
\ds{} A to jest miecz świetlny?
\ds{} To jest laserowa pałka, laserpała, taki przecinak.
Po uruchomieniu zaczyna wirować, wzdłuż pojawiają się promienie dasera. Czyli trochę jak miecz świetlny.
Daser (Death Amplification by Stimulated Emission of Radiation) przecina prawie wszystko jak masło, a pozostałe rzeczy jak ser. 
No, mózg przeciąłby jak powietrze.
Fajna zabawka.
\ds{} I szkło wymiarowe też przetnie? \ds{} Mateusz zapytał, widać że uważnie słuchał.
\ds{} Nadar, on nie przeszedł jeszcze wszystkich testów \dm{} wtrąciłam. \dm{} Nie zdradzaj mu tylu sekretów, bo nie wiadomo, czy na pewno z nami zostanie.
\ds{} No popatrz na niego, Kosma. \dm{} Nadar obchodził i studiował Mateusza ze wszystkich stron. \dm{} Myślisz, że sobie nie poradzi?
Poza tym, już pierwsze części testu przeszedł doskonale. 
Odpowiedział na abstrakcyjny list Profesora, ubrał się w najprawdziwszy strój francuski, a potem odważył się wsiąść do wielkiej, latającej kuli z kosmosu.
\ds{} Niby racja, ale doskonale wiesz, jak chory test potwory mogą tym razem wymyślić.
Pamiętasz, jak Mikołaj przetestował Ziemowita? Kazał mu się przebrać za klauna, przyjść na zabawę urodzinową dla dzieci i robiąc magiczną sztuczkę, zamordować jednego z nich, bo był złym owocem klonu.
Coś nie wyszło, morderstwo nie było czyste, wszyscy skąpali się w zielonej krwi tego podrabiańca.
Drugim zadaniem było uciec z więzienia, do którego go wrzucili po aresztowaniu.
\ds{} Przecież zdał.
\ds{} Albo tego, jak mu było, Błażeja, co Hdro zostawił w Capitalu i kazał jakimś sposobem wrócić na Ziemię.
Człowiek sam na planecie w całości zamieszkanej przez wszystkie gatunki smoków.
\dm{} Kontynuowałam rozmowę, zupełnie ignorując osobę, na której temat ją toczyliśmy.
\ds{} Nie zaliczył, bo ukradł rakietę jakiejś rodzince błękitnych celebritów, będącej wakacjach w Capitalu, zamiast rozegrać to w pokojowy sposób. 
Nawet się nie przejął że w środku statku wciąż były ich jaja!
A gdy rzucił się za nim opłacoony pościg bordowych pasowców, on ich bezwzględnie pozabijał, strzelając do nich z rakietowego działka.
Na szczęście nie miał klucza warstwy, nie mógł uciec z ich kwadry, doleciał do końca i rozbił się o ścianę wszechświata. 
W dodatku to była ściana drugiej kwadry, nie czwartej! 
Idiota tylko się oddalił od Ziemi, może postanowił zaatakować Potworan zamiast tego?
\dm{} Spostrzegłem, że Mateusz chyba przestał rozumieć. \dm{}
Jak się rozbił, to lewitował w smoczej przestrzeni, w chmurze resztek rakiety, przez pół megapulsa, prawie umierając z głodu.
W końcu te małe smoczki, których jaja były w rakiecie, się wykluły i zjadły go żywcem.
Dobrze mu tak.
\ds{} Przepraszam, że wam przerwę, ale co ze mną? To jakiś test zręczności, albo inteligencji? 
Zginę, pożarty przez coś? \dm{} Mateusz bezwstydnie przerwał. \dm{} Co mam zrobić, żeby go zdać?
\ds{} Masz być sobą. \dm{} Odpowiedziałam równocześnie z Nadarem.
\ds{} Ciebie chyba będzie testował Plazma. \dm{} Nadar się zamyślił. \dm{} On lubi militarne klimaty, pewnie trafisz na Planetę Wojny.
Albo będziesz wyżynał jakieś miasto, albo sam będziesz wyżynany. Musisz sam zdecydować.
To bardziej test charakteru, niż umiejętności. Jest to niebezpieczne miejsce.
Najlepsza śmierć... rozerwanie na kawałki przez jakąś futurystyczną wunderwaffe, najgorsza, pewnie wcielenie do Czarnej Armii.
Obedrą cię tam ze skóry, wyłupią oczy i zęby, wcisną w cyber-zbroję i zaleją powodującym szaleństwo smarem khaki, 
będziesz umierał powolną śmiercią, dobrze się bawiąc przy mordowaniu niewinnych. 
Ja tam wolę robić to samo, nie będąc rozpuszczanym przez czarny kwas.
\end{dialogue}

Tymczasem rozległ się dźwięk dzwonu, oznajmiającego posiłek.
Poszliśmy zgodnie na najwyższe piętro statku, Mateusz tym razem trzymał się z tyłu.
Przechodząc przez muzeum, Nadar poklepał radośnie gablotę z demonicznym mózgiem, który w odpowiedzi kłapnął groźnie szczerbatą protezą zębów.

Na najwyższym piętrze, znajdował się salon, biblioteka, scena teatralna, ogród i kapliczka.
Dach przyjmował tutaj miłą, półkulistą wypukłość, ze szczytu zwisał żyrandol na lampy oliwne.
Mateusz zapytał mnie, dlaczego w ogródku rosną tylko ziemskie kwiaty.

\begin{dialogue}
\ds{} Nie wiem czy wiesz, ale Ziemia jest uważana przez wielu za najpiękniejszą planetę wszechświata \dm{} odpowiedziałam. \dm{}
A przynajmniej na pewno przez naszego Profesora.
\ds{} Tos? \dm{} Mateusz zwrócił się w kierunku muzeum.
\ds{} Tos jest bardziej niezwykły, niż piękny. Poza tym, grzyby trzeba by hodować w amoniakowej szklarni.
No i nie powąchasz ich jak kwiatów \ds{} wyjaśniłam. \dm{} Aha, jeszcze Tosowe życie puszcza wszędzie zarodniki. 
Jeden wdech atmosfery tej planety i grzyby zaczną ci rozpuszczać żywcem nos. Dwa wdechy i spleśnieją ci płuca. Trzy wdechy i grzybnia wkręci się w mózg.
\ds{} A kaplica? \dm{} Zwrócił uwagę na mały budyneczek w rogu.
\ds{} Jak pewnie zauważyłeś, Profesor jest bardzo religijny. Poza tym, dobrze mieć miejsce, gdzie można pomodlić się o ratunek, będąc atakowanym przez kosmicznych bandytów. 
Każdy większy statek morski ma przecież kaplicę, to czemu statek kosmiczny także nie miałby mieć? 
\ds{} Kto zbudował ten statek?
\ds{} On jest bardziej strukturą wszechświata, niż konstrukcją, jest zapisany w Matrycy. 
Profesor otrzymał go w prezencie od Nieba, kiedyś zrobił coś bardzo wielkiego i dobrego, dzięki czemu zyskał nadzwyczajną przychylność aniołów. \dm{}
Rozejrzałam się, gdzie jest nasz gospodarz. \dm{}
Ale on nie lubi, gdy się o nim rozmawia.
\ds{} Tylko, kim on jest? \dm{} Mateusz nagle zapytał.
\ds{} W sumie nikt nie wie dokładnie, kim, lub czym, jest Profesor. 
Niektórzy mówią, że aniołem, inni, że dziwnym człowiekiem,
jest też teoria jakoby był ostatnim z jakiejś umarłej cywilizacji. 
Posługuje się jedynym w swoim rodzaju pismem i językiem, którego nikt inny we wszechświecie nie rozumie.
Widzi szerszy zakres barw, słyszy więcej dźwięków, nie wiem jednak, czy jest supersilny...
\end{dialogue}

Przed kontynuowaniem nieprzyjemnej rozmowy uratował mnie dzwonek, zwiastujący rozpoczęcie bankietu.

\divider{}

Antyrax podniósł worek, Piotr Lekter był sztywny, jego wyraz twarzy poskręcany był w dziwności. Tylko jedno oko lekko mu drgało.

Tymczasem wszyscy inni uciekli. Nie było już ani demonów, ani Neofantasora. Antyrax nie był aż taki głupi, żeby uwierzyć, że niebezpieczeństwo minęło. 
Na pewno czyhali na niego, pochowani w ruinach miasteczka.

Stąpał cicho. Pomimo to, jego kroki były doskonale słyszalne w grobowej ciszy, spowijającej dolinę.
Zero wiatru, zero ptaków, zero mieszkańców. Wszystko umarło.
Za chwilę on sam umrze, śmierć przyjdzie po cichu i autor nawet nie spostrzeże się, kiedy.

Wtedy wysuneła się zza winkla Winkla. 

\begin{dialogue}
\ds{} Pisze się ,,wysunęła'' \dm{} powiedziała i cisnęła w niego błędem ortograficznym. Ostrze wbiło mu się w pierś do połowy.
\end{dialogue}
Antyrax poczuł, jak trucizna dyktanda rozpływa mu się po żyłach. To był koniec. Żadna ilość abstrakcji nie wygra ze zwyczajną poprawnością językową.
Choćby stworzył kompletny, co do atomu, świat, to i tak niewiele by to dało. 
Powinien wrócić, i pisać programy komputerowe, zamiast tworzyć epikę. Przynajmniej tam kompilator powie mu o brakujących średnikach.

\begin{dialogue}
\ds{} Daj mi jeden powód, dla którego miałabym cię oszczędzić. \dm{} Demonica zawiesiła nad pisarzem olbrzymie ostrze z poprawnie zastosowanych dialogowych myślników.
\ds{} Mateusz musi dolecieć na miejsce, prawda? \dm{} odpowiedział.
\ds{} Jakoś leci i leci, a nadal nie wiadomo gdzie tak dokładnie jest. Rzucasz na prawo i lewo pojęciami, zupełnie ich nie tłumacząc.
Uniwersalność, światłografy, warstwy, potwory. O co chodzi?
\ds{} Chciałem opowiedzieć o nich później, akcja rozwija się powoli, ciekawiej jest najpierw rzucić hasło, a potem dopiero je opisać.
Poza tym, obowiązkową częścią fantastyki naukowej, jest nieopisywanie niektórych zagadnień.
\ds{} Od kilku stron ta opowieść to jeden wielki opis! I nie jest to żadna fantastyka naukowa, nie ma w tym nauki. Są kula z sacroterii.
\ds{} Nic byś nie zrozumiała, gdyby nie opisy. Jakbym napisał, że polecieli na Felicję, używając górnej warstwy, chociaż Kula nie posiadał do niej kluczy, to co byś sobie wyobraziła?
\ds{} Że to jakaś nadprzestrzeń, coś w stylu alternatywnego wymiaru. A klucze to pewnie wysokotechnologiczne urządzenia do wchodzenia w nią.
\ds{} Prawie. Górna i dolna warstwa odpowiadają za obieg energii we wszechświecie. Górna rozprowadza, a dolna zbiera.
Dolna warstwa czasami ma gejzery, niekontrolowane wybuchy energii, które tymczasowo uniemożliwiają korzystanie z niej w tym miejscu.
Bez opisu nie wiedziałabyś, że nie wolno korzystać z górnej warstwy, gdyż wprowadza to zawirowania w energii w czasoprzestrzeni pod nią.
Zawirowania powodują niedomiary i nadmiary Boskiej Energii, co się objawia większą skłonnością ludzi do popełniania grzechów, lub dobrych uczynków.
Każda ziemska wojna była spowodowana podobnym zawirowaniem.
W ogóle wszechświat składa się z czterech kwadr...
\ds{} Znowu aniołowie i chrześcijaństwo, wszyscy już o tym piszą. Nudne się to robi.
\ds{} Ale gdybym napisał, że statek Kula jest zasilany przez samego Belzebuba, a Profesor ma na swojej lasce czaszkę dziecka, oraz potwory były by tylko od niszczenia światów, to nie było by w tym nic niezwykłego, prawda? Może dostałbym porównania do Warhammera 40K? Otóż nie. Statek Kula działa na energię Boską, potwory służą aniołom, a ALOPP składa się z białych katolików polaków. 
Nikt nie opisał wcześniej takiego świata, a ja będę pierwszy.
\ds{} No więc opowiadaj. Masz kilka minut, zanim trucizna dotrze do twojego mózgu, a wtedy już nigdy więcej nie popełnisz żadnego błędu ortograficznego.
\end{dialogue}

\divider{}

Zasiedli do stołu. Srebrne sztućce z diamentowymi akcentami oraz ręcznie rzeźbione talerze, dobrze współgrały z tkanym obrusem ze złotych nici.
Nigdy nie widział tyle zastawy dla jednej osoby. 
Otrzymał po trzy noże i widelce, łyżkę, łyżeczkę, widelczyk, pałeczki, dziwny szeroki nóż, trzy kieliszki różnych kształtów, duży talerz, dwa małe talerzyki i głęboki talerz.
Do kryształowych kieliszków Profesor nalał wszystkim titanicowego wina.

Na przystawkę było sushi z kawiorem i truflami. Gdy przyszli, było już nałożone na talerzu. 
Profesor odmówił krótką modlitwę, dziękując za dar egzystencji, w imieniu wszelkiego życia, czasu i przestrzeni.

Jedzenie, jak wszystko, było bardzo kosztowne, a jakże, lecz w stu procentach pochodzenia ziemskiego.
Mateusz spodziewał się jakichś nieziemskich przysmaków, dziwacznych owoców, grzybów z Tosa, czy steku ze smoka.
Czy to była prawda, że wszechświat jest całkowicie pusty, a Ziemia jest jedynym znośnym miejscem w kosmosie?

Bardzo zaskoczyło Mateusza to, jak kulturalnie zachowywał się Nadar. Jest arogancki i nie leżą mu galowe ubrania, lecz z zachowania jakby wychował się na dworze królewskim.
Z kolei Katarzyna przywiązała olbrzymią wagę do ubioru, ale nie potrafiła poprawnie złapać pałeczek.
Profesor miał naturalnie i jedno i drugie.

Kula wstał i przemówił.

\begin{dialogue}
\ds{} Tradycją jest, że przy głównym daniu, wybieramy się w jakieś malownicze miejsce, dezaktywujemy tarcze i otwieramy dach, spożywając posiłek na świeżym powietrzu. \dm{}
Rozpoczął tajemniczo. \dm{} Proponuję, aby tym razem, nasz główny gość wybrał miłą okolicę, w której będziemy mogli najlepiej delektować się dzisiejszą pieczenią z bażanta.
\ds{} Tylko nie środek oceanu, ani malownicza plaża. \dm{} Nadar bezwstydnie się wtrącił. \dm{} Mam na jakiś czas dosyć wody. 
\ds{} To może pustynia? Piaskowe wydmy są bardzo ładne w promieniach zachodzącego słońca \dm{} odpowiedziała Kasia. \de{}
\ds{} Nie, piasek wpada do jedzenia i chrzęści w zębach.
\ds{} Szczyt Andów? Ładne widoki z jednej strony na puszczę, z drugiej na ocean.
\ds{} Zimno tam jest, będzie ci zamarzać zupa na talerzu. Potem będziesz cała mokra od topniejącego śniegu.
\ds{} Amazonia.
\ds{} Komary.
\ds{} Antarktyda.
\ds{} Pingwiny.
\ds{} To chyba zaleta.
\ds{} Nie, jeśli podkradają ci jedzenie z talerza.
\ds{} Nie wiem, może Paryż?
\ds{} Śmierdzi.
\ds{} Gejzery na Islandii.
\ds{} Dla mnie okej.
\ds{} Przepraszam bardzo, ale wyjątkowo nie przepadam za zapachem siarkowodoru. \dm{} Kula nagle się przyłączył.
\ds{} Sawanna?
\ds{} Za dużo turystów na safari, ciągle robią zdjęcia. Poza tym lwy łaszą się o kawałki ze stołu.
\ds{} A może Etna? \dm{} Mateusz niespodziewanie wypalił.
\ds{} Etna? \dm{} Kosma nie kojarzyła.
\ds{} Środek wulkanu, kula do połowy zanurzona w płynnej lawie, fontanny ognia, oświetlające otoczenie. Subtelne pomruki z wnętrza Ziemi.
Jeśli dobrze zrozumiałem, tarcze powinny nas przed magmą ochronić.
\end{dialogue}

Nastała niezręczna cisza. Czy Profesor śmiał się w duchu, że Mateusz przecenił zdolności statku, czy może rozpatrywał pomysł poważnie?

\begin{dialogue}
\ds{} To doskonały pomysł. Aktywujemy wszystkie warstwy na raz. Normalnie spowodowałoby to, że czulibyśmy się 
jak w szklanej kuli, bez wiatru, bez temperatury. Lecz w tym przypadku byłoby to i tak wskazane, ze względu na toksyczne wyziewy.
Nie ma nic lepszego od podziwiania lawowych rozbryzgów z odległości metra.
\end{dialogue}

Mateusz nie spodziewał się takiego obrotu spraw, ale cieszył się niemiłosiernie na myśl o zapatrzeniu się w przelewający się żywioł.

Tymczasem statek wynurzył się z Morza Śródziemnego. Profesor niepostrzeżenie przemknął przez Cieśninę Gibraltarską i skierował się prosto w kierunku włoskiego wulkanu.
Ponieważ jednak Kula nie posiadała żadnych okien, goście mogli jedynie uwierzyć mu na słowo. Przynajmniej do póki nie otworzył dachu salonu.

\begin{dialogue}
\ds{} Nie wierzę, że nie zapytałeś się jeszcze Profesora, jak działa ten niezwykły twór, Mateuszu. \dm{} Nadar zrobił sobie przerwę od jedzenia 
tłustego żurku i przykrył chlebową miskę, chlebową przykrywką.
\ds{} Czyli to nie jest tajemnica na równi z gabinetem Profesora? \dm{} Mateusz zapytał. \dm{} Myślałem, że nie wolno mi było tego wiedzieć.
Albo powiem inaczej. Nie chciałem być wywalony przez Profesora Kulę w kosmos tylko dlatego, że zadałem niewłaściwe pytanie w niewłaściwym momencie.
\ds{} Panie Mateuszu Mechalyczny. \dm{} Profesor uśmiechnął się tajemniczo. \dm{} Co innego wyciąganie od innych zakazanych informacji, a co innego rozlewanie barszczu. 
Ciekawość jest wysoko ceniona w ALOPP, a także przeze mnie. Proszę śmiało pytać.
\ds{} Zatem skąd to jedzenie? \dm{} Zaczął od najbliższej mu rzeczy. \dm{} Nie widziałem tutaj żadnej kuchni, nie widziałem spiżarni.
To jedzenie po prostu się tutaj pojawiło, jak przyszliśmy. Po przystawce z sushi, odwróciłem się na chwilę i znalazłem zaraz przed sobą bochen chleba z zupą.
Co to za czary? Co to za materiał? Co może być na tyle silne aby wytrzymać napór gorącej lawy? 
Skąd w skafandrach bierze się nieskończona ilość powietrza? Jaki jest tutaj obieg wody? 
Gdzie ucieka dym z pieca w łaźni? Jak się tym czymś w ogóle steruje? Skąd czerpie energię? Gdzie ma silniki?
\ds{} Odpowiedź na twoje wszystkie pytania znajduje się za tobą na ścianie. \dm{} Kula oznajmił, jakby to było oczywiste.
\end{dialogue}

Na ścianie wisiał elegancki zwój papieru, oprawiony w polaryzacyjnie mieniące się szkło.
Podobnie wyglądało do gabloty z diabelskim mózgiem i słoika na piklerze. Musiało więc to być szkło wymiarowe.

\niceframe{
\begin{Fontlukas}
\begin{center}
ŚWIATŁOGRAF
\end{center}
Z Mocy Najwyższego, potwierdza się nadanie specjalnej właściwości Profesorowi \weirdchar{profesor}.

Sacroteriowy twór, w formie uniwersalnego statku kosmicznego, jest niniejszym przekazany Profesorowi od zawsze na zawsze, dla dowolnych celów.
Dokładny projekt został nieodzownie zapisany w Matrycy.

Nie pobiera się żadnej opłaty od właściciela.

Z Bogiem.\\
Departament Światłografów.
\end{Fontlukas}}

Fragment papieru u dołu wyglądał na pozaplątywany w dziwaczne supełki.

\begin{dialogue}
\ds{} Światłograf to odwrotność cyrografu. Daje ci, jak to jest ładnie opisane, pewną właściwość. \dm{} Nadar zaczął opisywać, zamiast Kuli. \dm{}
Może dawać ci żyć wiecznie, strzelać promieniami z rąk, wygrywać w lotka, eksplodować innym mózgi za pomocą pstryknięcia palcami, czy właśnie posiadać taką oto okrągłą rzecz.
\ds{} A sacroteria? 
\ds{} Sacrum i materia. \dm{} Tym razem Katarzyna rozpoczęła wyjaśnienia. \dm{} 
Materia, która wygląda i reaguje tak samo, jak zwyczajna materia, lecz może zachowywać się w pewnych przypadkach całkowicie po swojemu. \dm{} 
Wzięła w palce końcówkę obrusu. \dm{} Z czego to jest zrobione? 
Powiedzielibyśmy że z nici i złota, może jakiś jedwab, albo z czego tam się robi obrusy.
Ale to jest sacroteria. Może i się zachowywać i plamić jak obrus, ale może równie dobrze zrastać po przerwaniu, jak żywa skóra, albo automatycznie solić leżące na niej potrawy.
W Matrycy jest zapisany algorytm działania tego obrusu, jak i zasady działania całej sacroterii we wszechświecie. 
Cała kula i to jedzenie także jest z sacroterii.
\ds{} Prawie, w Matrycy zapisano, że wytworzone jedzenie jest całkowicie zwyczajną materią. \dm{} Gospodarz poprawił. \dm{}
Jednak to nieistotne, gdyż nie byłby waćpan w stanie w żadnym stopniu doświadczalnie tego stwierdzić, jedynie zaglądając do Matrycy ma się całkowitą pewność. 
Matryca to prawdziwe miejsce, ma kształt wielkiej płyty, położonej nad całym wszechświatem, powyżej górnej warstwy. Naturalnie, nikt śmiertelny nie ma do niej dostępu.
\ds{} Zwykle za światłograf pobierana jest opłata, ilość dobra do wytworzenia. \dm{} Znowu Nadar zaczął. \dm{} 
Może opierać na liczbę uratowanych dusz, jakiś wielki czyn, czy właśnie, jak tutaj, na nic. 
Ale myślę, że jednak nasz Profesor, kiedyś coś tak fajnego wykonał, że Niebo go polubiło.
\ds{} A ta plamka?
\ds{} To odcisk duszy Profesora Kuli. Światłograf musi być podpisany. Nie jakąś przyziemną krwią, lecz czymś wiecznym i niezniszczalnym, twoją duszą.
\ds{} Pora na widowisko \dm{} właściciel kuli przerwał rozmowę.
\end{dialogue}

Stuknął laską i wtedy cały dach począł się otwierać na osiem stron, niczym kwiat. 
Żyradol poleciał na jedym płatku na bok i spoczął w specjalnym do tego miejscu, nad sceną.

Czerwona śmierć przelewała się przez otwarte kawałki dachu i obijała o niewidzialną barierę, spływając majestatycznie.
Płomienne światło tworzyło wspaniałą, bankietową atmosferę.
Rozsunięte fragmenty dostawały z pełną siłą żywiołu, lecz czerwone futro wcale się nie paliło, lawa po nim spływała, jak po mokrej kaczce.

Tak, jak Kula zapowiedział, na główne danie pojawił się pieczony bażant w sosie kurkowym.
Ponownie eleganckie jedzenie i ponownie z Ziemi. 
Czy dałoby się wygenerować potrawę idealną? 
Coś perfekcyjnego, lecz nieistniejącego?

A potem zdał sobie sprawę, że równie dobrze mógłby delektować się pizzą, pijąc aromat pizzy z próbówki.
To właśnie nieidealność nadaje prawdziwość.

\divider{} 

\begin{dialogue}
\ds{} Akcja, Antyrax! Daj mi akcję \dm{} przerwała mu Winkla. \dm{} Mam nadzieję, że w czasie tego bankietu zdarzy się coś ciekawszego.
\end{dialogue}

\divider{}

Wtem poczuli silne uderzenie. Wstrząsnęło statkiem, porozlewało wino z kieliszków i spowodowało, że Kula nie trafił widelcem w ziemniaka.
Zobaczyli na niebie całą armię latających helikopterów, jeden z nich wystrzelił drugi pocisk i zaraz druga eksplozja wstrząsnęła otoczeniem.

Profesor wstał, ukłonił się, i zbiegł po schodach. Zaraz też wrócił, niosąc olbrzymią, ozdobną tubę od gramofonu wmontowaną w taboret. 
Tuba była podłączona wężykiem do lejka, który to przyłożył sobie do ust.
Skierował otwór w górę.

\begin{dialogue}
\ds{} Jakim prawem przerywacie nam uroczystą konsumpcję bażanta, ciskając w nas wybuchowymi pociskami? \dm{}
powiedział pretensonajlnym tonem, a tuba wzmocniła jego dźwięk do potęgi megafonu. \dm{}
Proszę natychmiast opuścić krater wulkanu, inaczej komuś może stać się krzywda! \dm{} Odpowiedziała mu trzecia rakieta, kolidująca z tarczą.
\ds{} Wojsko Stanów Zjednoczonych, tajny oddział do walki z kosmitami. \dm{} Nocny wyciągnął z kieszeni okrągłą komórkę i podawał informacje.
\ds{} Proszę was. Nie musimy uciekać się do używania prądowych urządzeń. Schowajcie je z powrotem. \dm{} Nikt się jednak nie przejął uwagą starszego pana.
\ds{} Nadar. Masz coś mocniejszego, niż laserpała? \dm{} Katarzyna grzebała sobie pod suknią, niemal odwracając się na lewą stronę.
\ds{} Mogę tak ustawić dasery, aby strzelały w górę. Lecz wtedy potnie ich na kawałki.
\ds{} To może bierz tego dużego szyfratorem, wystraszysz resztę.
\ds{} Nie, bo spadnie do lawy i zginą ludzie.
\ds{} To ostatnie ostrzeżenie! \dm{} Kula nie brzmiał przekonująco nic a nic.
\ds{} Mam znikarkę, zniknę komuś pół kadłuba.
\ds{} Oszalałaś? Jeszcze uderzy w tarczę statku i sami znikniemy.
\ds{} ,,You are being arrested under UN law.'' \dm{} Dało się słyszeć stanowczy głos z góry.
\ds{} ,,You are being destroyed under ummm... our own law.'' \dm{} Nadar wyrwał Kuli lejek.
\ds{} Cokolwiek zrobimy, to spadną do wulkanu i umrą.
\ds{} No to zostaje pikler, potem trzeba będzie ustalić z Niebem, żeby cudownie wyciągnęli ich ze szkła wymiarowego.
\ds{} A co jeśli trafimy po drodze na uniwersalność? Wsadzisz ich razem, żeby się pozabijali? Nie mam zapasowych słoików.
\ds{} ,,Please leave your spaceship right now!''
\ds{} Sam jesteś ,,spaceship''. \dm{} Profesor nie dawał za wygraną.
\ds{} Profesorze, czy ma pan jakąś defensywną broń na pokładzie?
\ds{} Owszem. Dobre słowo i miłość do wrogów.
\ds{} ,,You have no power against army of the United Stat...''
\ds{} ,,Shut up!'' \dm{} Katarzyna przekrzyczała ryk lawowych fontann i wirników.
\ds{} Patrzcie! \dm{} Jednej z maszyn, pod wpływem gorąca, eksplodował silnik. Pilot wykatapultował się tak niefortunnie, że spadł prosto na szczyt półkulistej wypukłości tarczy.
\ds{} Kula wpuść go! Helikopter spada! \dm{} Profesor wskazał na żołnierza laską, który przesiąkł przez tarczę, zaraz w to miejsce uderzył wrak śmigłowca. 
Pomarańczowe światło przyćmiło na chwilę poświatę wulkanu. Kasia wystrzeliła szyfratorem w nieproszonego gościa.
\ds{} Mamy zakładnika, odpuście, albo go zab... coś mu zrobimy. \dm{} Profesor krzyczał do tuby.
\ds{} Oni nie rozumieją po polsku, Profesorze.
\ds{} ,,You out, or he die.'' \dm{} Mateusz nie spodziewał się po Kuli tak słabej znajomości angielskiego. Jednak amerykanie zrozumieli i przerwali ogień.
\ds{} ,,We will negotiate, please open your forcefield.'' \dm{} Statki powietrzne zniknęły z pola widzenia, zjawił się za to jeden mały helikopterek, z przywiązaną białą flagą.
Zbliżał się powoli i stanął w powietrzu, oczekując aż zniknie ochrona.
\ds{} To pułapka! \dm{} Nadar naskoczył Katarzynie na plecy i skulił razem ze sobą. Chwilę potem olbrzymia eksplozja wstrząsnęła światem. Wszyscy upadli tam, gdzie stali.
\ds{} Ja pierdolę. \dm{} Nocny pomógł wstać Kosmatej. \dm{} Drugiej nie przeżyjemy. Musimy uciekać.
\ds{} A zatem to koniec rozmowy. \dm{} Kula odłożył lejek na widełki, dotknął kulki na lasce, i wielki kwiat począł zamykać swe płatki. Zaczęli się wznosić.
\ds{} Stop! Tam, to chyba daser! \dm{} Katarzyna wskazała palcem zieloną linię nad kraterem.
\ds{} Skąd...?
\ds{} Zaraz przekroją nas na pół!
\ds{} A może uciekać do dołu? \dm{} Mateusz zaproponował.
\ds{} W dół?
\ds{} Do wnętrza Ziemi.
\ds{} Dałoby się. \dm{} Kula popatrzył w podłogę.
\ds{} Nie polecą za nami w lawę.
\ds{} I też spali się, cokolwiek wystrzelą.
\ds{} W spokoju wylecimy potem innym wulkanem, nawet na dnie oceanu.
\ds{} A zatem w dół. \dm{} Profesor opuścił laskę.
\end{dialogue}

Białe zwierciadło poczęło się zanurzać coraz głębiej w Etnie.
Po kilku sekundach, zniknęło pod fontannami płynnych skał.
Hałas kotłującego się wnętrza wulkanu wolno niknął i nastała absolutna cisza.
Panowała atmosfera przegranej walki, goście podziwiali tekstury dywanów i uciekali wrokiem od siebie nawzajem, jak gdyby chcieli się przepraszać za wyrządzone szkody.
Zaszyfrowane, pokryte niebieskawą poświatą, ciało żołnierza, leżało w kwietniku.
Połamane tulipany mizernie wyłaziły spod nietypowej osoby.

\begin{dialogue}
\ds{} Co powiecie wszyscy na deser? \dm{} Profesor przerwał milczenie.
\ds{} Jemu też? \dm{} Mateusz spojrzał na połamane kwiaty pod pilotem.
\ds{} Oczywiście. Tylko trzeba go odpowiednio ubrać.
\end{dialogue}

Wylecieli, rozbijając jeden ze szczytów Islandii.
Śnieżna chmura wzbiła się w powietrze.
Turyści cykali im zdjęcia, z których później i tak nic nie wyjdzie.

\divider{}

Winkla popatrzyła się krzywo.
\begin{dialogue}
\ds{} Eeee, nieee. To ma być akcja? 
\ds{} Tak na wymuszenie, ciężko coś dobrego stworzyć. \dm{} Antyrax kładł się powoli na ziemi. Trucizna robiła swoje.
\ds{} Daj mu spokój, mnie się podobało. \dm{} Z cienia wyszedł Everywhere Man. \dm{} Ale tylko trochę.
\ds{} Trochę za mało, musi być idealnie.
\ds{} Nigdy się nie da napisać nic idealnie.
\ds{} Przepraszam, ja tu umieram! \dm{} Pisarz ledwo siedział.
\ds{} Niech mu będzie. \dm{} Winkla westchnęła, wyciągając strzykawkę z antidotum. Wyciąg z prac pierwszoklasistów. \dm{} Ale czekamy na prawdziwą walkę, w której na prawdę mogą przegrać.
\end{dialogue}

\divider{}

Przed chwilą spadałem prosto w objęcia kulistego UFO. Zaraz potem obudziłem się, przywiązany do krzesła, ubrany w jakieś cyrkowe ubrania.
Siedziałem przy eleganckim stole, przede mną, na srebrnym tależyku, leżał kawałek czegoś, co przypominało ciasto. Jedną rękę miałem wolną.
Trzy osoby tajemniczo się mi przyglądały. Wolałbym standardowo trafić na stół operacyjny, z próbnikiem w dupie.
\begin{dialogue}
\ds{} ,,Here, have a dessert'' \dm{} powiedział gość z irokezem. Miałem ochotę rozsmarować mu to ciasto na twarzy, ale może rzeczywiście skończyłbym wtedy z próbnikiem.
\ds{} Nadar, co z nim zrobimy? \dm{} Jeden z nich, młodszy, zapytał go po polsku. Niesamowite, rozmawiają po polsku! Lepiej nie zdradzać, że rozumiem ten język.
\ds{} Damy mu zjeść, a potem rozkroimy i wsadzimy próbnik w dupę, żeby przeprowadzać chore eksperymenty.
\ds{} A nie warto wcześniej trochę go przepytać? Zaraz, co?
\ds{} Sam nam wszystko wyśpiewa, gdy będzie mutował się w krowę.
\ds{} Co, ale... auć... aha, tak najpierw w krowę, a potem w osła. Będzie boleć, oj będzie. \dm{} Młody nagle zmienił biegun.
\ds{} Potem podrzucimy jakiemuś niewyżytemu seksualnie farmerowi w Afryce.
\ds{} Mam lepszy pomysł. Wytniemy mu mózg, wsadzimy do słoika, i podłączymy do sztucznego ciała. Będzie prawdziwym cyborgiem.
\ds{} Sprawdzimy, ile taki wojak zniesie orgazmów na godzinę. Pewnie wytrzyma z dwa dni seksualnej męki, a potem wysiądzie mentalnie. \dm{} Dziewczyna się odezwała.
\ds{} Podobno całkiem dobrze sobie radzą w dziczy. Ciekawe, jak długo przeżyje sam w dżungli czerwonych kartaczy. \dm{} Starszy pan się odezwał. \dm{}
Te smoki bardzo lubią ludzi, najpierw pieką żywcem w swoim ogniu, a potem zjadają kawałeczek po kawałku.
\ds{} Albo od razu w całości na surowo, żeby ofiara utopiła się w kwasie żołądkowym.
\ds{} Nie, to za szybka i za prosta śmierć. Proponuję zawieść go na Tos, żeby wgryzły się w niego pasożytnicze grzyby.
\ds{} Super, przejmą nad nim kontrolę wystarczająco mocno, aby sterować ruchami i jednocześnie na tyle słabo, by zachować pełną świadomość.
\ds{} A może po prostu uwolnimy go? Odwieziemy prosto do domu.
\ds{} Doskonały pomysł, wsadzą go do wariatkowa i będą męczyć, żeby im coś powiedział. Wyręczą nas z roboty. Ciekawe, jak zareagują na jego opowieści o bankiecie w kuli?
\ds{} Pewnie trafi na stół operacyjny podziemnego laboratorium, z próbnikiem w dupie.
\ds{} Dość, zrobię wszystko, co mi rozkażecie! \dm{} zawołałem. Nie wierzyłem w ich opowieści, ale też nie miałem ochoty przekonywać się, czy rzeczywiście mam rację.
\ds{} Mówiłem, że to Polak? Swój, swojego wszędzie pozna. \dm{} Ten, którego nazwali Nadar, wykonał triumfalny gest. \dm{} Więc na początek zjedz ten przepyszny jabłecznik.
\end{dialogue}

Nie za bardzo miałem wybór. 
Gdyby chcieli mnie otruć, już by dawno to zrobili.
Poza tym, rzeczywiście wyglądał przepysznie.

Niepewnie wziąłem widelec w wolną rękę i zjadłem trochę kosmicznego jedzenia.
Smakował, jak ciasto które robiła moja babcia, gdy jeździłem z wizytą do Polski.
Miękkie, kruche, i lekko ciągliwe.
Zjadłem całe i czekałem, aż zacznę mutować w krowę.

\begin{dialogue}
\ds{} Teraz do rzeczy. Skąd do cholery macie daser? \dm{} Nadar wyjął długą pałkę i uruchomił. Zielone lasery wystrzeliły równolegle do trzonu, a całość zawirowała.
\dm{} Wnioskuję, że domyślasz się, co to robi?
\ds{} Nie. Nie mam pojęcia. Nic nam nie mówili, nasz oddział dostał ten laser całkiem niedawno, nie pozwalali nawet go przetestować \dm{} odpowiedziałem zgodnie z prawdą. 
Byłem pewien, że i tak nie uwierzą.
\ds{} Zademonstruję ci zatem, jak działa. \dm{} Wziął mój karabin i skrzyżował z pałką. 
Przeszła, jak przez masło, dzieląc moją broń na dwie części. Stalowe ścinki posypały się na stół. \dm{} Skąd macie naszą technologię?
\ds{} Przysięgam, nie mam nawet pojęcia, co ona robiła! Nie mówią nam tam niczego, wszystko jest w tajemnicy wojskowej. 
Lecąc na misję, nawet nie wiedziałem, z czym będziemy dzisiaj walczyć!
\ds{} Nadar, on chyba mówi prawdę, na pewno nie wtajemniczaliby go w zdobyte bronie kosmitów \dm{} dziewczyna powiedziała.
\ds{} Jak wyglądało pudełko dasera? Co w nim było? Jak je przewozili? Do czego podłączali? \dm{} kontynuował.
\ds{} Tylko przez chwilę mi mignęło. Było w specjalnym śmigłowcu. W asymetrycznym pudle z dziurą w środku. Dwóch naukowców je obsługiwało. Chyba nie podłączali do niczego zewnętrznego.
\ds{} Jedno pudło?
\ds{} Tylko jedno, mówili że bardzo cenne.
\ds{} Gdzie je przetrzymują?
\ds{} Ten śmigłowiec dołączył do nas później, od innej strony, nie leciał razem ze wszystkimi.
\end{dialogue}

Mój rozmówca się rozluźnił i nawet trochę uśmiechnął.
Zdziwiłem się, tak samo jak pozostali.

\begin{dialogue}
\ds{} Chyba mówisz prawdę. To by znaczyło, że musieli ukraść nam kiedyś jakieś daserowe urządzenie, ale wciąż nie wiedzą, na jakiej zasadzie działa.
\end{dialogue}

Milczałem.

\begin{dialogue}
\ds{} Mateusz, to będzie twoja pierwsza misja. Dowiesz się, gdzie trzymają ukradziony daser i odbijesz go z powrotem. 
\end{dialogue}

Mateusz przełknął ślinę.

\begin{dialogue}
\ds{} To co z nim w takim razie zrobimy? Teraz już na serio. Nie możemy go przecież tak po prostu wypuścić \dm{} dziewczyna zapytała poważnym tonem.
\ds{} Trzeba pokazać go potworom na sąd. Pewnie wsadzą naszego mordercę do kubistycznego więzienia.
\ds{} Czy to nie za ostro? Przecież trochę go armia do mordowania zmusiła. Jest poza tym ta nowa planeta koncentracyjna do zsyłek.
\ds{} SS-manni także byli zmuszani do mordowania, żadna ulga mu się nie należy. 
I nie może być umieszczony z innymi ludźmi, jest za dobrze wyszkolony, wymorduje wszystkich pozostałych. To ma być zsyłka, a nie raj dla psychopatów.
\ds{} To może, nie wiem. Zaszyfrować go aż do końca wszechświata. To będzie, jak podróż w przyszłość, nawet nie zauważy.
\ds{} Przecież to równa się śmierci. Obudzi się po bilionach pulsów tylko po to, aby zobaczyć Apokalipsę.
\ds{} Ja spróbuję go naprawić. \dm{} Starszy pan odezwał się po dłuższym czasie. Zdziwiłem się.
\ds{} Panie Profesorze, ta osoba jest niebezpieczna! Zdradzi i zabije pana.
\ds{} Ja wierzę, że każdy może się zmienić. Zrobimy z niego porządnego obywatela Felicji.
\ds{} Felicja jest przepełniona, nikt się tam więcej nie zmieści. Chyba nie chce pan wolny domowej?
\ds{} To zrobimy drugą Felicję, większą, dzikszą, o ustalonym prawie i dowolnej liczbie obywateli. Będzie równość i tolerancja dla wszystkich istot, będą mogły żyć w spokoju przed prześladowaniem.
Miejsce bezpieczne od przemocy, opresyjnych rządów i odrzucenia. Różnorodne i wspaniałe. Wszystkie kultury wszechświata stanowiące wspaniałą jedność.
Prawnie ustalę system który dla każdego będzie równy.
A to będzie jej pierwszy obywatel.
\end{dialogue}

Wszyscy, prócz Profesora parsknęli śmiechem.

\begin{dialogue}
\ds{} To już lepiej na zsyłkę. Przynajmniej będzie miał szansę na dożycie starości \dm{} Nadar zakończył.
\end{dialogue}

Starszy pan, czerwony ze wściekłości, roziązał mnie i poprowadził od stołu. 
Za plecami słyszałem tylko kolejne wizje, co by się na tej ,,lewackiej'' planecie działo.
Chłopacy rzucali obleśnymi pomysłami co do nowych praw, dziewczyna rozpatrywała kto, i jak musiałby ich przestrzegać.
\begin{dialogue}
\ds{} Zakaz jedzenia mięsa, bo zabijamy biedne zwierzątka.
\ds{} Ale takie zesłane smoki, jak mogłyby to przeżyć? One jedzą tylko mięso.
\ds{} Myślę, że potajemnie pożywiałyby się innymi obywatelami.
\ds{} Na pewno dałoby się napisać ustawę, rozwiązującą ten głodowy problem.
\ds{} No co ty, to przecież element ich kultury. Nie możemy zabronić innym być sobą, ty pieprzony rasisto, ha ha...
\ds{} ...nie wolno ci zabronić mi być rasistą... musisz tolerować moją nietolerancję...
\ds{} ...codzienne ćwiczenia seksualności...
\ds{} ...płeć będzie ustalona jako prosta?... jako przestrzeń może?
\ds{} Dzisiaj jestem w $49 + 12i$ procentach kobietą.
\ds{} ...kwaterniony lepsze...
\ds{} ...identyfikuję się, jako kamień... nie możesz mnie zjeść...
\ds{} ...jestem zjadaczem kamieni... mam takie prawo...
\ds{} ...czekaj, czekaj... czy kamienie mają prawo do aborcji?
\ds{} ...ustawą...
\end{dialogue}

Schodziliśmy w dół po schodach, śmiechy na górze stawały się coraz bardziej niewyraźne.
Ten statek był gigantyczny w środku. Profesor Kula, jak mi się przedstawił, zapytał o moje imię.
Zaprowadził mnie do własnego pokoju i zostawił, nawet nie zamykając drzwi. 
Powiedział, abym dowolnie korzystał z dobrodziejstw Kuli i nie bał się prosić o pomoc. 
Zaproponował nawet kąpiel w basenie, ale oczywiście odmówiłem.
Pokój był bardzo malutki i bardzo elegancki.
Wszytko w tym obrzydliwym, rokokowym stylu.
Większość stanowiło podwójne łóżko z daszkiem i firanką.
Do tego kilka krzeseł, stoliczek, szafka, toaletka.
To miejsce nie służyło do długotrwałego przesiadywania.
Gdyby urwać podpórkę od łóżka, rozbić lustro, związać firaną, rozłożyć na części krzesło, to mógłbym sobie stworzyć jakąś włócznię i tarczę.

Kogo ja oszukuję, jestem bezsilny wobec ich technologii.
Zamroziliby mnie ponownie w czasie, gdybym tylko próbował coś odwalić.
Albo przetną na pół tą laserową pałką.
Będę musiał jakoś uciec. Na każdym statku kosmicznym są kapsuły ewakuacyjne.
Trzeba się rozejrzeć pod pretekstem zwiedzania.

Uchyliłem drzwi. Po drugiej stronie był podobny pokój do mojego.
Ujrzałem tam, tą dziewczynę w sukni.
Malowała się przez lustrem, spojrzała na mnie, uśmiechnęła się i pokiwała palcem, jak małemu dziecku.
Nigdzie nie pójdę.

Czyli to jest UFO, a oni są kosmitami.
Ale nie byli kosmitami. Tego jednego byłem pewien, no może poza Profesorem Kulą.
Bardziej przypominali gości z Ziemi, tak inni od właściciela. Pewnie ich zaprosił do siebie.
Gdzie w takim razie lecą?
Na wakacje, na inną planetę, czy do swojego kosmicznego domu?
Na pewno nie mieszkają na Ziemi, bo używają abstrakcyjnych technologii.
To oznacza, że istnieje jakaś pozaziemska cywilizacja ludzi, a może nawet polaków, poza naszą planetą.
Skomplikowane to wszystko, nie mniej jednak, z pewnością nie zabiją mnie tak od razu.

Dopiero teraz zwróciłem dokładniejszą uwagę na swój ubiór.
Bardzo kosztowny i elegancki. Nie wiedziałem, gdzie jest mój oryginalny mundur. 
Nie miałem ze sobą nic innego, więc postanowiłem go zachować, może się przydać.
Usiadłem na łóżku. Było miękkie i wygodne, dawno nie leżałem na czymś takim, na chwilę położyłem się na plecach.
Nawet nie wiedziałem, kiedy zasnąłem.

Obudziło mnie głośne chrobotanie. Metaliczny dźwięk rozchodził się po całym statku.
Szklane ozdoby lekko dygotały, coś atakowało kulę.
Stwierdziłem, że skorzystam z zamieszania i wymknę się niepostrzeżenie.
Uchyliłem delikatnie drzwi, lecz na korytarzu nikogo nie było.

Moje eleganckie trzewiki hałasowały, jak na występie steperów.
Posuwałem się, zagłuszając swój ruch zewnętrznym dźwiękiem.
Jeśli schodziliśmy w dół, a nie widziałem po drodze żadnego wyjścia, to znaczy że musiało być ono na najniższym piętrze, na którym jeszcze nie byłem.
Udało mi się dojść do schodów w dół, ostrożnie wychyliłem głowę zza sufitu niższego piętra i zobaczyłem właz w ścianie kuli.
Tego szukałem.

Hałas wyraźnie był tu głośniejszy. Ktoś próbował przewiercić się przez drzwi.
Byłem pewien, że reszta mojej grupy przybyła mnie odbić.
Któż inny mógłby zaatakować latającą kulę?

Złapałem korbę do otwierania włazu i począłem kręcić, jakby od tego zależało moje życie.
W tym samym czasie usłyszałem za sobą zbieganie po schodach.
\begin{dialogue}
\ds{} Puść tą korbę! \dm{} Mateusz przybiegł, miał w ręce nóż do masła.
\ds{} Wypchaj się, wrócili tu po mnie \dm{} odpowiedziałem, odwracając głowę.
\ds{} Nikt po ciebie nie wrócił, jesteśmy pośrodku... \dm{} nagle zamilkł i otworzył szeroko oczy, jakby właśnie zobaczył ducha.
\end{dialogue}
Ostrożnie się odwróciłem, podejrzewając, że to wcale nie moja grupa przyszła mi z odsieczą.
To, co zobaczyłem, zmroziło mi krew.

W otwartym na oścież włazie, na tle rozgwieżdżonego nieba, lewitował dziadek w wannie.
Patrzył się na nas tajemniczo, uśmiechając się. Na głowie miał czepek kąpielowy, w ręce trzymał słuchawkę prysznica, wszędzie były góry piany.
\begin{dialogue}
\ds{} Witam panów. Czy nie macie może pożyczyć trochę szamponu? Lecę już tak milion lat i wciąż nie mogę dokończyć kąpieli.
\end{dialogue}
Pokręciłem lekko głową.
\begin{dialogue}
\ds{} Nie szkodzi \dm{} zaśmiał się. \dm{} Umyję się tobą.
\end{dialogue}
Zamachnął się słuchawką, jak lassem, rzucił do środka i owinął ją wokół mojej nogi.
Począł ciągnąć z nadludzką siłą, przewrócił mnie. Złapałem się korby w ostatnim momencie.
Owinięta wokół stopy końcówka prysznica była jak macka, próbowałem ją strząsnąć, lecz zahaczyła się o sznurówki butów.
Szarpnął mocniej i obrotowa rękojeść korby zaraz wyślizgnęła mi się z objęcia, poleciałem dalej, w kierunku próżni.
Czepiając się palcami puszystego dywanu zobaczyłem, jak Mateusz nadal stoi, zahipnotyzowany.
Włosia dywanu były za słabe, aby mnie utrzymać. Wciąż ciągnął mnie do siebie. Złapałem się ostatniej rzeczy przed śmiercią, framugi drzwi.
Wtedy też, połową mojego ciała, poczułem zimną pustkę kosmosu.
Ktoś delikatnie objął mnie dłonią za kostkę, szarpnął, i plusnąłem w ciepłą wodę.

Wanna lekko się zachybotała po moim wejściu.
Zobaczyłem obok siebie wyszczerzoną twarz staruszka.
Adrenalina nie pozwoliła mi poczuć, że duszę się w próżni kosmicznej.
Złapałem się boku wanny, aby wyskoczyć, lecz moja ręka ześlizgnęła się, jakby była z mydła.
Popatrzyłem na swoje dłonie, które topiły się jak wosk.
Dziadek przejechał gąbką po mojej twarzy, poczułem, jak zabiera mi cały policzek.

Nagle różowy promień wystrzelił ze środka kuli.
Znalazłem się w nurcie rwącej rzeki, która ciągnęła wszystko z powrotem.
\begin{dialogue}
\ds{} Ojojojoj! Nieszczęście \dm{} zawołał dziadziuś.
\end{dialogue}
Zobaczyłem Nadara z wyciągniętym pistoletem, jego różowy promień wciągał nasz obu. Wpadłem do lufy i zaraz uderzyłem we wklęsłą, szklaną, ścianę, rozpłaszczając się.
Wielka twarz Nadara obserwowała mnie czujnym wzrokiem.
Zaraz od tyłu dobiła mnie lecąca wanna, a potem wpadający do niej dziadziuś.
Woda przyszła ostatnia, zalewając wszystko, rozpuszczajac mnie.
Czułem, że umarłem, wciąż mając świadomość.

Po schodach zszedł Profesor Kula.
\begin{dialogue}
\ds{} Panie Mateuszu, specjalnie dla pana, wyszliśmy na chwilę z podróży górną warstwą, z powrotem do czasoprzestrzeni. \dm{} Nawet nie zauważył, co się tutaj przed chwilą stało.
\dm{} Znajdujemy się pomiędzy galaktykami,  w wielkiej pustce kosmosu. Najbliższa materia jest się trzy miliony lat świetlnych od nas. Proszę spojrzeć, 
o tam, to Droga Mleczna, nie jest niesamowita? Tak wygląda z zewnątrz, niczym biała spirala... \dm{} Przerwał, rozejrzał się. Popatrzył na Mateusza, Nadara i na mnie, zamkniętego w szklanej bańce.
\end{dialogue}

\divider{}

Winkla opuściła swój miecz i schowała. 
Nikt nic nie mówił, nikt nie reagował. 
Antyrax poczuł się obserwowany ze wszystkich, ciemnych zakamarków zaułka.
\begin{dialogue}
\ds{} Za bardzo skomplikowane i chaotyczne \dm{} odpowiedziała.
\end{dialogue}

\divider{}

Miałem dość. Omal nie zamieniliśmy się wszyscy w szampon.
Chciałem spać, chciałem wreszcie uciec od kosmosu, latającej po nim uniwersalności i kręcącego w nosie pudru.
Obudzić na miejscu, na Felicji.
Jednak nie byliśmy nawet w połowie drogi.

Zastałem Profesora i Katarzynę w salonie, zasnęli przy kawie.
Z biblioteczki dobiegał głos wertowania książek.
Na stole leżała zabezpieczona kulka wymiarowa. Mały dziadziuś w środku, mył się w najlepsze, niesamowitą ilością piany.
Od czasu do czasu kupka bąbelków poruszała się samoczynnie, jakby chciała uciec z wanny, lecz nic jej to nie pomagało.
Zasłużył na to.

Obudziłem śpiochów i poinformowałem, że mam dosyć i idę spać.
Kula coś jeszcze mówił o jakimś występie na scenie, cieście biszkoptowym i chińskiej herbacie. 
Katarzyna podniosła się i wolno poszła w kierunku schodów.
Znowu z nich spadła.

Miałem wrażenie, że wszyscy mieli dość, wszyscy z wyjątkiem Mateusza.
Męczył mnie, żebym mu wyjaśnił, jak złapałem uniwersalność do nieprzekraczalnego szkła wymiarowego,
co się stanie z żołnierzem, gdzie żyją smoki, kto to są potwory itp.
Siedziałem akurat w jacuzzi, uważając żeby nie zasnąć i się nie utopić, myłem się przed pójściem spać i nie miałem żadnej ochoty na rozmowy o infrastrukturze.
\begin{dialogue}
\ds{} Wasza praca zawsze tak wygląda? \dm{} męczył mnie.
\ds{} Tak.
\ds{} I często natrafiacie na takie dziwne zjawiska?
\ds{} Tak.
\ds{} I zawsze udaje wam się wygrać?
\ds{} Tak.
\ds{} Nie wierzę w to, kłamiesz.
\ds{} Tak.
\ds{} No dobra, a na górze w bibliotece jest Atlas Wszystkich Istot Wszechświata, ale jest trochę cienki. Większość zajmuje Ziemia.
\ds{} Tak.
\ds{} Czy to prawda, że istnieje tak mało żywych istot? Ledwo kilka planet? Kilkadziesiąt gatunków smoków?
\ds{} Tak.
\ds{} Zawsze będziesz odpowiadał mi ,,tak?''
\ds{} Tak.
\ds{} Jesteś chamski.
\ds{} Tak.
\end{dialogue}
Poszedł sobie w końcu.

Wtem fala zimnej wody wylała mi się na głowę. 
Mateusz trzymał puste wiadro z wodą do polewania pieca.
\begin{dialogue}
\ds{} Zabiję cię \dm{} wysyczałem przez zęby.
\end{dialogue}
Nie uciekł, a stał tylko z wyrazem twarzy, który ja sam bym przyjął w takiej sytuacji.
Nienawidziłem go za to, że potrafił być tak podobny do mnie.
\begin{dialogue}
\ds{} Dobra, właź. Pogadamy. \dm{} Już mi się nie chciało spać. więc równie dobrze mogłem obudzić się rozmową jeszcze bardziej.
\ds{} Co właściwie tam się stało? \dm{} zapytał, zdejmując ubranie.
\ds{} A jak myślisz? Powiedz, co wywnioskowałeś z tego zdarzenia?
\ds{} No więc, uniwersalność to... taka jakby magia. \dm{} Zanurzył się w bąbelkach.
\ds{} Poprawnie. Zachowuje się jak stereotypowa magia, jaką znasz z książek i filmów. Jednak z tą różnicą, że nie można jej w żaden sposób kontrolować.
\ds{} Ale... \dm{} Myślał przez chwilę. \dm{} Jak ją więc przechwyciłeś?
\ds{} Sacroteria jest silniejsza. Sacroteria może być wszystkim, w szczególności może sterować uniwersalnością. Jest sterowana bezpośrednio przez zasady Matrycy.
\ds{} To nie wyjaśnia, jak dziadzio znalazł się w zamkniętej kuli.
\ds{} Pikler ma mały moduł tunelowy. Działa trochę, jak kwantowy efekt teleportacji cząstek. Z jednej strony wchodzi dowolna rzecz, zaraz pojawia się kilka centymetrów dalej, w słoiku.
\ds{} A ściana jest ze szkła wymiarowego, przez którą nic nie przejdzie.
\ds{} Poprawnie.
\ds{} A czy ten dziadzio nie może sam stworzyć czegoś podobnego do modułu tunelowego po swojej stronie, i przeteleportować się z powrotem?
\ds{} Nawet jakby stworzył, nie będzie on działał, gdyż wewnątrz kuli nie działają zasady Matrycy, moduł tunelowy nie będzie miał swego rodzaju zasilania.
To trochę jak klucz bez zamka, albo silnik bez prądu. Matryca mówi, że w miejscu, gdzie jest nasz moduł, ma powstać tunel. A nie, że moduł sam z siebie tworzy tunel. Moduł jest jedynie wskaźnikiem dla Matrycy.
\end{dialogue}
Jak już opowiedziałem mu o Matrycy, to wyjaśniłem też całą budowę naszego świata.

%TODO Naprawa

Obudził mnie dźwięk opuszczanego włazu.
Wygramoliłem się z eleganckiego, acz niewygodnego łóżka, przebrałem w normalne ubrania i poszedłem na dół.
Spotkałem przy wyjściu Kulę.
Trochę nafukał na mnie, że jak śmiem chodzić po Kuli bez należytego stroju, ale miałem to gdzieś.
Wyszedłem.

Jak zwykle rześka atmosfera porannej Felicji obudziła mnie. 
Wylądowaliśmy na głównym placu, brukowana ulica mieniła się od rosy, gazowe lampy jeszcze nie były wygaszone.
W oddali wyłaniały się z mgły platformy lądownicze dla statków kosmicznych. Na jednej stał prawdopodobnie śmieć, sądząc po kształcie.
Na ławeczce obok siedział Mateusz i popijał poranną herbatę. 
Wszystko było w należytym porządku.
\begin{dialogue}
\ds{} I jak ci się podoba Felicja? \dm{} zapytałem.
\ds{} Wygląda bardzo przyjemnie, mógłbym tu mieszkać. Ale nie wiem, czy by mi pozwolili. Czytałem, że została stworzona sztucznie, jako oaza szczęścia dla ludzi.
\ds{} W teorii. W praktyce jest w niestabilnym stanie na skraju wojny domowej. Albo dokładniej, biorąc pod uwagę ilość obywateli, sprzeczki rodowej.
\ds{} Wiem, początkowo był tutaj komunizm, ale nie doszacowano ilości osób, dla jakich będzie działał i teraz wszyscy szukają alternatywnej ideologii do zastosowania.
\ds{} Mieliśmy już megapuls monarchii absolutnej, megapuls czystej demokracji, megapuls tyranii rasowej, megapuls jakiegoś czegoś gdzie każdy miał chodzić w aninimowej masce itp.
\end{dialogue}

Poprosiłem Mateusza ze sobą i poszliśmy w kierunku ratusza głównego.
Wszędzie było dziwnie pusto, jak gdyby nikt tutaj nie mieszkał. 
Usprawiedliwiłem sobie to tym, że była niedziela ranek.
Na wszelki wypadek trzymałem rękę na szyfratorze.
Jednak Mateusz był dziwnie niespokojny.

Wtem zza rogu wyskoczył na nas wielki potwór.
Wylądował metr od Mateusza i ryknął mu z całej siły prosto w twarz, strosząc kolce.
Mateusz jednak nawet nie zareagował.
\begin{dialogue}
\ds{} Plazma, nie wygłupiaj się \dm{} powiedziałem.
\ds{} Dziwne, nawet nie zareagował \dm{} ryknął w odpowiedzi.
\ds{} Może umarł ze strachu tak szybko, że nawet nie podskoczył.
\ds{} Ja cię kiedyś widziałem. \dm{} Nagle się odezwał. Niewzruszony Mateusz oglądał Plazmę, jak eksponat muzealny. \dm{} Byłeś kiedyś na targach fantastyki w Poznaniu, prawda?
\ds{} I to nie raz. \dm{} Plazma wyszczerzył trójkątne zębiska. \dm{} Widziałeś mnie kiedyś? Miło nie musieć używać kraba wśród obcych, chociaż raz.
\ds{} To ty zawsze wygrywasz konkurs na najlepszy cosplay, tak? Od początku byłem pewnien, że to jednak nie był strój. Zbyt realistyczny.
Do tego czasami zapominałeś się i nie chodziłeś sztucznie sztywno. Wyginałeś pod sobą parkiet, więc także nie byłeś stworzony z pianki. \dm{} Mateusz się rozgadał. \dm{} Ale oczywiście jak mówiłem wszystkim, że to nie jest strój, tylko prawdziwa istota
z kosmosu, to mnie wyśmiewali. 
\ds{} Widzę, że jesteś szczegółowy.
\ds{} Ja żyłem twoją tajemnicą przez kilka miesięcy \dm{} opowiedał \dm{} Ty nawet nie wiesz, jaką histerię spowodowałeś. Wszyscy chcieli wiedzieć kim jesteś i z jakiej gry była twoja postać. 
Właściwie to nawet ze zdjęć stworzyli twój trójwymiarowy model i sami chcieli się za ciebie przebierać, ale nie udawało im się otpowiednio zmieścić ludzkiej sylwetki w twój kształt.
\ds{} To by wyjaśniało, dlaczego ostatnio tak dużo zdjęć mi robią.
\ds{} Obejrzyj YouTuba czasami. Teraz wiem, że od początku miałem rację!
\ds{} To może następnym razem zrobię tak: \dm{} Plazma podniósł łapy, z których wystrzeliły ogniste płomienie, uformowały się w kształt Mateusza. 
Ognisty Mateusz spojrzał na prawdziwego Mateusza, prawdziwy Mateusz popatrzył z powrotem. Potem pomarańczowy kształt rozmył się na wietrze. \dm{} 
Ciekawe, ile wystawcy zapłaciliby mi za zrobienie tak z ich logami?
\ds{} Zrób to, nie mogę się doczekać jak YouTuberzy będą studiować książki od fizyki, żeby wyjaśnić te zjawisko. Może ktoś spali swój dom, próbując potem naśladować.
\ds{} Znamy się od niecałego kilopulsa, a ja już cię lubię. \dm{} Zamachał ogonem. \dm{} Nie myśl jednak, że dam ci fory. Lecimy na Planetę Wojny, dołączysz do Komodowej armii.
\ds{} Ta armia, co ubiera się w pancerze wspomagane, co wyglądają jak komody? \dm{} zapytał. \dm{} Kula miał u siebie książkę na temat tego świata.
\ds{} Dokładnie ta. Będziesz nowym rekrutem, odbędziesz wyczerpujące szkolenie na nowego żołnierza. 
Komody szykują się na ofensywę Hirten, po tym jak grupa imigrantów z północy wprowadziła się do wyludnionego miasta, uciekając przed wirusem rozpylonym nad ich krajem przez Czarnych.
Północnicy będą mieli ze sobą technologię, więc trzeba będzie ich wyrżnąć jeden po drugim, a nie spuszczać atomówkę na miasto.
\ds{} Ciekawie się tam dzieje.
\ds{} Niektórzy mówią też na nią Planeta Chaosu. I nie martw się, wydobędę cię z opresji, gdyby coś się działo...
\ds{} Przypomniało mi się. \dm{} Wręczyłem potworowi kulkę z dziadziem i półpłynnym szampo-żołnierzem. \dm{} Omało nas nie zmydlił.
\ds{} Cholera, to już drugi, mnożą się, czy coś?
\ds{} Ten ma różową słuchawkę, tamten miał zieloną. Znaleźliśmy go w pustce kosmicznej, kilkaset lat świetlnych od Drogi Mlecznej.
\ds{} Czyli może wylał się jakiś wszechświat wanno-dziadziusiów? Ilu ich jeszcze może być?
\ds{} Gorzej, że jego wąż nie jest uniwersalny, więc ignoruje tarczę uniwersalności wokół Kuli, złapał nim takiego żołnierza co uratowaliśmy i wyciągnął ze statku, jak lassem.
\end{dialogue}
Przerwały nam wołania dwójki osób z daleka. Przymoczarscy. 
\begin{dialogue}
\ds{} Hej, to zakazane pokazywać swoją sylwetkę! \dm{} Magda Przymoczarska dopadła naszą grupkę. \dm{} Wedle obecnego prawa, każdy obywatel ma obowiązek być niewidzialny.
\ds{} Co wy znowu odpierdalacie. \dm{} Przewróciłem oczyma. Gdzie był jej małżonek Michał?
\ds{} Nosząc pelerynę niewidkę, obywatel nie ma poczucia, że musi robić tak, jak większość. Nie sugeruje się czynami innych, w związku z tym osiąga lepsze decyzje przy popełnianiu własnych.
\ds{} Ale także nie zostanie nigdy sprowadzony do bezpiecznej normalości, gdyby odleciał, gdyż normalność nie jest teraz zdefiniowana przez nieistniejącą społeczność \dm{} Mateusz zauważył.
\ds{} Tak, ale skąd wiadomo, że powszechnie przyjęta normalność jest najlepszą rzeczą dla wszytkich? A może to właśnie ty masz rację w jakiejś sprawie, a nie większość?
\ds{} Wtedy każdy będzie robił wszystko po swojemu i powstanie niekończąca się wojna o tak nieznaczące zagadnienia, jak noszenie ubrań, czy używanie sztućców, gdyż każdy będzie chciał to robić na swój sposób.
\ds{} A czy konieczne jest, aby każdy nosił ubrania? Abstrachując od tego, że są niewidzialni.
\ds{} To co będzie bronić człowieka przez sraniem na ulicy, jeśli sobie wymyśli, że tak może? \dm{} Mateusz się już denerwował.
\ds{} Wykształcenie. Będzie wiedział, że zarazki z kupy dosięgną i jego, więc nie będzie popełniał autodestrukcyjnych czynów.
\ds{} A odpowiedzialość? Jeśli zrobię społeczeństwu coś źle, no nie wiem, zepuję przypdakiem latarnię, to nikt nie dowie się, że to ja to zrobiłem.
W efekcie nikt nie będzie czuł się odpowiedzialny za dobro ogólne.
\ds{} Sam korzysta z tego dobra ogólnego. I jeśli on je popsuje, to inni mogą także zrobić to samo. Zatem zakładając dobre intencje innych, postara się naprawić wspomnianą lampę, aby 
inni także naprawili rzeczy z których on sam korzysta.
\ds{} Nie można zakładać, że ludzie są dobrzy.
\ds{} Felicjanie są wybrani spośród wąskiej grupy osób. Wszyscy są wykształceni, mówią tym samym językiem, mają wspólną wiarę. Możemy zakładać rzeczy niezakładalne w prawdziwej mieszance społeczeństwa Ziemi.
\end{dialogue}
Mateusz chyba odpuścił. Plazma jednak nie.
\begin{dialogue}
\ds{} Jebcie się wszyscy. Wystarczy jeden zły, aby całe to społeczeństwo przewróciło się na kolana. Co, jeśli trafi się owoc klonu? \dm{} Zapytał rykliwym głosem.
\ds{} Zniszczy go pierwsza, lepsza osoba.
\ds{} Zatem ja jestem owocem klonu. Zniszcz mnie, proszę. A nie sama jesteś za słaba na to, nie masz innych osób do pomocy, nawet ich nie odnajdziesz, żeby poprosić o pomoc. \dm{}
Coś uderzyło Plazmę w nogę. Michał pewnie nosił pelerynę niewidkę. \dm{} No dajesz Michał, myślisz że nie widzę twojego ciepła? Wszystkich was widzę, wszyscy razem nawet mnie nie draśniecie. 
\ds{} Racja... \dm{} Magda zapisała coś w swoim notesiku. \dm{} Więc niech każdy nosi przy sobie szyfrator, aby samodzielnie pokonać każde zło.
\ds{} Głupiaś. Ja nie potrafię pokonać wszystkiego, myślisz, że wystarczy dać człowiekowi broń, a będzie wszechmogący? \dm{} Plazma skończył rozmowę. \dm{} Chodźcie, trzeba zdjąć z Mateusza 
te rokokowe błyskotki.
\ds{} Stać, musisz byś niewidzialny, inaczej zostaniesz usunięty z Felicji.
\ds{} To prędzej ja was usunę. \dm{} Zamyślił się. \dm{} Niniejszym podbijam zbrojnie Felicję i ustanawiam się absolutnym królem do końca dnia. Cośtam, cośtam. 
Możecie zbrojnie wystąpić, ale nie widzę waszego sukcesu z tym: \dm{} Stworzył sobie z ognia świecącą koronę. \dm{} A to są moi niewolnicy. Zakazuję wam rozmowy z nimi.
\ds{} Ale...
\ds{} Wprowadzam prawo, że Magda Przymoczarska nie może się odzywać do końca dnia. \dm{}
\end{dialogue}

Mateusz się chichrał, Magda wymachiwała rękoma, Michał wyłączył niewidzialność na pasie, jego wyraz twarzy mówił ,,a nie mówiłem?''

Poszliśmy do automatycznego krawca.
Na Felicji maszyny robią za ludzi większość roboty, lecz to ludzie muszą je obsługiwać. 
Jeśli chcą na przykład sobie zrobić ubranie, muszą najpierw posadzić bawełnę, zebrać ją, przygotować i na koniec wymyślić projekt do wydrukowania na maszynie.
Dzięki temu pracują, nawet jeśli mają jedzenie i mieszkanie. 
Na Felicji nie używa się pieniędzy, ale im ktoś pracowitrzy, tym bogatrzy, gdyż sam sobie swoje dobro wyprodukował.
Pieniędzmi jest niejako wymiana usług. Ktoś zbierze dwa razy bawełnę dla kogoś, kto wydrukuje ubrania dla obu.

Krawiec znajdował się na głównym placu.
Za szybą stała maszyna stylizowana na XIX wiek. 
Ekran komputera obsługującego miał wskakujące piksele z metalowej matrycy. 
Po oświetleniu ostrym światłem z góry, za pomocą cieni, tworzył czarno-biały lecz w pełni funkcjonalny, interfejs graficzny.
Doskonale to współgrało z ogólnym stylem planety.

\begin{dialogue}
\ds{} Nie mamy bawełny. \dm{} Zauważył Mateusz.
\ds{} Normalnie musiałbyś ją najpierw zebrać, żeby użyć w maszynie \dm{} Plazma ryknął \dm{} ale trochę sobie oszukamy system.
Nadar, masz swój komunikator?
\end{dialogue}

Włączyłem na komunikatorze tryb ściągania materii, kolorowe cząsteczki zaczęły tryskać z obramowania, żeby utworzyć przed ekranem kawałek nici.
Ostrożnie pociągnąłem nowoformowany sznurek i wsadziłem do maszyny. 

Tymczasem Mateusz na analogowym ekranie projektował swoje ubranie. 
Jak na kogoś, kto ma pierwszy raz styczność z naszą technologią, szło mu całkiem dobrze.
Tworzył sobie bluzę i dresy, podobne do moich. Przydadzą mu się na Planecie Wojny.

Gdy skończył, wyciągnął z maszyny bardzo ciekawy zestaw. Miał potworowy charakter, lekki zarys kolców na plecach i rękach, oraz zwiększoną ilość materiału na rękawach i piersi.
Dzięki temu Mateusz wyglądał na nieco większego i groźniejszego.
Do tego wyprodukował sobie sporo zapinanych kieszeni. 
Kaptur miał wbudowaną kominiarkę, żeby chronić przez zimnem, lub wzrokiem innych.
Spodnie były szerokie i nieuciskające, w sam raz do biegania i skakania przez przeszkody. Będę musiał spróbować tego samego.

Poszliśmy następnie na stację tramwajową. Pojedyńcze, zasilane sprężyną, platwormy na szynach automatycznie wychodziły z tunelu.
Wsiedliśmy do jednej, Plazma poszedł na tył, rozwinął skrzydła i swoimi silnikami odrzutowymi wziął nas na szybką przejażdżkę po planecie.
Było jak na kolejce górskiej, ale szybciej i mniej wyboisto.
Próbując przekrzyczeć hałas wiatru i Plazmy, pokazywałem mu mijane miejsca.

Felicja jest mniejsza, niż Księżyc, ale posiada kawałek gwiazdy neutronowej w jądrze, dzięki temu jej grawitacja jest podobna do ziemskiej.
Minęliśmy pola uprawne, na których rosły wszystkie rośliny do wyżywienia planety.
Dalej hodowle różnych zwierząt, nie tylko ziemskich.
Były automatyczne fabryki, na których można było dla siebie stworzyć dowolne przedmioty.
Istniały także elektrownie termojądrowe do zasilania całego systemu.

Ponadto małe morze z plażą i bezludną wyspą, ośnieżona góra do wspinaczki i jazdy na nartach.
Był lasek z prawdziwymi dzikimi zwierzętami, bagnem, polanami i jeziorkami.
Wszystko małe i symboliczne, idealne do szybkiego wypadu.
Na większe wakacje należało lecieć na inną planetę.

Zatrzymaliśmy się w katedrze na obowiązkową niedzielną mszę.
To zabawne, z tym nowym prawem niewidzialności byliśmy jedynymi w kościele. 
Otaczały nasz śpiewy z nikąd, a nad ołtażem lewitowały przedmioty.

Skończyliśmy wyprawę na wieżach do lądowania statków kosmicznych.
Rozsiane były luźno na dużym terenie, aby nie przeszkadzały sobie nawzajem.
Zatrzymaliśmy się pod tą właśnie wieżą, którą widziałem wsześniej. 
Obły kształ śmiecia majaczył na szczycie.

Wjechaliśmy windą na korbę, Plazma polecił Mateuszowi nią kręcić.
Dzielny wniósł mnie, siebie i Plazmę na sam szczyt, łącznie prawie siedem ton. 
Zmachał się tym całkiem nieźle, musiał przerywać kilka razy na odpodczynek, ale udało mu się.
Potwór był pod wrażeniem. Powiedział, że chciał go zahartować przed testem. 
To był oficjalny koniec bankietów, zwiedzań i przejażdżek. Teraz będzie ciężka praca.

\divider{}

Spostrzegł, że nie ma już ani Winkli, ani Everywhere Mana. 
Był sam w całym miasteczku.
Albo też demony chciały, żeby myślał, że jest sam.

Począł przechadzać się po wyludnionej wiosce, zaglądał do dziur, wołał do studni. Nic.
Postanowił zwabić pozostałe demony tym, co lubiły najbardziej.
Usiadł na środku i zaczął opowiadać. Tylko co lubiły?

\divider{}

\begin{dialogue}
\ds{} To jak tam u ciebie? Jak minęła podróż? \dm{} Zapytałem się Mateusza, puszczając stery śmiecia. W kosmosie i tak w nic nie uderzy.
\ds{} Jakoś.
\ds{} Słyszałem, że podziwiałeś mnie na targach w Poznaniu. Podobałem ci się?
\ds{} Taaak... \dm{} Mateusz zrobił dziwną minę.
\ds{} Przede mną nie musisz się ukrywać. Powiedz, co czujesz.
\ds{} Co ja czuję. \dm{} Skulił się w kąt rakiety.
\ds{} No to w takim razie powiem ci, co ja czuję. \dm{} Złapałem i zdjąłem mu majtki. \dm{} Pokażę, jak robią to kosmici.
\ds{} Puść, zostaw!
\ds{} W kosmosie nikt nie usłyszy twojego krzyku.
\end{dialogue}

\divider{}

\begin{dialogue}
\ds{} Dość, wystarczy! \dm{} Antyrax usłyszał wołanie ze studni. \dm{} Nikt nie chce pedalskiego, międzyrasowego kosmicznego gwałtu półjaszczura na człowieku!
\ds{} W nieważkości \dm{} dodał autor.
\ds{} Aaaahhh!
\end{dialogue}
No to wygląda na to, że to nie jest jedna z tych rzeczy, które demony lubią.
Spróbujmy jeszcze raz.

\divider{}

Śmieć był olbrzymim statkiem. 
Nazywały się tak, ponieważ były w pełni mechanicznie tworzone ze słabej jakości materiałów i na prostej mechanice.
Idealne na misje w niebezpieczne miejsca, gdzie ich ewnetualna strata nie będzie dużym problemem. 
A co najważniejsze, nikt nie wyciągnie z nich żadnej cennej technologii.

W śmieciu numer 877 wszystko trzeszczało. 
Ten statek dawno powinien się rozlecieć, lecz nadal się jakoś trzymał.
Połowa urządzeń nie działała. Od systemu wykrywania uniwersalności, po żarówkę na korytarzu.

Lewitowałem w ciemnościach, niesiony bezwładnością w nieważkości.
Kluczące korytarze ciągnęły się po całym kadłubie.
Dwa razy przebijałem się przez zasłonę z kabli, raz trafiłem na ślepy zauek.

W końcu doleciałem do kajuty Mateusza. 
Zapukałem, echo rozniosło się po ścianach.
Nikt nie odpowiedział, więc weszłem do środka.
Kajuta oświetlona była przepalającą się żarówką.
Poduszki, rzeczy osobiste i papiery unosiły się bezwładnie po przestrzeni.
Nigdzie nie było człowieka, wyglądało na to, że opuścił to miejsce w pośpiechu.

Moją uwagę przykuła świeża ściana, stojąca na środku pokoju.
Wykonana była z dziwnego, metalicznego materiału.
Jej obecność nie miała konstrukcyjnego sensu.
Podpłynąłem, aby się jej bliżej przyjrzeć.

Wtem usłyszałem daleki chrobot i padła elektryczność.
Ciemność pożarła mnie razem z całym statkiem.
Zapaliłem ognik w ręce. Pomarańczowe, migotliwe światło wydłużało cienie i poruszało ścianami.
Spostrzegłem, że tajemnicza ściana zniknęła, nie było po niej żadnego śladu.
Jak to możliwe?

Wypłynąłem na korytarz.
Ogień mógł oświetlić jedynie kika metrów w jedną i w drugą stronę.
Skierowałem się w prawo, bo z tamtąd przyszedłem.
Leciałem środkiem przez kilka kilopulsów, aż trafiłem na zamknięty właz. 
Nie przypominałem sobie, żeby tutaj był. 
Złapałem go pazurami, zaparłem się ogonem o ścianę i spróbowałem przekręcić koło. 
Piszczący dźwięk rozniósł się po korytarzach, echo odpowiedziało straszliwym jękiem. 
Aż mi się łuski zjeżyły na ogonie.

Pokręciłem jeszcze, odpowiedział mi syk. 
Za drzwiami musiała być próżnia.
Mateusza na pewno tam nie było.

Poleciałem więc w drugą stronę, minąłem jeszcze raz kajutę w której byłem i zamarłem.
Przed biurkiem siedział wyschnięty szkielet. Nie Mateusza.
Jednak skąd się tu wziął? Nie pamiętam aby ktoś na ostatnich misjach zaginął. 
W ręce miał długopis, w kartce wybita była dziura od nadmiernego pisania w jednym punkcie. 
Nie dało się rozczytać, litery były zapisane jedna na drugiej po kilka razy.

Poleciałem dalej, trafiłem na ścianę. Jak to możliwe?
Obmacałem ją, jakbym nie był pewien, czy jest prawdziwa.
Składała się z tego samego materiału, co ta w kajucie.

Wściekły uderzyłem pięścią w środek, odpowiedział głuchy dźwięk, pazury weszły mi do połowy.
Ze środka zaczęła kapać gęsta ciecz i spływać po ścianie.
Przecież panowała nieważkość!
Odbiłem się w drugą stronę. Poleciałem z powrotem.
Lecąc, trzeci raz zajrzałem do kajuty.
Była zamknęta, chociaż pamiętam, jak nie dodykałem drzwi.

Doleciałem do śluzy. Była otwarta na oścież.
Kto? Kiedy? Jak?
W śluzie latały papiery, pośrodku stała nienaturalna ściana. Kajuta.

Wtedy wróciła elektryczność.
W żółtym świetle żarówki zobaczyłem, że ktoś leży w łóżku.
Mateusz wstał i przetarł oczy.
\begin{dialogue}
\ds{} Plazma, wyglądasz jakbyś zobaczył ducha \dm{} powiedział.
\ds{} Coś się dzieje. Mamy uniwersalność na pokładzie!
\ds{} Nie opowiada...
\end{dialogue}

Światło ponownie zgasło.
Na miejscu Mateusza wisiał w powietrzu szkielet.
Ściana także zniknęła.

\divider{} 

Nikt nie wyszedł. Może horrory to nie jest ich domena. 
Szkoda jednak psuć taki dobry horror. 
Może trzeba go przerobić.

\divider{}

Dość!
Skupiłem w ręce promień standardowy i strzeliłem w lewitujące kości.
Laser przeciął czaszkę na pół.

Potem posłałem falę ognia, wszystkie papiery zapaliły się jasnym światłem.
Otoczony tańczącymi iskrami odwóciłem się do korytarza.
Wycelowałem w ciemność i skierowałem w nią ogniki.
Ściany zajęły się ogniem, zmuciłem niepalne żelazo do palenia się.

Rozgrzałem pięść do białości, rozciągnąłem strzydła i poleciałem w kierunku naprzeciwległej ściany.
Za mną rozciągało się ogniste piekło.
Z całej siły uderzyłem w to samo miejsce, gdzie wcześniej.
Rozciągnąłem na sobie pole siłowe.
Sztuczna ściana rozpadła się, jak szkło. Czarna ciecz ochlapała mnie, widziałem z wewnątrz kulistej bańki, jak spływa po ściankach, jak spala się w żywym ogniu.
Ostatkiem sił próbowała przebić się przez moją tarczę. Rozgrzałem ją wtedy do tysiąca stopni. Ciecz w mig wyparowała, jak woda wylana na patalnię.

Okazało się, że przebiłem się do maszynowni.
Języki ognia lizały mnie przez wybitą dziurę.
Wtedy zobaczyłem Mateusza. 
W jednej ręce miał karabin, w drugiej daser.
Daserem ciął wszystko, co się ruszało, karabinem strzelał naokoło, aby poruszać się siłą odrzutu.
Czarna maź formowała co chwila sztuczne ściany, pluła losowymi przedmiotami. 

Standardowym promieniem ciąłem ją na kawałki, nic to nie dawało.
Przynajmniej przestała atakować Mateusza.
Ogniste podmuchy widać trochę ją spowalniały, iskry przebijały się przez sztuczne ściany.

Zaczęło się robić gorąco. Pomyślałem o Mateuszu. 
Widać, że ocierał co chwila pot z czoła.
Ledwo dychał.

Na ścianie zobaczyłem skafander kosmiczny i butle z tlenem.
Ryknąłem i rzuciłem je człowiekowi, złapał.
Wtedy otoczyłem go kulistą tarczą, jak przed chwilą siebie i wypuściłem z wnętrza cały mój wewnętrzny ogień.
Ściany poczęły się topić.
Uniwersalność zapaliła się ogniem. 
Guma z kabli wyparowała. 

Ostatkiem sił uderzyłem standardowym promieniem w dolną ścianę.
Przebiłem się przez zmiękczony kadłub i otworzyłem dziurę na próżnię.
Wielki lej ognia uformował się w kierunku wyłomu.
Wystrzeliliśmy w kosmos, jak korek z butelki.

Popatrzyłem się za siebie. Tył rakiery jaśniał pomarańczonym światłem, gejzer ognia wylatywał przez wybitą przeze mnie dziurę.
Mateusz w kuli także podziwiał.

Gdy ubrał się w skafander, opuściłem pole siłowe i usnąłem z wyczerpania.

\divider{}

Nie?
Nikomu się nie podoba?
To może... nie wiem.

\divider{}

Lecę przez miliardy gwiazd. \\
Niczym martwy pingwin patrzący na spadający śnieg. \\
Nieobecnymi oczyma patrzy się w niebo. \\
Martwy. \\

Obca galaktyka. \\
Obce gwiazy. \\
Obcy człowiek przy boku. \\
Wkrótce martwy. \\

Minie milion pulsów. \\
Minie miliard pulsów. \\
Nie będę martwy. \\
Lecz czym jest nieśmiertelność w nieistnieniu? \\

Rozżażony statek. \\
Niczym iskra pośród gwiazd. \\
Niczym gwiazda pośród iskier. \\
Także zgaśnie. \\

Uniwersalność była. \\
Uniwersalność się skończyła. \\
Uniwersalność będzie na wieki. \\
Z wieczności uniwersalność powstała. \\

\begin{dialogue}
\ds{} Daj spokój. Zaczynasz przynudzać \dm{} Mateusz jęczał. 
\ds{} Trochę nudy potrzeba po tym, jak prawie zuniwersalizowało nas żywcem.
\ds{} Może. W każdym razie nie ma to znaczenia. Żyjemy, prawda? Poza tym jesteśmy już w układzie Planety Wojny. Za kila godzin trafimy na planetę.
\ds{} Nieźle sobie poradziłeś, tak w ogóle.
\ds{} Ja? Przecież to ty mnie uratowałeś. 
\ds{} No, ja to ja. Ale tym miałeś jedynie laser i przeżyłeś! Mało kto przeżywa atak aktywnej formy uniwersalności. \dm{} Chciałem mu teraz wytłumaczyć jej podział na typy, ale zmieniłem zdanie. \dm{}
Skąd ty w ogóle masz daser? To zbyt cenna technologia, aby instalować ją w śmieciach. Pomyśl, co by sie stało, gdyby uniwersalność ją przejęła.
\ds{} Wiem. Przeżyliśmy u Kuli atak wojsk USA, mieli ukradziony daser i prawie nas przecięli na pół.
\ds{} No, no.
\ds{} Katarzyna podarowała mi swój własny. Powiedziała, że ma być na czarną godzinę.
\ds{} Kto? Kosma? Pogadam z nią trochę. To nieodpowiedzialne.
\ds{} Przecież gdyby nie ona, mógłbym już nie żyć.
\ds{} Ale... \dm{} Miał rację, nie mogłem mu tego odmówić. \dm{} Sam widziałeś, co się dzieje z daserami w rękach wroga. Nie ma mowy, nie dostaniesz go na wyprawę.
Poza tym, i tak by ci go zabrali.
\ds{} Czy muszę na prawdę przechodzić ten test? To by nie wystarczyło? \dm{} Wskazał na oddalającą się iskrę.
\ds{} Już mówiłem. To test charakteru, masz się zachować poprawnie.
\end{dialogue}

\divider{}

Wygląda na to, że nawet jego własne postaci nie chciały słuchać jego poetyckiej twórczości.
Antyrax postanowił podejść postapokaliptycznie.

\divider{}

Już z daleka zobaczyłem ogniste pociski, lecące w naszym kieruku.
Wyciągnąłem łapę i pozbawiłem je w jednej chwili ognia.
Bez odrzutu, spadły z powrotem na ziemię.
Zaraz potem pojawiły się tam wielkie grzyby atomowe, co za idiota strzela pociskami atomowymi w przelatujące samoloty?
Chyba, że wiedzieli, że nie jestem żadnym samolotem.

W pustynnej, zakurzonej atmosferze, zobaczyliśmy jezioro Hirten i przybrzeżne miasto o tej samej nazwie.
Teraz to nie nazywało się oczywiście Hirten, ale kto by tam ogarnął obecną sytuację polityczną.
Automatyczne systemy obronne skierowały na nas lasery. 
Wsadziłem Mateusza na grzbiet i przyjąłem pełen strumień na klatę. Gilgotało.

Zlecieliśmy, aby sunąć tuż przy ziemi.
Z piasku wyskakiwały na nas co chwila technogony, ale ja byłem za szybki, żeby dać się złpać.

Wlecieliśmy nad zatrute jezioro, trujące opary przyjemnie drapały w gardle.
Jednak ze względu na człowieka, którego niosłem w rękach, trzymaliśmy się daleko od powierzchni.
Hirten mieniło się w świetle słońca. 
To niesamowite, że nigdy jeszcze nie zostało zbombardowane. 

Wylądowałem na szczycie najwyższego wieżowca. 
Szklany taras był kiedyś apartamentem jakiegoś bogacza. W wyschniętym basenie leżały resztki damskich strojów kąpielowych.
Palmopodobne rośliny, posadzone w donicach, straciły całkowicie liście.
Wyblakłe kafelki pokryte były niezmywalną warstwą kurzu.
Tak kiedyś może wyglądać Ziemia.

Wybiłem szybę, żeby wejść do środka.
Było ciemno i nieprzyjemnie, przypomniało mi się wnętrze naszej rakiety.
Zapaliłem ognik i zobaczyliśmy stosy wyschniętych ciał, pośrodku parkietu do tańca.
Widać było, że umarli nagle, dusząc się. Atak chemiczny.

Musiał złapać ich w trakcie przyjęcia.
Uciekli z balkonu, zamknęli drzwi, i mieli nadzieję przeczekać.
Muzyka musiała grać aż do końca, a kolorowe płyty podłogowe pewnie migały jeszcz jakiś czas po ich śmierci.

Przyjrzałem się bliżej, była to rasa północnych.
Wyglądali jak brzydcy, chudzi ludzie.
Mieli na sobie całe stosy elektroniki.
Transhumanizm pełną gębą.
To mi podsunęło myśl, że przeżyć musieli ci, którzy wcześniej zainstalowali sobie filtry w nosach.
Ale Komodowa Armia zapewne za kilka dni wkroczy do miasta, aby ich wykończyć.
Nie taki był plan, to miało się zdarzyć dopiero za rok.

Wyszliśmy na ulicę. Opadnięte samochoty latające, wyłączone hologramy, uduszeni ludzie.
Ani żywej duszy. Usłyszałem w oddali tupot, podskoczyłem i schowaliśmy się w jakimś mieszkaniu.
Zaraz zza rogu wyszła trójka Czarnych. To oni zabili ludność?
Szkoda, że nie rozumiałem ich języka.

Poleciłem Mateuszowi zostać, sam podskoczyłem do nich.
Naturalnie zaczęli strzelać, ale ja nie byłem bezbronny.
Jednemu urwałem nogę, czarny smar khaki, wymieszany z niebieską krwią, rozlał się po asfalcie.
Drugiego wyrzuciłem wysoko w powietrze, upadł kilakset metrów dalej. 
Trzeciego uniechuchomiłem, spajając mu ognistym promieniem łączenia w zbroi.

Głupio to musiało wyglądać, jak odpytywałem go, rysując obrazy na chodniku, pozostałościami jego kolegi.
Rozumiałem tylko jak przytakiwał i zaprzeczał.
Z tego się dowiedziałem, że rzeczywiście Komodowa Armia miała wykonać szturm na miasto, do którego przeprowadzili się imigranci z północy.
Jednak Kryształowa Królewna posanowiła potajemnie zatruć miasto, a potem ukryć w nim swoich podwładnych.
W ten sposób urządziła zasadzkę na Komodowych. Sprytne.

Podziękowałem mojemu informatorowi, a potem wróciłem do Mateusza i polecieliśmy pod osłoną nocy do najbliższej wioski pod władaniem Komodowców.

\divider{}

Co się dzieje? 
Ciągle nikt nie wychodzi z żadnych dziur.
Znowu nie trafił z typem opowieści?
Nie było rady, jak spróbować ponownie.

\divider{}

Pociąg lewitował na strunie, tuż nad powierzchnią ziemi.
Święcący promień przechodził przez środek urządzenia, wychodził w silniku.
Prowadził maszynę, od podpory z pierścieniem, do podpory.
Silnik jądrowy dawał bardzo ładne, wiśniowe światło, trzy ogniki zostawiały za sobą strugę przyjemnego, mocno rozgrzanego powietrza.
Po bokach widniały laserowe działka, lecz nie reagowały na naszą obecność.

Podleciałem od spodu, aby ukratkiem zajrzeć przez okno i upewnić się, że to poprawny pociąg.
Wnętrze spowite było niebieskim światłem.
Rzędy smutnych wieśniaków siedziały, patrząc nieobecnymi oczami w pustkę.
To musieli być poborowi.
Odebrano im całą elektronikę, bez gogli rozszerzonej rzeczywistości nie umieli się zająć niczym innym, a zwłaszcza rozmową.

Holograficzny wskaźnik na ścianie wskazywał następną stację.
Nie znałem tutejszych języków, ale byłem pewien, że nikt na pokładzie pociągu nie potrafił czytać i pisać.
Z obrazka dowiedziałem się także, że końcową stacją jest duże zagłębie militarne Komodowej Armii.

Poleciałem przodem, wyprzedzając pociąg, aż trafiłem na małą wioskę przy wyschniętym korycie rzeki.
Bardzo biedna i nawet bez elektryczności.
Jedynym oświetlonym miejscem był przystanek kolejowy.
Wyglądał, jak wmuszony w slumsową, lepiankową architekturę, jakby spadł z nieba. Może tak właśnie było.
Czerwone hologramy ostrzegały przed nadejściem składu za kilka chwil.

Grupę młodych ludzi ustawiono na placu.
Płaczący rodzice żegnali się z nimi, nie oczekując że zobaczą ich jeszcze kiedykolwiek.
Dawali im na drogę ubrania, jedzenie, a nawet zabawki.

Wylądowałem po cichu w ciemnym zaułku i wypchnąłem Mateusza, aby dołączył do grupy.
Zapewniłem, że będę go ubezpieczał i że wezmę go, zanim pójdą wyżynać Hirten, a raczej, zanim wpadną w pułapkę.
Na tle mieszkańców Planety Wojny będzie wyglądał, jak upośledzony grubasek. Musi tylko ograniczać pokazywanie się na golasa.
Nie będzie także bariery językowej, każda wioska mówi tutaj różnym akcentem, więc prawdopodobnie w wojsku będą się posługiwać jakimś pismem obrazkowym.
Idąc, miał mord w oczach.

Wszyscy mówią, że przesadzam z tymi testami, ale uważam że jak ktoś potrafi przeżyć ideologiczne piekło, to przeżyje wszystko.
Nie można dopuścić, aby agent ALOPP podjął złą decyzję przy ratowaniu innych istot.
Czy Mateusz będzie próbował ochronić ich przed wpadnięciem w pułapkę?
Czy wywoła powstanie?
A może zaatakuje przełożonych?
Do czego jest skłonny, tego wszystkiego się dowiem.

\divider{}

Ze studni wygramolił się Myestro.
Podszedł kilka kroków, niepewnie jak dzikie zwierzę.
Chyba był zainteresowany. Może należy kontynuować ten cyberpunkowy styl?

\divider{}

Drzwi otwarły się i smód wioski wdarł się do alemona.
Cała masa obrzydliwych wieśniaków zajęła ostatnie miejsca w pojeździe.
W tym jeden usiadł obok mnie, był jakiś dziwny.
Był ciemniejszy ode mnie, bardziej pomarańczowy.
Widać teraz biorą do armii kogo popadnie.

Dzwi się zasunęły i klimatyzator począł czyścić powietrze ze stęchlizny. Niewiele to oczywiście dało.

Większość osób harała, pelne kisy spływały im z hateriów, jak małym dzieciom.
Nie wierzę, że jestem tutaj z nimi. Zamiast cieszyć się, że będą bronić swojego kraju przed Mroźnikami, oni wylewają masę hariów po nic.

Wzięli mi mój geter, musiałem siedzieć i patrzeć się przez okno na nudny krajobraz.
Światło duzji oświetlało otoczenie harbestowym, miękkim światłem.
Piasek, piasek i trochę skał.
W oddali, na horyzoncie widziałem łunę Torten, to miasto zostało bezpodstawnie zabrane nam przez Mroźników.
Czułem się dumny, że będę brał udział w jego odbiciu.

Mój sąsiad wyraźnie był na coś wściekły. 
Jak można być złym w takim momencie?
A może był upośledzony?
Spróbowałem nawiązać z nim kontakt, ale oczywiście nie rozumiał, co do niego mówiłem.
Standard.

Zza horyzontu wynużyła się wielka forteca naszej armii.
Alemon zwolnił przed wrotami i zjechał na boczną duzję. 
Wielkie łapy złapały kadłub alomona i unieruchomiły.
Wielka rura przyssała się do drzwi, dał się słyszeć dźwięk spuszczanego powietrza.
Szkło się otwarło i czysta, pachnąca technologią atmosfera, zastąpiła wioskowy smród zgromadzonych.

Wychodzili pojedynczo, widać każdy chciał być ostatni.
Skorzystałem z okazji, przepchnąłem się przez śmierdzieli, żeby iść przodem, z dala od tej hołoty.
Co ciekawe, ten dziwny grubasek także chciał być na przedzie. Może jednak nie był z wioski?

Ruchomy regel prowadził nas przez odpowiednie korytarze.
Półprzezroczyte nizjeny wskazywały wszystkim drogę pismem obrazkowym.
Gdzieniegdzie przewinął się napis po zurdowsku.
Trochę głupio mi było przyznać, ale ja także nie rozumiałem zurdowskiego. 
Pomimo, że świetnie pisałem, czytałem i mówiłem w halat, behalat, po culsku i polivinowsku.

Dojechaliśmy do rzędu jaborysów. Wysoki zudloniec z pałeczką w ręce wskazywał każdemu z okienka, jakiego jaborysa ma użyć do obesztania się.
Poszedłem pierwszy, żeby pokazać innym, co to znaczy prawdziwa technologia.

Kabina była dość przestronna, jak tylko weszłem, szyby straciły przezroczystość.
Nizjen wskazywał, że mam ściągnąć ubranie i wrzucić do dziury, co też uczyniłem.
Zimna para ochlapała mnie, potem jakiś śmierdzący detergent, potem ponownie woda.
Trwało to kilka razy.
Na koniec, świecąca ściana przeskanowała całe pomieszczenie.
Czytałem o niej, to hutona, paliła wszystkie nieporządane substancje w ciele.
Ja nie miałem ich za dużo, ale tamtym na pewno się bardzo przyda.

Poszłusznie dałem się przeskanować, zapiekła mnie susetria, hateria i joty.
Lelonie paliły żywym ogniem, zarówno wewnątrz, jak i na zewnątrz. Ale to było zrozumiałe.
Straciłem wszystkie dulice, ale przecież odrosną.
Wykończony, otrzymałem nowe ubranie.
Było bardzo proste, szare, lekkie i wykonane z faflocji.
Na lelonii i susterii miałem wydrukowany jakiś symbol. 
Fala z przecięciem z lewej, wpisana w wejt.
Mógł być to mój numer porządkowy, litera a może liczba?

Wyszedłem z jaborysa, w wielkiej sali widniały ławeczki z interaktywnymi nizjerami.
Usadowiłem się przy jednej, wyświetlił się mój symbol.
Krótka animacja wskazała, że od teraz będzie to mój numer porządkowy.
Usłyszałem, jak się wymawia. 
Był pierwszy w kolejności, byłem numerem jeden, to prezent za odwagę?
Nizjer wprowadzał mnie w tajniki zurdowskiego, z chęcią przyswajałem nowy język.
Był niezwykle prosty.

Kolejne osoby korzystały z jaborysów, jedne bardziej krzyczały w trakcie hutonowania, inne były całkowicie cicho.
Mój pomarańczowy ulubieniec darł się w niebogłosy, aż mi się szkoda go zrobiło.
Wyszedł, wyglądając prawie tak samo. Widać miał tak niedorozwinięte dulice, że ich strata nie zmieniła mu wyglądu.
Jednak jego symbol był bardzo fajny. Symetryczny, podwójny fezel.
Wiedziałem już, jak się wymawia. Brzmiał groźnie.
Pomimo, że był w kolejności bardzo daleko od mojego, nie poprawiło mi to humoru.
Widać numery przydzielano losowo.

\divider{}

Myestro usiadł bardzo blisko.
To wystarczyło.
Antyrax zamachnął się na niego linią poleceć i złapał w powłokę.
ZSH trzymało go mocno i nie puszczało.
To wystarczyło, żeby zadać cios ostateczny.

Przyszykował program w BASHu, wbił mu go w ciało i uruchomił.
Wykryje on częste błędy w jego powieściach, rozładowywując demona słownikową perfekcyjnością.
Jednak program był póki co bardzo prosty, znajdował tylko błędy, które Antyrax popełnił wcześniej.
Mogło to nie wystarczyć.

Dla pewności kontynuował opowiadanie w cyberpunkowym kierunku.

\divider{}

Gdy wszystkich oczyszczono, na wielkim nizjerze wyświetlił się mój symbol, a po sali rozległa się jego wymowa.
Hos.
Poszedłem triumfalnie w kierunku wskazanych drzwi, wypinając na innych susterię z moim pierwszym numerem.

Przechodziłem wąskimi korytarzami, prowadzony przez nizjery. 
Tutaj już nie było regeli, ale przecież byliśmy w wojsku, to zrozumiałe.

Zaprowadziły mnie do sporej salki ze stołami.
Na talerzach dymiły przepyszne vinite.
Niesamowite, czym nas tutaj karmili.
Usiadłem przy swoim symbolu, złapałem yltona i zacząłem jeść, nie czekając na innych.

Jak się spodziewałem, zaraz zjawili się pozostali.
Nie była to cała grupa, dokładnie dwadzieścia osób.
W tym ten dziwak, Rohost.
Przydzielono mu miejsce na końcu mojego stołu.
Nie żebym się cieszył, że znalazł się w naszej grupie, ale może będzie z nim trochę ciekawiej.
Byle tylko nie opóźniał ćwiczeń.

Przycisk obok mojego talerza pozwalał dobrać sobie dokładkę.
Vinite wylatywały z zakrzywionej rurki.
Nieskończona ilość vinite! Gdybym wiedział, od razu zgłosiłbym się na pobór.

Wszyscy inni podchodzili do jedzenia z dystansem, wiadomo było, że we wsi trudno o składniki na vinite, zapewne większość z nich pierwszy raz widziała te pyszności na oczy.
Rohostowi absolutnie nie smakowało, jednak zmuszał się do jedzenia.
Co za marnotractwo.
Zjadł pięciokrotność mojej porcji, gdzie on to wszystko mieścił?

Na ścianie wisiał kolejny nizjer.
Tam wyświetlono animację, z której wynikała pozycja naszego oddziału w całej fortecy. Nie byliśmy pierwsi z kolei, szkoda.

Dostaniemy niedługo własne pancerze wspomagane, zwane hiperami.
Będziemy w nich ćwiczyć walkę, a także mieć trochę siłowych i zręcznościowych zajęć bez nich.
Każdy hiper ma wbudowany elektryczny taser, którym będą nas razić za niesubordynację.
Ćwiczenia potrwają cały sept, a potem szturmujemy Torten i wyżynamy Mroźników.
Dla mnie spoko.
Pierwszy raz w animacji użyto zurdowskich liter, ale w małych ilościach.
Jakiś wysoki rangą oficer powiedział coś po zurdowsku, a potem wszystko poszło do nowa.

Gdy wszyscy się najedli, ściana się otwarła, odsłaniając sporą salę z rzędami concatesów.
Surowo urządzone pomieszczenie oświetlone było białym światłem podłogowych żył.
Oddzielało concatesy od siebie, aby każdy miał swoją małą, pseudoprywatną przestrzeń.
Każdy concates miał odpowiedni symbol na futusi, oraz przydzieloną bulię z zestawem wszystkich potrzebnych urządzeń.
Było też wejście do personalnego jaborysu.
Obok każdego concatesa widniała wnęka w ścianie, prawdopodobnie tam będą stały nasze hipery.

Znalazłem przydzielony mi concates, przejrzałem zawartość bulji, gdzie znalazłem prosty geter.
Nie pozwalał na komunikajcę ze światem zewnętrznym, ale miał sporo książek do nauki zurdowskiego, a także szczegółowe instrukcje użytkowania hipera.
Nie mogłem się doczekać następnego dnia, nie sądzę, żebyśmy już juro otrzymali hipery, ale kto wie.
Wsunąłem się pod futusę i zafulałem, jak tylko zgasło światło.

\divider{}

Nagle coś uderzyło Antyraxa w głowę, tak że wypuścił z objęć Myesta.

Kolejny przyszedł się zabawić?

Antyrax wstał, lecz nie mógł nikogo zobaczyć.
Cisza spowijała wymarłą wioskę dokładkie tak samo, jak wcześniej.

\begin{dialogue}
\ds{} Pokażcie się, bo dokończę ten zboczony fragment! \dm{} Antyrax zawołał. Nikt mu jednak nie odpowiedział. \dm{} Dodam do tego transformacje płci, korki analne i masochistyczne tortury.
\ds{} Aaah... \dm{} Ledwo żywy Myestro jęknął z ziemi.
\ds{} No dobra, lepiej zatkajcie uszy...
\end{dialogue}

Kolejne uderzenie zwaliło go z nóg. 
WickedH stanął przed nim, wcale się nie kryjąc.
Chciał więcej abstrakcji, jeszcze więcej.

Antyrax posłusznie wziął się za opowiadanie.

\divider{}

Swiatło powoli się rozjaśniało, budząc nas delikatnie.
Zauważyłem, że mój ulubieniec od dawna nie śpi, tylko rysuje coś na swoim geterze, zakrywając tył, aby nie było widać.
Bardzo dziwne.

Od razu poszliśmy na śniadanie, nawet się nie przebierając. I tak nie było po co.
Dzisiaj w stołówce było gorzej, jafanele nadziewane duduzją.
Nikomu nie smakowało, nawet Rohostowi. Wybrzydza, jak Kryształowa Królewna.

W czasie jedzenia nie było komunikatu, za to ściana po drugiej stronie naszej sypialnie otwarła się i silne światło słoneczne oślepiło nas.

Otwór wychodził na spore patio w kształcie trójkąta, oddzielone wysokimi ścianami.
Wszechobecny piasek pokrywał je grubą wartstwą.
Znajdowało się wewnątrz militarnej fortecy, można było założyć, że pozostałe patia ustawione są promieniście po okręgu, oddzielone od świata głównym murem.
Od zewnętrznej strony widniały schody okalające wyjście, prowadziły do wielkich wrót.
Zapewne tędy będziemy opuszczać fortecę.

W piasku wysunął się kolejny nizjer, ledwo go było widać w tym słońcu.
Z nizjera ukazała nam się twarz znajomego generała.
Mówił, mocno gestykulując w akopamiamencie różnorakich animacji.
Kazał się nazywać dyrygentem, był przydzielony do naszej grupy.

Powiedział, a raczej pokazał nam, że pierwszy fodor zajmiemy się ogólnym treningiem siłowo-zręcznościowym zanim w ogóle mamy myśleć o hiperach.
Będą też zajęcia teoretyczne.

Z piasku wysunęły się pale zakończone poduszkami i schodki.
Naszym zadaniem było przeskoczyć na drugą stronę, trudne zadanie, jak na pierwszy raz.
\begin{dialogue}
\ds{} Zumi \dm{} powiedział, wyświetlając symbol jedego z nas. Zumi zaczął gramolić się po schodach.
\ds{} Ofte. \dm{} Wskazał drugiego.
\ds{} Sisto.
\ds{} Matart.
\ds{} Lukne.
\ds{} Hos. \dm{} O, to ja.
\end{dialogue}
Wszyscy inni pospadali w trakcie skoków, musiałem teraz pokazać się z jak najlepszej strony.
I udało mi się, spadłem dopiero na piątej poduszce, z dwudziestu.
Najdalej ze wszystkich.

Na koniec poszedł Rohost, przeskoczył wszystko, jakby od dawna to robił.
Niesamowite, na pewno nie był upośledzony. Kim w takim razie był?

Powtarzaliśmy skoki aż do znudzenia.
Komu się udało, miał do końca zadania wolne, mógł szyderczo obserwować pozostałych.
Przeskoczyłem jako piąty i korzystając z zamieszania, zakradłem się z powrotem do sypialni.

Nikogo nie było, porwałem więc ukradkiem geter Rohosta i począłem przeglądać.
Nie próżnował w nocy, zapisał całe mnóstwo kreskowych obrazków.
Jakieś mapy i matematyczne obliczenia.
Zaznaczył pozycję fortecy, wioski przy której wsiadł do alemona, przebieg duzji, kierunek łuny Torten, jezioro i... Torten? Ale to jest źle, Torten jest w innym miejscu.
Do tego były wykresy z naszą planetą i słońcem.
Ponadto sporo całkowicie niezrozumiałego pisma.

Skopiowałem sobie wszystkie jego dzieła na własne urządzenia i odłożyłem mu jego geter z powrotem tak, jak był.
Wyszedłem na zewnątrz, ostatnie osoby kończyły zadanie. Nikt się nie zorientował.

Następne ćwiczenie było siłowe. 
Poszło mi bardzo źle, ale to Rohost był na końcu.
Jak to się stało, że umiał tak dobrze skakać, a był słaby jak wejdel.

Było jeszcze kilka zadań testujących naszą prędkość, reakcję i suternę.
Rohost ponownie poległ, nie był suterny nic, a nic.

Zmęczeni, zjedliśmy kolację składającą się z pieczonych didisów.
Każdy z nas poszedł się wyhutonować w jaborysie.
Rohost zakrył wejście do jaborysa swoją futusą, rzeczywiście wstydliwy jak Królewna.

Potem była część teoretyczna, w stołówce stoły skryły się w podłodze, robiąc z niej salę wykładową.
Animacje pomieszane z zurdowskimi literami uczyły nas o naszych wrogach.
Było sporo o Czarnej Armii, o Mroźnikach, ich kraju i mieście Torten.
Przypomniały mi się notatki Rohosta.

Wieczorem był czas na naukę języka, co też wszyscy bez wyjątku posłusznie czynili.

\divider{}

\begin{dialogue}
\ds{} Ale się guzdrzesz! Daj życia do tej opowieści! \dm{} WickedH pogonił Antyraxa.
\end{dialogue}

\divider{}

Następny dzień był bardzo podobny do poprzedniego.
Śniadanie, ćwiczenia, obiad, wykłady, nauka języka.
Zaczęło się pokazywać, kto był najsłabszy z nas wszystkich.
Nie był to Rohost, gdyż w jednych dyscyplinach przodował, a w innych był na szarym końcu.
Nie byłem to też ja, na szczęście.

Wieczorem, zamiast nauki, postanowiłem przejrzeć ukradzione notatki.
Dyskretnie przeglądałem obrazy getera, rozumiejąc z nich coraz więcej.
Były mocno niekompletne.

Dwa dni później ponownie udało mi się szybko wygrać i ponownie zakradłem się do sypialni.
Przybyło sporo notatek, widać, że robił postępy.
Moją uwagę przykuł również nietypowy zapach jego concatesa, nigdy nie czułem wcześniej niczego podobnego.
Poszperałem pod futusą i na okinach został mi dziwny proszek.

W drugiej części dnia uczyliśmy się o technogonach zamieszkujących pustynię.
Te mechaniczne, drążące tunele maszyny, przejmowały władzę nad każdą elektroniką i jako takie były szalenie niebezpieczne dla naszej armii.
Dlatego nigdy nie należało zapuszczać się na otwartą przestrzeń samemu.

Ponownie pominąłem lekcje zurdowskiego, aby przejrzeć notatki.
Nasz dziwak obliczył pozycję fortecy w abstrakcyjnej siatce otaczającej planetę, korzystając z pozycji słońca i gwiazd na niebie.
Wyliczył kąt padania światła słonecznego w czasie.
Nanosząc te dane na mapy, wychodziły mu sprzeczności.
Chodziło o to, że łuna Torten była oddalona od samego miasta o dobre kilanaście... jakichś jednostek.
Przydałoby się to zweryfikować.

Znalazłem także kilka zwyczajnych szkiców, pokazywały opuszczone miasto z czarnymi ludzikami.
O co mogło chodzić? Przecież nie o Torten, bo było ono zamieszkane przez Mroźnych imigrantów.
To może miała być wizja przyszłości, jak już wybijemy miasto?

Przedostatniego dnia poprzenosili wiele osób do innych grup.
Połowę najlepszych zostawili.
Odbył się także konkus strzelecki, chociaż bardziej polegał na mordowaniu niewinnych.
Mieliśmy powyżywać się na kukłach Mroźników i Czarnych.
Przy czym były to całkiem dobrej jakości elektroniczne kopie prawdziwych istot.
Nawet krzyczały i fajtysiły, gdy zadawało im się ciosy.
Można było pomylić je z rzeczywistością, dopiero prawdziwie zmasakrowane ukazywały ukrytą elektronikę.

Ja rozumiem odbijać miasto z rąk oprawców, ale atakować niewinnych, tylko dlatego, że byli naszymi wrogami?
Nie sprawiało mi to przyjemności.
Właściwie tylko ja i Rohost mieliśmy moralne opory przed masakrowaniem robotów.
Wszyscy inni wyżywali się nieprawdopodobnie.

Co gorsza, zobaczyli naszą słabość.
W jednym pukju staliśmy się obiektem drwin i wyzwisk.
Jeszcze przed chwilą to my wywoływaliśmy poklask umiejętnościami, musiało to zagotować w pozostałych niezykłe pokłady złości, które mogli w końcu wypuścić.
Na szczęście był czas obiadu, więc ostateczna konfrontacja odsunęła się w czasie.

Wieczorem postanowiłem nadgonić zurdowskiego.
Rohost wrócił z patio późno, gdy pogasły wszystkie światła.
Był mocno poobijany, widać, że stoczył bójkę. 
Poszedł od razu do jaborysa i długo nie wychodził.

To myła moja szansa, nikt nie patrzył, wszyscy już spali.
Prześlizgnąłem się po podłodze, omijając concatesy i zajrzałem od dołu pod futusę zakrywającą jaborys tajemniczej postaci.
Zobaczyłem kosmitę, całkowicie obcą osobę, jedynie kszałtem podobną do nas.
Nia miał żadnych otworów na ciele, nie miał lelonii, czy sutratów.
Gdzie susteria, gdzie joty, a hateria?
Kim on do diaska był?

Uciekłem czem prędzej na swoje miejsce.
Powinienem to zgłosić.
Ale wtedy będę sam przeciw wszystkim.
Jeśli jest szansa, że Rohost wie cokolwiek więcej, to powinienem z tego skorzystać.

\divider{}

WickedH zaczął chrapać.

\divider{}

Nadszedł ten dzień.
Ustawiliśmy się grzecznie przed wyjściem, oczekująć na otwarcie drzwi na dziedziniec.
Rohost stał przygnębiony, z tyłu.

Najpierw pokazała się wąska szczelina.
Każdy próbował złapać przez nią jakikikolwiek widok dziedzińca.
Wtem drzwi się otwarły i wybiegliśmy na zewnątrz, niemal się tratując.

Na środku placu stało dwadzieścia nowiusieńkich hiperów.
Każdy, naturalnie opisany swoim symbolem.
Hiper Hos stał z tyłu, zaraz obok hipera Rohost.

Maszyny miały delikatną faukturę drewna, podkreślając dostojność i elegancję.
Kanciaste kształty podowowały, że wyglądaliśmy w nich znacznie masywniej, niż w rzeczywistości.
Gdzieniegdzie wychodził ręcznie rzeźiony ornament.
Środek susterii zdobiły szklane, szlifowane wstawki. Mieniły się w słońcu wszystkimi kolorami spektrum.
Po bokach przytwierdzone były rzędy srebrnych gałek, aby dopełnić uroku.

To właśnie ta elegancja naszych egzoszkieletów nadawała im taki postrach u wrog...
Rohost tarzał się ze śmiechu. Co go tak rozbawiło?
Próbował się powstrzymywać, ale ilekroć spojrzał na stojące maszyny, ponownie brało go na hihranie się.

Pora na przymiarki.
Wspiąłem się na swojego hipera i wszedłem do środka.
Był ciasny i niewygodny.
Od razu miałem w głowie lekcje, na których straszyli nas, jak to bez hipera nie przeżyjemy nawet kukusty na pustyni.
Spróbowałem się ruszyć, ale nie udało się.
Chyba hiper był wyłączony.

Gdy wszyscy weszli do swoich muszli, pojawił się mi przed oczyma nizjer.
Dyrygent mówił z niego po zurdowsku, zrozumiałem jednynie ogólny sens wypowiedzi, trzeba było bardziej przykładać się do języka.

Chodziło o to, że teraz dopiero zaczyna się prawdziwe szkolenie.
Od tej chwili mamy zakaz opuszczania hiperów, pod groźbą porażenia prądem.
Na pokaz, wszystkim nam to zaprezentował, to były jak tortury, nigdy wcześniej nie byłem rażony prądem.

W hiperach będziemy jeść, spać i hutonować się, gdyż ma wbudowany jaborys.
Możemy zapomnieć o naszych wygodnych concatesach.
Na te słowa zamknął wrota do sypialni. Na zawsze.

A teraz była pora na ćwiczenia, wszystko jak dawniej, ale z hiperami.
Z piasku wysunęły się kolumny z poduszkami.
Hos pierwszy.

Chodzenie w hiperze było trudne, ale dałem radę wspiąć się na schodki.
Skoczyłem na pierwszą poduszkę i odbiłem się tak mocno, że przeskoczyłem dwie.
Oczywiście spadłem w piasek, przynajmniej nie bolało.
Dyrygent się zaśmiał i poraził mnie prądem. Więc to tak teraz będzie.
Wszystkie vinite świata nie były tego warte.

Drugi był Rohost, spadł.
Trzeci Wewwt, ten nowy.
Bezbłędnie przeskoczył tor przeszkód.
Pozstali także radzili sobie całkiem nieźle.
Od razu widać było, kto teraz będzie najsłabszy, jak to możliwe?

W nocy załączył się nam jaborys na najmocniejszym trybie, pozbawił mnie ubrania.
Co oni planują? Zamknąć nas w tych maszynach na zawsze? Przecież nie jesteśmy Czarną Armią.
Od tego czasu wypatrywałem, czy przypadkiem nie podpinają nas do węża ze smarem khaki.

Treningi zmieniły charakter, więcej było strzelań i mordowań, a mniej historii świata.
Mieliśmy być zawsze posłuszni dyrygentowi. Codziennie pokazywali nam zdjęcia łuny Torten na horyzoncie, jako naszego jedynego sensu życiowego.
Ale ja wiedziałem, to nie mogło być Torten.

\divider{}

\begin{dialogue}
\ds{} Wiesz co? Mam tego dosyć \dm{} WickedH powiedział. \dm{} Dawaj mi to. \dm{} Wyrwał Antyraxowi z ręki worek z alternatywnym wszechświatem i wskoczył do środka.
\ds{} Co ty robisz? Zginiesz tam. \dm{} Antyrax próbował złapać go za znikający w ciemności ogon. \dm{} Zginiesz tam \dm{} powtórzył i uśmiechnął się.
\end{dialogue}

\divider{}

Noc przerwało wielkie uderzenie i wycie syreny.
Światła na głównym murze zapaliły się pełną mocą i skierowały się w górę.
Pośrodku fortecy stała wielka, czarna postać.

\begin{dialogue}
\ds{} ,,A teraz zabawimy się.'' \dm{} powiedziała donośnym głosem. Nic nie zrozumiałem.
\end{dialogue}

Poczęliśmy strzelać z naszych wbudowanych w stroje mekerów.
Gorące kukety wybuchały na ciele atakującego, jeden po drugim. Nic to jednak nie dawało.
Zacząłem się bać. Pierwszy raz w życiu nie byłem pewien mojej wyższości nad innymi.

\begin{dialogue}
\ds{} ,,Co powiecie na to?'' \dm{} Uderzył pięścią z całej siły w środek fortecy. Wewnętrzne ściany rozsypały się w proch. Zobaczyłem, że żadna inna grupa nie posiada jeszcze hiperów.
\ds{} ,,Wypierdalaj z mojego opowiadania!'' \dm{} Rohost odrzyknął. Nie wiedziałem, że zna jego język. \dm{} ,,Zabijasz niewinnych, mają wyprane mózgi, dlatego cię atakują!''
\ds{} ,,Jaka szkoda.'' \dm{} Przejechał wokół ręką po murze, zwalając wszystkie światła. 
\end{dialogue}
Nastała ciemność, rozświetlana jedynie xedowymi ekplozjami kuketów.
Po raz drugi zobaczyłem łunę domniemanego Torten.
Popatrzyłem w gwiazdy, przypomniałem sobie rysunki Rohosta.
To nie było Torten.

Zobaczyłem wielką, zbliżającą się nade mnie stopę.
Nie ucieknę.
Wtem czarny gigant został odrzucony z wielkim impetem w bok.

\divider{}

Antyrax z całej siły kopnął w worek.

\divider{}

Jednak nadal coś z całej siły uderzyło mnie. W okolice querta.
Odwróciłem się, to Rohost walił we mnie kawałkiem pukpula.
Ty zdrajco, pomyślałem.
On jednak uderzył mnie znowu tak, że aż zabolało.
Zamachnąłem się, aby mu oddać, ale on wywinął się i zdzielił mnie ponownie.
Prąd przeszył mi ciało, moduł do zadawania kar włączył sie niekontrolowanie. 
Co za geniusz, zabije mnie moim własnym hiperem.

On jednak nie przestawał bić, padłem na ziemię, a on walił dalej.
A niby tak bał się mordowania niewinnych, wiedziałem, że nie można mu było ufać.

Wtem prąd się wyłączył, a mój nizjer zasygnalizował uszkodzenie modułu hipera.
Popatrzyłem na Rohosta, stał, operacąc się o pukpul.
Wsadził mi go w rękę i pokazał miejsce na swojej zbroi, gdzie przed chwilą mnie bił.
Zrozumiałem.

Nie było mi łatwo i trwało to znacznie dłużej, niż w jego przypadku, lecz udało mi się ,,wyłączyć'' i jego taser.
Co teraz?
Teraz uruchomił tryb głośnomówiący.
\begin{dialogue}
\ds{} Nasza armia zdrada. \dm{} Próbował łamanego zurdowskiego. \dm{} Dyrgent Mroźnik.
\ds{} Atakować, nie rozmawiać! \dm{} Odpowiedział mu głos dyrygenta.
\ds{} Hirten pułapka... \dm{} kontynuował.
\ds{} Torten \dm{} poprawiłem go. \dm{} Hirten było przed najazdem Mroźników.
\ds{} Kaporten. \dm{} Usłyszałem nieznajomy, rykliwy głos. \dm{} Kaporten Czarna Armia trucizna imigranci północ wy pułapka.
\ds{} A światła? \dm{} Wysłałem zapytanie w eter nie mając pojęcia, kto słucha, a kto mówi. \dm{} Światła są obok Torten... Hirten... Kaport... tego miasta.
\ds{} Pułapka Mroźnicy my. \dm{} Rohost odpowiadał. \dm{} Dyrygent wie podstęp. Dyrygent nas śmierć.
\ds{} Hos, Rohost. To wasze ostatnie słowa. Do widzenia. \dm{} Nic się jednak nie stało. \dm{} Co zrobiliście z hiperami?
\ds{} Uwolniliśmy je. 
\ds{} To naruszenie zasad! Wszyscy, zabić Hosa i Rohosta. Rozkaz. \dm{} Wycelowała w nas osiemnastka mekerów.
\ds{} Pamiętajcie, kto jest prawdziwym wrogiem. To niekończąca się wojna. Wszyscy zostaliście osz... \dm{} Eksplozja ogłuszyła mnie. Zobaczyłem tylko pomarańczowe światło.
\end{dialogue}

Gdy się obudziłem, był już dzień. Wszędzie wokół tliły się szczątki fortecy i mojej byłej armii.
Musiał wybuchnąć pożar, jednak mnie jakimś cudem ominął.
Przeszukałem zgliszcza, lecz nic nie znalazłwszy, postanowiłem wrócić do cywilizacji wzdłuż duzji. 
Niektóre certy były poprzewracane, innym brakowało pierścieni, żaden alemon nigdy więcej tędy nie poleci.

Wtedy sobie przypomniałem, wychodziłem na pustynię. Technogony z pewnością zwęszą wkrótce mój hiper i wykopią się spod ziemi.
To koniec, chyba że zostawię hipera tutaj, ale wtedy pewnie i tak umrę z pragnienia.

\divider{}

Antyrax wyciągnął WickedH z worka.
\begin{dialogue}
\ds{} Żyjesz? \dm{} zapytał. 
\end{dialogue}
Nie żył.

Zostało jeszcze dwóch.
A jemu właśnie skończyły się pomysły.

Jak na zawołanie, stanęli przed nim Poriux i Mały Gołąb.
\begin{dialogue}
\ds{} Co? Właśnie skończyły ci się pomysły, nie? \dm{} Poriux zapytał.
\ds{} Tttak... \dm{} Antyrax się przyznał.
\ds{} Zapewne masz jeszcze wspaniałe zakończenie, ja z chęcią je przyjmę. Lubię zakończenia \dm{} Mały Gołąb wyszczerzył demoniczne zęby.
\ds{} Ty? Zakończenie? To mi się należy zakończenie! \dm{} Poriux się oburzył.
\ds{} To ja je dostanę, mam brązowe piórka u Neofantasora, a ty masz ledwo dwa opowiadania w bibliotece! \dm{} Zaczęli się szarpać.
\ds{} Tylko dlatego, że poprawnie piszesz, a nie dlatego, że ciekawie.
\ds{} Nie ma znaczenia myśl, jeśli nie możesz jej poprawnie przekazać!
\ds{} Nie ma znaczenia poprawne pismo, jeśli nie niesie w sobie żadnej myśli!
\end{dialogue}

Antyrax usiadł więc na ławeczce, pogrzebał zamaszyście w worku i wyciągnął tak oczekiwane przez wszystkich czytelników zakończenie.
Rzucił nim w sprzeczające się demony.
\divider{}

%TODO Mateusz wyrywa Plaźmie stery z pazurów i wraca ratować Hosa. Plazma za niesubordynację mówi, że wsadzi go do więzienia. Zostawia na Potworanie.




































