\chapter{Sam przeciw wszystkim}

\info{Wioska jest atakowana przez literackie demony. Tylko dzielny, acz niedoświadczony wojownik jest wstanie je pokonać, używając do tego
swojego miecza kreatywności i worka z alternatywnym wszechświatem.}

Gdy tylko czubek głowy Wielkiego Neofantasora wyłonił się zza gór, odpalono armaty załadowane gorącym atramentem.
Katapulty wystrzeliły litery, a papierowe samoloty wzniosły się w powietrze.
Wszystko na darmo, wielkolud ani drgnął.

Wojownicy z wioski, uzbrojeni w pióra i kałamarze, ruszyli do boju.
Atakowali wroga, używając całego swojego doświadczenia. Opowiadania cyberpunkowe, fantasy i fantastyki naukowej cięły powietrze.
Horrory, wiersze i dzieła detektywistyczne rozbijały się o jego ciało.
Jednak niewzruszony Neofantasor pozostawał niewzruszony.
Stanął przed wioską i wysypał z rękawa, niczym wytrzepując piasek z buta po całodniowym siedzeniu na plaży, dziesiątkę straszliwych demonów.
Demony szybko i sprawnie rozprawiły się z całą obroną, rzucając po jednej gwiazdce w każdego z wojaków.
Nikt z nich nie przeżył, wieś została bezbronna.

Ale jak to w baśniach zwykle bywa, znalazł się młody syn doktora --- Antyrax.
Trochę ułomny językowo i bez jakiegokolwiek doświadczenia, postanowił dzisiaj zginąć w walce.
Nie miał, jak wszyscy, zbroi, pióra, czy zapasu atramentu.
Posiadał jedynie szklany miecz, wypełniony płynną kreatywnością, dmuchawkę z serum śmiechu, lateksowe buty i worek z alternatywnym wszechświatem.

Najbliższa demonka, Obudzona, została kopnięta z lateksowego buta i nawet nie poczuła.
Dopiero serum śmiechu, wstrzyknięte w szyję, zwróciło jej uwagę.
Wtedy Antyrax sięgnął do swojego worka i wyciągnął pewną rzecz.
Obudzona zamrugała, nie rozpoznając zupełnie, co to jest i czy się tego bać.
Wojownik też oglądał rzecz ze wszystkich stron, ale i on nie rozpoznał, co właściwie właśnie wyciągnął.
Wyrzucił za siebie i spróbował znowu.

Po kilku minutach wyrzucania różnych, nieokreślonych obiektów z worka, spostrzegł że grupka demonów go otoczyła i z zaciekawieniem obserwuje jego zmagania z samym sobą.
Nie umiał używać własnego wszechświata. Co za matoł.
Postanowił więc wykonać swoją powinność w tradycyjny sposób i wbił swój miecz w najbliższego demona, który odskoczył z bólu. Przynajmniej to działało.
Nie długo cieszył się z sukcesu, zaraz wszystkie demony razem wzięły i dmuchnęły w niego strumieniem przecinków --- jego największą słabością.

Pierwszy lecący znak interpunkcyjny ominął, podskakując na lateksowych butach, drugi skontrował mieczem, lecz tysiąc kolejnych odrzuciło go na kilkaset metrów w bok.
Upadł obolały w błoto, po drugiej stronie wioski, ale na szczęście nic sobie nie połamał.
Na koniec, lecący błąd ortograficzny prawie rozciął go na pół, Antyrax zdążył odskoczyć, lecz puścił miecz, który pękł na drobne kawałeczki, zmiażdżony przez ,,Ó''.
Kreatywność wsiąkła w błoto, tworząc nowe królestwo błotnych mrówko-androidów zasilanych parą z wody basenowej.

Antyrax był bezbronny, wstał i zobaczył, jak jago wioska właśnie jest równana z ziemią.
Cień Wielkiego Neofantasora, wychylającego się zza góry, wyglądał, jak dłoń dziecka, które zagarnia klocki z ziemi, aby je zaraz obślinić i połknąć.
Czy to był już koniec dla niego i dla wszystkich?

\ds{} Nie, to nie koniec \dm{} pomyślał, sięgając po worek. \de{}

Jego lateksowym butom wyrosły śmigła wspomagane skryptem w Bashu i poniosły go prosto pod nos Obudzonej, zostawiając linuksowy ślad.
Zaciekawiona demonka rozpoznała niedojdę, którego wcześniej zmiażdżyła.
Antyrax nie miał już nic, tylko worek. Sięgnął do niego i długo nie wyciągał ręki.

\ds{} Mam coś specjalnie dla ciebie, Obudzona. \de{}

\divider{}

\ds{} Kolejny dzień w pracy, kolejny dzień robienia bezużytecznych czynności, dla bezużytecznej korporacji, w tym bezużytecznym świecie \dm{} pomyślałem. \de{}

Wziąłem komórkę i zacząłem przeglądać maile.
Spam z reklamą talerzy, zaproszenie do znajomych na Facebooku, oczywiście od obcej osoby. Powiadomienie o komentarzu na YouTube, 
powiadomienie o mailu na innej skrzynce pocztowej, przypomnienie o opłacie za subskrypcję tej bezużytecznej gry komputerowej i tysiąc innych bezużyteczności.
O, znowu rozpętałem gównoburzę na Mirko i zablokowali mi konto Twittera za niepoprawną politycznie myślozbrodnię.
Bezużyteczność do kwadratu.

Z nadmiaru bezużyteczności, zdrzemnąłem się w ubraniu na godzinę i obudziłem przez głośne gruchanie z okna.
Czyżby jakaś użyteczność w końcu mnie spotkała?
Na oknie siedział biały gołąbek, w dzióbku trzymał kopertę.
Nie będąc pewnym, czy nadal nie śnię, odebrałem przesyłkę i przyjrzałem się kopercie.
Gołąb zaraz zniknął, oddalając się bezszelestnie.

Koperta była z papieru czerpanego.
Adresowana była elegancką kursywą do mnie, do Mateusza Mechalycznego, zamieszkałego na ulicy Szerokiej w Gdańsku. 
Województwo Pomorskie, Polska, Ziemia, Układ Słoneczny, Galaktyka Droga Mleczna, Gromada Lokalna, Pierwsza Kwadra.
Z tyłu widniała pieczęć z wosku pszczelego.
Wypukły obraz kuli i jakiś napis z niezrozumiałych znaków.
Przecież był 2017 rok, kto jeszcze wysyła papierowe listy zamiast maili?
To był kawał. To musiał być kawał. Pytanie jeszcze, jak ktoś go mi wyciął?

Ostrożnie otworzyłem kopertę, trzymając ją obcęgami, spodziewając się że coś strasznego zaraz na mnie wyskoczy.
Nic takiego się jednak nie stało.
Pisany odręcznie atramentem list był tym, co zwykle znajduje się w kopertach zaklejonych pieczęcią.

\begin{em}
Szanowny Panie Mateuszu Mechalyczny.

Mam zaszczyt zaprosić Pana na uroczysty bankiet z okazji wyboru na jednego z przyszłych członków ALOPP.
To wielki zaszczyt, móc gościć nowych agentów tej organizacji na pokładzie mojego \emph{[słowo z dziwnych znaków]}.
Mam najskrytszą nadzieję, że zostanie Pan przyjęty i będzie mieć Pan we współpracy z \emph{[znak niczym wydrapany na ścianie]} udział w walce o wspólne dobro.

Zapraszam do siebie dnia 20 października, roku Pańskiego 2017.
Myślę, że marina w Głównym Mieście będzie doskonałym miejscem na lądowanie mojej kuli i tam się spotkajmy w lokalne południe.
Po obiedzie wybierzemy się w podróż na Felicję, gdzie pozna Pan swoich przyszłych, mam nadzieję, członków przybranej rodziny.

Przypominam, że w \emph{[to samo słowo z dziwnych znaków]} oprócz najwyższej kultury osobistej,
od zawsze obowiązywał elegancji ubiór francuski.
Wszyscy goście powinni przywiązać najwyższą dbałość o szczegóły swojego wyglądu.
Uprzejmie proszę także, aby nie posiadać na pokładzie żadnych urządzeń użytkujących elektryczność.

Z Bogiem.
--- Profesor \emph{[kolejne słowo z dziwnych znaków]}
\end{em}

Zaczynało się robić ciekawie. Autor tego dowcipu chciał, abym za trzy dni, w XVIII wiecznym stroju pałacowym,
udał się w środek miasta w dniach szczytu, nie zabierając ze sobą żadnej elektroniki.
Potem miałem przejść test na zostanie członkiem jakiejś organizacji.
Trzeba dokładniej przestudiować ten list.

Szybkie szukanie Felicji w internecie wskazało jedną stronę o teoriach spiskowych.
Felicja miała być planetką, stworzoną przez kosmitów, na której hodowano ludzi, aby przeprowadzać na nich straszliwe eksperymenty.
Jeśli to prawda, może zabrakło im tam królików doświadczalnych i porywają kolejnych? 
Ale wtedy przecież nie dawali by mi wolnej ręki.

Na tej samej stronie podano: ALOPP ma być organizacją terrorystyczną zrzeszającą ludzi w celu mordowania obywateli własnej planety.
Ale od czego był to skrót, to nikt nie wiedział.

Pierwsza Kwadra dawała za dużo losowych wyników, aby wywnioskować o co może chodzić.

Lokalne południe, czyli dwunasta godzina czasu słonecznego, uwzględniając jeszcze czas letni, to trochę przed trzynastą czasu strefowego.
Przyjdę o 12:00, najwyżej trochę poczekam.

Wikipedia natomiast wskazała, że gołębie pocztowe w żadnym wypadku nie mogłyby doręczyć listu bezpośrednio do odbiorcy.
Ich mechanika polega na wracaniu do macierzystego gołębnika z dowolnego miejsca na świecie, i tylko tyle.
Listów z pewnością nie wsadzano im do dzióbka.
Rozejrzałem się po pokoju, czy przypadkiem nie miałem w nim gołębnika, żeby hodować gołębie pocztowe, ale nie.
Moja teoria o dowcipie zaczęła się lekko sypać.

Kilka razy, w różnych częściach świata, widziano kuliste UFO i ludzi w strojach rodem z Wersalu jednocześnie.
Podobno zdjęcie kuli nigdy nie wychodziło poprawnie, a większość ludzi magicznie zapominała o zdarzeniu chwilę po odlocie tajemniczej struktury.
Nieliczni pamiętali, ale nikt im oczywiście później nie wierzył. Nikt nie znalazł ich wspólnych cech.
Kulę widywano w różnych miejscach, nie ograniczała się, jak na filmach, tylko do USA, przelatywała przez centra miast, pływała pod wodą
i toczyła bitwy z wojskami wszystkich krajów.

Następnego dnia zabrałem list do znajomego chemika. 
Potwierdził on moje obawy, list wykonany był oryginalną techniką sprzed kilkuset lat.
Skład chemiczny papieru i atramentu, odpowiadał tym używanym dawno temu.
W dodatku narzędzie pisania z pewnością było ptasim piórem.
Na myśl o bezużyteczności otaczającego mnie świata, postanowiłem pojutrze zrobić coś użytecznego.

\divider{}

Pod naporem nietypowości i kreatywności dzieła, Obudzona zaczęła krzyczeć, zwijać się w konwulsjach i palić czarnym ogniem.
Zmieniła się w mały księżyc i poleciała, jak frisbee, z powrotem w kierunku Wielkiego Neofantasora.

Triumf Antyraxa nie trwał długo, za chwilę od tyłu złapała go Dedirid.
Jej czarna ręka owinęła się wokół delikatnego światostwórcy, jak czarny worek na zwłoki.
Zaraz go wyciśnie, jak tubkę pasty do zębów.
Wywijając się rybio, Antyrax zanurkował do worka i długo nie wychodził.
Demonka przez godziny nachylała się nad otworem, aby capnąć go jak tylko wystawi głowę.
Gdy tylko coś się wysunęło, porwała to z ochotą.
Była to jednak kolejna opowieść, zabójczo eksperymentalna, niesamowicie abstrakcyjna.
Nietypowość poparzyła jej łapska.

\divider{}

Mateusz wypożyczył wymaganą górę z wypożyczalni kostiumów.
Zastanawiał się, czy nie podkraść jakiegoś szustokora z muzeum, ale to chyba nie było by zbyt poprawne.
Bał się, czy mierna jakość kaftana, spowodowana nieoryginalnością, nie będzie zwracać nadmiernej uwagi w świecie najprawdziwszych atłasowych pasów i perłowych guzików.
Postanowił kupić więc kilka ozdób ze sztucznej biżuterii, które wyglądały dość kosztownie, a stworzone były z 
byle-czego, i doszyć w losowych miejscach. Miał nadzieję, że Profesor i inni goście nie zauważą.

Z pończochami nie było żadnego problemu, znalazł je w damskim sklepie.
Tak samo coś, co można było podciągnąć pod dawną koszulę.
Musiała być flanelowa, z wystającymi rękawami.
Ogarnął także puder.

Peruka, cóż. Przynajmniej znalazł za szafą trójkątną czapkę piracką po poprzednich lokatorach.
Najgorzej, że zazwyczaj chodził na łyso, gdyż rodzice nie obdarzyli go mocnymi włosami.
Potrzebował więc na szybko przykleić coś sobie na łeb.
Liczył w głowie, ile lat może dostać za kradzież peruki sędziemu, gdy spostrzegł wyprzedaż starych futer.
Używając magii nożyczek, kleju i moli wytrzepanych z płaszcza, wygenerował coś, co po przykryciu trójkątną piracką czapką wyglądało dość znośnie.

U zegarmistrza kupił za grosze kopertę zegarka z brakującymi wskazówkami, całość zaczepioną na łańcuszku.
Zegarek nie musiał działać, ważne, aby był.

O dziwo, to buty przysporzyły mu najwięcej problemu.
Niby lakierki z klamrą nie są niczym bardzo skomplikowanym, a jednak nikt ich nie produkuje.
Może właśnie dlatego, że były modne trzysta lat temu.
Wpadł na pomysł, aby kupić coś podobnego i przerobić. 
Znalazł buty dla zakładów pogrzebowych, gdyż tylko te odpowiednio się błyszczały i przyszył im klamry od spodni.
Z daleka nie było widać różnicy.

W domu ubrał się i przejrzał w lustrze.
Połączenie Napoleona, Ludwika XIV i informatyka z Gdańska.
Muszą zrozumieć.

O jedenastej godzinie, owego wielkiego dnia, ubrał się w pełny strój.
Nie mógł się przecież tak pokazać w mieście.
Pończochy zatem przykrył spodniami dresowymi.
Na elegancki szustokor nałożył nieco za dużą bluzę z kapturem.
Lustrzane lakierki przykrył jakimiś szmatami, żeby nie zwracały zbytniej uwagi.
Tylko pseudo-perukę schował do plecaka.

To nie mogło pójść tak łatwo.
Z daleka zobaczył kordon policji i wojska, które blokowało wstęp każdemu wychodzącemu z Długiego Targu.
Ucieszył się i kamień spadł mu z serca. Oznaczało to, że jednak nie padł ofiarą żartu.
Znalazł w końcu promyk użyteczności w oceanie bezużyteczności.

Do mariny spróbował dostać się okrężną drogą, przebiegł przez Krowi Most na Wyspę Spichrzów.
Klucząc uliczkami, zbliżył się do portu, jednak tutaj też była blokada.
Widział w każdym bądź razie kawałek wody w basenie jachtowym, która była mocno niespokojna, więc coś nietypowego jednak tam stało..
Popatrzył w kanał i pomyślał, że zostanie mu wskoczyć do wody i popłynąć.
Ale przecież nie zostałby wpuszczony mokry do rakiety.
No i co z pudrem.

Szustokor był bardzo gruby, rozpiął więc bluzę żeby się nie ugotować, teraz wszystko było mu jedno, czy ktoś to zauważy.
Był tak blisko, a jednocześnie tak daleko.
Jak rybka w siatce wrzucona do oceanu.
Zaraz będzie widział swoją życiową porażkę jak na dłoni.
Co robić? Co robić?

Wybawienie przyszło nieoczekiwanie.
Oto bowiem mama z małą dziewczynką podpłynęły do niego skuterem wodnym, oferując szybką podwózkę.
Myśląc, że to pomyłka, zdjął bluzę, pokazując swój strój w połowie okazałości.
Kobieta jednak nie uciekła, nie przestraszyła się dziwaka, tylko się uśmiechnęła.

\ds{} Profesor Kula nie lubi spóźnialskich. Pospiesz się \dm{} powiedziała. \de{}

\ds{} Skąd... kim... \de{}

\ds{} Świat jest wielki, a zasięg potworów... znaczy tych właśnie... to jest taka jakby policja wszechświata... jest większy.
Jesteś nowy, wnioskuję, że zostałeś zaproszony na test. Pospiesz się.\de{}

\ds{} Test. Ale co mam robić, żeby go zdać? \de{}

\ds{} Być sobą. \de{}

\ds{} Jak mam ci się... \de{}

\ds{} Uratuj w przyszłości też komuś życie. \dm{} Pogłaskała swoją córeczkę po głowie. \de{}

Mateusz w XVIII wiecznym stroju wersalskim zasuwał kanałem Nowej Motławy na skuterze wodnym, ozdoby szustokora mieniły się tak samo, jak latające wokół 
krople wody i kule wystrzelone z mostu przez żołnierzy zabezpieczających
lądowanie wielkiej białej kuli w centrum miasta. 
Schował się na chwilę pod mostem, jak zagoniony przez wilki królik w norze, a gdy wypłynął z drugiej strony, wtedy ją zobaczył.

Była wielka, jak budynek, wypolerowana, biała i doskonale kulista.
Dołem dotykała lekko powierzchni wody, tworząc promieniste fale.
Odbijała w sobie cały Gdańsk.
Mateusz zobaczył w niej malutkiego siebie na łódeczce-zabawce, malutkie budynki, żołnierzyków, spichrze, basenik, niebo, helikopter jak muchę i blask pełnego słońca.
Już nikt nie strzelał, już tylko wszyscy patrzyli. I bali się.
On się nie bał. Przyszedł tu na bankiet.
Przyszedł we francuskim stroju.
Przyszedł tu, bo został zaproszony.

Zszedł ze skutera na pomost i poprawił perukę, wtedy też właz w dolnej części zaczął się otwierać.
Tak, jak się spodziewał, był to dźwięk ulatującej pary wodnej i szczęku łańcuchów, a nie elektrycznego silnika.
Ze środka powiał zapach kurzu, wosku i lekkiej stęchlizny.
W przejściu stanął On.
Nosił najwspanialszy strój, jaki Mateusz kiedykolwiek widział, tak inny od jego własnego, a przecież z tego samego okresu.
Przy jego ozdobach sztuczna biżuteria Mateusza rzeczywiście wyglądała na sztuczną.
Jego najprawdziwsza peruka przyćmiła wielkością cały futrzany twór.
Lakierki błyszczały się tak samo, jak jego statek kosmiczny.
W ręku trzymał laskę z białą kulką, pomniejszoną wersję tego, co znajdowało się tuż za nim.

\ds{} Jestem Profesor Kula. Miło mi pana gościć na moim statku. \de{}

\divider{}

Antyrax wyszedł z worka, gdy z Dedirid została już tylko kupka popiołu.
Jeszcze ośmiu.
Tym razem demony nie bardzo chciały go atakować.
Antyrax więc wskazał jednego z nich palcem, niczym sędzia młotkiem, właśnie skazanego na śmierć.
Lenna. Pora na mikropomysły.
Sięgnął do worka i od razu złapał to, czego szukał.

\divider{}

\ds{} Waćpan Mateusz Mechalyczny, jak mniemam \dm{} powiedziałem. \dm{} Waćpan Mateusz Mechalyczny niepewnie, acz szybko schował się za framugą włazu. \de{}

\ds{} Proszę wybaczyć mi mój ubiór i maniery, Panie Profesorze Kula. \dm{} Ukłonił się prawie do ziemi. \dm{} 
Musiałem przedrzeć się przez kordon wojska, aby przyjść do pańskiej Kuli. \de{}

\ds{} Nazywam się Kula, mości Mateuszu Mechalyczny, nie Kula \dm{} poprawiłem, kręcąc korbą włazu. \dm{} A ta kula nazywa się Kula. \de{}

\ds{} Kula... \dm{} niepewnie odpowiedział. \de{}

\ds{} Nie Kula, Kula. Moje nazwisko, nazwa statku, i bryła geometryczna. Kula, Kula i kula. To trzy różne słowa, zupełnie inaczej wymawiane. \de{}

Mateusz Mechalyczny podrapał się po głowie, ścierając puder.

\ds{} Ignoruj go, on mówi i słyszy na częstotliwościach poza naszym zakresem. \dm{} Katarzyna Kosmata przysunęła się do nas, nawet się nie witając.
\dm{} Jestem Kasia, cześć. \de{}

\ds{} Katarzyno Kosmata! \dm{} skarciłem ją. \dm{} Maniery! Nie prezentuj złego przykładu naszemu gościowi.\de{}

Gość jednak utopił wzrok w jej sukni, nachalnie gapiąc się na każdą część jej ubioru. 
Już się cieszyłem, że chociaż on będzie tu potrafił zachować maniery.
Najgorsze, że ona sama wręcz zaczęła tłumaczyć mu co i skąd na niej pochodzi.

Najpierw zawiesił oczy na jej biuście, wodząc źrenicami to w lewo to w prawo. 
Najprawdopodobniej podziwiał plecionkę z anielskich włosów, którą obszyta była suknia.
Anielskie włosy są całkowicie przezroczyste, gdy odpadną od właściciela.
Aby stworzyć ten element ubioru, trzeba było najprawdopodobniej obmacywać kąty w każdym z niebiańskich biur w czwartej kwadrze.

\ds{} Trafiłam przez to na dywanik ministra poprawnego zachowania, chciał to podciągnąć pod brak szacunku dla aniołów, ale wybroniłam się, że to przecież dla wspaniałej sukni, którą będę 
pokazywać wszystkim wokół i opowiadać o Królestwie Bożym. Nie mogli przecież zabronić mi nawracać. \de{}

Następnie zszedł niżej, aby przyjrzeć się lepiej pasowi.
Katarzyna lubiła się ubierać w nieprawdopodobnie kosztowny ubiór, lecz jej pas był wykonany ze zwyczajnego, ziemskiego jedwabiu.
Może chciała tym pokazać, że jej suknia jest w równym stopniu stworzona ze składników z całego wszechświata?

Po pasie, przyszedł czas na szyję. Katarzyna założyła tym razem kolię ze zmutowanych pereł Khaliniskali...
czy to była Rezurma... nie pamiętam, kto ostatnio przejmował stolicę tej... Planety Wojny, jak ją wszyscy nazywają.
Mieniły się i pulsowały wszystkimi kolorami tęczy. Od podczerwieni, po nadfiolet.
Te perły można znaleźć tylko w małżach, żyjących w skażonym jeziorze, na północy kontynentu Terb.
Zdaje się że to albo Czarna Armia, albo Komodowa utopiła tam zbiorniki z kancerogennym żelem, w celu wewnętrznego wyniszczenia przybrzeżnego miasta Hirten.
Nie udało się, mieszkańcy Hirten wyczuli podstęp i zamiast umrzeć na nowotwory od picia skażonej wody, poumierali z pragnienia.
W każdym razie życie w jeziorze przeszło nieprzyjemne zmiany fizyczne.

Buty. Był to wspólny wytwór czterech Khrnzrhkh.
Najpierw poprosiła Chrrkrhkrrkk o stworzenie lodowej podstawy.
Potem pewnie Iłiścirr obudował to swoją czarną rkkizniisi, Buffsirr dodał czerwone wkładki ze swoich buffzerda, a Fluszszrisss utwardził swoim ogniem.

\ds{} Te buty zostały zrobione przez potworów, najpierw Mikołaj stworzył lodową podstawę, Psychit zalał ektoplazmą, Pyrroq dodał bomby, a Plazma utwardził ogniem. \dm{}
Katarzyna Kosmata właśnie spowodowała, że kolejna osoba będzie nazywać Khrnzaalk potworami, zamiast normalnie w ich własnym języku. \de{}

Zahaczył o wachlarz.
Ten był stworzony z półprzezroczystych łusek białego cyrkowca.
Te smoki wyginęły doszczętnie po ataku czerwonych kartaczy na ich planetę.
Właściwie jedyne cyrkowce można teraz znaleźć w zoo w Capitalu.
Szkoda ich, miały wspaniałą kulturę. 
Cyrkowe baśnie do dziś opowiada się małym smoczkom do legowiska, a cyrkowi malarze są niedoścignionym przykładem talentu we właściwie całym wszechświecie.
Pokonani przez bandę zwierząt. Żeby chociaż bordowe pasowce, ale nie. Najgorsi z najgorszych zarżnęli i pożarli najlepszych.

Ostatecznie popatrzył się na jej twarz.
Nie, nie na twarz, a na makijaż.
Oczywiście, Katarzyna Kosmata nie mogła spocząć na prawie wymarłych smokach.
Puder był stworzony ze zmielonych ciosów mamuta.
Jak ona odkopała je z syberyjskiego błota i wybieliła, tego nie wiem.

\ds{} Przekonałam Chronosa, żeby odwrócił trochę czas i przywrócił im świeżość. \de{}

Nikomu się nie udało przekonać kiedykolwiek Chronosa do dobrowolnego używania mocy! Jak ona to zrobiła?

Poszedł wyżej.
Jej fryzura była przeogromna. A wszystko to naturalne włosy, ich wzrost był stymulowany tą ich obrzydliwą światłową technologią, laserem bodajże.
Można na pewno by osiągnąć ten sam efekt, używając normalnej pary i mechanizmów, a nie diabelskiego prądu.

Na włosach siedziały jej skrzące się motyle. Co jakiś czas, któryś wzbijał się w powietrze, robił pętlę wokół jej głowy i lądował z powrotem.
Były to najprawdziwsze motyle, hodowane i tresowane w tajnej placówce pod motylarną w Burggarten.
Ciekawe, jak je zdobyła. Pewnie jak gdyby nigdy nic przyszła tam do nich w tej sukni, z szyfratorem w ręce, i powiedziała:
,,Dajcie mnie tresowanych motyli na głowę, bo zaraz mam bankiet we wielkiej, latającej kuli.''
Być może tylko po to w ogóle przyjechała do Wiednia.
Po motyle i po to, żeby wsiąść do Riesenrad i pojechać wagonikiem na sam szczyt, gdzie specjalnie umówiła się ze mną, abym podleciał Kulą.
Następnym razem pewnie stanie na szczycie Empire State Building, a Kula będzie robiła za King Konga.
Czy można się uzależnić od amnezji, którą pokryty jest statek? 
Uzależnić od siania paniki w ludziach, którzy i tak za chwilę o wszystkim zapomną?

Przeczyściłem gardło.

\ds{} Znaczy... witam... bardzo mi miło, dzień dobry... eee... Kasiu-ażyno. \dm{} Stał bez ruchu kilka pulsów, aż zdecydował się delikatnie ująć jej dłoń i pocałować. 
Ważne, że się stara. \dm{} Ja Mateusz... jestem. \de{}

Katarzyna Kosmata odprowadziła nas do głównej sali, gdzie czekał już zastawiony stół dla czterech osób.

\ds{} Na naszym bankiecie spodziewamy się trzech gości \dm{} oznajmiłem zgromadzonym. \dm{} Zatem dołączy do nas jeszcze jedna osoba.
Będzie to Nadar Nocny, którego odbierzemy z Atlantyckiego Laboratorium Dna Morskiego.\de{}








%TODO strój francuski, nie wersalski
%TODO Opis tego, że test składa się z segmentów i że każdy z nich ma więcej, jak jedną odpowiedź









