\chapter{Na pewno nie jest pan wilkołakiem} 
\info{Tematem jest "`Mój pierwszy raz"'. Opowieść pewnego człowieka, dla którego właśnie mija
pierwszy miesiąc, od ugryzienia przez wilkołaka.
A może nie przez wilkołaka?}

Podobno nie każde ugryzienie wilkołaka naznacza człowieka.
Ja jednak wolę dzisiejszej nocy nie ryzykować.
Właśnie teraz mija miesiąc od tamtego pamiętnego dnia, rana się już prawie zagoiła, ale blizna zawsze będzie mi przypominać o mojej skazie.
Pierwsza pełnia księżyca, tej nocy wszystko się wyjaśni.

Z samego rana spakowałem namiot, trochę jedzenia i wyruszyłem w las, aby uciec jak najdalej od wioski.
Nie mogę narażać jej mieszkańców i rodziny, gdyby okazało się, że jednak jestem... tym czymś.

Rozbiłem namiot pośrodku jeziora, na małej wysepce, do której przypłynąłem kłodą drewna. 
Mogę mieć tylko nadzieję, że wilkołaka nie utrzyma.
Dla bezpieczeństwa przywiązałem ją łańcuchem, a klucz od kłódki opaliłem w ogniu, aby pozbawić go ludzkiego zapachu i zakopałem pod drzewem.

Zasnąłem, gdy był jeszcze dzień, miałem cichą nadzieję, że rano obudzę się w tej samej pozycji.

\divider{}
I tak się stało, nadal byłem w namiocie. Jednak już po kilku sekundach wiedziałem, że coś było nie tak.
Kłoda drewna nadal była przywiązana do brzegu, ale z drugiej strony wyspy.
Pod drzewem widniała wykopana pazurami dziura, ale klucz leżał w środku.
I moje paznokcie, były jakby dłuższe?

Czy możliwe było, żebym w nocy odwiązał kłodę, popłynął nią na brzeg, a potem wrócił i przywiązał z powrotem?
Znacznie ciekawsze było jednak, co robiłem w lesie, polowałem, wyłem? 
W końcu dzień drogi to nie tak daleko od wioski... przeszły mnie dreszcze ze strachu.
Trzeba wracać. Szybko.

\divider{}
Już z daleka wiedziałem, że jednak tu byłem.
Echo gwarnej wioski niesie się przez las, ale tym razem było zupełnie cicho.
Nawet ptaki zamilkły, a wiatr ustał.

Tlące się pozostałości słomianych dachów dopełniały plamy krwi na drodze.
Odwracałem wzrok od zmasakrowanych ciał nie dlatego, że mnie obrzydzały, ale dlatego, że bałem się zobaczyć w nich członków mojej rodziny.

Każda chałupa miała wyważone drzwi, z każdej futryny wiała śmierć.
Tak oto kolejna wioska zniknęła z map.
Czy za miesiąc jeszcze jedna podzieli jej los? 

Ale była jedna osoba, która mogła znać prawdę.
Przybysz. Nigdy nie opuszczał domu i z pewnością teraz też tego nie zrobił.
Pewnego razu przyszedł do naszej wioski, za pomocą dziwnych sztuczek leczył rany, odpędzał wrogów i budował mechanizmy.
Zdobył nasze zaufanie i pozwoliliśmy mu osiąść nieopodal.

Wie, że zostałem ugryziony i z pewnością, jak tylko mnie zobaczy, strzeli do mnie jednym ze swoich dziwactw i zada bolesną śmierć.
Może to i lepiej, tacy, jak ja powinni być usuwani ze społeczeństwa. Moja wina, trzeba było oddalić się od wioski na większą odległość.
Ale jeśli istnieje cień szansy, że dowiem się prawdy, to trzeba iść.

\divider{}
Kwadratowy domek zwyczajowo świecił niebieskawą łuną.
Na dachu były takie specjalne deski, które kradną światło słońca, a potem część oddają w nurcie elektroko-jakimśtam.
Nowa, kanciasta gałąź dziwacznego drzewa antony wbita była w rogu, już czwarta z kolei.
Płot iskrzył się jak pocierany koc, ale wiedziałem, że przy dotyku może zabić silnym nurtem.
Magiczne ogrodowe krasnale zaczną do mnie strzelać promieniami słońca, gdy wejdę nieproszony zamiast użyć przycisku przy furtce.

Niepewnie dotknąłem kółka na małym pudełku. 
W domu rozległo się bicie dzwonków, chociaż byłem w domu Przybysza już wcześniej i wiem, że na pewno nie miał żadnych dzwonków przy wejściu.
Zamknąłem oczy i czekałem, aż gorący promień dokona mojego żywota.

\ds{} Witam naszego wspaniałego bohatera, ale im pan pokazałeś, no, no. 
\dm{} Przybysz chyba oczekiwał mojej wizyty. \dm{} Mam to wszystko nagrane, chodź, pokażę, jak ratuje pan wioskę.
\dm{} Odczytał zakłopotanie z mojej twarzy.
\dm{} Nic pan nie pamięta? Ciekawe. Akurat wczoraj, tej samej nocy Imperium zaatakowało.
Przyjechali z wozami, żeby porwać wszystkich na niewolników kopalni.
\dm{} Odsunął się zapraszając do środka, chyba jednak nie chciał mnie zabić.
\dm{} Wtedy ty się zjawiłeś, w samą porę. Wszystkie te trupy na podwórzu należą do imperialistów, naszych nie tknąłeś. \de{}

\ds{} Ale truchło naszej sąsiadki, tam leży! \dm{} odpowiedziałem z niedowierzaniem. \de{}

\ds{} To dlatego, że kilku mieszkańców okazało się agentami, to oni nas zdradzili dla Imperium. 
Nie martw się, każdego z nich także zmieniłeś w krwistą papkę, a domy spaliłeś biegając z pochodnią w zębach.\de{}

Czyste, perfekcyjnie wykonane wnętrze domu przypominało gładkością mleczny lód.
Magiczne przedmioty, których znaczenia mogłem się jedynie domyślać walały się wszędzie.
Na stoliku siedziała wielka, tresowana mucha z gatunku drą. Pokazywał mi kiedyś, jak głośno lata i jak potrafi przekazać potem, co widziała.

Przybysz wskazał na niezwykłe okno w ścianie, nazywał je kranem E.
Pokazywało nieistniejące widoki, ale nie można było go otworzyć.
Tym razem zobaczyłem na nim nocną scenę, z góry oświetloną blaskiem księżyca. 
Ludzie z pochodniami biegali i wchodzili do domów, wyciągając siłą mieszkańców.

Wtedy pojawił się on. Znaczy ja.
Nienaturalna pozycja, lśniące białe futro, wielkie pazury i dzikie spojrzenie.
Zamknąłem oczy, nie chciałem tego oglądać.

Gdy krzyki ucichły, wilkołak sprawdzał dokładnie każdy z domów i czasami podpalał.
Mój dom także musiał zostać zniszczony. Moja własna rodzina trzymała z Imperium? 
Znaczy, że... musiałem ją zabić. Rozszarpałem moje własne dzieci!
Łzy napływały mi do oczu i nie wiem czy dlatego, że ich straciłem, czy dlatego, że prawdopodobnie oszukiwali mnie tak długi czas, a całą wioskę sprzedali Imperium.
Podejrzewałem to już od dawna, ale sercem uwierzyć nie mogłem, jak widać niesłusznie.

\ds{} Z tego, co wiem wilkołaki nie zachowują się w tak logiczny sposób.
Jeśli pan pozwoli, zamierzam przeprowadzić kilka badań i pomóc Panu nad tym zapanować.
\dm{} Było mi już wszystko jedno.
\dm{} Możemy to wykorzystać do naszych celów, proszę sobie wyobrazić, cała armia wspaniałych bestii, może nawet obalimy Imperium!
Nie będzie więcej wojny, palenia domów i najazdów, nie tylko dla naszej wioski, ale każdej! \dm{} Przybysz zachwycił się, patrząc w sufit, gdybym do wszystkiego miał taki optymizm, jak on.
\dm{} Tu jest stół z klamrami na ręce i nogi.
Przywiążę pana i zobaczymy, zapewniam, nic się panu nie stanie.
Pełnia trwa dwie noce, do zachodu słońca zostało mało czasu. \de{}

Obojętnie.
Nie stanie się, to dobrze, stanie się, to i nawet lepiej.
A jak będę przywiązany, to nikogo więcej nie zamorduję.

Położyłem się na stole, Przybysz zapiął klamry i przyczepił mi jakieś sznurki do ciała.
Na wszelki wypadek, zostawił ognistą kuszę na widoku.

Byłem za bardzo pobudzony, aby zasnąć. Wkrótce jednak zacząłem tracić przytomność. Zaczyna się.

\divider{}
Obudziłem się także na stole, ubrania porwane na strzępy, sznurki pozrywane, klamry przegryzione.
Przybysz siedział w rogu i trzymał się za czerwony bandaż.

\ds{} Mamy mały problem.
\dm{} Sięgnął do deski guzików.
\dm{} Gdybyś po prostu się zerwał i mnie ugryzł, to by było jeszcze w porządku.
Ale po transformacji zacząłeś ze mną rozmawiać i przekonywać do swoich racji. \de{}

Ze sztucznych ust przy kranie E dało się słyszeć głos Przybysza, oraz drugi, zupełnie nieludzki.
Niski i ryczący, słychać było w nim echo kosmosu, dźwięk wielu zamieszkanych planet i ich słońc.
Czerwony karzeł, żółty karzeł i wielka czarna dziura. 
Dwa księżyce, pierścienie i sztuczny twór.
Miasta bogate i zdobione, pokój i szczęście mieszkańców, śmiech i radość.
Ale ziemski język, którym rozmawiał sztuczny, ograniczony, na pewno nie polski.
Skąd ja to wszystko w ogóle wiem?

\ds{} Od razu rozpocząłeś rozmowę ze mną po wspólnym, zwanym również nowomową. 
\dm{} Przybysz nieufnie spoglądał to na mnie, to na kran E. 
\dm{} Ten język powstanie dopiero za trzy tysiące lat w Ameryce, jako uniwersalny sposób porozumiewania się ludzi z różnych krajów. Nie ma jeszcze nikogo w tej chwili na świecie, kto potrafiłby się nim posługiwać.
Z wyjątkiem mnie. I chyba też ciebie? \de{}

Pokręciłem głową, ja także go nie rozumiem.

\ds{} Interesujące.
\dm{} Zamyślony wstał i podał mi nowe ubrania.
\dm{} Proszę wyobrazić sobie bardzo inteligentne zwierzęta. 
Przychodzą z bardzo daleka i chcą się od nas nauczyć części naszej kultury.
Po ugryzieniu, w ciągu kilku lat stajesz się jednym z nich. 
\dm{} Wskazał palcem na moje włoski na rękach, były teraz dłuższe i jaśniejsze.
\dm{} Tak podbijają świat za światem, usuwają to, co złe, zostawiają tylko to, co w ich przekonaniu dobre i przyjmują u siebie.
\dm{} Wyciągnął książkę z szuflady.
\dm{} To nie zadziała. Sztuczne mieszanie kultur nie zadziała. Nie można mieszać.
\dm{} Uniósł się, jakby chciał mnie przekonać, a ja nie rozumiałem.
\dm{} Uciekłem przed wymuszonym mieszaniem, uciekłem przed Nowomową, przed sztucznie poprawnym zachowaniem.
To wymaga czasu, kulturę trzeba wytworzyć, a nie zdobyć.
Oni upadną, tak, jak my upadniemy. Na własne żądanie. \de{}

W książce były ryciny spalonych domów, brudnych ulic i wojujących ludzi, ale nadal nic z tego nie rozumiałem.
Przybysz patrzył się przez okno. Pierwsi ludzie wracali do wioski.

\ds{} Ale może gdyby wpierw oczyścić tę kulturę z brudu, to by zadziałało? \dm{} Patrzył na mnie, szukając potwierdzenia racji.
\dm{} Jedno jest pewne. \dm{} Zgasił kran E. \dm{} Na pewno nie jest pan wilkołakiem. \de{}

\ds{} Nie jesteśmy my \dm{} poprawiłem. \de{}













