\chapter{Na pewno nie jest Pan wilkołakiem} 

Podobno nie każde ugryzienie wilkołaka naznacza człowieka.
Ja jednak wolę dzisiejszej nocy nie ryzykować.
Dzisiaj mija miesiąc, od tamtego pamiętnego dnia, rana się już prawie zagoiła, ale blizna zawsze będzie mi przypominać o mojej skazie.
Pierwsza pełnia księżyca, tej nocy wszystko się wyjaśni.

Z samego rana spakowałem namiot, trochę jedzenia i wyruszyłem w las, aby uciec jak najdalej od wioski.
Nie mogę narażać jej mieszkańców i rodziny, gdyby okazało się, że jednak jestem... tym czymś.

Rozbiłem namiot pośrodku jeziora, na małej wysepce, do której przypłynąłem kłodą drewna. 
Mogę mieć tylko nadzieję, że wilkołaka nie utrzyma.
Dla bezpieczeństwa przywiązałem ją łańcuchem, a klucz od kłódki opaliłem w ogniu, aby pozbyć się ludzkiego zapachu i zakopałem pod drzewem.

Zasnąłem, gdy był jeszcze dzień, aby nic w nocy nie czuć i mieć cichą nadzieję, że rano obudzę się w tej samej pozycji.

\divider{}
I tak się stało, nadal byłem w namiocie. Jednak już po kilku sekundach wiedziałem, że coś było nie tak.
Obudziłem się nagi, lecz zamiast spodziewanych strzępów ubrań, zobaczyłem je nietknięte, złożone w kostkę zamiast poduszki.
I moje paznokcie, były jakby dłuższe?

Kłoda drewna nadal była przywiązana do brzegu, ale z drugiej strony wyspy.
Tam, gdzie zakopałem klucz, widniała wykopana pazurami dziura.
Klucz wisiał na wejściu, żebym z pewnością go znalazł.

Czy możliwe było, żebym w nocy odwiązał kłodę, popłynął nią na brzeg, a potem wrócił i przywiązał z powrotem?
Znacznie ciekawsze było jednak, co robiłem w lesie, polowałem, wyłem? 
W końcu dzień drogi to nie tak daleko od wioski, zdesperowany wilkołak dobiegłby w kilka pacierzy.
Muszę jak najszybciej wrócić i sprawdzić, czy wszyscy żyją.

\divider{}
Już z daleka wiedziałem, że jednak tu byłem.
Echo gwarnej wioski niesie się przez las, ale tym razem było zupełnie cicho.
Nawet ptaki kompletnie zamilkły, a wiatr ustał.

Tlące się pozostałości słomianych dachów współgrały z plamami krwi na drodze.
Odwracałem wzrok od zmasakrowanych ciał nie dlatego, że mnie obrzydzały, ale dlatego, że bałem się zobaczyć w nich członków mojej rodziny.

Każda chałupa miała wyważone drzwi, z każdej futryny wiała śmierć.
Tak oto kolejna wioska zniknęła z map.
Czy za miesiąc kolejna podzieli jej los? 

Jednak wiedziałem, że jest jedna osoba, która może znać prawdę.
Przybysz. Nigdy nie opuszczał swojego domu i z pewnością teraz też tego nie zrobił.
Wie, że zostałem ugryziony i z pewnością, jak tylko mnie zobaczy, strzeli do mnie swoją magią i zada bolesną śmierć.
Może to i lepiej, tacy, jak ja powinni być usuwani ze społeczeństwa. 
Ale jeśli istnieje cień szansy, że dowiem się prawdy, to trzeba iść.

\divider{}
Kwadratowy domek zwyczajowo świecił niebieskawym światłem.
Kratkowany, ciemny szklany dach odbijał popołudniowe słońce, nowa kanciasta gałąź drzewa antony wbita była w rogu.
Płot iskrzył się, jak pocierany koc, ale wiedziałem, że może zabić.
Magiczne ogrodowe krasnale zaczną do mnie strzelać, gdy wejdę, a nie użyję przycisku przy furtce.

Niepewnie dotknąłem kółka na małym pudełku. 
W domu rozległo się bicie dzwonków, chociaż byłem u Przybysza już wcześniej i wiem, że na pewno nie miał żadnych dzwonków przy wejściu.
Zamknąłem oczy i czekałem, aż gorący promień dokona mojego żywota.

\ds{} Witam naszego wspaniałego bohatera, ale im Pan pokazałeś, no, no. 
\dm{} Przybysz rozsunął drzwi na boki bez dotykania. \dm{} Mam to wszystko nagrane, choć, pokażę, jak ratuje Pan wioskę.
\dm{} Odczytał zakłopotanie z mojej twarzy.
\dm{} Nic Pan nie pamięta? Ciekawe. Akurat wczoraj, tej samej nocy Imperium zaatakowało.
Przyjechali z wozami, żeby porwać wszystkich i zmusić do niewolnictwa w kopalniach.
Wtedy ty się zjawiłeś, w samą porę. Wszystkie te trupy na podwórzu należą do imperialistów, naszych nie tknąłeś.
Ale kilku mieszkańców okazało się agentami, to oni nas zdradzili dla Imperium. 
Nie martw się, każdego z nich także zmieniłeś w krwistą papkę, a domy spaliłeś. Nieczęsto widzi się wilkołaka biegającego z pochodnią w zębach.\de{}

Czyste, perfekcyjnie wykonane wnętrze domu przypominało gładkością lód.
Na surowych z wyglądu kwadratowych półkach stały magiczne przedmioty, których znaczenia mogłem się jedynie domyślać.
Na stoliku siedziała wielka, tresowana mucha z gatunku drą. Pokazywał mi kiedyś, jak lata i jak potrafi przekazać potem, co widziała.

Przybysz wskazał na niezwykłe okno w ścianie, nazywał je kranem E.
Zobaczyłem na nim nocną scenę, z góry, oświetloną blaskiem księżyca. 
Ludzie z pochodniami biegali i wchodzili do domów, wyciągając siłą mieszkańców.

Wtedy pojawił się on. Znaczy ja.
Nienaturalna pozycja, lśniące białe futro, wielkie pazury i dzikie spojrzenie.
Zamknąłem oczy, nie chciałem tego oglądać.

Gdy krzyki ucichły, wilkołak sprawdzał dokładnie każdy z domów i czasami podpalał.
Mój dom także musiał zostać zniszczony. Moja własna rodzina trzymała z Imperium? 
Chociaż... to by wiele wyjaśniało, czasami tak dziwnie się zachowywali.

\ds{} Z tego, co wiem wilkołaki nie zachowują się w logiczny sposób.
Jeśli Pan pozwoli, zamierzam przeprowadzić kilka badań i pomóc Panu zapanować nad tym.
Możemy to wykorzystać do naszych celów, proszę sobie wyobrazić, cała armia wspaniałych bestii, może nawet obalimy Imperium!
\dm{} Przybysz zachwycił się, patrząc w sufit.
\dm{} Tu jest stół z klamrami na ręce i nogi.
Dzisiaj także jest pełnia, do zachodu słońca zostało mało czasu.\de{}

Nie byłem skory, ale skoro mogę wykorzystać moją... przypadłość w dobrym celu, to czemu nie.
Przybysz jest taki mądry, na pewno wie, co robi.
No i będę przywiązany, więc na pewno tym razem, już nikogo więcej nie zabiję.

Położyłem się na stole, Przybysz przykleił mi do ciała końcówki sznurków, a na kranie E pojawiły się drgające kreski.
Chyba reagowały na mój oddech i bicie serca. Szkoda zachodu, mógł po prostu przystawić ucho do brzucha, żeby dowiedzieć się tego samego, bez marnotrawienia magii.
Na wszelki wypadek, zostawił mrożącą kuszę na widoku.

Byłem za bardzo pobudzony, aby zasnąć. Wkrótce jednak zacząłem tracić przytomność. Zaczyna się.

\divider{}
Obudziłem się także na stole, ubrania porwane na strzępy, sznurki pozrywane, klamry przegryzione.
Przybysz siedział w rogu i trzymał się za czerwony bandaż.

\ds{} Mamy mały problem.
\dm{} Sięgnął do deski guzików.
\dm{} Gdybyś po prostu się zerwał i mnie ugryzł, to by było jeszcze w porządku.
Ale po transformacji zacząłeś ze mną rozmawiać i przekonywać do swoich racji. \de{}

Ze sztucznych ust przy kranie E dało się słyszeć głos Przybysza, oraz drugi, zupełnie nieludzki.
Niski i ryczący, słychać było w nim echo kosmosu, dźwięk wielu zamieszkanych planet i ich słońc.
Czerwony karzeł, żółty karzeł i wielka czarna dziura. 
Dwa księżyce, pierścienie i sztuczny twór.
Miasta bogate i zdobione, pokój i szczęście mieszkańców, śmiech i radość.
Ale język, którym rozmawiał przyziemny, ograniczony, na pewno nie Polski.
Skąd ja to wszystko wiem?

\ds{} Przybysz spoglądał to na mnie, to na kran E. 
\dm{} Od razu rozpocząłeś rozmowę ze mną po Wspólnym, zwanym również Nowomową. 
Ten język powstanie dopiero za trzy tysiące lat w Ameryce, taki kontynent, nie ma jeszcze nikogo w tej chwili na świecie, kto potrafiłby się nim posługiwać.
Z wyjątkiem mnie. I chyba też ciebie? \de{}

Pokręciłem głową, ja także go nie rozumiem.

\ds{} Interesujące.
\dm{} Wstał i podał mi nowe ubrania.
\dm{} Proszę wyobrazić sobie bardzo inteligentne zwierzęta. 
Przychodzą z bardzo daleka i chcą się od nas nauczyć części naszej kultury.
Po ugryzieniu, w ciągu kilku lat stajesz się jednym z nich. 
\dm{} Wskazał palcem na moje włoski na rękach, były teraz dłuższe i jaśniejsze.
\dm{} Tak podbijają świat za światem, usuwają to, co złe, zostawiają tylko to, co w ich przekonaniu dobre i przyjmują u siebie.
\dm{} Wyciągnął książkę z szuflady.
\dm{} To nie zadziała. Mieszanie kultur nie zadziała. Nie można mieszać.
Uciekłem przed mieszaniem, uciekłem przed Nowomową, przed poprawnym zachowaniem.
Ziemski multikulturalizm to nowotwór, a co dopiero galaktyczny?
Oni upadną, tak, jak my upadniemy.
\de{}

W książce były ryciny spalonych domów, brudnych ulic i wojujących ludzi, ale nic z tego nie rozumiałem.
Przybysz patrzył się przez okno. Pierwsi ludzie wracali do wioski.

\ds{} Ale może gdyby wpierw oczyścić kulturę z brudu, to by zadziałało? \dm{} Patrzył na mnie, szukając potwierdzenia racji.
\dm{} Jedno jest pewne. \dm{} Zgasił kran E. \dm{} Na pewno nie jest Pan wilkołakiem. \de{}

\ds{} Nie jesteśmy my \dm{} poprawiłem. \de{}













