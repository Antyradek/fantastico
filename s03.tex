\chapter{Ksiądz} 
\ds{} Dzień dobry panu, przyszliśmy zabrać pana na operację \dm{} powiedziało dwóch agentów Poczty stojąc w drzwiach mojego domu. \de{}

\ds{} Jaką znowu operację? \dm{} Dobrze wiedziałem, jaką operację. W końcu to się musiało stać. Jednak nadal nie wiedziałem, dlaczego akurat dzisiaj. \de{}

\ds{} Nie wie pan? \dm{} Agenci spojrzeli po sobie. \dm{} No tak, jak mógłby pan to wiedzieć, nie ma pan przecież jeszcze mrówki. Nie ogląda pan telewizji.
Tak, tak.
W związku z ostatnimi niewytłumaczalnymi atakami spazmów u obywateli w całym mieście, Poczta stwierdziła, że dla bezpieczeństwa, 
wszyscy nieochipowani za niemowlaka powinni obowiązkowo mieć zainstalowaną mrówkę. \de{}

\ds{} Ale przecież tylko osoby z właśnie takimi mrówkami doznają ataków, bezpieczniejsi będziemy bez nich. \dm{} Próbowałem się bronić, ale wiedziałem, że
to i tak nie ma sensu. \dm{} Poza tym operacja wszczepienia u dorosłych ma wyższe ryzyko niepowodzenia, nie opłaca się. \de{}

\ds{} Tak, tak. Niefortunne narodzenie w samolocie, nie można było wszczepić mrówki od razu, przykro nam.
Ale teraz może być pan spokojny, wynaleziono nową, bezpieczną metodę. Dzięki niej każdy może cieszyć się swoją mrówką w dowolnym wieku.
Czy to dorośli, czy dzieci, teraz operacja jest bezbolesna, całkowicie bezpieczna i trwa tylko godzinę.
W pełni refundowana przez Pocztę. Zamów już dziś \dm{} agent wyrecytował reklamę, która prawdopodobnie leci w telewizji co pięć minut. \dm{}
Zapraszamy z nami, chip jest dla bezpieczeństwa wszystkich. Przyczyny ataków nadal są nieznane i jest szansa, że nosicielami mogą być niezachipowani.
Poczta musi wyeliminować taką możliwość.
Chyba nie chce być pan oskarżony o celowe wywoływanie zamachów? \dm{} Odsunął się pokazując miejsce w prywatnej kapsule, która zawiezie mnie prosto do 
najbliższego szpitala. Zastanawiałem się, czy jeszcze wrócę do swojego domu.

To było dzisiaj rano. W tej chwili wracam niebusem ze szpitala na stację górnego dworca, gdzie poczekam dziesięć minut na pociąg magnetyczny do domu.
Jadę sto-trzydzieści-jeden metrów nad ziemią w wagoniku zawieszonym na szynie, z prędkością pięćdziesięciu-pięciu kilometrów na godzinę. 
Pode mną są slumsy, miejsce od którego powinienem się trzymać z daleka. 
To składowisko podejrzanych osób, niebezpiecznej technologii, mafii i wszystkich odpadów z miasta.
Wyznają nawet swoją zakazaną \censor{}.
Dziwne, nie mogłem znaleźć odpowiedniego słowa.
Niebus dojedzie do stacji za dwadzieścia-trzy minuty.

Wiedziałem wszystko, nie musiałem nawet spoglądać na ekran z rozkładem, aby znać kolejność stacji.
Dworzec górny, fabryki jedzenia, jezioro pocztowców, centrum handlowe klasy średniej, rezerwat.
Byłem wszystkim, mogłem wszystko. 
Dzięki ci, Poczto za tą mrówkę. Ty, która zaczęłaś od dowożenia listów, tak się rozrosłaś przejmując firmę za firmą, że sprawnie zarządzasz teraz całym miastem.
Jedyną myślą, która nie pozwala mi być w pełni szczęśliwym, są ataki. Przecież prawdopodobieństwo dostania spazmów jest tak małe.
Jest większa szansa na wygranie w loterię --- kupon kupię na stacji za dychę --- niż to, że jakiś atak nastąpi teraz w tym szynobusie.
I nie stwierdzono, jakoby były zaraźliwe.

To były moje myśli? Jeszcze rano bałem się wyjść z domu bojąc się tej tajemniczej choroby, ale teraz coś mi mówi, że byłem w dużym błędzie.
Tysiące myśli potwierdzających, że to nic groźnego, przelatuje przez moją głowę.
Kierują mną, abym myślał, że sam doszedłem do takich wniosków.
Albo też nic mną nie kieruje, wszystko jest tak, jak dawniej. To ja mam całkowitą kontrolę nad sobą.

Kupię sobie ładną nakładkę na czoło, w tej chwili goła mrówka jest brzydkim, wystającym trójkątem.
Widzę, jacy inni noszą ciekawe rozwiązania, nie muszę szukać ich po całym mieście, od razu wiem, gdzie sprzedają, który model.
O, ten ma taki fajny irokez, który świeci kolorowo w zależności od myśli. Patrzy w okno, różowy oznacza miłość.

\censor{}. Ostatnimi siłami spróbowałem przywołać na myśl \censor{}. 
Zarówno to, jak i \censor{}, oraz \censor{} były zakazane.
Same myśli, jakby były wymazywane z głowy jeszcze przed interpretacją przez mózg.
Ale przecież to nie problem, nie powinienem o tym myśleć. 
Tylko mieszkańcy slumsów robią \censor{}, jestem porządnym obywatelem, trzymam się od tego z daleka.

Słabość, czuję się słabo.
Tak, to pierwszy objaw ataku, ale nie ma się czego bać.
Każdemu może się zrobić słabo, tak duszno jest w tym wagoniku, 30\% stężenia $CO_2$.
Nogi odmawiają posłuszeństwa, dźwięk zanika, tracę kontakt z otoczeniem.
Tracę władzę nad ciałem! Irokez zmienił kolor na czerwony.
Karetka nie podleci do wagonika, jest zajęta innym atakiem!
Nic mi nie będzie.

Wtem wychodzi z tłumu wysoki człowiek w płaszczu.
Na nogach sutanna, pod szyją ma biały kołnierzyk, to musi być \censor{}.
Patrzy na moją mrówkę, przeszywa ją na wylot, patrzy się na gniazdo do połączenia.
Słyszę teraz jego słowa, to stary język, łacina. Treść jest nieistotna, nie muszę jej znać.
Surowe spojrzenie, ręką wykonuje znak \censor{}, w lewej dłoni trzyma wisiorek z \censor{}.
W prawej butelkę z wodą \censor{}, otwiera ją, wylewa na nas.
Piecze, piecze! To kwas. Palimy się. 
Ciszej, ciszej! Tylko nie Ojcze Nasz, nie wypowiadaj Jego Imienia, nie przyzywaj Jej, nieee!
Uciekać, trzeba uciekać. Zabije nas. Z tego śmiertelnika już nic nie będzie, odwołać atak.
Gdzie uciekać? W dół, do slumsów. Tam, dwójka ma sprawne mrówki, rozdzielić legion!
Ha, żegnaj ojczulku, nie uratujesz go już. Chwila, czy to pendrive z Mrówixem? Cholera, ale to nic. Ten też wkrótce padnie.

\ds{} Wolne oprogramowanie, tworzone przez wszystkich, dla wszystkich. Napisane z miłości, waszej największej słabości. \de{}

Ciemność. A potem szturchanie przez nieznajomego.
Wysoka osoba w płaszczu była pochylona nade mną, uśmiechała się, jakby właśnie uratowała człowieka przed śmiercią.
Pomógł mi wstać i utrzymać się na nogach, choroba ustąpiła natychmiastowo.
Miałem mokrą twarz, ale nie od potu. Po co mnie oblał wodą?

\ds{} Czy dobrze się czujesz? \dm{} Jego głos pełen był nadziei. \dm{} Udało się, Bogu dzięki. Przy okazji jestem Ojciec Tomasz, franciszkanin. \de{}

\ds{} Przepraszam, chyba trochę zasłabłem. \dm{} Ludzie wokół stali, jak zamrożeni. Nikt się nie poruszał, nie mrugał. \dm{}
Czy... \de{}

\ds{} Dobrze, potem wyjaśnimy, goni nas czas. A dokładniej Poczta. \dm{} Ksiądz pociągnął rękojeść awaryjnego otwierania drzwi.
\dm{} Tak na szybko. Zostałeś zaatakowany przez demona, udało mi się go przepędzić i zainstalować Mrówixa, wspinamy się na górę i uciekamy po szynie.
Masz pełną kontrolę nad chipem, Poczta nie będzie wiedzieć, gdzie jesteś.
Ci ludzie zostali zdalnie spauzowani, żeby nic nie widzieli i nic nie słyszeli. Oczywiście Poczta nadal wszystko przez nich widzi, pomachaj im.
Za kilka minut włączą ręczne sterowanie i ci ludzie nas rozszarpią, dlatego spiesz się. Tu jest rączka, podnieś się zaprzyj nogą tam.\de{}

\ds{} Przecież szyna jest pod napięciem! \dm{} Tak stwierdziłem wedle mojej wiedzy. \dm{} Sto-ileśtam metrów w dół! \de{}.

\ds{} Tylko niektóre części są niebezpieczne, masz w głowie plany budowy niebusa, poszukaj dokładnie, puki co rób to, co ja i nie dotykaj niczego więcej. \de{}

Wyszliśmy ostrożnie na dach i zaczęliśmy biec po szynie do najbliższej podpory.
Niebus. Szynobus. Mam. Dokładne plany linii, przekroje, schematy budowy i działania.
Już nie byłem wszystkim. Byłem książką, dokładniej biblioteką. I to dość zakurzoną.
Brak bezpośredniego połączenia z internetem. Już nic mi informacji nie podsuwało, musiałem je sobie sam znaleźć.

Przęsło numer 23, będą schody na dół. Zamek szyfrowy przy wejściu to PoczLok-2122, zamknięty, ale zabezpieczenia dziurawe, jak ser. 
Znalazłem kilka exploitów i program do hackowania. Jest też instrukcja obsługi, trzeba ręcznie przeczytać.

Akurat dopadliśmy schodów, gdy pocztowy grawitowiec wysunął się zza budynku. 
Zbiegliśmy, prawie zeskoczyliśmy ze schodów. Zamek poszedł w kilka sekund, byliśmy w centrum slumsów.
Miałem mapę, miała zaznaczone kryjówki, dopadliśmy pierwszej z nich i zniknęliśmy w sieci starej kanalizacji.
Notatka mówiła, że Poczta nie zna jej przebiegu. Jeszcze.

\divider{}
Mieszkanie Ojca było bardzo biedne, ale czyste i zadbane.
Usiedliśmy pod oknem patrząc na ostatnie pocztowe grawitowce opuszczające zawieszony w górze wagonik.
Gdy ruszył, wszystko wróciło do normy, ludzie zostali odpauzowani, nikt nic nie pamiętał, nikt nie zauważył mojego zniknięcia.
Z wyjątkiem mnie.

\ds{} ...i taka właśnie jest historia \dm{} Ojciec Tomasz skończył opowiadać o atakach. \dm{}
Tym razem to ani nie wirus, ani nie choroba, ani żadna broń biologiczna, ale demony nie z tego świata.
Przez setki lat ludzie mieli w mózgu naturalną obronę przed tymi pomiotami, ale wraz z nadejściem mrówek to się zmieniło.
Kiedyś trzeba było interesować się voodoo, odprawiać czarne msze i wróżyć z kart, aby zaprosić szatana do siebie.
Dzisiaj sam wchodzi atakując chip i przez niego mózg. Nowoczesne czasy, nowoczesne opętanie.\de{}

\ds{} I może jeszcze szatan wgrywa tam swoje oprogramowanie? \dm{} Zażartowałem. Ciężko mi było uwierzyć w to wszystko. \de{}

\ds{} Nazywa się BorutOS, mamy jego obraz i inżynierią wsteczną opracowaliśmy exploity, aby przełamać go i zastąpić otwartym projektem Mrówixa. \dm{}
Ksiądz skrzywił się, widząc śmiech na mojej twarzy. \dm{} Nie wierzysz? To dzięki temu właśnie cię uratowałem, wykorzystałem lukę w systemie, który
demony wysyłały do twojej mrówki i zastąpiłem naszym. Puki co nie mogą znieść faktu, że ten system jest tworzony przez różne niepowiązane osoby, z miłości.
Puki co, jesteśmy bezpieczni. \de{}

\ds{} No dobrze. W każdym bądź razie żyję, jestem uodporniony na ataki, ale co dalej? \dm{} Doszło do mnie, że nie mogę tak po prostu wrócić do domu.\de{}

Ojciec otworzył okno, smród stęchlizny, spalonych silników elektrycznych i smarów wdarł się do pomieszczenia.

\ds{} Czekamy na koniec świata \dm{} powiedział, wskazując ręką na miasto. Grupa pocztowych grawitowców ponownie gdzieś zmierzała. \dm{}
Albo pomożesz mi ratować ludzi. Każdy się przyda, trzeba zdobywać informacje, ulepszać nasz system, pisać wirusy, dbać o słabszych... czasem zabijać pocztowców. 
Nie żebyś miał jakiś większy wybór, do domu nie pójdziesz, legalnie nie popracujesz, właściwie to oficjalnie umarłeś. \de{}

\ds{} Pomogę. \dm{} Nie zastanawiałem się długo. \dm{} I tym razem to mój świadomy wybór. \de{}












