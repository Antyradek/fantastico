\chapter{O tym jak mały Jasio znalazł żyrandol, a potem wszyscy się utopili}

\info{Grafomaniowy tekst, gdzie słowa są dosłownie.}

Pewnego dnia, Jasio znalazł w domu magiczną lampę dżina.
Jednak ponieważ akcja się dzieje w Polsce, a w Polsce nie ma lamp, to trzeba uznać, że znalazł taką normalną lampę.
Znaczy, są w Polsce lampy, ale takie na prąd, a nie takie na ogień, więc nie mógł znaleźć takiej na ogień, bo by nie był w Polsce.
Więc dla uproszczenia uznajmy, że to będzie normalna lampa dżina. Taki żyrandol, jak wisi w domach na suficie, i że Jasio go tak po prostu znalazł. 
To się stało, gdy pewnego dnia bawił się w pokoju.
Spadł mu prosto w ręce, gdy bawił się na dywanie.

No i go to zabiło, bo gruzy z sześciu ścian są trochę ciężkie. Więc zacznijmy od początku.
Jasio bawił się na środku pokoju, gdy w ręce wpadła mu magiczna lampa dżina. Tfu żyrandol, taki Polski Żyrandol.
Właściwie, to żyrandol nie jest polskim słowem, więc chyba powinien być to ,,ozdobny zwis sufitowy''.

Normalnie jest tak, że lampę trzeba potrzeć, żeby wyskoczył z niej dżin, ale to takie niemoralne, no bo kto moralny wyciera żyrandole, które znajduje?
Znaczy, jak nie jest się pedałtem. Ale Jasio miał już dziewczynę, więc nie pocierał.
Do tego nie chciał robić holokaustu roztoczom z kurzu.
Więc uznajmy, że skoro to była lampa na światło, to dżin wyszedł z lampy, jak tylko Jasio ją włączył, wsadzając ją do kontaktu.

Wtem nagle wyskoczył dżin. Był taki umięśniony i niebieski, jak w bajkach Disneya, bo nikt nigdy nie widział innego dżina w prawdziwym życiu, więc nie mógłby sobie wyobrazić żadnego innego.
\begin{dialogue}
	\ds{} Witaj mały Jasiu. Jestem dżin Dźn Dzinn z ozdobnego swisu zufitowego \dm{} powiedział dżin Dźn Dzinn.
\end{dialogue}
	Mały Jasio, trzęsąc się z porażenia prądem, poszukał tego motocykla, co mu wjechał do pokoju, ale go nie znalazł, bo to nie motocykl tak robił, tylko dżin Dźn Dzinn.
\begin{dialogue}
	\ds{} Spełnię twoje trzy życzenia, bo zwykle dżiny tak robią, a ja jestem dżinem. \dm{} Dumnie wyszczerzył swoją masę mięśni. \dm{} Dżinem Dźn Dzinn.
	\ds{} A dlaczego nie jedno, jak inni? \dm{} pytał Jasio. \dm{} Jak Złota Rybka ostatnio dała mi.
\end{dialogue}
Dżin Dźn Dzinn wzdrygnął się na tę myśli i szczęknął mięśniami, ale nie dlatego, że go obrzydzało, a dlatego że ostatnia osoba była... no... miała... takie życzenia... jakby... co to w internecie są czasami... jak się wpisze adres... może nie przy dzieciach.
\begin{dialogue}
	\ds{} Widzisz, ja jestem lepszy, niż rybka, bo jestem dżinem Dźn Dzinnem i mogę spełniać trzykrotnie więcej życzeń.
	A także pozwalam na życzenie sobie więcej życzeń, na co cyganki i czterolistne koniczynki nie pozwalają. \dm{} Pokręcił się na popękanej od obciążenia podłodze,
	jak bączek, co go małe dziecko puszcza.
	\ds{} To ile życzeń mam w ostateczności? \dm{} Jasio zatknął nos i wyciągnął kalkulator z plecaka i zaczął na nim liczyć ilość życzeń. Zabrakło mu guzików do liczenia, bo nie miał palców, a nie miał palców, bo potrzebował na czymś liczyć.
	\ds{} Daj, pomogę ci. \dm{} I magiczny niebieski duch dodał mu całe mnóstwo guzików w celu pomocy liczenia.
	\ds{} Dziękuję, ale taki łuskowaty plecak to ciężko teraz założyć, uwiera w plecy.
	\ds{} Proszę bardzo. Oprócz życzeń, pomagam za darmo w codziennych problemach. \dm{} Wyciągnął rękę. \dm{} Masz trzy życzenia. \dm{} Pokazał czy palce.
	\ds{} Tak, to są palce, dlaczego pytasz?
	\ds{} Nie to, tylko trzy. TY-RZY!
	\ds{} Ja Jasio, nie żaden Rzy \dm{} odpowiedział nieprawidłowo Rzy, któremu Dźn Dzinn nieopatrznie zmienił imię.
	\ds{} Masz już dwa życzenia. Bo się zagapiłem. Przemyśl dokładnie, co chcesz powiedzieć.
	\ds{} No nie wiem, hmmm. \dm{} Jasio, który to był już teraz Rzy, zaczął dokładnie myśleć.
	\dm{} Nie, nie zastanawiaj się, nie to miałem na myśli!
\end{dialogue}
Ale Rzy padł trupem, jak wyłączona lalka, z której wyciągnięto baterie. 
Albo bardziej jak lalka z wyciągniętą baterią, którą wyłączono.

Rozumiecie już chyba, jak działają dżiny, prawda? No więc gdy dżin zrozumiał swój błąd, że za dużo mówił, to przywrócił Rzy do pierwotnej postaci.
Zaczęli od początku, to jest od lampy dżina. Od ozdobnego zwisu sufitowego dżina, takiego na prąd.

\begin{dialogue}
	\ds{} Witaj, jestem dżin Dźn Dzi...
\end{dialogue}
Małpa wieszała się na ozdobnym zwisie dżina i wcale nie odpowiadała, a do tego odłączyła dżinowi wtyczkę z kontaktu.
Chciałem powiedzieć... napisać, nie mogła się wieszać, bo lampa była na na suficie, a na ziemi.
I nie mogła też odpowiadać, bo małpy przecież nie znają zasad poprawnego formułowania zdań.

\begin{dialogue}
	\ds{} Mam tego już dość! \dm{} Rzeczywiście miał już tego dość! \dm{} Autorze, pisz poprawnie to opowiadanie, bo mi nie wychodzi spełnianie życzeń. 
	Co chwila przekręcasz słowa i nie mogę poprawnie dżinować! Otrzepał się z piachu i zdmuchnął brud ze swojego zwisu.
\end{dialogue}

Autor postanowił przyznać dżinowi rację i powtórzył całość.
	
Pewnego dnia Jasio znalazł w domu magiczną lampę dżina.
Jednak ponieważ akcja się dzieje w Polsce, a w Polsce nie ma lamp, to trzeba uznać, że znalazł taką...

\begin{dialogue}
	\ds{} Nie! \dm{} grzmotnął dżin, którego jeszcze nie miało być w tej scenie. Daj mi jasia!
\end{dialogue}

Załapał się za głowę, popatrzył na wybitą dziurę w ścianie i zaczął się po niej masować.

\begin{dialogue}
	\ds{} Witaj dżinie Dźn Dzinnie, jestem mały jasio \dm{} powiedział mały jasio, szeleszcząc puchem.
	\ds{} Taa. \dm{}Dżin Dźn Dzinn otrzepał ręce z tynku. \dm{} Mam cię w dup... znaczy... mów życzenia, szybko.
	\ds{} Chcę wiertarkę i cztery żółte kulki. \dm{} Jasio zawiedziony wyciągnął jeden ze swoich rogów z czterech liter dżina.
	\ds{} Po co ci wiertarka? \dm{} zapytał dżin Dźn Dzinn jasia, wyciągając sobie resztki puchu z ucha. (A myślałeś, że z czego?) \dm{} Masz.
	\ds{} Dżżżnnn Dźnnn Dziiinnn \dm{} zrobiła wiertarka.
	\ds{} Słucham \dm{} odpowiedział dżin. A potem zrozumiał, że odpowiedział wiertarce. Niestety, jego prawo kazało mu spełniać życzenia również przedmiotom. \dm{} No niech będzie, byle rach ciach ciach. \dm{} W pokoju nastały przez chwilę Święta Bożego Narodzenia.
\end{dialogue}

I tutaj pewnie czytelnik się zdziwi, ale wiertarki nie potrafią mówić. Jednak ta powiedziała. Bo bez tego ta opowieść by mi się zatrzymała w połowie, bo jak ja piszę, to nie myślę o przyszłości. Nie chcę przestawać pisać...

W każdym razie wiertarka poprosiła sobie własnego królestwa na obcej planecie, gdzie żyła długo i szczęśliwie, będzie o tym inna opowieść.
Prawdziwa taka, z laserami, smokami, ludźmi w statkach kosmicznych i problemami społecznymi, jakie mogą występować w cywilizacji wiertarek, które to potem można zastosować w codziennym życiu ludzkim. Już nie mogę się doczekać tych sklepów zoologicznych.

A cztery kulki były potrzebne jasiowi po to, żeby czytelnik zastanawiał się po co mu były. 
Jak w tym żarcie o Jasiu i czterech kulkach, co prosił rodziców przez całe życie o nie, a jak miał powiedzieć po co mu były, to wykitował.

\begin{dialogue}
	\ds{} Zostało ci jeszcze jedno życzenie. \dm{} Dżin wyprał jasia w pralce, aby pozbyć się resztek kleju i złożył do kupy igłą i nitką.
	\ds{} Jedno?
	\ds{} JE-DY-NO.
	\ds{} No dobrze, daj mi się zastanowić. \dm{} Ukroił sobie kawałek NO na bankiecie DY, z którego wszyscy uciekli, ze względu na smród.
	\ds{} Cokolwiek powiesz, to się spełni. W końcu będzie koniec tego beznadziejnego opowiadania.
	\ds{} No to na trzecie życzenie to morze chcę...
\end{dialogue}









