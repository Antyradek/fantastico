\chapter{O tym jak mały Jasio znalazł żyrandol, a potem wszyscy się utopili}

\info{Grafomaniowy tekst, gdzie słowa są dosłownie.}

Pewnego dnia Jasio znalazł w domu magiczną lampę dżina.
Pozwalała ona, znaczy jej zawartość, spełniać dowolne życzenia posiadacza.
Jednak ponieważ akcja się dzieje w Polsce, a w Polsce nie ma lamp, to trzeba uznać, że znalazł taką normalną lampę.
Znaczy, są w Polsce lampy, ale takie na prąd, a nie takie na ogień, więc nie mógł znaleźć takiej na ogień, bo by nie był w Polsce.
Więc dla uproszczenia uznajmy, że to będzie normalna lampa dżina. Taki żyrandol, jak wisi w domach na suficie, i że Jasio go tak po prostu znalazł. 

Znalazł żelazny, złocony żyrandol. Jak pewnego dnia bawił się w pokoju, który spadł mu na ręce.
No i go to zabiło, bo gruzy z sześciu ścian są trochę ciężkie. Więc zacznijmy od początku.
Jasio bawił się na środku pokoju, gdy w ręce wpadła mu magiczna lampa dżina. Tfu żyrandol, taki Polski Żyrandol.

Nawiasem mówiąc, to żyrandol nie może być polski.
Nie jest polskim słowem, w oryginale nazywał się po polsku w inny sposób, więc trzeba by przetłumaczyć mu tutaj nazwę na polską.
Chyba powinien być to ,,ozdobny zwis sufitowy'', jak to mówiły nasze dziadki.
Ogólnie wiele jest takich niepolskich słów po polsku. Polewka, przydeptywanie manipulatora kulkowego, czy pisanie na kluczplacie.
A nie, to ostatnie to był kaszubski.

W każdym razie, wróćmy do opowieści.
Normalnie jest tak, że lampę trzeba potrzeć, żeby wyskoczył z niej dżin, ale to takie niemoralne, no bo kto moralny wyciera żyrandole, które znajduje?
Znaczy, jak nie jest się pedałtem. Ale Jasio miał już dziewczynę, więc nie pocierał.
I był bardzo normalny, kiedyś rwał sobie koszulę i tarzał się po ziemi z rozpaczy na widok kogoś trzymającego papierosa w zakazanym miejscu.
Więc uznajmy, że skoro to była lampa na światło, to dżin wyszedł z lampy, jak tylko Jasio ją włączył. Wsadzając ją do kontaktu.
To spowodowało, że mieszkaniec lampy postanowił wyjść, gdy zrodziła się w głowie taka myśl do której szedł i jak doszedł.

Wtem nagle z lampy wyskoczył dżin. 
Był taki umięśniony i niebieski, jak w bajkach Disneya, bo nikt nigdy nie widział innego dżina w prawdziwym życiu, więc nie mógłby sobie wyobrazić prawdziwego.
\begin{dialogue}
	\ds{} Witaj mały Jasiu. Jestem dżin Dźn Dzinn z ozdobnego zwisu sufitowego \dm{} powiedział dżin Dźn Dzinn z ozdobnego zwisu sufitowego.
\end{dialogue}
	Mały Jasio, trzęsąc się ze strachu i porażenia prądem z kontaktu, poszukał tego motocykla, co mu wjechał do pokoju, ale go nie znalazł, bo to nie motocykl tak robił, tylko dżin Dźn Dzinn.
\begin{dialogue}
	\ds{} Spełnię twoje trzy życzenia, bo zwykle dżiny tak robią, a ja jestem dżinem. \dm{} Dumnie wyszczerzył swoją masę mięśni. \dm{} Dżinem Dźn Dzinn.
	\ds{} A dlaczego trzy, a nie jedno życzenie, jak inni? \dm{} pytał Jasio. \dm{} Jak Złota Rybka ostatnio mi dała.
\end{dialogue}
Dżin Dźn Dzinn wzdrygnął się na tę myśli i szczęknął mięśniami, ale nie dlatego, że go obrzydzało, a dlatego że ostatnia osoba, której spełniał życzenia, była... no... miała... takie życzenia... jakby... co to w internecie są czasami... jak się wpisze adres... i takie obrazki... może nie przy dzieciach.
\begin{dialogue}
	\ds{} Widzisz, ja jestem lepszy, niż rybka, bo jestem dżinem Dźn Dzinnem i mogę spełniać trzykrotnie więcej życzeń.
	A także pozwalam na życzenie sobie więcej życzeń, na co cyganki i czterolistne koniczynki nie pozwalają. \dm{} Pokręcił się na podłodze,
	jak bączek, co go małe dziecko puszcza.
	\ds{} To ile życzeń mam w ostateczności? \dm{} Jasio wyciągnął kalkulator z plecaka i zaczął na nim liczyć ilość życzeń. Zabrakło mu guzików do liczenia, bo nie miał na dłoniach palców, a nie miał palców, bo potrzebował na czymś innym liczyć.
	\ds{} Daj, pomogę ci. \dm{} I magiczny, niebieski duch dodał mu całe mnóstwo guzików w celu pomocy liczenia.
	\ds{} Dziękuję, ale taki łuskowaty plecak to ciężko teraz założyć, uwiera w plecy.
	\ds{} Proszę bardzo. Oprócz życzeń, pomagam także zwalczać inne rzeczy. Codzienne problemy za darmo. \dm{} Wyciągnął rękę. \dm{} Masz trzy życzenia. \dm{} Pokazał czy palce. Jasio uważnie się przyglądał.
	\ds{} Tak, to są palce, dlaczego pytasz?
	\ds{} Nie to chciałem powiedzieć, tylko trzy. TY-RZY!
	\ds{} Ja Jasio, nie żaden Rzy \dm{} odpowiedział nieprawidłowo Rzy, któremu Dźn Dzinn nieopatrznie właśnie zmienił imię.
	\ds{} Masz już dwa życzenia. Bo się zagapiłem. Przemyśl dokładnie, co chcesz powiedzieć.
	\ds{} No nie wiem, hmmm. \dm{} Jasio, który to był już teraz Rzy, zaczął dokładnie przemyśliwać.
	\ds{} Nie, nie zastanawiaj się tak dokładnie, nie to miałem na myśli!
\end{dialogue}
Ale Rzy padł trupem jak martwy trup.
Trochę jak wyłączona lalka, z której wyciągnięto baterie. 
Albo bardziej jak lalka z wyciągniętą baterią, którą wyłączono... w każdym razie przestał myśleć.

Rozumiecie już chyba, moi drodzy czytelnicy... internauci, jak działają dżiny, prawda? 
Cokolwiek powiedzą, to się zaraz dzieje.

I tak sobie też teraz pomyślałem, że dżina można także przecież przedstawić jako świecącą kulkę światła. 
Jednak kulka światła w żyrandolu jest mało realistyczna, widział ktoś kiedyś cokolwiek takiego?
No i nie ma przynajmniej trzech palców.
Zostawmy więc niebieskiego mięśniaka.

No więc gdy dżin zrozumiał swój błąd, to jest że za dużo mówił, to przywrócił Rzy do pierwotnej postaci.
Zaczęli od początku, to jest od żyrandola dżina. 
Od ozdobnego zwisu sufitowego dżina, takiego na prąd.

\begin{dialogue}
	\ds{} Witaj, jestem dżin Dźn Dzi...
\end{dialogue}
Małpa wieszała się na ozdobnym zwisie dżina i wcale nie odpowiadała.
To znaczy... chciałem powiedzieć... napisać inaczej. 
Nie mogła się wieszać, bo lampa była nie na na suficie, a na ziemi. I nie miała wcale depresji.
I nie mogła ona też odpowiadać, bo one przecież nie znają zasad poprawnego formułowania zdań, małpy również.

\begin{dialogue}
	\ds{} Mam tego już dość! \dm{} Rzeczywiście miał już tego dość! \dm{} Autorze, pisz poprawnie to opowiadanie, bo mi nie wychodzi spełnianie życzeń. 
	Jesteś beznadziejnym pisarzem, nie dziwię się, że nikt nigdy nie chce nie zagłębiać się w nie niepłytko.
	Co chwila przekręcasz słowa i nie mogę poprawnie dżinować! Otrzepał się z kurzu i zdmuchnął brud ze swojego zwisu.
\end{dialogue}

Autor postanowił przyznać dżinowi rację i powtórzył całość.

\bigskip
	
Pewnego dnia Jasio znalazł w domu magiczną lampę dżina.
Pozwalała ona, znaczy jej zawartość, spełniać dowolne życzenia posiada...

\begin{dialogue}
	\ds{} Nie! \dm{} grzmotnął dżin, którego jeszcze nie było w tej scenie. \dm{} Auć, moja głowa. \dm{}
	Załapał się za głowę, popatrzył na wybitą dziurę w ścianie i zaczął się po niej masować. \dm{} Daj mi jasia!
	\ds{} Witaj dżinie Dźn Dzinnie, jestem mały jasio \dm{} powiedział mały jasio, szeleszcząc puchem. \dm{} Jestem jasiek, taka poduszka. Wyjaśniam, jakby w niektórych stronach Polski to określenie nie było znane. Bardzo ważne jest, aby w tym opowiadaniu wszystkie słowa były klarownie widoczne.
	\ds{} Taa. \dm{} Dżin Dźn Dzinn otrzepał ręce z tynku. \dm{} Mam cię w dupie... znaczy... nie, nie mam! Mów życzenia, szybko.
	\ds{} Chcę wiertarkę i cztery żółte kulki. \dm{} Jasio klarownie oznajmił, będąc wyraźnie zawiedzionym z naprawy pomyłki słownej dżina. Wyciągnął swój róg z czterech liter niebieskiego ducha, którego przed chwilą tam wsadził.
	\ds{} Po co ci wiertarka? \dm{} zapytał dżin Dźn Dzinn jasia, wyciągając sobie resztki puchu z ucha (a myślałeś, że z czego?) \dm{} Masz, co mi tam.
	\ds{} Dżżżnnn Dźnnn Dziiinnn \dm{} zrobiła wiertarka.
	\ds{} Słucham \dm{} odpowiedział dżin. A potem zrozumiał, że odpowiedział wiertarce. Niestety, jego prawo kazało mu spełniać życzenia również przedmiotom. \dm{} No niech będzie twoje życzenie, byle rach ciach ciach. \dm{} W pokoju nastały przez chwilę Święta Bożego Narodzenia, gdy puszyste wnętrzności jasia rozlały się po wnętrzu.
\end{dialogue}

I tutaj pewnie czytelnik się zastanawia, co mogła powiedzieć wiertarka na swoje życzenie.
Niestety, muszę go zawieść, ale wiertarki nie potrafią mówić. Więc nie mogła nic powiedzieć, bo by to nie była realistyczna opowieść.
Taka opowieść, jak moje pozostałe.
Jednak tak się złożyło, że ta powiedziała. Powiedziała, że zapisana w niej wiertarka powiedziała życzenie.

W każdym razie, wiertarka zażyczyła sobie własnego królestwa na obcej planecie.
Żyła potem długo i szczęśliwie, dopóki bateria nie padła, nie mając jednego słońca, a trzy. 
Będzie o tym nowa, inna opowieść.
Prawdziwa taka, z laserami, smokami, ludźmi w statkach kosmicznych i problemami społecznymi, jakie mogą występować w cywilizacji wiertarek, które to potem można wykorzystać w codziennym życiu ludzkim. 
Zaspoileruję, że doskonale nadają się u nas jako tania siła robocza przy konstrukcji statków.

A cztery kulki były potrzebne jasiowi po to, żeby czytelnik zastanawiał się po co mu były. 
Jak w tym żarcie o Jasiu i czterech kulkach, co prosił rodziców przez całe życie o te kulki, a jak miał powiedzieć po co mu były, to wykitował.

\begin{dialogue}
	\ds{} Zostało ci jeszcze jedno życzenie. \dm{} Dżin wyprał jasia w pralce, aby pozbyć się resztek kleju i złożył do kupy igłą i nitką.
	\ds{} Jedno?
	\ds{} JE-DY-NO.
	\ds{} No dobrze, daj mi się zastanowić. \dm{} Ukroił sobie kawałek NO na bankiecie DY. \dm{} Powinienem chyba poprosić o wyciągnięcie z tej kupy, w której mnie zaszyłeś, ale nie wiem.
	\ds{} Cokolwiek powiesz, to się spełni. W końcu nie będzie tego opowiadania z beznadziejstwem i przekręconych w nim słów. Nadejdzie kres. Tylko mówiąc, sprawdź dokładnie pisownię wypowiedzi.
	\ds{} No to na trzecie życzenie to morze chciałbym...
\end{dialogue}









