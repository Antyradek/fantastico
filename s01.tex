\chapter{Kontakt bezpośredni}
\info{Ziemowit i Psychit lądują na dziwnej, czarno-białej planecie i próbują zrozumieć zastaną cywilizację. 
Wszystko jest takie oczywiste, że aż zostają aresztowani za bycie za mało oczywistymi.}

\illustration{img/blackwhite.png}

Dźwięki doskonale zsynchronizowanej opery na dwanaście głosów wypełniały przestrzeń kokpitu.

\begin{dialogue}
\ds{} Dziennik pokładowy, wpis numer... och, jeden. \dm{} Psychit uderzał trójpazurzastymi szczypcami rytmicznie w klawiaturę. \dm{}
Misja typu kontakt bezpośredni z niezbadaną wcześniej cywilizacją. Uczestniczą potwór Psychit i Ziemianin Ziemowit.
\ds{} Urodziłem się na Felicji \dm{} poprawił potwora Zimi. \dm{} Od ostatniej wizyty na Ziemi nie chcę mieć z nią już nic wspólnego.
\ds{} I tak nikt tego nie będzie czytał \dm{} odpowiedział Psychit, nie odrywając wzroku z ekranu \dm{} ...Felicjanin Ziemowit.
Statek typu strzałkowiec piątej generacji, napęd na pręty operowe. Nazwa?
\ds{} Jeszcze nie wymyśliłem. Myślisz, że tak prosto nadać imię, gdy przed chwilą dostało się swój własny, ukochany statek? To trzeba przemyśleć \dm{}
wyjaśnił Ziemowit. \dm{} Poza tym, jakie to ma teraz znaczenie? Chyba nie wierzysz w te przesądy, że latanie nienazwanym statkiem przynosi pecha?
\ds{} Nie, ale łatwiej jest później skompletować jego fragmenty, gdy eksploduje. \dm{} Psychit wyszczerzył srebrzyste zęby. \dm{}
Powierzchnia planety jest całkowicie biała. Na razie brak reakcji ze strony mieszkańców. Rozpoczynamy podchodzenie.
Ziemowit prawdopodobnie znowu połamie wszystkie pręty.
\ds{} Nie pisz tego! \dm{} oburzył się Zimi i pociągnął tak nieuważnie stery, że jeden z dwunastu prętów rozsypał się z dźwiękiem mordowanej śpiewaczki operowej.
\end{dialogue}

Reszta lądowania odbyła się bez niedogodności, nie licząc złośliwego uśmieszku na paszczy Psychita.

\divider{}

Planeta okazała się przykryta grubą warstwą chmur, światło centralnej gwiazdy rozpraszało się całkowicie i produkowało depresyjny, jesienny dzień.
Pod nimi, na powierzchni równie białej, jak niebo, zobaczyli niewielkie skupiska czarnych domków. 
Ziemowit postanowił postawić swój stateczek na skraju małego miasteczka.

Wszytko na tym świecie było czarno-białe. Biała trawa, kamienie i chodniki kontrastowały z murami z czarnej cegły i mieszkańcami ubranymi we wzorzyste ubiory.
Nigdy się jeszcze nie zdarzyło, aby przy kontakcie bezpośrednim nie powstał chaos, a mieszkańcy nie uciekali przed intruzami w popłochu. 
Odpowiedniki wojska pojawiały się zazwyczaj w przeciągu dwóch kilopulsów.
Tym razem wszyscy kontynuowali swoje czynności.
Co jakiś czas, niektórzy ukradkiem przyglądali się gościom, ale zaraz wszechobecni policjanci zwracali im uwagę, głośno bucząc.
Zimi postanowił nazwać ich dyrygentami, bo tak śmiesznie wymachiwali patyczkami.

Mieszkańcy planety nie wyróżniali się niczym szczególnym. Mieli po dwie nogi, zginane w trzech miejscach, kręcone rogi na trójkątnych głowach i cztery mackowate ręce.
Dość normalne, zważywszy na różnorodność całego wszechświata.
Nienormalne natomiast było ich zachowanie. Każdy chodził prosto, jak po linie, ze stałą prędkością, lub całkowicie stał w miejscu.
Wszystkie czynności mieszkańcy wykonywali grupowo, do taktu.

Co ciekawe, to Zimi i jego statek przyciągali najwięcej spojrzeń, a nie stojąca przy nich, kolczasta, pięciometrowa, zębata bestia.
Tak się złożyło, że Psychit także był czarny, ze srebrnymi pazurami i kolcami w kształcie wielkich igieł chirurgicznych, 
widocznie to uchodziło za całkowicie normalne w tym świecie, w przeciwieństwie do wściekle pomarańczowego strzałkowca i
ubranego specjalnie pod jego kolor pilota. Ziemowit nieśmiało poprawił kaptur swojej bluzy.

Oprócz dyrygentów z pałeczkami i różnorodnie ubranych obywateli, po miasteczku kręciło się także całe mnóstwo malarzy.
Każdy z nich dźwigał dwa wiadra z farbą i biegając po całym mieście, malował na bieżąco wszystkie obtłuczenia, pobrudzenia i szarości świata.
Mała grupka nieustraszenie zebrała się obok przybyszów, chyba aby rzucić się na Ziemowita i go ,,naprawić'', lecz Psychit warknął na nich tak, że uciekli, rozlewając farbę.
Jednak jeden, najmniejszy malarz, zaraz wrócił i starannie zamalował ślad uciekinierów.

\divider{}

Dwójka dziwaków (a właściwie tylko jeden dziwak) szła środkiem chodnika z równą prędkością, taką jak wszyscy.
Zwolnili na chwilę, przechodząc na bok, aby obejrzeć wspaniałą fontannę w centrum.
Mała rzeczka wpływała u dołu i wytryskiwała z pomnika na szczycie w nudnobiałe niebo.
Być może fontanna uczestniczyła w produkcji tych całych chmur.
Gdyby nie rozproszenie światła przez niebo, woda mieniłaby się zapewne wszystkimi kolorami tęczy.
Wielu mieszkańców, po kolei, nabierało z niej wody do wiader i od razu duszkiem wypijało, pod czujnym okiem dyrygentów.
Całe szczęście że woda jest przezroczysta i nie trzeba jej barwić.

Naraz wokoło przybyszów zerwało się wielkie buczenie.
Wszyscy dyrygenci skakali wokół, krzycząc coś w niezrozumiałym języku, a z daleka nadbiegała tym razem chyba prawdziwa policja, z elektrycznymi włóczniami.
Ziemowit sięgnął do pasa po szyfrator, aby zamrozić ich w splocie czasoprzestrzeni, ale potwór nie chciał się bronić.
Postanowili więc pokojowo rozwiązać konflikt.

Kłując swoich gości w plecy, policjanci zaprowadzili ich do budynku z kratami w oknach i zamknęli w dużej celi.
Na szczęście środek budynku był biały, więc nie było to aż tak depresyjne miejsce, na jakie wyglądało z zewnątrz.

\begin{dialogue}
\ds{} Ciekawe, co z nami zrobią? Mają tu sąd, będą chcieli się porozumieć jakimś językiem migowym, a może zaraz nas wypuszczą i ugoszczą w pałacu? Zdarzało się to już przecież
\dm{} Zimi bombardował pytaniami swojego znajomego.
\end{dialogue}

Potwór wzruszył ramionami i wskoczył na łóżkopodobną strukturę w kącie, która zarwała się pod jego ciężarem.

\begin{dialogue}
\ds{} Demolowanie im mebli na pewno nam nie pomoże \dm{} Zimi skarcił Psychita. \dm{} Myślę, że nikt ich nie budował, żeby wytrzymywały trzy tony.
\ds{} Wypraszam to sobie, nie jestem aż tak wielki. Ważę jedynie dwie i pół tony, a to spora różnica \dm{} ryknęło monstrum. \dm{}
I to ich wina, powinni byli być przygotowani na przyjęcie członków jednej z czołowych organizacji wszechświata. 
Większość cywilizacji słyszała o nas i zawsze jest gotowa na nasze przyjście.
\ds{} Większość cywilizacji słyszała o was tyle, że jesteście morderczymi jaszczuropodobnymi istotami, zabijającymi w okrutny sposób całe populacje i jedzącymi małe dzieci \dm{}
poprawił Zimi, gapiąc się przez okno. \dm{} Podobno jest was tyle, że niszczycie sprawnie całe galaktyki. Niezłe legendy jak na dziewiątkę istot na cały wszechświat.
\ds{} To Pyrroq zjada chętnie dzieci, nie ja \dm{} bronił się. \dm{} Poza tym, jak wiesz, zabijanie kogokolwiek jest od ponad osiemdziesięciu gigapulsów, 
znaczy tych... waszych dwóch tysięcy lat zabronione, zwłaszcza dla nas.
Teraz mamy być posłusznymi zwierzątkami i nieść pokój i szczęście. Już raz Niebo ukarało mnie utratą mocy na... em... miesiąc. Straszne przeżycie.
Ty jeszcze nie masz mocy, więc co możesz wiedzieć. To tak jakby ci ktoś odciął ogon, skrzydła, wszystkie kończyny i zostawił dryfującego w przestrzeni kosmicznej.
\ds{} Mieszkam na Felicji! \dm{} Zimi się odwrócił. \dm{} My też liczymy czas w pulsach, mam już gdzieś ziemskie lata! Zapamiętaj, że...
\end{dialogue}

Przerwał w połowie zdania, bo trójka policjantów z włóczniami przyszła i otworzyła kratę, wyprowadzając więźniów.
Psychitowi zamknęli na rękach kajdany, między pierwszym a drugim kolcem, tak sprawnie jakby aresztowali istoty z kosmosu regularnie.

Poprowadzili ich do wielkiego pokoju, na środku którego widniała spora konstrukcja.
Miała ona wiele igieł, świdrów i harpunów skierowanych do środka, gdzie przymocowana była krata z pasami.
Malarze sprawnie zacierali czerwone ślady po ostatnim użyciu urządzenia.

\begin{dialogue}
\ds{} Śmierć za przejście na nieprawidłową stronę chodnika \dm{} zaśmiał się Psychit. \dm{} Dość tej zabawy \dm{} ryknął nisko i zerwał kajdany, jakby były z plasteliny.
\end{dialogue}

Zaraz rzucili się na niego policjanci, ale on podciął im nogi ogonem, w locie łapiąc ich włócznie w łapę i wbił z całej siły w podłogę, 
jakby chciał pokazać, że mógł równie dobrze w nich.

Popatrzył słodko na przewodniczącego, który ukrywając strach, krzyknął coś niezrozumiałego do podwładnych.
Zaraz otworzyli główne drzwi i zaprosili byłych więźniów do wyjścia, jak gdyby nic się przed chwilą nie stało.

\divider{}

\begin{dialogue}
\ds{} Nie mam pojęcia, co się właśnie wydarzyło, a ty? \dm{} zagadnął Zimi, wracając w stronę rakiety. \dm{} Oczywiście dziękuję za ratunek. Jak zwykle.
\ds{} Myślę, że ten świat nie jest jedynie czarno-biały. Jest także czarny i biały nawet u podstaw \dm{} Psychit głośno rozmyślał. \dm{} 
Nie ma tu miejsca na szarości, nawet w zachowaniu obywateli. Śmierć, albo wolność, środek chodnika, albo stanie w miejscu. 
Wypicie całego wiadra wody na raz, albo wzięcie całego ze sobą. Wszyscy robią rzeczy albo doskonale, albo wcale. Za zejście z linii od razu śmierć.
\ds{} Taki perfekcjonizm może powodować dużo dobrego, ale jakoś nie widać po ludziach, żeby byli szczęśliwi. Myślę że jednak ich system działa gorzej, niż lepiej.
\ds{} Już nie raz widywaliśmy zniewolone światy. Ale ci mogą zawsze przeskoczyć na inny kolor, zrobić dokładnie odwrotnie, niż robili wcześniej, więc technicznie rzecz biorąc, mają więcej wolności od nas.
\dm{} Potwór zmrużył ślepia. \dm{} Ja nie mogę jednego dnia burzyć miast, a drugiego odbudowywać bez konsekwencji.
\ds{} Nasze prawo nakazywałoby naprawić ten system. ALOPP zostało przecież przez was powołane tylko po to, żebyśmy my, ludzie, pomogli wam, potworom nieść dobro, a tutaj nie za dużo go widzę. 
Ale tym razem może postaraj się nie rozwalić wszystkiego w drobny mak.
\ds{} To jedna wielka kolorowanka. Możemy użyć tylu kolorów, ilu chcemy. \dm{} Psychit przystanął. \dm{} A zaczniemy od przywrócenia twojej rakiety z powrotem do pomarańczowego.
\end{dialogue}

Nieuważnie zostawili statek nieupilnowany, doskonały kąsek dla malarzy.
Ci z największą dokładnością pokryli każdą powierzchnię białą farbą, nawet okna.

\begin{dialogue}
\ds{} Mój stateczek. Moje dziecko! \dm{} Zimi pobiegł do swojego nowego nabytku. \dm{} Nawet nie zdążyłem się tobą wystarczająco nacieszyć!
\end{dialogue}

Ziemowit biegał w kółko wokół urządzenia.

\begin{dialogue}
\ds{} Farba się zmyje, lepiej sprawdź czy... \dm{} Przerwał mu krzyk.
\end{dialogue}

Wygląda na to, że malarze aż za bardzo przyłożyli się do swojej pracy. 
Zajrzeli w każą dziurkę, w każdy zakamarek.
Twardymi pędzlami potłukli wszystkie delikatne, zasilające pręty operowe.
Znaczy prawie wszystkie, bo jeden zbił wcześniej Zimi.

Ziemowit pokazał garść niebiesko-białych kryształków.

\begin{dialogue}
\ds{} Masz zapasowe, prawda? Masz całe dwa zestawy. \dm{} Psychit się nie przejął. \dm{}
W najgorszym razie to wyniosę cię poza atmosferę na rękach, a tam otworzymy przejście do dolnej warstwy.
\end{dialogue}

Dał się słyszeć drugi krzyk. Czarna farba wylała się ze środka, gdy tylko Zimi otworzył właz.

\begin{dialogue}
\ds{} Ty ostatni wychodziłeś, nie zamknąłeś włazu, prawda? \dm{} wysyczał przez zęby. \dm{} Chyba czeka cię nieco dłuższy lot do domu na własnych skrzydłach. 
I będziesz trzymał mnie, statek i pięć ton powietrza w bańkach energii, bo farba zalała grubą warstwą wszystkie ściany i nie mam ochoty siedzieć w środku w czasie podróży.
A przedtem jeszcze sam zanurkujesz do środka i wyłowisz klucz dolnej warstwy, żebyśmy nie spędzili na podróży tryliona pulsów.
\ds{} Dobrze, popsułem. \dm{} Potwór podwinął ogon. \dm{} Ale nie mam wystarczającej siły na lot do domu, jeszcze rozciągając tak wielkie tarcze. To wykańcza, wiesz? 
A gdybym ja kazał ci teraz biec podwójny maraton ze mną na plecach? Zrobiłbyś to bez problemu? My też się męczymy \dm{} bronił się. \dm{} 
W dodatku ten system gwiezdny jest na skraju kwadry, podróż dolną warstwą, nawet prosto przez środek wszechświata, zajęłaby mi kilka kilopulsów. 
Lepiej wezwij pomoc, Floria z Mojmirą miały zwiedzać sąsiednią galaktykę, na pewno szybciutko tu przylecą swoją latającą szklarnią.
Może nasza mistrzyni życia wyrośnie tu kilka kolorowych kwiatów, ciekawe jak zareagowaliby mieszkańcy.
\end{dialogue}

Ziemowit wyjął z kieszeni swój okrągły komunikator, aby wezwać pomoc, ale okazał się on być odłączony od korpusu.
Te urządzenia używają wewnątrz małych portali do dolnej warstwy, jak baterii, dzięki czemu mają nieograniczoną moc obliczeniową.
Czasami zdarza się, że połączenie się zerwie i zostajemy wtedy jedynie z podstawką pod szklankę.

\divider{}

\begin{dialogue}
\ds{} Skoro i tak spędzimy tu trochę czasu, to równie dobrze moglibyśmy trochę z nimi poeksperymentować \dm{} westchnął Psychit, patrząc w stronę fontanny. \dm{}
Zacznijmy od czegoś prostego.
\end{dialogue}

Złapali więc jednego z obywateli i zmusili do patrzenia na żarówiastą bluzę Ziemowita.
Psychit wypalił laserem ze swojej dłoni dziurę w ziemi i pokazywał mu barwne przekroje skał, oraz jak świeci rozgrzana farba.
Obsypywali go niebieskim proszkiem z prętów.

Kosmita nic sobie z tego nie robił. 
Na każdy kolor zawsze zamykał oczy i siedział w milczeniu.
Czy dlatego że nie chciał, czy dlatego że się bał, bo za doświadczanie koloru groziło podziurawienie na wylot? Na dedykowanej ku temu maszynie.

Próbowali podobnie robić z kilkoma innymi istotami.
Od buczenia dyrygenta więdły im uszy, najwięcej zainteresowania wykazał jednak malarz, który opryskał wszystko pokazywane farbą.

\begin{dialogue}
\ds{} Pora na krok ostateczny. Dość mam już tego cackania się \dm{} powiedział czarny potwór z białe kropki.
\ds{} Nie róbmy im krzywdy, nie wolno nam! \dm{} odpowiedział pomarańczowy człowiek w czarne kropki. \dm{} Co zamierzasz wyprodukować swoją mocą?
\ds{} Jeśli nie chcą patrzeć na kolory, to zrobimy tak, żeby widzieli je nawet przy zamkniętych oczach \dm{} powiedział, wkładając pazury do fontanny. \dm{}
Zjedz tę grudkę, to antidotum na to, co się zaraz stanie.
\end{dialogue}

Każdy z potworów ma jakąś podstawową moc. Psychit na przykład potrafi wytworzyć swoim ciałem dowolne lekarstwo, narkotyk, truciznę, czy halucynogen. 
Na niego samego jednak, jego substancje nigdy nie działają.
Nośnikiem tych związków jest czarny smar nazywany potocznie ektoplazmą.

Psychit najeżył się i wypuścił z kolców na rękach czarną substancję, która natychmiast zmieszała się z wodą.
Dumny ze swojego dzieła był zaskoczony, że nawet sikawka zaczęła pompować jego ektoplazmę, rozpylając czarne chmury, brudząc nieskazitelnie białe chodniki.

Z wrednym uśmiechem obserwował istoty nabierające czarną wodę do kubłów, pijące ją z trzęsącymi się mackami i przerażeniem w oczach.
Nie odważyliby się zaprotestować, nawet gdyby kazano im pić płynne żelazo.

\divider{}

Potwór i człowiek wygodnie usiedli na końcach strzałek statku i obserwowali rozwój wydarzeń.
Pierwsi zareagowali malarze. Pędzlami malowali i tak doskonale białe substancje. 
Rozlewali gwałtownie farby na siebie nawzajem i wymachiwali pędzlami w powietrzu, chcąc widocznie je także pomalować.

Wkrótce wszyscy obywatele wewnętrznie spanikowali. Chodzili drżącymi krokami, zakrywali oczy, ale nikt nie odważył się zejść z wytyczonych linii.
Nikt nie zwolnił, nikt nie wyszedł poza binarność.
Pomimo że w ich głowach i przed oczyma tańczyły wszystkie kolory wszechświata, wytworzone jednym z najsilniejszych halucynogenów
jakie Psychit potrafił wyprodukować, nikt nie dał tego po sobie poznać.

Wyglądało to śmiesznie, dopóki jedna osoba nie wykłuła sobie oczu, i szczęśliwa dalej po omacku nie szła w kolejce środkiem chodnika.
Kilkoro pozostałych poszło w jej ślady, zanim Psychit nie podbiegł i nie zalepił wszystkim wkoło oczu swoją ektoplazmą w przeciwbólowym trybie, oślepiając ich jedynie tymczasowo, 
żeby nie robili sobie krzywdy.

Środek rozpylony w chmurze podziałał na wszystkich, ale nie chcieli oni reagować na jego efekty.
Jedni próbowali pomalować niebo, więc wygląda na to że wierzyli w widziane kolory.
A z kolei ci, którzy wyłupili sobie oczy, prawdopodobnie byli świadomi, że wszystko powstaje tylko w ich głowach.
Nie mogli wewnętrznie znieść nawet tego.

Ziemowit w milczeniu zastanawiał się, czy nie posunęli się za daleko.
Być może wyrządzili nieświadomym istotom niepotrzebną krzywdę.
Chociaż z drugiej strony, wytwór Psychita niedługo przestanie działać i wszystko wróci do normy.
Chodniki zostaną pokryte kolejną warstwą bieli, a fontanna na nowo odmalowana.
Statek się weźmie na hol i nie pozostanie tu po nich żaden ślad.
Nawet wspomnienia obywateli zostaną szybko wyplewione przez wiecznie wiszący topór śmierci za każde odstępstwo od normy.

A Psychit? On smakował farbę. 
Ze wszystkich dziwności wszechświata, ta planeta nawet go nie ruszyła.
Ziemowit wziął trochę czerni na palec i pomyślał, że sporo musi się jeszcze o życiu nauczyć.

Fuj, obrzydlistwo.







