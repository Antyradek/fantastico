\chapter{Zatruty Skarb} 

\info{Dwójka znajomych wysyła łazika do wnętrza opuszczonej kopalni, aby znaleźć rzekomo ukryte tam złoto, skradzione przez Nazistów.}{Horror}{14 000}{\undefineduniversum{}}

\begincapital{S}ztabki złota.
Pełnowymiarowe. Każda o wartości kilku milionów złotych. Nie trzeba płacić od nich podatku. I nie trzeba zanosić do muzeum.
Wystarczy sprzedać w darknecie za Monero.
Gruby widział już przed oczami skarb, który niechybnie uda im się za chwilę wydobyć z ciemnych podziemi opuszczonej kopalni.
Już miał przygotowany w głowie plan upłynnienia zdobyczy.
\dialog{Dlaczego akurat kopalnia, i dlaczego akurat ta?}{Leszek się zapytał.}{Chyba większa szansa na znalezienie Złotego Pociągu na Helu.}{}
\dialog{Gdybyś miał do ukrycia złoto, to gdzie byś je ukrył, jak nie w opuszczonej kopalni, której połowa jest wiecznie zalana dwutlenkiem węgla?}{Gruby przekonywał.}{}
\dialog{W opuszczonej kopalni, której połowa jest wiecznie zalana dwutlenkiem węgla na terenie III Rzeszy?}{odpowiedział.}{}
\dialog{Ty się lepiej skup na rozstawianiu sprzętu.}{}{}

Odkrywcy przytargali ze sobą szpulę przewodu, laptopa i coś, co wyglądało jak pęk kabli z dna szuflady na kable.
To była gwarancja niedołączenia do kolejnych ofiar, które utonęły w zatrutym labiryncie podziemi.
\dialog{Witamy w roku \the\year, przyszłość jest dziś.}{Gruby usiadł na środku wejściowej komnaty i otworzył komputer.}{Pospiesz się z tym łazikiem, nie mamy całej nocy.}
Pod koniec wojny Niemcy ewakuowali się z kraju, zabierając ze sobą co cenniejsze rzeczy.
Zestaw złotych sztabek zniknął wtedy bez śladu, jak wiele innych skarbów.
Historycy potwierdzają, że te sztabki, jak i masa cenności, nigdy nie dostały się na granicę.

\dialog{Droga idzie prosto tym chodnikiem}{odkrywca skarbów kontynuował.}{Następnie zaczyna się lekki spadek i... odwrotnie wsadzasz ten akumulator!}
\dialog{Dobrze wsadzam, zobacz}{Leszek dawał wymówki.}{}
\dialog{Dopiero jak ci o tym powiedziałem.}{Rozejrzał się po komorze, z której wychodziło kilka przejść.}{Pod nami jest dolny poziom, na którym poziom tlenu jest zabójczo niski. Bez aparatu tlenowego to pewna śmierć. No chyba, że wcale się tam nie... czerwony do czerwonego, a czarny do czarnego!}
\dialog{Przecież wiem, jak podłączyć silniki.}{}{}
\dialog{Nie wiesz.}{Gruby poruszał kursorem myszy na komputerze, jakby chciał przyspieszyć czas.}{Wyziewy dwutlenku węgla robią robotę. Dlatego ta kopalnia została porzucona jeszcze przed wojną. Wielu śmiałków poległo, błądząc w tym labiryncie.}
\dialog{Ktoś tu zginął? Dlaczego mi nie powiedziałeś?}{Leszek postawił złożonego łazika na ziemi.}{}
\dialog{Bo byś nie poszedł. Poza tym źle przykręciłeś pokrywę.}{}{}
\dialog{Dobrze jest!}{}{}
\dialog{Zaraz zobaczymy.}{Gruby podłączył szpulę przewodu do komputera, a łazik zaświecił kolorowymi diodami. Na ekranie laptopa zobaczył nogi Leszka w rubinowym świetle podczerwieni.}{Dobrze jest.}

Już na wejściu do kopalni zawartość tlenu była dwa procent niższa od przeciętnej.
Odkrywca popchnął małego joysticka, a samochodzik mozolnie ruszył w stronę ciemnego korytarza.
Leszek znudzony rozwijał szpulę przewodu, który łazik ciągnął za sobą w celu zachowania komunikacji z nimi.
Na ekranie korytarz prowadził na razie prosto, a mały odkrywca prowadził się równo. Ostatecznie urządzenie zniknęło im ze światła latarek.

Około 50 m dalej, zgodnie z naszkicowaną w zeszycie Grubego mapą, chodnik schodził w dół i tam zaczynała się zabójcza strefa.
Zdalny kierowca po krótkim namyśle kontynuował wjazd do strefy śmierci. Wskaźnik poziomu tlenu zaczął szybko maleć i przekroczył zabójczą dla człowieka wartość.
Jednak niewzruszony robot dzielnie pchał naprzód.

Obaj gapili się w szary ekranik, jakby zaraz miał na nich wyskoczyć jakiś podziemny zombie.
Nie spodziewali się jednak, że stanie się dokładnie odwrotnie. To oni wyskoczyli na podziemnego zombie.

Ciało leżało tutaj od jakiegoś czasu. Jeden z nieszczęśników, który tak jak oni, najwyraźniej chciał się wzbogacić.
Zabłądził w labiryncie korytarzy, a miał ograniczony zapas tlenu.
\dialog{Ciekawe, czy znalazł cokolwiek ciekawego}{mruknął Gruby, nie zwalniając tępa. Objechał trupa szerokim łukiem, jakby robotowi zadawanie się ze zmarłymi robiło jakąś różnice.}{}
\dialog{Dobrze że my tak nie skończymy}{Leszek odpowiedział.}{}
\dialog{No chyba, że łazik się gdzieś zawiesi i trzeba będzie po niego pójść.}{Zamrugał z niedowierzaniem.}{O właśnie tak jak teraz.}
Robot jechał po równym i nagle przewrócił się na bok jak nieporadny żółw.
\dialog{No to by było na tyle.}{Leszek westchnął.}{Może dałoby się go wciągnąć z powrotem kablem.}
\dialog{Wciągniemy go z powrotem Leszkiem. Zakładaj aparat tlenowy i leć.}{}{}
\dialog{Co? Żeby skończyć obok jak ten trup? Dlaczego, ja? A kto sterował?}{}{}
\dialog{Jesteś mi winien za ostatni raz. Może mam ci przypomnieć, ile był warty tamten dron?}{Gruby podniósł głos.}{Po prostu trzymaj się kabla i postaw go z powrotem na nogi. Co za problem. Widziałeś drogę. Poza tym to twój zestaw.}

Leszek mozolnie ubrał ciężki zestaw do nurkowania, który wygrzebał gdzieś z piwnicy.
Poza środowiskiem wodnym olbrzymia butla przechylała go do tyłu, a paski wrzynały się w ramiona.
Nie potrzebował aż tak wielkiej maszynerii do wizyt w dusznych korytarzach, ale zaletą było że nie musiał nic nowego kupować.
Zaciągnął się i sprawdził ciśnienie. Nawet przez usta poczuł zapach zdezelowanego kompresora z garażu.
Włączył latarkę i nie oglądając się za siebie, poszedł śladami łazika.

Tymczasem Gruby został sam. Gdy kroki Leszka ucichły, słyszał tylko własny oddech i wiatraczek laptopa.
Pomyślał o wszechobecnym kurzu.
Przypatrzył się uważnie każdemu z chodników, które wychodziły z sali. Większość z nich nie schodziła do innych poziomów, była krótka i kończyła się ślepo.
Gdyby kopalnia znajdowała się bliżej miasta, mógłby znaleźć tu kupy śmieci i bezdomnych.
Może któryś z nich znalazłby skarb. Podobno denaturat daje supermoce.
Po ile chodzi złoto na skupie?

Usłyszał kroki i zobaczył światło latarki. Leszek wrócił, sapiąc z aparatu nurkowego.
\dialog{No, nawet szybko ci poszło.}{Gruby rzadko był pod wrażeniem.}{Widzę, że mogę jechać dalej.}
\dialog{Co?}{Leszek wypluł ustnik.}{}
\dialog{Dzięki, że postawiłeś łazika do pionu}{wycedził przez zęby.}{Myślałem, że będę musiał zbudować nowego.}
\dialog{Ale ja...}{Podszedł i zajrzał na ekranik, na którym widać było dalszą podróż robocika po ciemnym korytarzu.}{Kabel był urwany.}
\dialog{Kabel? Urwany? Przecież widzę, że działa.}{}{}
\dialog{Przysięgam. Szedłem prosto. Potem w równią pochyłą w dół. Znalazłem trupa, a potem kabel się urywał.}{}{}
\dialog{Chyba to twoje ustrojstwo przecieka i nawdychałeś się gazów.}{Gruby spojrzał na Leszka jak na idiotę.}{Widziałem cię na ekranie. Podszedłeś do robota przekręciłeś go z powrotem, a potem odszedłeś.}
Leszek zmarszczył brwi, jakby robił kalkulacje w głowie.
\dialog{Sprawdziłeś, czy ten trup nie niósł naszego skarbu?}{Gruby wypalił.}{}

Nurek podziemi popatrzył ze wstydem i pokiwał głową.
To było chyba bezczeszczenie zwłok, ale ciekawość wzięła górę.
Leszek pamiętał, jak pochylił się nad nim, i zbadał z każdej strony.
Był to młody chłopak, wygląda że zmarł niedawno. Pewnie jeszcze nie zaczęli go szukać.
Miał całkiem drogi sprzęt. Szkoda że niewystarczająco rozumu.
Jednak pójście tam z aparatem nurkowym najmądrzejsze także nie było.
W ręku trzymał nabazgraną mapę, wycinek tej samej co Leszek miał ze sobą.
Nie było tam wielkiego znaku ,,X'' na skarbie.
Widocznie nieszczęśnik wracał z nieudanych poszukiwań.
Tak blisko wyjścia. Straszna śmierć.
\dialog{Tylko...}{Zastanowił się, czy mówić, że aparat tlenowy nieszczęśnika nadal miał ciśnienie.}{...nic takiego.}
Zdjął przyrząd, zakręcił butlę i usiadł z powrotem, obracając w ręku ustnik.
Czy faktycznie urządzenie przeciekało dwutlenek węgla, a jego zamroczonemu umysłowi zdawało się, że nie zrobił tego, co zrobił?

W każdym razie misja trwała dalej.
Korytarz kręcił się w różne strony. Szpula odwijała się, pociągana \emph{czymś}.
Na ekranie podpory sufitu wykrzywiały się w różne strony.
Obu przeszło przez myśl, że zapomnieli o ryzyku zawałów. Dobrze że użycie robota rozwiązuje i tę kwestię.
Jednak chyba nie będą przekopywać się przez gruzy, aby odkopać jego szczątki.

Korytarz kończył się rozstajami.
Pojechali w lewo i zamarli.
W ciemności majaczyło światełko.
Czy ktoś tam był? A może to zgubiona latarka tego nieszczęśnika?
Współczesne konstrukcje z wydajnych diod i kilka dni mogą się świecić.

Gruby zgasił i zapalił latarkę. Światełko na ekranie zrobiło to samo.
Oboje się zaśmiali, strachliwie.
Najwyraźniej przez przypadek zawrócili i kierowali się do korytarzy na niższym piętrze dokładnie pod sobą.
Gdzieś w okolicy musiała być jakaś dziura, przez którą wpadało światło.

Gruby polecił Leszkowi jechać prosto, a sam pokręcił się po okolicznych wejściach, świecąc tu i ówdzie.
\dialog{Mów jak będzie jaśniejsza}{polecił.}{Może dało by się wyciągnąć robota gdzieś tędy i sprawdzić ten twój kabel przy okazji.}
Leszek wziął kontrolę w palce i jechał naprzód, wpatrując się w punkcik na ekranie.
Kabel odwijał się ze szpuli z podobną prędkością.
Gruby zaglądał w każdą szparę wokół, ale światełko na złość nie zmieniało jasności.
\dialog{I jak tam, coś się zmienia?}{zapytał.}{}
Jednak nie otrzymał odpowiedzi.
Zajrzał przez ramię na ekran, a potem na Leszka, a potem na ekran. I nie rozumiał, czemu Leszek miał taką grozę wypisaną na twarzy.

Na ekranie widniała stojąca w kącie latarnia. Mały płomyczek jasno tańczył w środku, rzucając złowieszcze cienie przez pękniętą szybkę.
Gruby prześledził kierunek wzroku Leszka, który wcale nie patrzył na ogień, a na miernik w kącie ekranu.
Miernik tlenu.
Wskazywał całe 3\%.
W takich warunkach żaden ogień nie mógł się utrzymać, oboje to wiedzieli.
Ale o ile Gruby stwierdził prawdopodobny błąd miernika, o tyle Leszek, doświadczony już przez wcześniejszą wyprawę, miał zupełnie inne przypuszczenia.

Stali tak długi czas, aż w końcu Leszek wypalił.
\dialog{A gdyby tak to bliżej zbadać? Użyję szczypiec i weźmiemy ten fenomen do nas.}{}{}
\dialog{Daj spokój. Przyjechaliśmy po skarb, szczypce są dla zbierania skarbów, a nie jakichś śmieci.}{Gruby ponaglił.}{}
\dialog{Ogień przeczący prawom fizyki nie jest skarbem?}{}{}
\dialog{Chyba wizyta tam na dole ci wypaliła styki, tak samo jak temu czujnikowi.}{Przewrócił oczyma.}{Jedziemy dalej.}

Ale nie pojechali dalej, ponieważ nagle obiekt ich rozmów tajemniczo podniósł się w górę i zniknął za kadrem.
Leszek przekręcił robota w bok, ale nic nie zobaczył.
Zaraz potem przekręcił w drugi bok, ale mignęła im jedynie jakaś postać w poszarpanych szatach znikająca za rogiem razem z latarnią.

\dialog{Nie ma szansy, aby ktokolwiek przeżył w tym środowisku.}{Leszek był blady.}{}
\dialog{Może to ten trup, którego wcześniej widzieliśmy, nagle ożył?}{Gruby zażartował, ale cofnął słowa, widząc Leszka na skraju paniki.}{}

Gruby postanowił jechać dalej. Co prawda był trochę niespokojny, że ten tajemniczy ktoś ukradnie mu łazika, ale skoro jeszcze tego nie zrobił, to może ma inne plany.
Natomiast zapewne także szuka skarbu. I warto by było go znaleźć przed nim.
Kazał Leszkowi jechać prosto i nie śledzić tajemniczego jegomościa, na co pilot trochę odetchnął z ulgą.
Przycisnął prąd do dechy, aby jak najszybciej zapomnieć o tym dziwnym zjawisku.

Droga była prosta, a na końcu wydawało się że widzą jakąś poświatę.
Leszek przycisnął twarz do ekranu, ale rozmazany obraz stał się jeszcze bardziej rozmazany.
Usłyszeli także coś w tunelu obok, jakby szuranie i jęczenie, zmieniało się wraz z ruchami gałki.
Do tego w ciemności obok małe światełko majaczyło na wysokości podłogi. Łazik.
Musieli zrobić pętlę w tym przeklętym labiryncie.
Chociaż Leszkowi różnica wysokości trochę się nie zgadzała.

Gruby wstał i poszedł w jęczącą niewydolnością motorów ciemność przynieść łazika.
Na ekranie ciemna sylwetka zrobiła to samo.
\dialog{Odepnę kabel, a ty go zwijaj.}{}{}
Leszek posłusznie podszedł do szpuli i zaczął zwijać.
Kręcenie powodowało mu podejrzanie mało oporu.
Aż w końcu dowiedział się dlaczego tak było, gdy urwany koniec drutu zamajtał mu w ręce.
A do tego ilość skręconego przewodu mniej więcej zgadzała się z odległością, jaką wtedy przeszedł z aparatem nurkowym.
To by znaczyło, że jednak nie miał przewidzeń i faktycznie kabel był urwany.
Próbował sobie przypomnieć, czy pamiętał czy szpula dalej się odwijała, gdy jechali dalej.

Doskoczył ekranu, ale łazik nadal wyświetlał obraz.
Nie było tam żadnego modułu bezprzewodowego, żeby mógł to robić. Sam to konstruował, to wiedział.
Na wideo zobaczył sylwetkę siebie samego, z ruchu ruchomej kamery jakby ktoś niósł filmującego robota w rękach.
Popatrzył w kierunku tego tunelu na Grubego, ale to nie Gruby tam szedł z robotem.

Zobaczył postać starego dziada w poszarpanym odzieniu.
Jego oczy były ślepe, a twarz wykrzywiona w horrorze.
Z ust wydobywały się dźwięki przypominające pracę silnika elektrycznego.
W ręku niezgrabnie trzymał łazika.

Starzec stanął w drzwiach, położył delikatnie nieznaną mu technologię na ziemi przed Leszkiem, spojrzał na niego niewidzącymi oczyma, po czym odwrócił się i poszedł z powrotem.
Wkrótce nastała absolutna cisza, zakłócona jedynie głośnym, biciem serca Leszka.

Wtem zerwał się, złapał w garść wszystko, co leżało wokół, i wybiegł z kopalni, potykając się o kabel kilka razy.
Był środek nocy. 
Nie zdał sobie sprawy, że powinien był być już ranek.
Dobiegł do swojego zaparkowanego obok samochodu, wrzucił wszystko na tylne siedzenie i odjechał.
Pędził ze 100 km/h w terenie zabudowanym, przejechał 3 wioski zanim stwierdził, że może lepiej zastanowić się czy nie wrócić po swojego znajomego.

Zjechał na pobocze, aby przemyśleć sytuację i powody dla których to on jest tutaj, a nie w kopalni, zamiast żeby było na odwrót.
Może wszystko mu się śni, a to są jakieś halucynacje z niedotlenienia?
Leży tam właśnie obok tego trupa i umiera.
Czy powinien za chwilę zobaczyć światełko w tunelu?

Światełko zobaczył, ale nie w korytarzu śmierci, a na siedzeniu obok.
Popękana latarenka stała oparta o oparcie. Jej płomień był gładki i niewzruszony.
