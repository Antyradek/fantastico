\chapter{Uniwersalny tort}

\info{W czasie wesela do tortu wpada mały, uniwersalny meteoryt. Gdy zakażony uniwersalnością tort zostaje zjedzony, zaczynają się dziać dziwaczne rzeczy.
Tematem pracy jest obraz Salvadora Dalíego ,,Gala desnuda mirando el mar que a 18 metros aparece el presidente Lincoln'' (Gala podziwiająca morze, w odległości 18 metrów zmieniająca się w Lincolna).}

Wesele rozkręciło się już na dobre. 
Upici goście powoli przestali zwracać uwagę na dwie, wielkie istoty z kosmosu, przemykające zwinnie między stołami.
To był doskonały czas na najważniejsze wydarzenie wieczoru --- tort.
Gigantyczna góra lukru wjechała na środek sali w akompaniamencie fajerwerków i gradu brokatowego konfetti. 
Malowany złotem lukier pokryty brokatem mienił się w świetle świec, półprzezroczyste żelki rozpraszały tęczowo światło, a w perłowych szlaczkach można się było przejrzeć.
Było to wszystko na tyle mocno przesadzone, że jedna z rakiet wpadła do tortu i wysadziła w powietrze ostatnie, dziesiąte piętro.
Malutcy pan i panna młoda, rozerwani, polecieli w cztery strony świata.
Nie żeby ktoś to zauważył, ludzie byli zajęci tratowaniem się, aby dorwać jak najwięcej przepysznego deseru. 
Po kilku chwilach, została już tylko wylizany patera. To musiał być bardzo dobry tort.

Do czerwonego kosmity mało osób się zbliżało, z wyjątkiem panny młodej i jej siostry.

\ds{} Ty musisz być jednym z tych sławnych potworów, tyle nam o was opowiadali, to zaszczyt was spotkać \dm{} siostra panny młodej zagadała do Pyrroqua. \dm{}
Jestem Magda, a to mój mały synek. Tymek. \de{}

\ds{} Miło mi poznać rodzinę Anastazji. Jestem... \dm{} Pyrroq wydał z siebie niezidentyfikowany dźwięk, przypominający eksplozję planety.
Potem wyszczerzył ostre ząbki i wyciągnął pazurzastą łapę, z której wysunęły się trzy macki, chcące uścisnąć rękę Magdzie. Odruchowo cofnęła wyciągniętą prawicę. \de{}

Mojmira, stojąca obok wtrąciła się.

\ds{} Pyrroq, nazywamy go Pyrroq. Musisz mu wybaczyć, ma bardzo trudny i wredny charakter, ale spokojnie, nie jest groźny. 
Pyrroq, przestań proszę być sobą, zachowuj się po ludzku, nie rób mi wstydu.
\dm{} Uśmiechnęła się. \dm{}
Wybacz, ja jestem Mojmira, przyjechałam tu z tymi dwoma potworami, pilnuję ich, żeby nie zrobili niczego głupiego, jak teraz. 
A ten drugi, zielony, co tak się trzyma z boku, to Chronos. \de{}

\ds{} Więc ty także jesteś spoza Ziemi? Mieszkasz z nimi na jakiejś planecie? \dm{} Magda zapytała. \dm{} Nie przewidywałam, że kosmici będą tak podobni do ludzi. \de{}

\ds{} Dobrze przewidywałaś. \dm{} Mojmira się zaśmiała. \dm{} Ja się urodziłam na Turandii, tej wysepce na Atlantyku. Jestem w stu procentach człowiekiem i Ziemianinem.
\dm{} Zauważyła, że Magda wyraźnie się rozluźniła. \dm{} W kosmosie wcale nie ma tak wielu myślących istot, jak to mówią filmy. 
Najbliższe od nas, pozaziemskie życie, jest bardzo daleko, poza naszą gromadą galaktyk. A i tak to jakieś niedorobione szczuropodobne cosie, głupsze od tych ziemskich szczurów.
Jednak niestety, żadne istoty nie są zielonymi ludzikami z wielkimi oczyma.
\de{}

\ds{} Więc moja siostra, Anastazja, was zaprosiła. Od jak dawna znacie się z panną młodą? Jak to się stało, że poznała tak ekscentryczne osobistości? Musi to być bardzo nietypowa znajomość. \de{} 

\ds{} Bardzo standardowo. Uratowaliśmy jej życie z płonącego pociągu, jak wielu innym osobom. 
Ci ,,obrońcy'', jak sami siebie nazywają we własnym języku, są dostatecznie rozwinięci technologicznie i kulturowo, aby móc bezproblemowo podróżować po wszystkich galaktykach.
No i zdarzyło się, że byli tu na Ziemi i akurat uratowali twojej siostrze życie.
A że wtedy i ja byłam z nimi, to panna młoda wzięła sobie za życiowy cel mi się jakoś odwdzięczyć. 
I tak oto zaprosiła mnie, Pyrroqua i Chronosa na swoje wesele, nie zważając na nic, strasząc przy okazji większość gości.\de{}

\ds{} Więc to wy ją uratowaliście? Nic mi nie mówiła. \de{}

\ds{} Oficjalna wersja to chyba wybuch pobliskiego wodociągu, który zgasił pożar. 
Znaczy, wodociąg rzeczywiście zgasił pożar, ale nie sam z siebie.
\dm{} Mojmira westchnęła. \dm{}
Wiesz, co by się działo, gdyby tak po prostu, wszystkim nagle objawić prawdę.
Cześć, jesteśmy potężną, ludzko-potworową organizacją kosmitów, będziemy wam zwalczać zło.

\ds{} To pewnie dlatego Anastazja powiedziała gościom, że zaprosi dwóch klaunów w kostiumach. Ciężka dola superbohaterów. Ale nadal wam trochę zazdroszczę.\de{}

\ds{} Jestem taką samą superbohaterką, jak ty. Właściwie, to w moim przypadku, sama przyszłam do potworów, a nie oni do mnie.
Odkryłam, że Turandia to sztuczna wyspa, osiedlili tam garstkę ludzi w 1818 roku, stylizując to na katastrofę statku.
Celem miało być stworzenie szczęśliwej cywilizacji od nowa. Moja wścibskość doprowadziła mnie do prawdy i nawiązałam z nimi kontakt.
Wyszło na to, że ten eksperyment dawno był zapomniany i od lat nikt się nim nie interesował!
Ale zamiast wyczyścić mi mózg, pogratulowali i pozwolili dołączyć. 
Okazało się, że w swoim świecie prowadzą całą akademię ludzi. Zrzeszają wiele osób, podobnych do mnie, bo po prostu chcieli mieć pomoc i podzielić się potęgą.
Dali nam technologię i kazali razem ratować światy. Czasami wolałabym nadal siedzieć z rodzicami w chatce na Turandii.

Mały Tymek podszedł do Pyrroqua. Miał małe, niebieskie oczka, blond włosy i słodki garniturek.
Buzia umazana tortem.
W ręku trzymał wymięty rysunek Pyrroqua.

\ds{} Jesteś smokiem? \dm{} zapytał \dm{} weźmiesz mnie do na lot do smoczego królestwa?\de{}

\ds{} Pewnie. \dm{} Pyrroq odwrócił się i nachylił. \dm{} Wskakuj. \de{}

Tymek spróbował wejść, po jego kolcach na grzbiet, ale jak tylko złapał się półkolistego noża, zaraz przeciął sobie rączkę. 
Krew i łzy pociekły na podłogę, a płacz poszedł w nocne niebo. 

\ds{} Co ty zrobiłeś mojemu dziecku! \dm{} Magda dopadła potwora, żeby okładać go torebką, ale zaraz się cofnęła ze strachu i złapała płaczącego synka. \de{}

\ds{} Ja? Nic. To twoja wina, że masz głupie dziecko, które nazywa potwora smokiem i łapie się jego ostrych kolców. 
I jeszcze traktuje mnie, jak jakiegoś konia, może specjalnie siodło mam założyć? \de{}

Mojmira złapała rączkę Tymka, wyciągnęła kamienną buteleczkę i psiknęła kilka razy na ranę, która zaraz zniknęła.

\ds{} Uwierz mi, ale on nie jest aż taki zły, na jakiego wygląda. Jest tylko wredny. Nie zabiłby twojego dziecka, przysięgam. \dm{} Mojmira uspokajała Magdę i Tymka. \dm{}
Wiedział, że mam trochę antyczasu i zaraz uleczę Tymka i nic się nie stanie, jak zatnie się o kolce. \de{}

Magda jednak już zapomniała o cierpieniu swojego dziecka i utkwiła wzrok w magicznie leczącej rany, kamiennej buteleczce.

\ds{} Co to za magia? Gdzie można to kupić? Tymek potrzebowałby tego codziennie, to przebojowe dziecko, a ja wydaję fortunę na plastry.
Co tam, każdy tego potrzebowałby, będziecie to sprzedawać? Sprzedawajcie.\de{}

\ds{} Tego nie można kupić, to trzeba wytworzyć. \dm{} Mojmira zaczęła tłumaczyć. \dm{} To jest antyczas. Środek który, gdy się go dotknie, odwraca upływ czasu. 
Tak naprawdę, to nie uleczyłam rany Tymka, a cofnęłam jego dłoń o kilka minut wstecz. Spokojnie, to bardzo małe stężenie.
Powstaje poprzez dodanie antymaterii do płynnego czasu, który z kolei postarza przedmioty przy kontakcie. 
Jest w kamiennej buteleczce, właśnie dlatego, że kamień jest stary i długo może być odmładzany, a szklana butelka zaraz zamieniłaby się w piasek.
A pewnie zapytasz, jak powstaje ten dodatni, płynny czas? 
\dm{} Odwróciła się do Chronosa, który zabawiał jakichś gości, staniem na ogonie. \dm{} 
Każdy z potworów ma jakąś moc, Pyrroq potrafi wytworzyć bomby i zdalnie je eksplodować, a Chronos... cóż. Teoretycznie potrafi sterować czasem.
\dm{} Nachyliła się nad ucho Magdy. \dm{} A w praktyce jest tak przerażony używaniem swojej mocy, że nigdy tego nie robi.
Tyle się nasłyszał o paradoksach, rozerwaniach czasoprzestrzeni, czarnych dziurach, że jest sparaliżowany tym kompletnie.
Teoretycznie więc, najsilniejszy ze wszystkich, w praktyce najsłabszy.
Używa jedynie standardowych mocy: lasera i pola siłowego.
Ale jak się go przymusi, żeby jakoś spróbował sterować cykaniem zegara, to po nieudanych próbach, zostaje taka zielona substancja, płynny czas. \de{}

\ds{} Czyli to jest taka jakby jego krew? Krew potwora? \de{}

\ds{} Potwory nie mają krwi. Bardziej to jako pot. Pot potwora. Co się wykrzywiasz, chciałaś wiedzieć, to wyjaśniam. \dm{} Zwróciła się do Tymka. \dm{}
A Chronos też potrafi latać, jak nadal chcesz, to z chęcią zabierze cię do królestwa smoków. I specjalnie założy siodło. \de{}

\ds{} Zielony smok to nie smok, ja chcę czerwoneeeegoooo. \de{}

\divider{}

Sala bankietowa nie miała dachu, ciepła noc i lekki wiaterek orzeźwiały gości.
Chmury zaczęły się przerzedzać i niedługo miał wyjść księżyc.
Mojmira, po spróbowaniu wszystkich dań, zaczęła żałować, że nie dopchała się do tortu, ale pocieszała się tym, 
że na pewno nie był lepszy, niż niektóre kosmiczne przysmaki, jakie miała okazję skosztować na swoich misjach.
Wtedy też zauważyła, jak ludzie zaczynają się dziwnie rozglądać na boki. 
Podszedł do niej Chronos.

\ds{} Mojmira, czy to normalne u ludzi? Widzę, że wielu na sali nabrało jakichś niespokojności. Co się z wami dzieje? Co to za anormalność? Wytłumacz. \de{}

\ds{} Spokojnie, w sumie. Na pewno nic groźnego. Ja nie czuję nic dziwnego. A czy ty coś czujesz? \de{}

\ds{} Wszystko czuję. Patrz, połowa ludzi nie umie poprawnie tańczyć, przecież dla ludzi tańcowalność jest wbudowana w mózg.
To pewno owoce klonu, sprawdź, czy nie są roślinami podszywającymi się za ludzi... \de{}

\ds{} Skończ. \de{}

\ds{} Księżyc wychodzi zza niebobrudów, w telepogodoprzepowiadarce było to przepowiedziane dopiero na za dwie godziny, czy to dlatego, że za horyzontem szaleje jakieś burzodziwactwo modyfikujące pogodę? Musimy to zweryfikować. \de{}

\ds{} Proszę. \de{}

\ds{} Albo celowo źle nam oznajmili pogodę, pewnie mają w tym jakąś mordochęć. Lecimy to potwierdzić. \de{}

\ds{} Przestań. \de{}

\ds{} Racja. Za dużo teorii spiskowych. Pewnie po prostu ludziopaliwo jest zatrute. Nie dotykałaś, mam nadzieję, niczego co leży na stołach? Ani nie wąchałaś? Ani nawet nie podchodziłaś do nich?
W rakiecie mamy sporo bezpiecznych ludzkogębowkładów, specjalnie aby nie jeść niezweryfikowanego. \de{}

\ds{} Dość. 

\ds{} No więc dlaczego posiadasz brak wiedzy na temat tego, co się tu aktualnie dzieje? 
Agent ALOPP musi być zawsze czujny, niespodziankośmierć czyha wszędzie. Może ktoś wypuścił coś do powietrza? \de{}

\ds{} Jedzenie nie jest zatrute. Zjadłam po trochu wszystkiego, co tu leży i nic mi nie jest. \de{}

\ds{} To koniec. Trujomateria o zwolnionym zapłonie, to ostatnia godzina twojej egzystencji, zaraz skrócę ci męki, nadstaw tętnicę. \de{}

\ds{} Chronos! Koniec! Czemu ze wszystkich potworów, akurat wy dwaj musieliście ją uratować? Nie mogę was znieść. Nic poważnego się nie dzieje. 
Jeden panikuje, drugi kroi dzieci.
Ludzie są niespokojni, tak, ale przecież jedli to samo, co ja poza...
\dm{} Mojmira odwróciła się w kierunku środka sali \dm{} tortem. \de{}

\ds{} Zatruty tort, klasyczne. \de{}

\ds{} Nie śmiej się, to poważne. To mogłoby się zdarzyć, na sali jest wiele ważnych osobistości. Doskonałe miejsce na zamach. \de{}

\ds{} A dlaczego dowiadujemy się tego godzinę po tym, jak ów tort zniknął w jedzenioworkach ludzi, których masz za zadanie chronić? Słucham? 
Czemu od początku nie spacerujesz z omnimetrem, nie testujesz wszystkiego pod kątem każdej śmiercionośności? \de{}

\ds{} Nie spodziewałam się, zaprosili nas jako gości, uznałam że wszystkiego pilnują. Poza tym, ten tort tak szybko zniknął, nawet nie było się jak mu przyjrzeć. 
I hej, jestem od niedawna w ALOPP, jeszcze nie mam doświadczenia. \de{}

\ds{} Musimy ewakuować ludzi, brudny tortopodpieracz nadal emanuje śmiercią. To pewnie trucizna lotna. \de{}

\ds{} To się nazywa patera. Naucz się polskiego. Torty podaje się na paterach! Bycie kulturalnym człowiekiem polega na rozróżnianiu wszys... \de{}

\ds{} Patera, ciastobaza, nóżkotalerz, podcukrogrzybek, deserotrzymadło. Co za różnica. Nie znam się na sztućcologii, potwory nie używają paterów. \de{}

\ds{} Paterek. \de{}

\ds{} Idź już. \de{}

Mojmira wyciągnęła omnimetr i jeździła nim po resztkach ciasta, ale nie wykrył on żadnej trucizny, poza nieprawdopodobną ilością cukru.
Jednak to nie uspokoiło wcale Chronosa. Pyrroq się zaciekawił.

\ds{} Uniwersalność. Sprawdzaj zawsze uniwersalność. O ile trucizny są tylko na ludzi, to uniwersalność jest łatwozabijaczem dla całej Ziemi. \de{}

Wyciągnęła miernik uniwersalności i wypuściła zaraz z ręki, gdy wskazał wartości spoza skali.

\ds{} Uniwersalność, więc tak zginiemy. Zawsze to podejrzewałem. To koniec. Nie. To gorzej. Nieprawidłowomateria, spoza naszego świata, jest obok nas i w ludziach wokół.
Sztuczny wytwór, łamiący dowolne prawa fizyki, znowu trafił na Ludziokulkę.
Cokolwiek może się zdarzyć, wyrosną wszystkim dodatkowe kończyny, a potem te kończyny zmienią się w koniczyny, rzucą się na nas i przywołają armię nożyczek, która wytnie wszystkie okoliczne drogopokazywacze w śnieżynki. 
A te z kolei zaczną się obracać, jak dyskopiły i założą swój własny biznes tartaków, zniszczy on gospodarkę kraju i zmusi ludzi do rzucenia się z głodu w ocean, ale nie będą mogli wodoodetchnąć, 
gdyż podmorskie lawapluje, zasilane wiatrem z krów, które zjadły za dużo koniczyny, podniosą dno morskie i... \de{}

\ds{} Czy możesz wreszcie przestać? \de{}

\ds{} Przestać? Ale przecież tak właśnie działa uniwersalność. Magia, dowolnobyt, nieposkromienie, łamanie praw fizyki. Wszystko. Wszystko. Wszystko. \de{}

\ds{} Chronos ten jeden raz ma rację, Mojmiro. \dm{} Pyrroq był podejrzanie spokojny. \dm{} Uniwersalność nie ma granic. Żadnych. 
Już pominę, że sam zjadłem całkiem sporo tego tortu. Cóż nie sądziłem, że tak zginę. Ale uniwersalność to nawet zabawna śmierć, nie boję się. 
A jak i tak wszyscy zginą, to może, ach, może w końcu będę mógł bezkarnie mordować! \de{}

\ds{} Co oczołapiesz? Powiedz! \dm{} Chronos potrząsał Pyrroquiem. \de{}

\ds{} Sam nie wiem, wydawało mi się, jakby ludziom rozpraszało się lekko ubranie. A jak chciałem podejść, to wszystko wracało do normy.
Panna młoda miała dłuższą suknię, tamten gość był jakiś większy, a ten dostał wisiorek na szyi. To się działo przez chwilę, a potem zniknęło.
Lepiej sprawdź, tym miernikiem, co było źródłem. Może ktoś, czegoś w kuchni dosypał? Celowo, albo i nie. \de{}

Mojmira machała miernikiem na wszystkie strony. W kuchni nie było niczego groźnego. Był jednak cienki szlaczek na podłodze, który szedł do leżącego pod stołem, małego kamyczka.

\ds{} Och. Czy to możliwe? \dm{} Pyrroq się zastanowił. \dm{} Pamiętacie, jak do tortu wpadła rakieta? To nie była rakieta, tylko meteoryt. 
Ten właśnie kamyczek, znalazłem go w swoim kawałku... znaczy piętrze i wyrzuciłem w cholerę. \de{}

\ds{} Dotykałeś uniwersalności! Nie zbliżaj się! \dm{} Chronos schował się za Mojmirę. \de{}

\ds{} Chodźmy na zewnątrz. \dm{} Mojmira rozejrzała się po ludziach, którzy przyglądali się, zwabieni rykami Chronosa. \dm{} 
Dlaczego nie powiedziałeś? To nietypowe, meteoryty nie wpadają na Ziemi do tortów, jak muchy do zup. \de{}

\ds{} Nie? Myślałem, że jak latają po kosmosie, to trafiają w planety i nie zawsze się palą w atmosferze. Mogą wtedy wpadać w różne, dziwne miejsca. 
I że to się zdarza cały czas. \de{}

\ds{} To się zdarza bardzo rzadko, a ich prędkość rozwaliłaby cały tort, razem z paterą, stolikiem i kawałkiem podłogi. Ale to uniwersalność i nie musi słuchać się fizyki. 

Mojmira ukradła szczypce z sałatki i podniosła nimi meteoryt.
Wyglądał, jak całkowicie normalny kamyczek z tym, że stawiał dziwny opór w trakcie ruchów.
Trzeba będzie go dokładnie zbadać i wymyślić sposób na neutralizację.

\divider{}

Byli akurat na zewnątrz, nad stawem i obgadywali plan działania. 
Pyrroq formował płaskie bomby, a potem puszczał nimi kaczki, pomagając sobie mocą zdalnej manipulacji.
Na końcu eksplodował je pod wodą, w majestatycznych wybuchach.

Wtedy zza chmur wyszedł księżyc w całej okazałości, razem z gwiazdami.
Zbiegło się to z krzykami ludzi na weselu. 
Uniwersalność reagująca na światło serokulki? Chronos miał obawy.
Lecz Mojmira stwierdziła, że równie dobrze może być to zależne od fal grawitacyjnych, powstałych przy zderzeniu się jakichś czarnych dziur, akurat będących 
w pewnej odległości lat świetlnych, równej liczbie ziarenek piasku w doniczce z 
tulipanem, w jakimś domu o adresie, która jest liczbą pierwszą i na ulicy pamięci człowieka, który umarł w 1966 roku.
Chronos pogratulował Mojmirze, że zaczyna rozumieć, na czym polega uniwersalność.

Szli szybko, z powrotem do drzwi, gdy wypadła z nich panna młoda, ścigana przez mały tłumek.

\ds{} Pokaż tę suknię, jest taka piękna. \de{}

\ds{} Muszę się przyjrzeć tym świecącym diamentom. \de{}

\ds{} Czy to białe złoto oplecione platyną? \de{}

Aby uratować pannę młodą, Pyrroq zrobił to, w czym był najlepszy. Tupnął łapą, ryknął i wystraszył wszystkich z powrotem, aż jeden się przewrócił.

\ds{} Nie proszę, nie zjadaj mnie! Nie nadziewaj mojej głowy na swoje kolce! Nie chcę dołączyć to twojej kolekcji! \dm{} krzyczał. \de{}

\ds{} O czym ty mówisz? Przecież on nie mógłby niczego bolcoprzebić na swoje ciało, bo ma półkoliste kolce\dm{} zdziwiony Chronos odpowiedział. \dm{}
Zobacz, przecież nie ma żadnej kolekcji ludziomózgokulek. \de{}

\ds{} Czego? \dm{} Wystraszony gość wstawał. \dm{} Jak to nie ma tych głów! Jest wielki, brudny i śmierdzący. Pokryty zaschniętą krwią swoich ofiar.
Ma wielkie szpikulce na całym ciele, na które powbijał głowy tych, których zjadł! Jeden, dwa, trzy... sześć wyschniętych czaszek, dziecięcych.
I ten błysk śmierci w oku. Zabije mnie i całą moją rodzinę w męczarniach. \de{}

\ds{} Chyba za dużo wypiłeś. Ja nie widzę w nim nic z tego, co opisujesz. \dm{} Mojmira podeszła do niego, żeby pomóc mu wstać, ale sam wstał szybciej i uciekł. \dm{}
Czy ta uniwersalność polega na tym, że ludzie widzą... Pyrroq! \dm{} Mojmira aż się przewróciła w tył, tak samo, jak gość. \dm{} On miał rację, patrząc stąd, jesteś przerażający! I obrzydliwy. \de{}

\ds{} O co chodzi, ja nadal widzę go normalnego. \dm{} Chronos nie dowierzał. \de{}

\ds{} Czyli ma to związek z odległością? \de{}

Mojmira wstała i zaczęła powoli podchodzić do Pyrroqua, którego kolce z każdym krokiem stawały się coraz mniejsze. Zaschnięta krew znikała, a czaszki parowały. 
Będąc na wysokości Chronosa widziała już dawnego potwora.
Ale postanowiła iść dalej. Wtedy ponownie zaczął się zmieniać.
Tym razem porósł futrem, półkoliste już kolce zmieniły się z ostrych noży, na pluszowe.
W końcu go dotknęła.
Nie był twardy i kanciasty, jak zwykle, a miękki i przytulaśny, a do tego ciepły, co było dziwne, gdyż potwory są zmiennocieplne.
Ścisnęła go.

\ds{} Weź przestań. Nie jestem pluszakiem. \dm{} Pyrroq zaczął ją delikatnie odklejać od siebie. \dm{}
Sama słyszałaś, jestem groźny i noszę czaszki zjedzonych dzieci na kolcach. \de{}

\ds{} Rzeczywiście, jesteś dzieciościskakiem. \dm{} Chronos także podszedł do niego. \dm{} Mojmira, nie wtulościskaj się do uniwersalności, na miłość Boską.
To wygląda na jakąś zależność od odległości, z której aktualnie oczołapiemy. Ani ja, ani Mojmira nie zjedliśmy tortu, ale nadal widzimy zmianę wyglądu Pyrroqua, który zeżarł cały tortosegment. \de{}

\ds{} To nie tylko wygląd, ja czuję rzeczywiste futro. I czułam także jego smród z daleka. A ty, Pyrroq, widziałeś jakąś zmianę w nas? \de{}

Pyrroq zaprzeczył. A potem spojrzał na przestraszoną pannę młodą, stojącą na małym półwyspie.
W tej nienaturalnie uniwersalnej sukni, wyglądała absolutnie kosmicznie.
W świetle księżyca, jej ubiór mienił się, prezentując wszystkie odcienie bieli.
Tysiące diamentów, każdy dokładnie oszlifowany i wpleciony w jedwabne nici od jedwabników, karmionych jedynie kwiatami paproci.
Na tle czarnej wody, wyglądała jak słowiańska bogini. Jak zaklęty łabędź, albo tajemnicza zjawa.
Jej welon, targany przez wiatr, niczym rosa na pajęczynie, zostawiał w powietrzu srebrny pył.
Poświata sukni była tak jasna, że można było przy niej czytać.
Odbicie w wodzie wręcz przestało się marszczyć, aby tylko nie skalać wyglądu postaci.
Wokół latały świetliki i dopełniały wspaniałości.

Jednak Pyrroq, Chronos i Mojmira widzieli wystarczająco dużo skarbów wszechświata, żeby nie rzucać się bezmyślnie na białe złoto oprawione platyną, 
jak robili to przed chwilą goście.
Podobnie, jak to było z Pyrroquiem, podchodzili do niej, a jej suknia gasła z każdym krokiem.
Z bliska nie miała na sobie żadnej sukni, a jedynie proste ubranie. Całkowicie niepozorna, szara myszka.

\ds{} Czy ktoś mi wytłumaczy, co tutaj się do cholery dzieje? To wasza sprawka? \dm{} zapytała. \de{}

\ds{} Nic szczególnego, jedynie najgroźniejsza forma materii we wszechświecie, przez przypadek wpadła do weselociasta i zatruła wszystkich gości tak, że widzą iluzjobyty \dm{}
Chronos odpowiedział.\de{}

\ds{} Próbujemy ogarnąć, jak działa. Sytuacja jest poważna, ale jeszcze nikt nie umarł, więc mogło być gorzej. \dm{} Mojmira patrzyła to na szarą myszkę, to na krwawego mordercę w oddali. \dm{}
Wiemy, że patrząc na osobę, która zjadła trochę tortu, widzi się trzy różne rzeczy. Pyrroq z daleka jest wściekłym mordercą, ze średniej odległości normalny, a z bliska słodkim kotkiem.
Ty natomiast z daleka nosisz nieprawdopodobną, diamentową suknię, ze średniej odległości jesteś ubrana w twoją zwyczajną suknię ślubną, a z bliska nosisz bardzo niepozorne ubranie.
Coś musi łączyć te trzy rzeczy. \de{}

\ds{} Pyrroq, chodź tu i pomóż nam! \dm{} Chronos zawołał. \dm{} Nie stój tam i nie sercozawałuj ludzi. \de{}

\ds{} Mnie nie przeszkadza. Jest dobrze. \dm{} Patrzył na lustra zdobiące zewnętrze sali i robił groźnie pozy. \dm{}
Poradzicie sobie beze mnie. Na co wam bezużyteczny kociak. I zawołajcie kilku gości, niech się przestraszą.\de{}

\ds{} Byłam w sali bankietowej, gdy to się zaczęło, ludzie otrzymywali góry pieniędzy, kobiety, samochody. Ale zawsze, jak podchodziłam bliżej, to wszystko znikało.
Zmieniało się w coś odwrotnego. Te osoby, które dobrze znałam, one z bliska zmieniały się w... siebie? Prawdziwych siebie. \dm{} Panna młoda wpatrywała się w swoje odbicie w wodzie.
\dm{} Zawsze chciałam mieć niewyobrażalnie doskonałą suknię ślubną, ale w głębi duszy... chyba właśnie jestem takim nikim. 
Trzy odległości. To, jacy chcemy być, to jakimi widzą nas inni i to, jacy jesteśmy wewnątrz. \de{}

\ds{} Czy może być to aż tak proste? \dm{} Mojmira się zastanowiła. \dm{} Wiem, że Pyrroq chciałby być taki, taki obrzydliwy, budzić postrach. 
A jednocześnie jest aż za dobry wewnątrz i do końca nie skrzywdzi nikogo. 
A to całkiem logiczne, że chcąc mieć diamentową suknię, wewnątrz, wybacz mi, wcale nie jesteś diamentowa.
Uniwersalności pokazującej pragnienia i wygląd serc jeszcze nie spotkaliśmy. \de{}

\ds{} Wszytko to nieprawda. To, jacy chcemy być, nie ma znaczenia, bo i tak nigdy się to nie stanie.
Nigdy nie będę mogła kupić takiej wymarzonej sukni i nosić gigantycznych diamentów, a przecież biedna nie jestem. 
Marzenia nie mogą przyćmić naszego aktualnego życia, gdyż nigdy nie marzymy o realistycznych rzeczach.
Suknia, której nie da się wyprodukować, samochody niespalające paliwa, doskonałe żony i dyplomy uczelni bez przykładania się do nauki, ludzie o tym marzą, ludzie tego nie mogą dostać. \dm{} 
W odbiciu luster, jej suknia zaczęła marnieć i się rozpadać. \de{}

\ds{} Ale nie tym razem \dm{} Chronos odpowiedział \dm{} wszystko to się dzieje naprawdę. Uniwersalność rządzi się własnymi prawami. Tutaj wszystko jest możliwe.
Czujesz słodki zapach śmierci Pyrroqua? Zobacz, na trawie został jeszcze srebrny pył z twojej wstążkoczapki. 
Zdejmiesz suknię, przejdziesz kawałek i nadal tam będzie. \dm{} Chronos zdjął polem siłowym czaszkę z kolca Pyrroqua i przyciągnął ją do siebie. \dm{}
Dziewczynka, pięcioletnia, jeszcze żyła, gdy przebijano jej głowę. Możesz ją palcokontaktnąć, możesz jej doświadczyć, nie jest żadną iluzją. 
Poza tym. Czy nie dałoby się wyprodukować takiej sukni atomołącząc ją w jakimś teleporterze? Albo podobnie, zmienić ci mózgoośmiornice w głowie, abyś zdobyła potrzebną wiedzę?
To, co dziś wydaje się niemożliwe, często może być uzyskane w przyszłości. Zapewniam, że dawno temu ludzie marzyli o niestworzonych rzeczach, całkowicie normalnych dzisiaj.\de{}

\ds{} No właśnie. Zawsze marzymy o rzeczach, dla nas aktualnie nieosiągalnych.
A kiedy jednak jest okazja do spełnienia arcymarzenia, to okazuje się jakąś sztuczną uniwersalnością. \dm{} Na te słowa panny młodej, Chronos wypuścił z rąk czaszkę,
przypominając sobie swoją awersję do uniwersalności. \dm{} Suknia może i będzie mogła być ubrana, ale co dalej? Jeśli akurat się zdarzy, że nie będzie powodowała
jakichś klątwowych efektów, to co ze mną? Jeśli będę miała suknię, to będzie koniec. Największe marzenie spełnione, pora umierać.
To właśnie dążenie do sukni, trzyma nas przy życiu. Spełnienie najwyższego marzenia, koniec drogi. Nacieszysz się nim dzień, a potem ono uświadomi cię, że nie masz już w życiu celu.
Wesele nie może wiecznie trwać, nie ma nieskończonej ilości dróg do jazdy supersamochodem, a dyplom nie służy do powieszenia na ścianie. \dm{} Odbicie w lustrze przedstawiało makabryczną pannę w 
sukni zjedzonej przez mole. Ale jej bliski wygląd wracał z szarej myszki do normalności. \de{}

\ds{} Ale jeśli nasze najwyższe marzenie nie może być nigdy osiągnięte, to po co żyć? \dm{} Mojmira nie płakała. \de{}

\ds{} Po to, żeby zastępować je mniejszymi. To nas pociąga do działania, nie sam główny cel, ale pomniejsze. 
Nie można mieć diamentowej sukni, ale można mieć mniej magiczną. Można mieć bardzo ładną, normalną suknię.
Taką, jak ta. \de{} 

Diamentowa suknia w odbiciu rozpadła się zupełnie, a ubranie myszki ewoluowało do standardowej sukni, jaką panna młoda początkowo nosiła.
Mojmira przyłożyła do panny miernik uniwersalności, który teraz wskazywał całkowite zero.

\divider{}

Już nikt nie krzyczał. 
Ludzie stali w równej odległości od siebie, patrzyli w lustra i na siebie nawzajem, robiąc dziwne ruchy.
Każdy podziwiał swoje odbicie. Panika się skończyła. Była jedynie martwa cisza.

Nikt nie zwrócił uwagi na pannę młodą, która była jedyną ruszającą po sali postacią. Chronos i Mojmira patrzyli z okna, jak jej idzie.
Sunęła w swojej sukni przez surrealistyczne widoki, nierealistycznych postaci, skąpanych w świetle księżyca.
Wysunęła się zza sportowca, obwieszonego medalami z olimpiady ze wszystkich dziedzin, zwłaszcza tych, które odbywały się jednocześnie.
Przejechała ręką po karoserii termojądrowego samochodu sportowego, swojego wujka.
Doszła do wielkiej góry mięśni, która była bardziej szersza, niż wyższa i kilkukrotnie większa od potwora. Wszyscy bohaterowie komiksów złączeni razem.
Postanowiła jednak przejść przez środek. Mięśniak rozmywał się pod jej obecnością, w centrum zobaczyła chudego patyczaka, nie rozpoznała, który to był z gości.
Jej stryj siedział na kupie papierów i coś pisał, wszystkie to były badania medyczne i przepisy na lekarstwa do leczenia każdej choroby.
W oczy rzucił jej się dokument naprawiający uniwersalność, dokument naprawiający samego siebie.
W końcu jej siostra Magda, z bliska nie wyglądała niezwykle, ale panna wiedziała, jaka będzie z daleka. 
Odchodziła, a u stóp siostry wyrastała mównica, ona pozdrawiała swoich wyborców.
Widziała swojego ojca, jak obraca jej nowego męża na ruszcie, nad ogniskiem. Sama czasami z chęcią by to zrobiła.
Nagle z powietrza wyłonił się jej kuzyn. Niczym upiór z mgły.
Ale gdy przeszła bliżej, zniknął ponownie. Depresja.
Mały Tymek bawił się klockami i całkowicie nie zważał na to, co się dzieje wokół. W żaden sposób także nie zmieniał się, pod wpływem odległości.
W końcu doszła do końca sali. Pan młody spoglądał na lustro. Zobaczyła w nim samą siebie. Drugą siebie.
Weszła w swoje odbicie, a jej mąż odwrócił się i uśmiechnął. Powiedział, że szukał jej wszędzie, aż zobaczył jej postać w lustrze i nie mógł oderwać wzroku.
A może niektóre marzenia rzeczywiście mogą być prawdziwie spełnione?
Trzy osoby i potwór, została jeszcze reszta gości i drugi potwór.

Anastazja miotała się po sali.

\ds{}
Tego nie ma naprawdę, nie istnieje.
Nigdy nie wygrasz całej olimpiady, to niemożliwe.
A nawet, jeśli byś wygrał. Jakim wrakiem człowieka się staniesz?
Umrzesz rok później.
\dm{} Medale odpadły i potoczyły się po podłodze. \dm{}
Wujku, nigdy nie będziesz mieć samochodu na wodę, a nawet jeśli.
Ile by ważył? Gdzie byś go bezpiecznie parkował? Kto by go naprawiał?
Przecież nawet nie ma porządnej drogi z waszego domu do miasta.
Gdzie byś nim pojechał, mając radioaktywny tokamak pod maską?
Pewnie prosto do kostnicy, cięższy o całe mnóstwo dodatkowych neutronów, przewoziliby cię w pudle do radioaktywnych odpadów.
\dm{} Wujek zamrugał, wyprostował się. Samochód zmienił się w jego własnego poloneza i zniknął. Już sześć dusz. \dm{}
Siostrunio, wybacz. Ty akurat nigdy nie będziesz rządzić krajem. I nie dlatego, że nie będziesz umiała, a dlatego, że ludzie są za głupi, aby zaakceptować 
twoje odważne zmiany. Jak myślisz, jak zareagowaliby wyborcy na twój pomysł tego dziwnego, wspaniałego prawa, o którym mi ciągle tyle opowiadasz?
\dm{} Mównica przewróciła się z łoskotem. Siedem. \dm{}
Wasze marzenia nigdy się nie spełnią tak, jak tego oczekujecie. \dm{} Osiem. \dm{}
Po spełnieniu ich i tak byście żałowali. \dm{} Dziewięć. \dm{}
Nie da się, nie możesz, nie zrobisz. \dm{} Wszyscy. \dm{}

Mojmira i Chronos powoli podchodzili do Pyrroqua. Jego kolce zmniejszały się tak, jak wcześniej.

\ds{} Dlaczego uważasz, że nie jesteś dość groźny? \dm{} Mojmira zapytała. \de{}

\ds{} Bo ciągle słyszę, jak to mało spiczasty jestem. Małe dzieci wolą mieć na mnie przejażdżkę, niż uciekać z krzykiem.
,,Ha, ha, Pyrroq. Twoje kolce wyglądają, jak noże do pizzy''. Ile razy to słyszałem. \de{}

\ds{} I jak pizzotnarkami, jesteś wstanie nimi obracać i ciąć wszystko z chirurgiczną doskonałością. \dm{} Chronos opowiadał. \dm{} A do tego jeździć, jak na kółkobutach.
Każdy z nas jest w czymś lepszy, a w czymś gorszy. A ja? \dm{} Westchnął. \dm{} Ciągle mi dźwiękoprzekazują. ,,Jesteś najbardziej bezużytecznym potworem ze wszystkich''. 
,,Jak można bać się używać własnej mocy''? ,,Co to znaczy, że nie umiesz cofnąć czasu i uratować mu życia''?
,,Szkodzisz nie tylko sobie samemu, ale nam wszystkim''.
,,Sam potrafię bardziej nagiąć czasoprzestrzeń magnetyzmem, niż ty dedykowaną zdolnością''.
,,Wstyd, wstyd''. ,,Prędzej wynajdziemy wehikuł czasu, niż ty sam nauczysz się używać własnej mocy''. Ile razy już to uchoodbierałem. \de{}

\ds{} Widzisz? Obaj nie jesteście doskonali i nigdy nie będziecie dokładnie tacy, jacy chcecie. \dm{} Mojmira jeszcze bardziej się zbliżyła.
Spiczaste kolce zniknęły zupełnie. \dm{} Ale czy to powód, żeby nie mieć w ogóle marzeń?
Zapewniam cię, Pyrroq, że większość kosmitów trzęsie się na wasz widok ze strachu. I nie ma znaczenia wygląd, wasza zła sława niesie się wszędzie i działa znacznie potężniej, niż jakieś igły.
Mniejsze marzenia, Tymek nadal czeka na swojego czerwonego smoka.
\dm{} Mojmira ostatecznie ścisnęła Pyrroqua. 
Był zimny, twardy i kanciasty. Taki, jak zwykle. \de{}

Goście po tej przygodzie na pewno nie chcieli kontynuować wesela. 
Wszyscy, z wyjątkiem najbliższej rodziny Anastazji, pakowali się do odjazdu.
Ale czy wesele okazało się klapą? Ludzie powyciągali lekcje z tej przygody, co jest znacznie ważniejsze, niż dotańczenie tańca do końca.















