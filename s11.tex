\chapter{Zielona tajemnica} 

\info{Dwójka podróżujących statkiem kosmicznym ludzi zauważa za oknem nową gwiazdę. Próbują dowiedzieć się o co chodzi, bo jakoś wszyscy robią im na przekór.}

Czytnik kart zapikał i moim oczom ukazał się mały pokoik. 
Ta mała klitka będzie mi służyć przez najbliższe kilka dni podróży.
Inni pasażerowie statu także zajmowali swoje miejscówki w innych pokojach, podobnych do mojego.
Ukradkiem popatrzyłem na kajutę sąsiada, czy aby nie była większa. 
Na szczęście nie była.

Jeśli tak wyglądała druga klasa, to jak musiała wyglądać piąta? 
Trzymali ich na zewnątrz w próżni? W namiotach?

Zostawiłem bagaż, dokładnie zamknąłem drzwi i rozejrzałem się po pokładzie.
Stella Grande posiadała dziesięć pięter, wypełnionych restauracjami, kasynami i pubami.
W sumie nic więcej treściwego nie było, albo inaczej, wszystko inne było dość zakurzone i dawno nieużywane.
Trafiłem na jakąś salę kinowo-teatralną, której nikt nie sprzątał od kilku kursów.
Była i siłownia w której brakowało większość sprzętu.

Poszedłem najwyżej jak się dało.
Stąd rozpościerał się przepiękny widok z pokładu obserwacyjnego.
Sferyczna czasza z grubego pleksiglasu miała chronić wnętrze przed zabójczą próżnią.
Na razie chroniła jedynie przed atmosferą wiecznie spowitego w smogu miasta.
Nie mogłem się doczekać, aby w końcu zobaczyć ponownie słońce.

Gdy rozpoczęło się odliczanie, wstałem ze stojącego tam leżaka i podszedłem do szyby, żeby po raz ostatni pożegnać się ze swoją planetą.
Chciałem także się upewnić, czy nadal jestem w stanie ustać przy przyspieszeniu 3 G.

Pokład zadrżał, a kłęby pary całkowicie przysłoniły widok.
Zrobiło się na tyle ciemno, że zapaliły się dyskretnie ukryte pod siedzeniami lampy.
Oglądanie mlecznego sufitu nie było tym, co chciałem oglądać po raz ostatni na tej planecie.

\begin{dialogue}
	\ds{} Trochę jak w wariatkowie, prawda? \dm{} Usłyszałem za sobą damski głos. \dm{} A pan jest zapalonym astronautą, widzę.
	\ds{} Dzień dobry, \differentlan{mademoiselle} \dm{} Obróciłem się i ukłoniłem grzecznie do nagle zmaterializowanej za mną osoby.
	\ds{} Dzień... \dm{} Młoda kobieta była wyraźnie zmieszana \dm{} ...dobry. Nazywam się Galiza, to mój pierwszy lot w kosmos.
	\ds{} Doktor Wizgr... Po prostu Wizgrant. \dm{} Akademickie odruchy nadal mną rządziły \dm{} Miło mi panią poznać.
\end{dialogue}

Otwarło się nad nami niebieskie oko nieba.
Poczułem także, że zrobiłem się trzykrotnie cięższy.
W dodatku ten przeklęty reumatyzm.
Pani Galiza także ledwo trzymała się na nogach.

\begin{dialogue}
	\ds{} Może usiądziemy? \dm{} zaproponowałem. \dm{} Przy takim przyspieszeniu łatwo się przewrócić.
\end{dialogue}

Usiedliśmy, a właściwie zwaliliśmy się na leżaki.
Nad nami był już tylko ciemniejący błękit.
Galiza jednak wpatrywała się w niego jak w pokaz fajerwerków.
Potem zaczęły wychodzić pojedyncze gwiazdy.
Gdy w końcu przyspieszenie zmalało do normalnego, wstaliśmy aby obejrzeć Ziemię z orbity.

\begin{dialogue}
	\ds{} Ciekawe, czemu każdy z tych statków musi mieć taką nudną nazwę. Stella Grande, Trans-Galactica, czy Saturn Brava.
	\ds{} Przecież morskie wycieczkowce także mają podobnie beznadziejne nazwy \dm{} odrzuciła, jakby była to oczywistość.
	\ds{} No ale teraz mogliby zrobić inaczej... \dm{} Jestem taki beznadziejny w rozmowach z kobietami. \dm{} 
			A propos, od spodu statku jest podobna szyba, co tutaj i tam lepiej widać będzie oddalającą się Ziemię.
\end{dialogue}

Galiza przystała na propozycję.
Udaliśmy się przez labirynt głośnych imprez i krzyczących dzieci do najniższego pokładu.
Tutaj także było całkowicie pusto.
Usiadłem na przezroczystej podłodze, tysiąc kilometrów nad powierzchnią, i zaprosiłem Galizę, żeby zrobiła to samo.

Siedzieliśmy przez chwilę, obserwując oddalający się dom.

\begin{dialogue}
	\ds{} Więc... \dm{} Zacząłem rozmowę, bo trochę dziwnie się robiło. \dm{} Byłem swego czasu, w sumie nadal jestem, doktorem astronomii.
		Z chęcią poopowiadam pani o każdej z tych gwiazdek osobno. Chociaż niestety, wiele osób o szybko nuży.
\end{dialogue}

Co za nietakt. Pomyśli sobie teraz, że się chwalę swoją wiedzą.
Pewnie będzie mnie miała za jakiegoś snoba.

\begin{dialogue}
	\ds{} To bardzo zabawne spotkać takiego kogoś w jego własnym środowisku \dm{} zaśmiała się. \dm{}
		Czyli pan pewnie leci na Tytana w celu jakichś badań, prawda? \dm{} A potem posmutniała. \dm{} Ja wybieram się jedynie na wakacje, tak po prostu.
	\ds{} To ciekawie się składa, bo ja także. Już kiedyś byłem na Tytanie, pomyślałem że to będzie dobre miejsce na śmier...
	\ds{} Co proszę? \dm{} Popatrzyła się na mnie oczyma tak głębokimi, jak głębokie pole Hubbla.
	\ds{} Nic, nic. \dm{} Szybko się poprawiłem.
	\ds{} Hmm... \dm{} szepnęła cichutko, niczym przelot ćmy w próżni.
	\ds{} Chciałem powiedzieć... że to takie miłe miejsce. Żeby dokończyć żywota. \dm{} Zabrzmiało to jeszcze gorzej. \dm{} Widzi pani, 
		nowotwór mnie trawi. Podróżujemy w kosmos, a współczesna medycyna nie jest w stanie naprawić problemu dręczącego ludzi od zarania dziejów.
		Zostało mi kilka lat życia, to pani pierwsza podróż kosmiczna, a moja ostatnia.
\end{dialogue}

Zapadła niezręczna cisza, przerywana jedynie wybuchami śmiechu gdzieś powyżej.
Oddech Galizy uspokoił się.

\begin{dialogue}
	\ds{} Więc, co to jest za gwiazdka? \dm{} Dziewczyna zmieniła temat błyskawicznie. 
	\ds{} Spica. 
	\ds{} Spica? Jest trochę niebieskawa.
	\ds{} Bo jest bardzo gorąca, emituje dużą ilość wysokoenergetycznych fal elektromagnetycznych przez co nasze oko... \dm{} Zauważyłem że Galiza przestaje nadążać.
		\dm{} Poza tym jest podwójna. To tak na prawdę dwie gwiazdy, które krążą wokół siebie.
	\dm{} Ciekawe, ciekawe. \dm{} Ale nie było zaciekawienia w jej głosie.
\end{dialogue}

Rozmowa się nie kleiła, albo to ja nie potrafiłem nic z siebie wydusić.

\begin{dialogue}
	\ds{} Pamięta pan przedwojenny świat? \dm{} Nagle się mnie zapytała. 
	\ds{} Nie jestem aż tak stary \dm{} zaśmiałem się. \dm{} Akurat gdy się urodziłem, król Hegezot przejmował władzę nad wszystkimi krajami.
	\ds{} Och, przepraszam. Zawsze mnie ciekawiło, jak ludzie podróżowali kiedyś w kosmosie. Myślałem że pan wie. Wszystkie przedwojenne teksty zostały przecież obowiązkowo zakazane.
	\ds{} Wiem tyle samo, co pani. Poza tym... \dm{} Nachyliłem się do jej ucha \dm{} w kosmosie ściany także mają uszy.
	\ds{} Nie te. \dm{} Wskazała na przezroczystą posadzkę na której środku siedzieliśmy.
	\ds{} To nadal ryzykowne.
	\ds{} Tak samo, jak życie, prawda?
\end{dialogue}

Zapamiętałem te słowa, dzwoniły mi w uszach jeszcze przez pewien czas, po tym jak rozeszliśmy się.
Szeptanie przeciwko obecnej władzy nie było mądrym posunięciem.
Po zakończeniu czwartej wojny światowej, tak zwanej Ostatniej Wojny, nad Ziemią zapanował wspólny rząd z ,,demokratycznie'' wybieranymi przywódcami.
Każda przedwojenna księga została zakazana, jedynie naukowe twory były przefiltrowane przez bezwzględną cenzurę i wydane w elektronicznej postaci.

Każda dziedzina życia stała się na nowo, a obywatele byli teraz rybkami w akwarium.
Miało to swoje minusy, to fakt.
Jednak wbrew pozorom, ludzkość rozwijała się szybciej niż kiedykolwiek.
Tak masowe podróże kosmiczne nie byłyby możliwe bez wspólnego rządu.
Ale czy na pewno? Takie przynajmniej jest oficjalne stanowisko nauk historycznych.

Na niepodległym Tytanie na pewno znajdę odpowiedzi na wszystkie pytania.
Zresztą, co mnie to obchodzi, jeśli mnie aresztują, to przecież w aktualnym stanie nic nie tracę.
Za to Galizę powinno, ma przed sobą całe życie, lepiej żeby nie spędziła go w zakładzie naprawczym do prania mózgów.

Wziąłem z bagażu przewodnik po Tytanie i usiadłem na leżaku na tarasie widokowym.
Mało było w tej książce napisane, ale zawsze lepsze to niż nic.
Natomiast informacje z książki dość mocno odbiegały od moich wspomnień, kiedy kilkanaście lat temu ostatni raz odwiedzałem ten księżyc.

Obudził mnie dzwonek zwiastujący obiad.
Z leżaka obok poderwała się Galiza.

\begin{dialogue}
	\ds{} Widzę, że pani także nie może sobie znaleźć miejsca na dolnych pokładach \dm{} zapytałem się wrednie.
	\ds{} Nie, tylko. \dm{} Trochę się zmieszała. \dm{} Pan ma doświadczenie, a ja jestem tutaj taka sama. Wcale nie przyszłam tutaj za panem.
	\ds{} Oczywiście, że nie. \dm{} Uśmiechnąłem się w duchu. \dm{} W każdym razie zadzwonił dzwonek na obiad. Czy dałaby się pani zaprosić na kosmiczny bankiet?
	\ds{} Ależ, ja nie mam pieniędzy na obiad drugiej klasy!
	\ds{} Nie szkodzi, niech będzie na mnie. \dm{} Zmrużyłem oczy. \dm{} Skąd pani w ogóle wie, że jestem z drugiej klasy?
	\ds{} Tak, jakoś. Zobaczyłam pana bilet. \dm{} Skłamała. Nigdy nie trzymałem przy niej swojego biletu. Musiała śledzić mnie do mojej kajuty.
\end{dialogue}

Wziąłem ją pod rękę i poszliśmy razem do głównej restauracji.
Idąc, patrzyła jeszcze na gwiazdy.
Może jednak interesowała się nimi.

\begin{dialogue}
	\ds{} Więc, gwiazdy mogą być czerwone, albo żółte jak nasze Słońce, albo też białe czy fioletowe?
	\ds{} Zgadza się, zgodnie z zasadami promieniowania ciała doskonale czarnego. \dm{} Znowu zacząłem przynudzać. \dm{} To jak rozgrzany metalowy pręt.
		Im mocniej rozgrzany, tym jaśniejszy.
	\ds{} A czy może być więc zielona gwiazda?
	\ds{} Nie słyszałem o czymś takim, szczerze powiedziawszy. W dodatku żadna planeta, czy księżyc w naszym układzie gwiezdnym nie są zielone, skąd taki pomysł?
	\ds{} No bo tam jest. \dm{} Wskazała palcem gwiazdozbiór Łabędzia.
\end{dialogue}

I rzeczywiście, zielony punkt jasno świecił na tle drogi mlecznej.
Nie przypominałem sobie, żeby w tamtym miejscu znajdowała się jakaś gwiazda.
A także nie było to w osi ekliptyki, więc wszystkie planety były od razu wykluczone.
Może asteroida, albo kometa?
Ale te także nie mogły być zielone.

\begin{dialogue}
	\ds{} Szczerze powiedziawszy, pierwszy raz od dawna widzę coś takiego. Może supernowa?
	\ds{} Patrzyłam na mapy nieba, próbując określić, co to za gwiazdka, ale nie nic w tym miejscu nie znalazłam.
	\ds{} Ciekawe, kiedy się pojawiło.
	\ds{} Kilka godzin temu, gdy pan smacznie spał. Przeglądałam elektroniczny atlas i porównywałam z widokiem, gdy zobaczyłam tę gwiazdkę.
	\ds{} No, to że ją pani właśnie wtedy zobaczyła, to nie znaczy że nie było jej tutaj wcześniej. \dm{} Otwarłem drzwi na korytarz. Sześciokątne okna oświetlały korytarz światłem drogi mlecznej.
	\ds{} Nie, ja pamiętam, że patrzyłam się na ten gwiazdozbiór, gdy na początku się spotkaliśmy. I wtedy nic takiego tam nie było.
	\ds{} Przykro mi, gwiazdy nie pojawiają się tak znikąd. Musiało się pani przewidzieć. Ale mnie się to dość często zdarza. \dm{} Poprowadziłem ją do drugoklasowej restauracji.
\end{dialogue}

Usiedliśmy na stoliku na samym środku, bo tylko taki był wolny.
Wręczono nam karty dań z trzema do wyboru, wszystkie wegańskie.
Zamówiłem, jak większość, sojowego placka po francusku.
Galiza chyba nie chciała sprawiać wrażenie kosztownej, dlatego wzięła kotlet warzywny.
Do tego podano nam chłodzoną próżnią wodę.

Jedzenie spędziliśmy na rozmowach o naszych dzieciństwach.
W sumie głównie ja mówiłem, gdyż miałem chyba więcej do powiedzenia.
Galiza wychowała się w centrum miasta i była grzeczną dziewczynką, która zawsze słuchała się rodziców i nauczycieli.
W sumie głównie dlatego postanowiła się teraz przeciwstawić i wybrać na wakacje w kosmos.

\begin{dialogue}
	\ds{} Wizgrant... \dm{} zapytała tajemniczo.
	\ds{} Słucham? \dm{} odpowiedziałem cicho, oczekując jakiegoś personalnego pytania.
	\ds{} Patrz. \dm{} Ostrożnie uniosła rękę nad stolik, na obrusie zarysował się cień.
\end{dialogue}

Wolno zadarliśmy głowy w górę, przez trójkątne okno jaśniała zielonym światłem ta nowa gwiazda.

\begin{dialogue}
	\ds{} Zrobiła się jaśniejsza \dm{} szepnęła.
	\ds{} I zmieniła pozycję \dm{} dopowiedziałem. \dm{} Teraz jest w gwiazdozbiorze Lutni.
\end{dialogue}

Rozejrzeliśmy się po innych stolikach. Nikt inny się nie przejmował niezwykłym fenomenem.
Jak to możliwe, czyżby nie widzieli?

Wtedy cały pokój zajaśniał na zielono.
Reflektory skupiły swój kolor na małej scenie w rodu, na którą wyszedł rudy człowieczek w stroju Leprechauna. 
\begin{dialogue}
	\ds{} Czym różni się kosmonauta, od astronauty? \dm{} zaczął swój występ bez słowa powitania \dm{} Jedni polecieli na Księżyc, a drudzy z Księżyca spadli!
\end{dialogue}

Publiczność wybuchnęła śmiechem. Galiza zaczęła jeść swój kotlet szybciej.

\begin{dialogue}
	\ds{} Czemu próżnia jest taka zimna? \dm{} Na kolejny dowcip poszedłem w ślady mojej pary. \dm{} Bo się przestraszyła czarnej dziury!
\end{dialogue}

Tym razem to zielony ludek śmiał się najbardziej ze wszystkich.

\begin{dialogue}
	\ds{} Dlaczego UFO jest okrągłe?
	\ds{} ...chłodź-my \dm{} Galiza wskazała wyjście, plując jedzeniem. Pokiwałem głową, zatykając usta palcem, żeby mój placek też nie wyleciał.
\end{dialogue}

Uciekliśmy w samą porę, jak już byliśmy w bezpiecznym korytarzu, cała sala zaryczała ze śmiechu.

Skierowaliśmy swe kroki w stronę ogólnodostępnego teleskopu.
Można na nim dokładniej oglądać kosmos, przez który właśnie przedziera się Stella Grande.
Mgławice w rzeczywistości są znacznie bardziej kolorowe, niż jak się je ogląda z Ziemi, bo atmosfera pochłania część kolorów.
Liczyłem na to, że uda nam się bliżej przyjrzeć tajemniczej gwiazdce.

Jednak tym razem zastała nas informacja, że teleskop jest w renowacji.
To było dziwne, gdyż jeszcze przed startem był otwarty.

Udałem się do biblioteki, z nadzieją na znalezienie jakiejś informacji o przelatujących w pobliżu kometach.
Cóż innego mogło to być? Takie jasne.
Zostawiłem Galizę w korytarzu prowadzącym do kajut piątej klasy.
Nie chciała, żebym widział, gdzie spała.

Biblioteka stylizowana była na starodawny zbiór książek.
Elektroniczne ekrany oprawione były w papierowe, postarzane okładki i poustawiane na drewnianych półkach.
Każda z takich ,,książek'' łączyła się z głównym komputerem i mogła wyświetlić wszystko to, co każda inna.

Wziąłem jakąś taką z najmniej zużytą okładką i zacząłem przeglądać dostępne tytuły.
Tylko naukowe i jakieś bełkotliwe śmieci.
Ani tu, ani tam nie było żadnej wzmianki o zielonych gwiazdach, kometach, czy innych podobnych zjawiskach.
Nie chciało mi się wierzyć, że to zjawisko zaszło pierwszy raz w historii, na moich oczach.
Dlatego uznałem, że informacja o tej komecie musiała znajdować się w ocenzurowanej części literatury.

Poszedłem poszukać dziewczyny, od razu znalazłem ją na dolnym pokładzie, centralnie na środku jej ulubionej, przezroczystej podłogi.
Ziemia spadała pod nami, Galiza tęsknie patrzyła się na niebieskawą kropkę.

Wyjaśniłem sytuację z biblioteką, jednak nim dokończyłem opowieść, otrzymałem karteczkę z gwiazdozbiorem Łabędzia i zaznaczoną na niej kreską.
Wokół niej zapisane były godziny.
Moja teoria o naturalnym pochodzeniu tego obiektu legła w gruzach.

\begin{dialogue}
	\ds{} Wierzysz w kosmitów? \dm{} zapytała się tajemniczo \dm{} bo ja wierzę.
	\ds{} Nie spotkaliśmy jeszcze żadnych, a poznaliśmy wszystkie zakamarki Układu Słonecznego. W każdej chwili w kosmos patrzą się tysiące teleskopów, monitorujących każdy skrawek nieba. To trochę niemożliwe, żeby się tak nagle przybysze z zewnątrz pojawili niezauważeni.
	\ds{} Weź poprawkę na rząd światowy.
	\ds{} Wziąłem, nie wszystkie teleskopy przecież są pod ich kontrolą. Na samym Tytanie pewnie pracuje z tuzin.
	\ds{} Ale gdyby...
	\ds{} W fizyce nie ma gdyby. Ten obiekt musi pochodzić z Ziemi. Musi być to statek kosmiczny, albo jakaś sonda. Inaczej się nie da.
\end{dialogue}

Galiza wyraźnie posmutniała.

\begin{dialogue}
	\ds{} Weź poprawkę na rząd światowy \dm{} dodałem tajemniczo. Galiza pokazała doskonale białe ząbki.
	\ds{} Ale co my zrobimy bez teleskopu?
	\ds{} A kto powiedział, że nie mamy teleskopu?
\end{dialogue}

Rozejrzałem się i wyjąłem ze ściennego lichtarza szkiełko w kształcie małej soczewki skupiającej.
Ustawiłem się tak, aby zielone UFO było w linii ze mną i ozdobnym kawałkiem podłogi, który także okazał się być soczewką.
Spojrzałem przez szkiełko i zacząłem się posuwać w przód i w tył. W końcu zobaczyłem wyraźniejszy obraz kanciastego stateczka.
Dałem popatrzyć dziewczynie.

\begin{dialogue}
	\ds{} Dlaczego obraz jest odwrócony? \dm{} Zadała to pytanie, które zadaje 99\% osób pierwszy raz spoglądających przez teleskop.
	\ds{} Bo soczewka załamuje promienie, które przecinają się w ognisku, więc wyjściowy obraz jest odwrócony jednocześnie w poziomie i w pionie.
	\ds{} Ale lornetka.
	\ds{} Lornetka ma pryzmaty, które załamują... dobra, przyszłaś tu rozprawiać o optyce, czy oglądać statek? \dm{} Trochę się zniecierpliwiłem.
	\ds{} Jejku, to rzeczywiście statek. Ma rogi, zupełnie jak gwiazdy.
	\ds{} Gwiazdy nie mają... \dm{} Pociągnąłem ją ze sobą w stronę biblioteki. \dm{} Twoja rogówka ma rogi, znaczy żyły na których zachodzi dyfrakcja...
\end{dialogue}

Oczywiście w bibliotece nie było także nic o zielonych statkach kosmicznych.
Z naukowych pism dowiedziałem się jedynie tyle, że statek tej wielkości nie mógłby długo przeżyć w próżni bez jakiegoś nuklearnego zasilania.
Zasilanie fuzyjne Stella Grande zabiera większość pojemności statku, a i tak jest jedno z najmniejszych możliwych. Co więc zasilało tamten statek?
Reakcja rozpadu? Kto jeszcze używał tej brudnej i drogiej technologii w dzisiejszych czasach?
Pomyślałem, że rząd światowy przecież wyklął i zabronił używać uranowych reaktorów, więc też ciężko by było, żeby zastosował je w swoich tajnych projektach.

\begin{dialogue}
	\ds{} ,,...ratowali statki, w których żagle wiatr dąć przestał, i dawali im trochę swojego wiatru, który wcześniej razem łapali na wietrznej wyspie.''
	\ds{} Galiza, co ty czytasz? \dm{} zapytałem się, zrezygnowany.
	\ds{} ,,Ich rycerskie zbroje rany leczyły, a miecze cięły wrogów jak papirus. Walczyli z piratami morza nicości, żeby handel między wszystkimi wyspami żółtego oceanu kwitł.''
	\ds{} Bajki dla dzieci? \dm{} Bajki były jednymi z niewielu treści, których rząd nie cenzurował. Bo i po co?
	\ds{} ,,W zamian za wiatr, uratowane łodzie dawały im trochę niepotrzebności, aby rycerze mroźnej pustyni mogli wyczyścić swoje zbroje do połysku.'' 			\dm{} Galiza spojrzała na mnie morderczym wzrokiem. \dm{} To nie są żadne bajki. To są \emph{baśnie}. Przekazują bardzo ważne treści.
	\ds{} Niby jakie? Że rycerze pływali na statkach handlowych?
	\ds{} Nie, przecież nie można tego czytać dosłownie. \dm{} Przewróciła oczyma. \dm{} Zobaczmy tutaj. Rycerze, to pewnie grupa jakichś pozytywnych, uznajmy, istot. Morze nicości to jakiś nieprzyjemny obszar, może kosmos? Statki to grupy, które przemierzały ten obszar, a tamci im w tym pomagali. I brali od nich jakąś nietypową zapłatę. Chyba chodzi o zasilanie.
	\ds{} I co z tego wynika? Przecież statki kosmiczne są samowystarczalne. Jedno napełnienie zbiornika z wodorem pozwoli na podróż do Alfa Centauri i z powrotem. Po co miałby ktoś statkom pomagać i wozić im paliwo?
	\ds{} A czy zawsze tak było?
	\ds{} No, od czasów Ostatniej Wojny. A wszystko, co było wcześniej, zniknęło w cenzurze.
	\ds{} To nie zniknęło. \dm{} Podniosła triumfalnie urządzenie. \dm{} Wiesz przecież, że napędy fuzyjne nie pojawiły się od razu.
	\ds{} Ale napęd nuklearny jest nieprzydatny do podróży kosmicznych. Zabójcze promieniowanie, mniejsza energia reakcji, niebezpieczne śmieci z którymi nie ma co zrobić, katastrofalne skutki awarii. \dm{} W czasie wymieniania znalazłem odpowiedni tytuł w książce. \dm{} Proszę, cała rozprawa o tym, dlaczego uranowe zasilanie to zły pomysł. Weź sobie.
	\ds{} Nie dotknę się tego, to zostało ocenzurowane!
	\ds{} W przeciwieństwie do twoich opowieści o kosmicznych piratach.
\end{dialogue}

Najwyraźniej nic sobie z naszej rozmowy nie robiła, bo nadal nie wyciągnęła nosa z sekcji opowiastek na dobranoc.


















