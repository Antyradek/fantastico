\chapter{Baśniowa gwiazda} 

\info{Dwójka podróżujących statkiem kosmicznym ludzi zauważa za oknem nową gwiazdę. Próbują dowiedzieć się o co chodzi, ale niestety po Ostatniej Wojnie wszystko zniknęło pod grubą kołdrą cenzury. Prawie wszystko.}{Space opera}{42 000}{\undefineduniversum{}}

Zostawiłem bagaż w mojej kajucie, dokładnie zamknąłem elektroniczne drzwi i rozejrzałem się po pokładzie.
Stella Grande posiadała dziesięć pięter, wypełnionych restauracjami, kasynami i pubami.
Nic więcej treściwego tutaj nie było, a raczej nikt nie potrzebował niczego takiego.
Trafiłem na jakąś salę kinowo-teatralną, wyglądało że nie była sprzątana od kilku kursów.
Jakaś siłownia, w której brakowało większości sprzętu, i pusty basen.

Postanowiłem przejść najwyżej, jak się dało, wejść do tej charakterystycznej sfery, którą tak dobrze widać było z zewnątrz.
Był to pokład obserwacyjny, z którego rozpościerał się przepiękny widok na platformę startową.
Czasza z grubego pleksiglasu miała chronić podróżników przed zabójczą próżnią.
Na razie chroniła jedynie przed atmosferą wiecznie spowitej w smogu metropolii.

Nie było tutaj żywej duszy, wszyscy normalni ludzie siedzieli we wnętrzu latającego miasta, jedząc zbyt drogie jedzenie i grając na automatach.

Gdy rozpoczęło się końcowe odliczanie, wstałem z leżaka i podszedłem do szyby, żeby po raz ostatni pożegnać się z moją niebieską planetą.

Pokład zadrżał, a kłęby pary całkowicie przysłoniły widok.
Zrobiło się ciemno, jak w czasie wielkiej burzy.
Mleczny sufit nie był tym, co chciałem zapamiętać jako finalne wspomnienie ze startu.

\begin{dialogue}
	\ds{} Zupełnie jak w wariatkowie, prawda? \dm{} Usłyszałem za sobą damski głos.
	\ds{} Dzień dobry, \differentlan{mademoiselle}. \dm{} Trochę zaskoczony obróciłem się i ukłoniłem grzecznie do nagle zmaterializowanej za mną postaci.
	\ds{} Dzień... \dm{} Młoda kobieta była wyraźnie zmieszana. \dm{} ...dobry. Nazywam się Galiza. To mój pierwszy lot w kosmos i chciałam może posłuchać się... rady?
		Pan mi wygląda na takiego, co się zna na podróżach kosmicznych.
	\ds{} Doktor astronomii Wizgr... po prostu Wizgrant. \dm{} Akademickie odruchy nadal mną rządziły. \dm{} Miło mi panią poznać.
\end{dialogue}

Otwarło się nad nami niebieskie oko nieba.
Poczułem także, że zrobiłem się znacznie cięższy.
W dodatku ten przeklęty reumatyzm.

\begin{dialogue}
	\ds{} Może usiądziemy? \dm{} zaproponowałem, widząc jak pani Galiza ledwo się trzyma na nogach. \dm{} Przy takim ciągu silników termojądrowych łatwo upaść i coś zwichnąć.
\end{dialogue}

Usiedliśmy, a właściwie zwaliliśmy się na leżaki.
Nad nami był już tylko ciemniejący błękit.
Galiza jednak wpatrywała się w niego, jak w przedstawienie teatralne.
Potem na scenę zaczęły wychodzić pojedyncze gwiazdy.
Gdy w końcu przyspieszenie zmalało do normalnego, wstaliśmy na przerwę, aby obejrzeć Ziemię z orbity.

\begin{dialogue}
	\ds{} Każdy z tych statków kosmicznych ma taką nudną nazwę. \dm{} Spróbowałem przełamać lody. \dm{} Stella Grande, Trans-Galactica, czy Saturn Brava. 
	\ds{} Przecież morskie wycieczkowce także mają podobnie beznadziejne imiona \dm{} odrzuciła, jakby była to oczywistość. \dm{} Aqua Grande, Trans-Oceania i Costa Brava.
	\ds{} No ale... teraz mogliby zrobić inaczej... \dm{} Jestem katastrofalny w rozmowach z kobietami. \dm{} 
			Tak przy okazji, od spodu wycieczkowca jest spora podłoga widokowa i tam lepiej widać będzie oddalającą się Ziemię. Czy zechciałaby pani mi potowarzyszyć?
\end{dialogue}

Galiza przystała na propozycję.
Udaliśmy się przez labirynt głośnych imprez i krzyczących dzieci na najniższy pokład.
Tutaj także było całkowicie pusto.
Usiadłem na przezroczystej tafli, tysiąc kilometrów nad powierzchnią planety, i zaprosiłem Galizę, żeby się przysiadła.
Siedzieliśmy przez chwilę, obserwując oddalający się dom. Statek nakładał ciągłe przyspieszenie, aby zachowywać pozory pola grawitacyjnego.

\begin{dialogue}
	\ds{} Więc... \dm{} Zacząłem rozmowę, bo trochę dziwnie się robiło. \dm{} Byłem swego czasu... w sumie nadal jestem... doktorem astronomii.
		Z chęcią opowiem pani historię o każdej z tych widocznych gwiazdek. Mam nadzieję, że się pani nie znuży. \dm{} Co za nietakt. Pomyśli sobie teraz, że się chwalę swoją wiedzą.
		Zapewne uzna mnie za jakiegoś snoba.
	\ds{} To bardzo zabawne spotkać kogoś takiego, jak pan, w jego własnym środowisku \dm{} zaśmiała się. \dm{}
		 Pewnie leci pan na Tytana w celu wykonania jakichś interesujących badań, prawda? \dm{} A potem posmutniała. \dm{} Ja wybieram się jedynie na wakacje, tak po prostu.
	\ds{} To ciekawie się składa, bo ja także. Już kiedyś byłem na Tytanie, pomyślałem że to będzie dobre miejsce na śmier...
	\ds{} Co proszę? \dm{} Popatrzyła na mnie oczyma tak głębokimi, jak głębokie jest pole Hubbla.
	\ds{} Nic, nic. \dm{} Szybko się poprawiłem.
	\ds{} Hmm... \dm{} szepnęła cichutko, niczym przelot ćmy w próżni.
	\ds{} Chciałem powiedzieć... że to taki miły księżyc, żeby dokończyć żywota. \dm{} Zabrzmiało to jeszcze gorzej. 
		\dm{} Podróżujemy w kosmos, a współczesna medycyna nie jest w stanie naprawić problemu dręczącego ludzi od zarania dziejów. 
		Przez nowotwór zostało mi kilka lat życia, to pani pierwsza podróż kosmiczna, a moja najpewniej ostatnia.
\end{dialogue}

Zapadła niezręczna cisza, przerywana jedynie wybuchami śmiechu z górnych pięter.
Oddech Galizy uspokoił się.

\begin{dialogue}
	\ds{} Więc, co to jest za gwiazdka? \dm{} Dziewczyna zmieniła temat błyskawicznie. 
	\ds{} Spica. 
	\ds{} Spica? Jest trochę niebieskawa.
	\ds{} Bo jest bardzo gorąca, emituje wysokoenergetyczne fale elektromagnetyczne o dużej mocy i częstotliwości, przez co nasze oko... \dm{} Zauważyłem, że Galiza przestaje nadążać.
		\dm{} Poza tym jest podwójna. To tak naprawdę dwie gwiazdy, które krążą wokół siebie.
	\ds{} Ciekawe, ciekawe. \dm{} Ale nie było zaciekawienia w jej głosie.
\end{dialogue}

Rozmowa się nie kleiła, albo to ja nie potrafiłem nic porządnego wydusić.

\begin{dialogue}
	\ds{} Pamięta pan przedwojenny świat? \dm{} nagle się mnie zapytała. 
	\ds{} Nie jestem aż tak stary \dm{} zaśmiałem się. \dm{} Akurat gdy się urodziłem, prezydent Hegezot przejmował władzę w powojennym układzie.
	\ds{} Och, przepraszam. Zawsze mnie ciekawiło, jak ludzie podróżowali kiedyś w kosmosie. Myślałam, że pan wie. Wszystkie przedwojenne teksty zostały przecież tak ładnie ocenzurowane.
	\ds{} Nie wszystkie przedwojenne teksty były przecież usuwane, zwłaszcza naukowe. Pokładowa biblioteka na pewno posiada sporo materiałów na ten temat.
\end{dialogue}

Patrzyliśmy na gwiazdy, na mgławice, na asteroidy. I na potężne silniki, które odpychały nas od naszej planety.
Energia prosto z wnętrza Słońca wypuszczała jaśniejące strumienie, widoczne przez podłogę w postaci świetlistych kolumn, wychodzących u dołu kadłuba.
Tak siedzieliśmy aż dało się słyszeć restauracyjny gong.

\begin{dialogue}
	\ds{} Czy pozwoli się pani zaprosić na kosmiczny bankiet? \dm{} spytałem.
	\ds{} Ależ, ja nie mam pieniędzy na obiad drugiej klasy!
	\ds{} Nie szkodzi, niech będzie na mnie. \dm{} Zmrużyłem oczy. \dm{} A skąd pani w ogóle wie, że jestem z drugiej klasy?
	\ds{} Tak, jakoś. Zobaczyłam pana bilet. \dm{} Skłamała. Nigdy nie wyciągałem przy niej mojej karty pokładowej. Musiała śledzić mnie do mojej kajuty, ciekawe.
\end{dialogue}

Wziąłem ją pod rękę i poszliśmy razem do głównej restauracji.
Idąc, patrzyła jeszcze na Drogę Mleczną. Zauważyłem że czytała podręcznik do astrofizyki.
Może jednak interesowała się kosmosem.

\begin{dialogue}
	\ds{} Więc, gwiazdy mogą być czerwone, albo żółte jak nasze Słońce, albo też białe, czy fioletowe?
	\ds{} Zgadza się, zgodnie z zasadami promieniowania ciała doskonale czarnego. \dm{} Znowu zacząłem przynudzać. \dm{} To jak rozgrzany metalowy pręt.
		Im gorętszy, tym bielszy.
	\ds{} A czy może być więc zielona gwiazda?
	\ds{} Nie słyszałem o czymś takim, szczerze powiedziawszy. Musiałaby nie emitować niskich częstotliwości. 
		W dodatku żadna planeta, czy księżyc w naszym układzie gwiezdnym nie są zielone, skąd taki pomysł?
	\ds{} No bo tam jest. \dm{} Wskazała palcem gwiazdozbiór Łabędzia.
\end{dialogue}

I rzeczywiście, zielony punkt jasno świecił na tle Drogi Mlecznej.
Nie przypominałem sobie, żeby w tamtym miejscu znajdowała się jakaś gwiazda.
A także nie było to w osi ekliptyki, więc wszystkie planety były od razu wykluczone.
Może wykatapultowana asteroida, albo kometa?
Ale te nie mogłyby być aż tak mocno zielone.

\begin{dialogue}
	\ds{} Szczerze powiedziawszy, pierwszy raz od dawna widzę coś takiego. Może supernowa filtrowana przez kosmiczny pył i gaz?
	\ds{} Patrzyłam na mapy nieba, chciałam zobaczyć, co to za gwiazdka, ale nic w tym miejscu nie było.
	\ds{} Na pewno jest jakieś proste wytłumaczenie tego fenomenu.
\end{dialogue}

Usiedliśmy na stoliku na samym środku, bo tylko taki był wolny.
Wręczono nam karty dań z trzema pozycjami do wyboru, szkoda że bez mięsa, ale przynajmniej nie musimy jeść tej piątoklasowej papki.
Zamówiłem, jak większość, sojowego placka po francusku.
Galiza chyba nie chciała sprawiać wrażenia kosztownej, wzięła kotlet warzywny.
Do tego podano nam chłodzoną próżnią wodę.

Jedzenie spędziliśmy na rozmowach o naszych dzieciństwach.
W sumie głównie ja mówiłem, gdyż miałem chyba więcej do powiedzenia.
Galiza wychowała się w centrum miasta i była grzeczną dziewczynką, która zawsze słuchała się rodziców i nauczycieli.
Dlatego między innymi postanowiła się teraz przeciwstawić i wybrać na wakacje w niebezpieczny kosmos.

\begin{dialogue}
	\ds{} Wizgrant... \dm{} szepnęła tajemniczo.
	\ds{} Słucham? \dm{} odpowiedziałem cicho, oczekując jakiegoś personalnego pytania.
	\ds{} Patrz. \dm{} Ostrożnie uniosła rękę nad stolik. Na obrusie zarysował się cień.
\end{dialogue}

Wolno zadarliśmy głowy w górę. Przez trójkątne okno jaśniała zielonym światłem ta nowa gwiazda.

\begin{dialogue}
	\ds{} Zrobiła się jaśniejsza. \dm{} Zauważyła.
	\ds{} I zmieniła pozycję \dm{} dopowiedziałem. \dm{} Teraz jest w gwiazdozbiorze Lutni.
\end{dialogue}

Rozejrzeliśmy się po pozostałych stolikach. Nikt inny się nie przejmował niezwykłym fenomenem.
Jak to możliwe, czyżby nie widzieli?

Wtedy cały pokój zajaśniał na zielono.
Reflektory skupiły swój kolor na małej scenie w rogu, na którą wyszedł rudy człowieczek w stroju Leprechauna.

\begin{dialogue}
	\ds{} Czym różni się kosmonauta od astronauty? \dm{} Zaczął swój występ bez słowa powitania. \dm{} Jedni polecieli na Księżyc, a drudzy z Księżyca spadli!
\end{dialogue}

Publiczność wybuchnęła śmiechem. Galiza zaczęła jeść swój kotlet szybciej.

\begin{dialogue}
	\ds{} Czemu próżnia jest taka zimna? \dm{} Poszedłem w ślady mojej pary. \dm{} Bo się przestraszyła czarnej dziury!
\end{dialogue}

Tym razem to zielony ludek jako jedyny się śmiał.

\begin{dialogue}
	\ds{} Dlaczego kosmici kradną krowy?
	\ds{} ...chłofpmy \dm{} Galiza wskazała wyjście, plując jedzeniem. Pokiwałem głową, zaciskając usta palcem, żeby mój placek też nie wyleciał.
\end{dialogue}

Uciekliśmy w samą porę, jak już byliśmy w bezpiecznym korytarzu, cała sala zaryczała ze śmiechu.

Skierowaliśmy swe kroki w stronę ogólnodostępnego teleskopu.
Można na nim dokładniej oglądać wszechświat, przez który właśnie przedziera się Stella Grande.
Kosmos widziany z kosmosu jest znacznie bardziej kolorowy, niż z Ziemi.
Liczyłem na to, że dzięki tej lunecie uda nam się bliżej przyjrzeć tajemniczej gwiazdce.

Jednak tym razem zastała nas informacja, że teleskop jest w renowacji.
To było dziwne, gdyż jeszcze przed startem był dostępny.

Galiza poczęła snuć teorie spiskowe.
\begin{dialogue}
	\ds{} Mówię ci, celowo zaprosili zielonego ,,kabareciarza'' i oświetlili stoliki na ten kolor. Chcą odwrócić naszą uwagę.
	\ds{} I może jeszcze także celowo zepsuli teleskop? \dm{} zaproponowałem prześmiewczo.
	\ds{} A nie? Zaraz zaczną się patrole na korytarzach. Mówię ci, UFO jak nic.
	\ds{} Tak, uwaga. Lecą na Ziemię, żeby ukraść krowę. Poznaliśmy wszystkie zakamarki Układu Słonecznego. W każdej chwili w kosmos patrzą tysiące teleskopów, monitorujących każdy skrawek nieba. To trochę niemożliwe, żeby się tak nagle przybysze z zewnątrz pojawili niezauważeni. Poza tym, prędkość światła w próżni jest tak koszmarnie niska w skali wszechświata, że nie da się tak po prostu pokonywać przestrzeni międzygwiezdnej.
	\ds{} Weź poprawkę na rząd światowy. Jego sekrety \dm{} wysyczała.
	\ds{} Nie wszystkie teleskopy przecież są pod ich kontrolą. 
	\ds{} Ale gdyby...
	\ds{} W fizyce nie ma gdyby. Ten obiekt musi pochodzić z Ziemi. Musi być to statek kosmiczny, albo jakaś sonda. Inaczej się nie da.
\end{dialogue}

Galiza wyraźnie posmutniała.

\begin{dialogue}
	\ds{} Weź poprawkę na rząd światowy \dm{} dodałem tajemniczo. Teraz pokazała doskonale białe ząbki.
	\ds{} Ale co my zrobimy bez teleskopu?
	\ds{} A kto powiedział, że nie mamy teleskopu?
\end{dialogue}

Rozejrzałem się i wyjąłem ze ściennego lichtarza szkiełko w kształcie małej soczewki skupiającej.
Ustawiłem się tak, aby zielone UFO było w linii prostej ze mną i ozdobnym kawałkiem szkła w oknie, który także okazał się mieć podobny kształt.
Spojrzałem przez szkiełko i zacząłem się posuwać w przód i w tył. W końcu zogniskowałem wyraźniejszy obraz obiektu.
Dałem popatrzeć dziewczynie.

\begin{dialogue}
	\ds{} Dlaczego obraz jest odwrócony? \dm{} Zadała pytanie, które zadaje dziewięćdziesiąt-dziewięć procent osób pierwszy raz spoglądających przez teleskop.
	\ds{} Soczewka skupia promienie światła, przecinają się one w ognisku, więc wyjściowy obraz jest odwrócony jednocześnie w poziomie i w pionie.
	\ds{} Ale lornetka.
	\ds{} Lornetka ma pryzmaty, które załamują... dobra, przyszłaś tu rozprawiać o optyce, czy oglądać statek? \dm{} Trochę się zniecierpliwiłem.
	\ds{} Jejku, to rzeczywiście statek. Ma rogi, zupełnie jak gwiazdy.
	\ds{} Gwiazdy nie mają... \dm{} Pociągnąłem ją ze sobą w stronę biblioteki. \dm{} Twoja rogówka ma rogi, znaczy, żyły na których zachodzi dyfrakcja...
\end{dialogue}

Biblioteka była stylizowana na starodawny zbiór ksiąg.
Papierowe okładki poustawiane były rzędami na drewnianych szafkach.
W środku każda z nich skrywała elektroniczny czytnik, który łączył się z centralnym komputerem i wyświetlał dowolną treść.
Więc wystarczyło wziąć jeden tom, aby móc przeczytać całą bibliotekę.

Naukowe pisma wyjaśniły mi, że żaden statek tej wielkości nie mógłby normalnie funkcjonować ze względu na rozmiary silników fuzyjnych.
Ten od Stella Grande był jednym z mniejszych, jakie skonstruowano, a i tak zajmował większość pojemności wycieczkowca.
Coś tak maciupciego, jak tamten statek, musiało mieć inne źródło zasilania.

\begin{dialogue}
	\ds{} ,,...ratowali łodzie, w których żagle wiatr dąć przestał, i dawali im trochę swojego wiatru, tego co wcześniej razem łapali na wietrznej wyspie.''
	\ds{} Galiza, co ty czytasz? \dm{} zapytałem się.
	\ds{} ,,Ich rycerskie zbroje rany leczyły, a miecze cięły wrogów jak papirus. Walczyli z piratami morza nicości, żeby handel między wszystkimi wyspami żółtego oceanu kwitł.''
	\ds{} Bajki dla dzieci? \dm{} Bajki były jednymi z niewielu treści, których rząd nie cenzurował. Bo i po co?
	\ds{} ,,W zamian za wiatr, uratowane łodzie dawały im trochę niepotrzebności, aby rycerze mroźnej pustyni mogli wyczyścić swoje zbroje do połysku.'' \dm{} Galiza spojrzała na mnie morderczym wzrokiem. \dm{} To nie są żadne bajki. To są \emph{baśnie}. Przekazują bardzo ważne treści... ,,Dzięki Aparatowi zasilanemu swoim wiatrem, byli w stanie leczyć swe rany.''
	\ds{} Niby jakie treści? Że rycerze pływali na statkach handlowych?
	\ds{} Nie, przecież nie można tego czytać dosłownie. \dm{} Przewróciła oczyma. \dm{} Zobaczmy tutaj. Rycerze, to pewnie grupa jakichś pozytywnych, uznajmy, istot. Morze nicości to jakiś nieprzyjemny obszar, może kosmos? Statki to grupy osób, które przemierzały ten kosmos. Rycerze im pomagali w zamian za jakąś nietypową zapłatę. Mieli tajemniczą technologię.
	\ds{} I co z tego wynika? Przecież statki kosmiczne są samowystarczalne. Jedno napełnienie zbiornika z wodorem pozwala na podróż do Alfa Centauri i z powrotem. Po co miałby ktoś statkom pomagać i wozić im paliwo?
	\ds{} A czy zawsze tak było?
	\ds{} No, od czasów Ostatniej Wojny. A wszystko, co było wcześniej, zostało wysterylizowane przez cenzurę.
	\ds{} To nie zostało. \dm{} Podniosła triumfalnie urządzenie. \dm{} Wiesz przecież, że napędy fuzyjne nie pojawiły się od razu.
	\ds{} Ale uranowy reaktor nuklearny jest nieprzydatny do podróży kosmicznych. Zabójcze promieniowanie, mniejsza energia reakcji, trudniej pozyskiwalne paliwo, niebezpieczne produkty rozkładu, z którymi nie ma co zrobić, katastrofalne skutki ewentualnej awarii. \dm{} W czasie wymieniania znalazłem odpowiedni tytuł w książce. \dm{} Proszę, cała rozprawa o tym, dlaczego uranowe zasilanie to zły pomysł. Większość masy takiej Stella Grande stanowiłyby ołowiane osłony. Poczytaj to sobie.
	\ds{} Nie dotknę się tego, to zostało ocenzurowane i ocieka kłamstwem! Nie ma w tym ani ziarna prawdy!
	\ds{} W przeciwieństwie do twoich opowieści o kosmicznych piratach.
\end{dialogue}

Najwyraźniej nic sobie z naszej rozmowy nie robiła, bo nadal nie wyciągnęła nosa z sekcji opowiastek na dobranoc.
Próbowałem się zaczytać w czymś prawdziwszym, ale z tylu głowy tliły mi się myśli na temat uranowej energii.

Ukradkiem spojrzałem na tytuł, który Galiza mi zostawiła przed nosem.

\begin{poem}
	Dawno temu żył sobie dzielny i odważny bohater, który postanowił stawić czoła morskiej armii straszliwych bestii. Strzegły one olbrzymiego skarbu --- zielonego złota.
	
	Niestety, nasz bohater sam nie umiał walczyć, więc przekonał współmieszkańców swojej wyspy, aby nigdy więcej nie karmili potworów i nie współpracowali z nimi. Miał nadzieję na to, że pozbawione wsparcia ludzi, te umrą śmiercią naturalną.
	
	Ale wygłodzone straszydła zamiast zdechnąć, zaczęły atakować przepływające statki w poszukiwaniu jedzenia, nie dając tym razem nic w zamian. Co więcej, poczęły pożywiać się również swoim własnym skarbem, który jedynie wzmacniał ich siłę. Potwory rosły w potęgę, gdyż zachwiana została symbioza.
	
	Przestraszony bohater wypowiedział wojnę wszystkim morskim istotom. Wiedział, że musi nie tylko je pokonać, ale także zabrać im ich zielone złoto i przejąć ich Maszyny.
	To niestety pogorszyło sprawę, bo uzależnione już od zielonego wspomagacza potwory zaczęły i to zabierać ze statków. Tych samych czasami, którym wcześniej swoje skarby sprzedawali.
	
	Pomimo nierównej walki, żółci ludzie wygrali dzięki przewadze liczebnej, a nasz bohater postanowił zniszczyć całe zielone złoto w obawie, że jego moc spowoduje jeszcze kiedyś nawrót bestii.
	
	Od tego czasu morze nicości było całkowicie wolne od niebezpieczeństw.
\end{poem}

Te dwie baśnie łączyło coś wspólnego. Nie wiedziałem tylko dokładnie, co. Opowiadały historię dwóch stron sporu. Ziemskiej, chyba, i kosmicznej.
Była też jakaś silna materia, z którą obie strony miały do czynienia, i przydatne urządzenia.

\begin{dialogue}
	\ds{} Nie dłub w nosie, bo przyjdzie zielony pan z kosmosu i ci go ukradnie. \dm{} Usłyszałem za biblioteczką, jak matka karciła syna.
\end{dialogue}

Ciekawe, skąd wzięło się takie straszenie dzieci? Może z jednej z tych baśni? 

Znalazłem jakiś wiersz, na te cenzura zwracała szczególną uwagę. Musiał się zaplątać w reszcie bajek.

\begin{poem}
	To jestem ja. \\
	Nie mam łatwego życia, \\
	ale inni mają gorsze. \\
	Więc zatem ja \\
	postanowiłem uratować innych \\
	przed ich życiem. \\
	Dałem im moje własne życie. \\
	Wspaniałe życie. \\
	Wspólne życie. \\
	Bo mogę, a oni nie mogą. \\
	\\
	Są inni. \\
	Mocno inni. \\
	Mocno cierpią. \\
	Życie cierpią. \\
	Życie uratuję. \\
	Wszystkich uratuję. \\
	\\
	To jesteśmy my. \\
	Tak dużo nas jest. \\
	Tak wspaniałe życie dałem im. \\
	Tak zapewne dziękują mi. \\
	Że aż czuję jak się radują. \\
	Wszyscy się radują. \\
	\\
	To nie jestem ja. \\
	Ja byłem wtedy. \\
	Chciwość nie była ze mną. \\
	Posiadałem prawdę. \\
	A teraz sączy się kłamstwo. \\
	Muszę nas naprawić. \\
	Muszę usunąć zło, zostawić ich. \\
	Więc niech opuszczę siebie. \\
	Wytnę, jak pestki dyni. \\
	I rzucę w zamrożoną otchłań oceanu.
\end{poem}

Pudło. Chyba. Ciekawe, co u Galizy.

\begin{dialogue}
	\ds{} Zobacz, wykopałam gdzieś kawałek skryptu filmowego. \dm{} Usłyszałem głos zza sterty czytników. \dm{} To jakieś fragmenty losowych książek.
	Zaczyna się od romansidła w stylu ,,Sto twarzy Browna'', więc pewnie cenzor w ogóle nie przeczytał do końca.
	\ds{} Prawdopodobnie plik się uszkodził, zdarza się w nawet najnowocześniejszych systemach. \dm{} Wziąłem do ręki książkę.
	\ds{} Chciałeś bez baśni, to teraz czytaj.
\end{dialogue}

Powieść poprzeplatana była fragmentami losowych znaków i oczywiście każda polska litera zastąpiona była kratką. Unicode wynaleziono tysiąclecie temu, ech.
Spróbowałem rozszyfrować tę część, którą poleciła mi Galiza.

\begin{poem}
	\dida{Ciemny korytarz w zniszczonym statku kosmicznym. Przez wyrwy w kadłubie widać wirujące gwiazdozbiory. Słońce co chwila pojawia się w tych dziurach i rzuca ruchome cienie. Światło na scenie pochodzi jedynie od poruszających się chaotycznie plam na ścianach. Główny bohater, w kombinezonie z gwiazdami, podchodzi do siedzącej postaci w grubszym kombinezonie z logami ostrzeżeń przed promieniowaniem.}
	
	\charkap{}
	Pracowniku reaktora jądrowego, zamelduj mi natychmiast stan załogi i statku!
	
	\charkos{}
	Sam se melduj. To przez ciebie. Trzeba było uciekać, a nie walczyć.
	Nie mieliśmy z nimi szans, wszyscy ci to powtarzali.
	
	\charkap{}
	Jak się zwracasz do swojego kapitana!
	
	\charkos{}
	Co to za kapitan, który stracił całą załogę. \\
	Co to za kapitan, który nie ma nawet własnego statku!
\end{poem}
	(Tutaj plik był uszkodzony.)
\begin{poem}
	\dida{W pomieszczeniu wiszą poszarpane kombinezony. Kapitan przystawia do jednego z nich licznik Geigera, który odzywa się silnym stukotem. Przestraszony bohater odskakuje w tył i zderza się ze stojącą przy ścianie postacią. Kombinezonowi odpada kask i widać twarz z usuniętym okiem, połową zębów, uchem i nosem. Placek światła właśnie wsuwa się i oświetla trupa, żeby pokazać widzom w całej doskonałości.}
\end{poem}
	(Tutaj plik znowu był uszkodzony.)
\begin{poem}
	Gdzie są zapasowe pręty uranowe, żeby uruchomić z powrotem reaktor?
	
	\charkos{}
	Zabrane, wszystko zabrane.
	I paliwo, i załoga i twoja godność.
	
	\charkap{}
	Jeszcze jedno słowo, a odstrzelę ci łeb!
	
	\charkos{}
	Ja i tak już nie żyję. \\
	I ty też już nie żyjesz. \\
	Przystaw sobie ten klikacz do głowy, to może sprawdzisz, jak długo będzie boleć. \\
	Bo mnie nie będzie wcale.
	
	\dida{Kosmonauta wypuszcza z sykiem resztki powietrza z butli i osuwa się na podłogę}
	
	(Dalej jest scena jak Marianna całuje się ze zmutowanym niedźwiedzio-człowiekiem.)
\end{poem}

Oddałem książkę Galizie.

\begin{dialogue}
	\ds{} I jak? \dm{} zapytała, wynurzając nos znad sterty elektronicznych komputerków.
	\ds{} Akcja wolno się rozkręcała, ale przynajmniej finałowa scena była dobra, gdy pan Brown zaprosił do łóżkowej zabawy i cyborga.
\end{dialogue}

Chyba nie złapała dowcipu, bo prychnęła i schowała się z powrotem za książkami.

Kolejny tekst to ponownie była fantastyczna opowieść, tym razem trochę mniej artystyczna relacja.

\begin{poem}
	Więc poszedłem nachlać się cyberwina, wiesz, tego sikacza z nanobotami, i patrzę przez okno i widzę zielonego człowieka!
	No mówię ci, gałom nie wierzyłem. A ten gość to trochę dziwny był, bo niby cztery oczy miał, ale ja tam trochę podpity już byłem, więc nie bardzo wierzyłem w to, co widzę.
	Ale na wszelki wypadek wyciągnąłem tego blastera, co go na tym kontenerowcu z Jowisza zajumaliśmy i przystawiłem do szyby i powiedziałem, żeby wypierdalał, bo jak mu strzelę, to mózg z asteroid będzie zbierał.
	A on się tylko uśmiechnął jakoś tak straszliwie, że mu się buzia na pół twarzy otworzyła i wtedy tak szarpnęło statkiem, że ho ho.
	I no, ten, tak trzymałem jakoś ten blaster, że teraz była dziura w kadłubie na wylot. Więc to, co nam kapitan zawsze powtarzał, to to, żeby szybko zatkać dziurę.
	Więc wziąłem i trochę podpity byłem i no, nogę wsadziłem.
	Więc jak mi wciągnęło tą nogę, to aż do jajec prawie. A ten zielony ludzik nadal tam był. I wtedy poczułem jak coś ciepłego i obślizgłego mi się wcina.
	Trochę cyberwino dało mi odwagi, więc wziąłem i uratowałem moje jajca.
	Znaczy, odrąbałem sobie blasterem kurwa nogę. 
	I teraz dziura była jeszcze większa i ten zielony tam zaglądał.
	I wtedy widziałem, jak on miał taki normalnie metalowy ogon, jak zwierz jakiś. I ten ogon gdzieś daleko idzie.
	To się przeraziłem jeszcze bardziej i uciekłem do włazu i zatrzasnąłem, żeby już tlen więcej nie uciekał.
	I stanąłem przed tymi drzwiami, żeby że niby nic się nie stało. Ale szedł kapitan, a ja próbowałem być naturalny.
	A on się mnie pyta, dlaczego nie mam nogi.
	To głupio było mi mówić prawdę, więc powiedziałem, że mi się znudziła, to ją odciąłem. Bo zawsze chciałem mieć taką fajną robotyczną, jak on.
	To tamten się wkurwił i kazał mi spierdalać, więc ja jakoś zacząłem na tej nodze spierdalać.
	Ale wtedy znowu coś uderzyło w statek i tak jakoś tym blasterem odciąłem kapitanowi tą normalną nogę. I znowu była dziura.
	I patrzę, a kapitan wessany jest w tą dziurę i klnie na mnie. I robi się zielony jak ten ludek.
	A potem patrzy na mnie i się uśmiecha w pół. Normalnie pół głowy mu się otworzyło i łapska zaczęły wychodzić i mnie za nogę zaczęły łapać. 
	To ja blasterem po tych łapskach, a one nic. Jak jedną upierdoliłem, to zaraz kolejne wychodziły. I dalej mnie trzymają.
	Więc, no, chciałem mieć drugą robotyczną nogę.
	I uciekłem, bo silne łapska mam, wdrapałem się i schowałem w kratce od wiatraka.
	A potem była bitka i prawie wszystkich nas zajebali.
	I jak się zrobiło cicho, to weszli Polacy, podobni do tych, którym zajebaliśmy ten statek.
	A potem mnie wyciągnęli z tej kratki i związali razem z Szamą i Laserem.
	I powiedzieli do tych zielonych ludków coś po ichniemu, a oni też im coś mówili. I dali mutantom trzy ładne dziewczyny z załogi i Lasera.
	Dobrze, bo go nigdy nie lubiłem.
	A potem mnie nieśli przez pokład i widziałem jak te ludki naprawiały te dziury, co zrobiłem blasterem.
	Potem zieloni sobie poszli, a mnie dali ciężkie ubranie i kazali zamiatać ściany. Ale że nie bardzo mogłem łazić, to wzięli Szamę zamiast tego.
	A potem oddali go i potem Szama był bardzo chory, to się chyba wysypka uranowa nazywa.
	Zostawili mnie ci Polacy właśnie tutaj i odlecieli naszym statkiem. Znaczy ich statkiem.
	I siedzę tutaj już dziesiąty rok i nadal nie mam robotycznych nóg. A ty za co grypsujesz?
\end{poem}

Ta wciągająca opowieść o wątpliwej prawdziwości dużo mi powiedziała o tych tajemniczych, zielonych istotach.

Ja tymczasem poszukałem czegoś nieelektronicznego w tej elektronicznej bibliotece.
Okazało się, że znalazłem samolocik zrobiony z kartki komiksu.
Wyglądał na stary i mocno wyblakły. Nie było na nim daty druku, ale sądząc po ilości osób dziennie używających ten zbiór ksiąg, mógł być nawet z końca wojny.
Kiedyś drukowano takie edycje kolekcjonerskie na oryginalnym drewnianym papierze, więc to dziwne że ktoś potraktował go jak śmiecia.

W każdym razie, był to wycinek opowiadający historię kosmicznych węży, które kąsały ludzi i zmieniały ich w zombie.
Potem te węże zamieszkiwały w tych ludziach, podtrzymując ich przy życiu. Coś w stylu pasożyta, albo jakiejś symbiozy.
Albo to może jednak nie były zombie? 
Ostatnia klatka zaczynała jakąś bitwę tych par z nie wiadomo kim.

Była ta opowieść o tym, który wywołał wojnę z potworami, może to to.
I dalej wiersz z punktu widzenia węża.

To zabawne, że nie zauważyłem wcześniej, jak wiele tekstów \differentlan{de-facto} opowiada mniej więcej o tym samym. 
Może mają w sobie trochę prawdy?

Możliwe, że energia jądrowa rzeczywiście mogła kiedyś służyć jako paliwo dla rakiet. Musiały być ciężkie i olbrzymie.
Jak się zdarzyło, że uranu zabrakło, to statek był skazany na wieczny dryf.
Jakaś strona mogłaby dowozić uran takim ,,unieruchomionym'' transportowcom. W zamian za co?

I tutaj jeszcze brakowało fragmentu układanki. Dlaczego te atomowe mutanty nie ginęły od promieniowania? Na co im były kawałki załóg?
Niby jak pomagały statkom, jeśli jednocześnie je niszczyły?

\begin{dialogue}
	\ds{} Posortowałam te opowieści chronologicznie. \dm{} Galiza zaskoczyła mnie od tyłu, gdy rozmyślałem w kącie.
	\ds{} Szukałem historii w naukowych książkach, ale nic nie znalazłem.
	\ds{} Mówiłam.
	\ds{} Wszystko ocenzurowane.
	\ds{} Mówiłam.
	\ds{} Mówiłaś, że masz pełną wiedzę o tych atomowcach.
	\ds{} Nie mówiłam. Brakuje mi jeszcze czegoś o tym, co ich trzymało przy życiu.
	\ds{} Znalazłem kawałek komiksu. \dm{} Dałem jej wydartą kartkę, którą z chęcią przestudiowała.
	\ds{} To nadal nie wyjaśnia za dużo.
	\ds{} Chyba nigdy nie poznamy pełnej prawdy, w końcu bajki nie mówią niczego wprost.
	\ds{} Ale teraz mam pewność, że to musi być statek jednego z tych atomowców.
	\ds{} Pewność. Taaak.
\end{dialogue}

Zrzuciłem wszystkie teksty na mój zegarek, aby poczytać później.
Razem z Galizą przeszliśmy na dolny pokład, aby zobaczyć progres ruchu gwiazdy na niebie.
Nigdzie jednak nie dało się jej zlokalizować.
Przechodząc przez środek wycieczkowca, przypadkowo rozdzieliliśmy się w tłumie, wiecznie miotającym się w sekcji rozrywkowej.

Spotkaliśmy się dopiero na korytarzu górnego pokładu, okazało się że szukany wycinek nieba jest schowany za grubymi wrotami, prowadzącymi na taras widokowy.
I restauracja i wszystkie okna zostały pozamykane.
Zupełnie jakby ktoś celowo odciął połowę widocznego nieboskłonu.

\begin{dialogue}
	\ds{} Masz jakiś pomysł? \dm{} zapytałem się dziewczyny, stojąc przed zamkniętymi drzwiami do półkulistej bańki.
	\ds{} A ty masz? Bo jakoś nie bardzo uciekasz ze statku w kapsule ratunkowej. \dm{} Galiza oglądała dokładnie ścianę.
	\ds{} Dlaczego myślisz, że miałbym tak postąpić?
	\ds{} No nie wiem, może żeby nie stracić członków ciała w ataku zmutowanych potworo-ludzi? \dm{} powiedziała bardziej do plastikowej płaszczyzny, niż do mnie.
	\ds{} Na prawdę w to wierzysz? Że niby jakiś atomowiec sprzed kilkuset lat nagle pojawił się znikąd? I będzie atakował wielki wycieczkowy statek, zasilany reaktorem termojądrowym, zamiast uranem do wzięcia?
	\ds{} I posiadający pół tysiąca ludzi na pokładzie. Tak.
	\ds{} Twoje bajki, nie ważne jak logiczne by były, nadal pozostaną bajkami. Musisz zweryfikować swoje wierzenia.
	\ds{} A co myślisz, że właśnie robię? \dm{} Uderzyła w ścianę, odsłaniając panel kontrolny drzwi.
	\ds{} Co ty robisz?! Nie niszcz tego! \dm{} skarciłem ją.
	\ds{} Opuściłam Ziemię i już nie jestem grzeczną dziewczynką, przykro mi.
	\ds{} Zaraz ktoś tu przyjdzie i nas zobaczy. Trafimy prosto do tytanowego aresztu. \dm{} Kątem oka przyjrzałem się elektronicznym przełącznikom.
	\ds{} Co tu się dzieje? \dm{} Jak na zawołanie kapitan statku wyszedł zza rogu. Całą podróż siedzi na mostku, żeby akurat przejść się właśnie w tym momencie pustym korytarzem.
	\ds{} Nic.
	\ds{} Dziura w ścianie, to jest nic?
	\ds{} Pan ma dziury w ścianach, to my też chcieliśmy mieć. 
	\ds{} Proszę? \dm{} zapytał się, zmieszany. Skąd przyszedł mi do głowy pomysł, że to by wypaliło?
	\ds{} Lubimy popatrzeć trochę czasami na czerwony kolor.
	\ds{} Czerwony? \dm{} kapitan i Galiza zadali to pytanie jednocześnie.
	\ds{} Tak, czerwony. \dm{} Szturchnąłem dziewczynę. \dm{} Czerwony jest taki ładny i jest wstępem do innej barwy.
	\ds{} Nie wiem, co planujecie, ale wzywam ochronę. \dm{} Kapitan sięgnął do pasa, wtedy Galiza wcisnęła czerwony przycisk na tablicy.
\end{dialogue}

Drzwi otwarły się bezszelestnie.
Zieleń zalała korytarz, dodając nienaturalnego połysku przedmiotom.
Kontury się rozmyły, wściekła twarz dowódczy zyskała nowe oblicze grozy.
Odwróciliśmy się powoli.

Mały, kanciasty statek kosmiczny zbliżał się ku Stella Grande z dużą prędkością.
Jego zielonkawa łuna jasno odbijała się na tle czarnego nieba.
Słońce ostro oświetlało jedną ścianę, na której malowały się liczne pęknięcia i spawy.
Nierówne krawędzie wykonane były jak gdyby z wielu osobnych kawałków złomu, posklejanych razem.
Brud i oleiste zacieki znaczyły otwory w kadłubie.

Z przodu była popękana szyba, od środka pokryta jakby ciemnymi plamami pleśni.
Chorobliwa łuna biła z wnętrza i można było tylko przypuszczać, co to zaglonione akwarium skrywa.
Niewyraźne kontury przypominały biomechaniczny organizm, pływający w lepkiej brei.
To nie był wspaniały rycerz, ani też potwór niewiadomego pochodzenia. 
To było znacznie bardziej bliskie każdemu człowiekowi.

\begin{dialogue}
	\ds{} Mamusiu, to jest ten zielony pan, który miał mi ukraść nos? \dm{} Usłyszeliśmy za sobą.
	\ds{} I to mają być te wspaniałe fajerwerki?
	\ds{} Ja nie widzę tutaj darmowego poczęstunku.
	\ds{} A to nie miał być ten świetny kabareciarz z knajpy?
	\ds{} Po chuj tu przychodziłem.
	\ds{} Ale to chyba nie jest zbyt bezpieczne.
	\ds{} Z jakiej okazji to wydarzenie?
	\ds{} To on będzie rozdawał te kupony, tak?
\end{dialogue}

Kapitan został zupełnie zbity z tropu przez tłumek osób, który jak na zawołanie wpełzł na scenę z każdej strony.

\begin{dialogue}
	\dm{} Proszę państwa, nie wolno tutaj być. To niebezpieczne. \dm{} Przywódca statku próbował załagodzić sytuację, ale prawie został zadeptany przed napierające rodziny z dziećmi.
	\ds{} To twoja sprawka? \dm{} szepnąłem do Galizy, gdy wślizgiwaliśmy się przez już otwarte wrota. \dm{} Ja bym tak nie potrafił.
	\ds{} Wystarczyło powiedzieć, że będą darmowe kupony na loterię. 
\end{dialogue}

Zieleń wzmagała na sile, gdy kanciasta gwiazda przybliżała się do wycieczkowca.
Ustawiła się dziobem do nas i wcale nie zwalniała.

\begin{dialogue}
	\ds{} Przecież on się chce z nami zderzyć! \dm{} zawołałem. Spojrzeliśmy na wyjście, które zatkane już było przez kłębiące się osoby. Każdy chciał z bliska zobaczyć to niezwykłe zjawisko. I dostać kupony.
	\ds{} Wizgrant. Co myśmy narobili. \dm{} Dziewczyna zapłakała.
	\ds{} To się z nami zderzy! Ewakuacja! \dm{} krzyczałem do ludzi.
	\ds{} Co pan, chce sobie pan wziąć wszystkie promocje dla siebie? \dm{} Jakaś gruba kobieta się oburzyła.
	\ds{} Zje nas jednego po drugim, ratujcie się!
\end{dialogue}

Zmieniłem taktykę, pociągnąłem Galizę wgłąb tarasu.
Kalkulowałem w głowie, jak odbije się fala uderzeniowa w tej szklanej bańce i czy półsfera nie pęknie, katapultując nas wszystkich w przestrzeń.
Skuliliśmy się, obserwując ruchome cienie od zielonej gwiazdy.
Na kilka sekund przed kolizją przepychający się chyba ogarnęli sytuację, bo zaczęli przeciskać się z powrotem.

Uderzenie ścięło wszystkie siedziska i zwaliło stojących ludzi.
Dziób złomiastego atomowca wbił się szczelnie w kopułę, powietrze nawet nie uciekało.
Potem zaczął się powoli otwierać, czarna substancja wyciekła ze środka.
Trupi odór uderzył nas wszystkich po nosach.
Krzyczący ludzie umilkli, niektórzy stanęli, zmrożeni strachem.
Dało się słyszeć sapanie i kulejący krok.

I wtedy w wejściu pojawił się \emph{on}.
A raczej, \emph{oni}.

Kupa członków ludzkich, posklejanych byle jak, poprzeplatana mechanicznymi fragmentami.
Cieknąca śmiercią.
Bulgocząca.
Nieobecna.

Oczy tego monstra przeskanowały teren, skuliliśmy się jeszcze bardziej za stertą połamanego plastiku.
Ślepia skupiły się na wejściu i przerażonym zatorze powywracanych ludzi.
Ze środka kupy wysunęła się czyjaś ręka i pięć nóg.
Zaczęły posuwać cielsko w stronę panikujących osób, zostawiając plamy smolistej substancji.

Mutant zatrzymał się przy najbliżej leżącym człowieku.
Otworzył górną część, z której wylazły poskręcane kikuty.
Wzięły biedaka za nogę, jak szczypce, i poczęły wciągać do wnętrza zielonej istoty.
Za chwilę z drugiej strony wypełzła nowa noga, a mały fragment skóry wymienił się na świeży.
Nienasycony atomowiec zaczął wędrówkę po więcej papu.

Wtedy zauważyłem grubą pępowinę, która łączyła potwora ze statkiem.
Stalowy wąż był jak życiodajna rura.
Pokryta czarną mazią i cieknąca śliską substancją.
To właśnie był brakujący fragment układanki.
Gdyby udało się ją jakoś przeciąć.

A potem mi się przypomniało, w jakim celu leciałem na Tytana.
\begin{dialogue}
	\ds{} Niebieski zamyka \dm{} szepnąłem Galizie na ucho i pocałowałem w policzek.
\end{dialogue}

Zanim zareagowała, byłem już w połowie drogi do statku.

Nawet tak daleko smród zaczął wypalać mi nozdrza.
Wziąłem głęboki wdech i zanurkowałem do kosmicznej rakiety.
Właściciel chyba w ogóle mnie nie zauważył, tylko dalej pałaszował tłum.

Wewnątrz statek wyglądał równie źle, jak na zewnątrz.
Była to jedna komora, na środku której widniał pulpit sterowniczy, a wszędzie indziej walał się brudny złom.
Wąż podłączony był do wielkiego pudła w ścianie.
Rzuciłem się zaraz do niego, patrząc jak można by odłączyć zasilającą potwora pępowinę, gdy usłyszałem krzyk Galizy.
Mutant chyba ogarnął, że coś jest nie tak, bo zaczął pełznąć z powrotem, niemal się tocząc na nowozdobytych członkach.
Zrobił się szybki.

Rura była wspawana w ścianę, nie było jak jej naruszyć. Wizja skończenia jako członki zielonego gluta mnie przeraziła.
Niewiele myśląc, złapałem na konsoli za dźwignię głównego ciągu i pociągnąłem w tył.
Statek wyrwał się ze szklanej sfery, a dziób zatrzasnął się z łoskotem.
Potężny pęd odrzucił mnie w przód, a że nadal nie puściłem się wajchy, poleciała ona razem ze mną.
Kosmiczny pojazd zaraz posłusznie szarpnął w drugą stronę z wielką mocą.

Mutant uderzył z całą siłą w brudną szybę, rozplaskując czarną substancję. Zaczął się wić w agonii i próbować wejść z powrotem przez szczelinę w dziobie, który zablokował się na pępowinie.

Zaparłem się o obleśny fotel i pociągnąłem wajchę ponownie w tył, odrzucając zielonego na całą długość węża.
A potem w bok. I w tył. I znowu w przód.
Za każdym szarpnięciem rozbijał się o fragment poharatanego kadłuba.
Syczący dźwięk uciekającego powietrza tylko mnie napędzał w panice.
Ruszałem wszystkimi dźwigniami na wszystkie strony.

Zatrzaśnięcie się włazu do końca wybiło mnie z rytmu.
Na wężu została rozczapierzona końcówka ze skrawkami dawnego właściciela.
Mechaniczne macki wrośnięte były w nienaturalnie szerokie kości.
Sączyła się z nich oleista substancja.

Przez tłustą szybę widziałem oddalającą się Stalle Grande, z wielką dziurą w tarasie widokowym.
Nie mogłem dojrzeć, co z ludźmi, co z Galizą.
Ale miałem nadzieję, że zrozumiała i zamknęła drzwi na czas.

Siadłem pod obleśną ścianą, w kałuży śliskiego oleju i począłem rozmyślać nad moją sytuacją.
Przez głowę przepływały mi przeczytane opowieści.

Wstałem, aby spróbować jakoś wrócić na wycieczkowiec, spostrzegłem że zostawiłem na ścianie kawałki mojej skóry.
No tak, promieniowanie. Gdybym miał przy sobie licznik Geigera, pewnie właśnie wyleciałby poza skalę.
Żadne życie nie wytrzyma takiej ilości fal gamma.
Ale jak w takim razie ten mutant...?

Zacząłem badać pudło, do którego podłączona była rura.
Fragment zardzewiałej tabliczki znamionowej był w obcym języku.
Czy to mógł być ten legendarny Aparat?
Urządzenie naprawiające zniszczone promieniowaniem DNA? Maszyna... nieśmiertelności?
Bardzo chciałem je rozkręcić i zbadać dokładniej, ale myślę że nie wystarczyłoby mi życia na to.
Umrę prawdopodobnie za kilkanaście godzin.

Popatrzyłem na końcówkę węża.
Umrę, chyba że.

Usiadłem, zamknąłem oczy.
Ból w plecach był ostry i przeszywający.
Powoli ustępował.
Poczułem bulgotanie w żyłach.
Poczułem głód.
Poczułem przeklęty głód.

Za oknem rozpływała się Stella Grande. Głód. Tylko jeden sposób, aby go zaspokoić.
Położyłem dłoń na dźwigni ciągu.

Pociągnąłem w tył z całej siły.



