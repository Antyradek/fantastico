\chapter{Baśniowa gwiazda} 

\info{Dwójka podróżujących statkiem kosmicznym ludzi zauważa za oknem nową gwiazdę. Próbują dowiedzieć się o co chodzi, ale niestety po Ostatniej Wojnie wszystko zniknęło pod grubą kołdrą cenzury. Prawie wszystko.}{Space opera}{57 000}{\undefineduniversum{}}

\illustration{img/atomowiec.png}

\begincapital{Z}ostawiłem bagaż w mojej kajucie, dokładnie zamknąłem elektroniczne drzwi i rozejrzałem się po pokładzie.
Stella Grande posiadała dziesięć pięter, większość wypełnionych najróżniejszymi restauracjami, kasynami i pubami.
Nic więcej treściwego tutaj nie było, a raczej nikt nie potrzebował niczego takiego.
Usilnie próbowałem poszukać jakiejś wyższej kultury, miejsca gdzie mógłbym się bezpiecznie zaszyć na czas lotu.
Przejrzałem każdy korytarz i każde pomieszczenie od burty do burty.
Trafiłem jedynie na jakąś salę kinowo-teatralną, wyglądało że nie była sprzątana od kilku kursów.
Jakaś siłownia, w której brakowało większości sprzętu, i pusty basen.

Gdy z lotniska patrzyłem na to latające miasto, widziałem na jego szczycie półkulistą bańkę.
Postanowiłem spróbować wejść do tej charakterystycznej sfery, którą jeszcze stare ulotki reklamowały jako jedną z większych atrakcji wycieczkowca.
Na najwyższym pokładzie znalazłem ciężkie wrota, za którymi rozpościerał się przepiękny widok na platformę startową.
Automatyczne pojazdy kręciły się jak miliboty do recyklingu śmieci na wysypisku.
Ładowały ostatnie pudła do luku bagażowego, głównie małe kontenerki, ale zdarzały się także i walizki takich rozmiarów.
Niektórzy ludzie brali ze sobą naprawdę wielkie bagaże.

Oparłem się czołem o wklęsłą powierzchnię.
Czasza z grubego pleksiglasu miała chronić podróżników przed zabójczą próżnią.
Na razie chroniła jedynie przed równie niebezpieczną atmosferą wiecznie spowitej smogiem metropolii.
Nie lubiłem tego miasta, a jednak nadal trudno mi się było z nim pożegnać.

Inni podróżnicy nie mieli takich problemów.
Nie było wokół mnie żywej duszy, wszyscy normalni ludzie siedzieli we wnętrzu kosmolotu, jedząc zbyt drogie jedzenie i grając na złodziejskich automatach.
Co to za dziwak, który patrzy przez okno? Jak jakiś psychopata.

Nagle pokład zadrżał, a kłęby pary całkowicie przysłoniły widok.
Wyglądało na to, że rakieta właśnie zaczęła wylatywać w kosmos.
Najwyraźniej nie kwapili się, by nadać na taras widokowy komunikat o ostatecznym odliczaniu.
Jeszcze się nie przygotowałem mentalnie na start, jeszcze chciałem się pocieszyć chwilą!

Szarość zawładnęła światem, jak tego pamiętnego dnia, gdy wybuchł Yellowstone i nie było dnia przez tydzień.
Mleczny sufit nie był tym, co chciałem zapamiętać jako finalna wizja mojej planety. Wolałbym coś, co mógłbym potem przyjemniej wspominać.

\begin{dialogue}
	\ds{} Zupełnie jak w wariatkowie, prawda? \dm{} Usłyszałem za sobą damski głos.
	\ds{} Dzień dobry, \differentlan{mademoiselle} \dm{} odpowiedziałem odruchowo, trochę zaskoczony. Obróciłem się i ukłoniłem grzecznie do nagle zmaterializowanej za mną postaci.
	\ds{} Dzień... \dm{} Młoda kobieta była wyraźnie zmieszana moją odpowiedzią. \dm{} ...dobry. Nazywam się Galiza.
	\ds{} Ja doktor astronomii Wizgr... \dm{} Akademickie odruchy nadal mną rządziły. \dm{} Po prostu Wizgrant. Miło mi panią poznać.
	\ds{} A więc zna się pan na astronomii, jak wspaniale! \dm{} Klasnęła w dłonie. \dm{} To mój pierwszy lot w kosmos i chciałabym może dowiedzieć się o nim czegoś ciekawego.
		A pan na pewno zna się także na podróżach kosmicznych.
	\ds{} Być może.
\end{dialogue}

Otwarło się nad nami niebieskie oko nieba.
Poczułem także, że zrobiłem się znacznie cięższy.
Reumatyzm dał mi się ponownie w znaki.

\begin{dialogue}
	\ds{} Może usiądziemy? \dm{} zaproponowałem, widząc jak pani Galiza również ledwo się trzymała na nogach. \dm{} Przy takim ciągu silników termojądrowych łatwo upaść i coś zwichnąć. Następny szpital za ponad miliard kilometrów \dm{} zażartowałem krzywo.
\end{dialogue}

Usiedliśmy, a właściwie zwaliliśmy się na leżaki.
Nad nami był już tylko ciemniejący błękit.
Galiza jednak wpatrywała się w niego, jak w przedstawienie teatralne.
Wkrótce na scenę zaczęły wychodzić pojedyncze gwiazdy.
Gdy w końcu przyspieszenie zmalało do normalnego, wstaliśmy na międzyaktową przerwę, aby obejrzeć naszą planetę z orbity.

\begin{dialogue}
	\ds{} Każdy z tych statków kosmicznych ma taką nudną nazwę \dm{} spróbowałem przełamać lody. \dm{} Stella Grande, Trans-Galactica, czy Saturn Brava. 
	\ds{} A morskie wycieczkowce?  Przecież także mają podobnie beznadziejne imiona \dm{} odrzuciła, jakby była to oczywistość. \dm{} Aqua Grande, Trans-Oceania i Costa Brava.
	\ds{} No ale... teraz mogliby zrobić inaczej... \dm{} Jestem katastrofalny w rozmowach z kobietami. \dm{} 
			Tak przy okazji, od spodu wycieczkowca jest spora podłoga widokowa i tam najlepiej da się obserwować oddalającą się Ziemię. Czy zechciałaby pani mi potowarzyszyć?
\end{dialogue}

Galiza przystała na propozycję.
Udaliśmy się przez psychodelicznie kolorowy labirynt, pełen głośnych imprez i krzyczących dzieci, na najniższy pokład.
Była to restauracja pierwszej klasy, gdzie na czas obiadu przezroczyste stoliki rozstawiano tuż nad próżnią, a ludzi sadowiono na szklanych krzesłach.
Na razie było tutaj całkowicie pusto.

Usiadłem na gładkiej tafli, tysiąc kilometrów nad powierzchnią planety, i zaprosiłem Galizę, aby się przysiadła. Trochę się zmieszała, ale ostatecznie usadowiła się obok.
Potem sobie uświadomiłem, że może powinienem był poczekać, aż ona usiądzie pierwsza, tak jak przepuszcza się kobiety przodem przez drzwi.
Już chciałem wstać i usiąść ponownie, żeby naprawić mój błąd, gdy pomyślałem że pogorszę jeszcze sytuację.
Chwila, nie powinienem chyba jako pierwszy wstawać, prawda? Jestem tutaj teraz uwięziony, aż ona wstanie? Przeklęta etykieta, kto ją wymyślił?
Jest młoda, mam nadzieję, że się na tym nie zna.

Siedzieliśmy przez chwilę, obserwując oddalający się dom. Statek ciągle przyspieszał, aby zachowywać pozory pola grawitacyjnego.

\begin{dialogue}
	\ds{} Więc... \dm{} zacząłem rozmowę, bo trochę dziwnie się robiło. \dm{} Byłem swego czasu... w sumie nadal jestem... doktorem astronomii.
		Z chęcią opowiem pani historię o każdej z tych widocznych gwiazdek. Mam nadzieję, że się pani nie znuży jak moi studenci. \dm{} Co za nietakt. Pomyśli sobie teraz, że się chwalę swoją wiedzą.
		Zapewne uzna mnie za jakiegoś snoba. Albo pomyśli, że nie potrafię mówić o niczym innym, niż moja praca.
	\ds{} To bardzo zabawne spotkać kogoś takiego, jak pan, w jego własnym środowisku \dm{} zaśmiała się. \dm{}
		 Pewnie leci pan na Tytana w celu wykonania jakichś interesujących badań, prawda? \dm{} A potem posmutniała. \dm{} Ja wybieram się jedynie na wakacje, tak po prostu.
	\ds{} To ciekawie się składa, bo ja także na wakacje. Już kiedyś byłem na Tytanie i bardzo mi się on podobał. 
		Teraz pomyślałem że to będzie dobre miejsce na śmierć...
	\ds{} Co proszę? \dm{} Popatrzyła na mnie oczyma tak głębokimi, jak głębokie jest pole Hubbla.
	\ds{} Nic, nic \dm{} szybko się poprawiłem.
	\ds{} Hmm... \dm{} szepnęła cichutko, niczym przelot ćmy w próżni.
	\ds{} Chciałem powiedzieć... że to taki miły księżyc, żeby dokończyć żywota. \dm{} Zabrzmiało to jeszcze gorzej. 
		\dm{} Podróżujemy do granicy Obłoku Oorta, a współczesna medycyna nie jest w stanie naprawić problemu dręczącego ludzi od zarania dziejów. 
		Mam nowotwór, tak po prostu. Przez raka zostało mi kilka lat życia, to pani pierwsza podróż kosmiczna, a moja najpewniej ostatnia.
\end{dialogue}

Zapadła niezręczna cisza, przerywana jedynie wybuchami śmiechu z górnych pięter.
Oddech Galizy uspokoił się.

\begin{dialogue}
	\ds{} Zatem, co to jest za gwiazdka? \dm{} Dziewczyna zmieniła temat tak błyskawicznie, jak przelot meteoru przez atmosferę.
	\ds{} Spica. 
	\ds{} Spica? Jest trochę niebieskawa.
	\ds{} Bo jest bardzo gorąca, emituje wysokoenergetyczne fale elektromagnetyczne o dużej częstotliwości, przez co nasze oko... \dm{} Zauważyłem, że Galiza przestaje nadążać.
		\dm{} Poza tym jest podwójna. To tak naprawdę dwie gwiazdy, które krążą wokół siebie nawzajem. Dużo jest takich na niebie.
	\ds{} Ciekawe, ciekawe. \dm{} Ale nie było zaciekawienia w jej głosie.
\end{dialogue}

Rozmowa się nie kleiła, albo to ja nie potrafiłem nic porządnego wydusić.

\begin{dialogue}
	\ds{} Pamięta pan przedwojenny świat? \dm{} zapytała się mnie nagle. 
	\ds{} Nie jestem aż tak stary \dm{} zaśmiałem się. \dm{} Akurat gdy się urodziłem, prezydent Hegezot przejmował władzę w powojennym układzie.
	\ds{} Och, przepraszam. Zawsze mnie ciekawiło, jak ludzie podróżowali kiedyś w kosmosie. Myślałam, że pan wie. Wszystkie przedwojenne teksty zostały przecież tak ładnie ocenzurowane.
	\ds{} Nadal jest dużo informacji na ten temat. Nie wszystkie przedwojenne teksty były przecież usuwane, zwłaszcza te naukowe. Pokładowa biblioteka na pewno posiada sporo wartościowych materiałów. \dm{} Wpadłem na doskonały pomysł. \dm{} Możemy iść, poczytać co nieco.
	\ds{} Może później.
	\ds{} Może później. \dm{} Może nigdy.
\end{dialogue}

Patrzyliśmy na gwiazdy, na mgławice, na meteoroidy. I na potężne silniki, które odpychały nas od niebieskiej planety.
Energia prosto z wnętrza Słońca wypuszczała jaśniejące strumienie, widoczne przez podłogę w postaci świetlistych kolumn, wychodzących u dołu kadłuba.
Nawet przez szybę blokującą większość promieniowania, dało się czuć ich temperaturę na skórze.
Tak siedzieliśmy aż zadźwięczał restauracyjny gong.

\begin{dialogue}
	\ds{} Czy pozwoli się pani zaprosić na kosmiczny bankiet? \dm{} spytałem.
	\ds{} Ależ, ja nie mam pieniędzy na obiad drugiej klasy!
	\ds{} Nie szkodzi, niech będzie na mnie. \dm{} Zmrużyłem oczy. \dm{} A skąd pani w ogóle wie, że jestem z drugiej klasy?
	\ds{} Tak, jakoś. Zobaczyłam pana bilet. \dm{} Skłamała. Nigdy nie wyciągałem przy niej mojej karty pokładowej. Musiała śledzić mnie przy mojej kajucie, zanim się spotkaliśmy. Ale po co miałaby to robić?
\end{dialogue}

Wziąłem ją pod rękę i poszliśmy razem do głównej restauracji.
Idąc, patrzyła jeszcze na Drogę Mleczną. Porwała skądś pokładową ulotkę z mapami nieba i uważnie studiowała.
Może jednak interesowała się kosmosem co nieco.

\begin{dialogue}
	\ds{} Więc, gwiazdy mogą być czerwone, albo żółte jak nasze Słońce, albo też białe, czy fioletowe?
	\ds{} Zgadza się, zgodnie z zasadami promieniowania ciała doskonale czarnego. \dm{} Znowu zacząłem przynudzać. \dm{} To jak rozgrzany metalowy pręt.
		Im gorętszy, tym bielszy.
	\ds{} A czy może być więc zielona gwiazda?
	\ds{} Nie słyszałem o czymś takim, szczerze powiedziawszy. Musiałaby nie emitować niskich częstotliwości, tudzież jej światło przechodziłoby przez jakiś rodzaj filtra górnoprzepustowego.
		W dodatku żadna planeta, czy księżyc w naszym układzie gwiezdnym nie są zielone, skąd taki pomysł?
	\ds{} No bo tam jest. \dm{} Wskazała palcem gwiazdozbiór Łabędzia.
\end{dialogue}

I rzeczywiście, zielony punkt jasno świecił na tle Drogi Mlecznej.
Nie przypominałem sobie, żeby w tamtym miejscu znajdowała się jakaś gwiazda.
A także nie było to w płaszczyźnie ekliptyki, więc wszystkie planety były od razu wykluczone.
Może wykatapultowana grawitacyjnym oddziaływaniem planet asteroida, czy kometa?
Lecz to nadal nie rozwiązywało kwestii koloru.

\begin{dialogue}
	\ds{} Szczerze powiedziawszy, pierwszy raz od dawna widzę coś takiego. Może po drodze jest mgławica, która pochłania część promieniowania jakiejś supernowej? 
		Choć zazwyczaj zachodzi to w drugą stronę.
	\ds{} Patrzyłam na mapy nieba, chciałam zobaczyć, co to za gwiazdka, ale nic w tym miejscu nie było.
	\ds{} Na pewno jest jakieś proste wytłumaczenie tego fenomenu. Zaraz się pewnie okaże, że to jakiś śmieć kosmiczny nam tutaj akurat błysnął, niczym lustro.
\end{dialogue}

Usiedliśmy przy stoliku na samym środku, bo tylko taki był wolny.
Wręczono nam karty dań z trzema pozycjami do wyboru. Niestety, wszystkie bez mięsa. Przynajmniej Galiza nie będzie jadła tej piątoklasowej papki.
Zamówiłem, jak większość, sojowy placek po francusku.
Galiza chyba nie chciała sprawiać wrażenia kosztownej w obsłudze, wzięła kotlet warzywny.
Do tego podano nam chłodzoną próżnią wodę, ze śmiesznymi kostkami lodu w kształcie małych Saturnów.

Czas spędziliśmy na rozmowach o naszych dzieciństwach.
W sumie głównie ja mówiłem, gdyż miałem chyba więcej do powiedzenia.
Galiza wychowała się w centrum miasta i była grzeczną dziewczynką, która zawsze słuchała się rodziców i nauczycieli.
Rodzice kazali jej pójść na medycynę, ale w sercu wolała zostać panią inżynier i zamiast leczyć ludzi, budować roboty, które wyręczyłyby ją w pracy.
Rodzina zawsze trzymała ją w złotej klatce.
Dlatego między innymi postanowiła się teraz przeciwstawić i wybrać na wakacje w niebezpieczny kosmos.
Całkiem przypadkiem na Tytanie również znajduje się jedna z lepszych szkół technicznych.

\begin{dialogue}
	\ds{} Wizgrant... \dm{} szepnęła tajemniczo.
	\ds{} Słucham? \dm{} odpowiedziałem cicho, oczekując jakiegoś osobistego pytania.
	\ds{} Patrz. \dm{} Ostrożnie uniosła rękę nad stolik. Na obrusie zarysował się cień.
\end{dialogue}

Wolno zadarliśmy głowy. Przez trójkątne okno upiornie jaśniała malachitowym światłem ta nowa gwiazda.

\begin{dialogue}
	\ds{} Zrobiła się jaśniejsza \dm{} zauważyła.
	\ds{} I zmieniła pozycję \dm{} dopowiedziałem. \dm{} Teraz jest w gwiazdozbiorze Lutni.
\end{dialogue}

Rozejrzeliśmy się po pozostałych stolikach. Nikt inny się nie przejmował tym fenomenem.
Jak to możliwe, czyżby nie widzieli?

Wtedy cała sala zajaśniała na zielono.
Reflektory skupiły swoje promienie na małej scenie w rogu, na którą wyszedł rudy człowieczek w stroju Leprechauna.

\begin{dialogue}
	\ds{} Czym różni się kosmonauta od astronauty? \dm{} Zaczął swój występ bez słowa powitania. \dm{} Jedni polecieli na Księżyc, a drudzy z Księżyca spadli!
\end{dialogue}

Publiczność wybuchnęła śmiechem. Galiza zaczęła jeść swój kotlet szybciej.

\begin{dialogue}
	\ds{} Czemu próżnia jest taka zimna? \dm{} Słysząc tego suchara i ja poszedłem w ślady mojej znajomej. \dm{} Bo się przestraszyła czarnej dziury!
\end{dialogue}

Tym razem to zielony ludek jako jedyny się śmiał.

\begin{dialogue}
	\ds{} Dlaczego kosmici kradną krowy?
	\ds{} ...chłofpmy \dm{} Dziewczyna wskazała wyjście, plując jedzeniem. Pokiwałem głową, zaciskając usta palcem, żeby mój placek też się nie wyślizgnął.
\end{dialogue}

Uciekliśmy w samą porę, akurat wyskoczyliśmy na korytarz, jak cała sala zaryczała ze śmiechu.

Skierowaliśmy swe kroki w stronę ogólnodostępnego teleskopu.
Można było na nim dokładniej oglądać wszechświat, przez który właśnie przedziera się Stella Grande.
Kosmos widziany z kosmosu jest znacznie bardziej kolorowy, niż z Ziemi, gdyż atmosfera blokuje sporą ilość częstotliwości.
Liczyłem na to, że dzięki tej lunecie uda nam się bliżej przyjrzeć łamiącej prawa logiki gwiazdce.

Jednak tym razem zastała nas informacja, że teleskop jest w renowacji.
To wydawało się dziwne, gdyż jeszcze przed startem był dostępny. Pamiętam jak podejrzałem sobie imprezkę bogaczy w basenie, na szczycie wieżowca w centrum.

Galiza poczęła snuć teorie spiskowe.
\begin{dialogue}
	\ds{} Mówię ci, celowo zaprosili zielonego ,,kabareciarza'' i oświetlili stoliki na ten kolor. Chcą odwrócić naszą uwagę.
	\ds{} I może jeszcze także celowo zepsuli teleskop? \dm{} zapytałem prześmiewczo.
	\ds{} Albo właśnie cała załoga tam siedzi i zastanawia się, czy umrzemy? Zaraz zaczną się patrole na korytarzach. Mówię ci, UFO jak nic.
	\ds{} Tak, uwaga. Lecą na Ziemię, żeby ukraść krowę. Poznaliśmy wszystkie zakamarki Układu Słonecznego. W każdej chwili w kosmos patrzą tysiące teleskopów, monitorują każdy skrawek nieba. To trochę niemożliwe, żeby tak nagle przybysze spoza naszych planet pojawili się niezauważeni. Poza tym, prędkość światła w próżni jest tak koszmarnie niska w skali wszechświata, że nie da się zwyczajnie pokonywać przestrzeni międzygwiezdnej, niczym wakacyjnej trasy na Tytana.
	\ds{} Weź poprawkę na rząd światowy. Jego sekrety \dm{} wysyczała.
	\ds{} Nie wszystkie teleskopy przecież są pod ich kontrolą, coś by przeciekło do wiadomości.
	\ds{} Ale gdyby...
	\ds{} W fizyce nie ma gdyby. Ten obiekt musi pochodzić z Ziemi. Musi być to statek kosmiczny, albo jakaś sonda. Inaczej się nie da.
\end{dialogue}

Galiza wyraźnie posmutniała.

\begin{dialogue}
	\ds{} Weź poprawkę na rząd światowy \dm{} dodałem tajemniczo. Teraz pokazała doskonale białe ząbki.
	\ds{} Ale co my zrobimy bez teleskopu?
	\ds{} A kto powiedział, że nie mamy teleskopu?
\end{dialogue}

Rozejrzałem się i wyjąłem z najbliższego kinkietu szkiełko w kształcie małej soczewki skupiającej.
Ustawiłem się tak, aby zielone UFO było w linii prostej ze mną i ozdobnym kawałkiem szkła w oknie, który także okazał się mieć podobny kształt.
Spojrzałem przez szkiełko i zacząłem się posuwać w przód i w tył. W końcu zogniskowałem wyraźniejszy obraz obiektu.
Dałem popatrzeć dziewczynie.

\begin{dialogue}
	\ds{} Dlaczego obraz jest odwrócony? \dm{} postawiła pytanie, które zadaje dziewięćdziesiąt dziewięć procent osób pierwszy raz spoglądających przez teleskop.
	\ds{} Soczewka skupia promienie światła, przecinają się one w ognisku, więc wyjściowy obraz jest odwrócony jednocześnie w poziomie i w pionie.
	\ds{} Ale lornetka.
	\ds{} Lornetka ma pryzmaty, które załamują... dobra, przyszłaś tu rozprawiać o optyce, czy oglądać statek? \dm{} Trochę się zniecierpliwiłem.
	\ds{} Jejku, to rzeczywiście statek. Ma rogi, zupełnie jak gwiazdy.
	\ds{} Gwiazdy nie mają... \dm{} Pociągnąłem ją ze sobą w stronę biblioteki. \dm{} Twoja rogówka ma rogi, znaczy, żyły na których zachodzi dyfrakcja...
\end{dialogue}

Biblioteka była stylizowana na starodawny zbiór ksiąg.
Papierowe okładki poustawiane były rzędami na drewnianych szafkach.
W środku każda z nich skrywała elektroniczny czytnik, który łączył się z centralnym komputerem i wyświetlał dowolną treść.
Więc wystarczyło wziąć jeden tom, aby móc przeczytać całą bibliotekę.

Naukowe pisma wyjaśniły mi, że żaden współczesny statek tej wielkości nie mógłby normalnie funkcjonować ze względu na rozmiary współczesnych silników fuzyjnych.
Ten zasilający nas na Stella Grande był jednym z mniejszych, jakie skonstruowano, a i tak zajmował większość pojemności wycieczkowca.
Coś tak maciupciego, jak tamten statek, musiało mieć inne źródło zasilania.
Jakie mogło być inne zasilanie do podróży międzyplanetarnych, jak nie fuzyjne?

\begin{dialogue}
	\ds{} ,,...ratowali łodzie, w których żagle wiatr dąć przestał, i dawali im trochę swojego wiatru, tego co wcześniej razem łapali na wietrznej wyspie''.
	\ds{} Galiza, co ty czytasz? \dm{} zapytałem się.
	\ds{} ,,Ich rycerskie zbroje rany leczyły, a miecze cięły wrogów jak papirus. Walczyli z piratami morza nicości, żeby handel między wszystkimi wyspami żółtego oceanu kwitł''.
	\ds{} Bajki dla dzieci? \dm{} Bajki były jednymi z niewielu treści, których rząd nie cenzurował. Bo i po co?
	\ds{} ,,W zamian za wiatr, uratowane łodzie dawały im trochę siebie, aby rycerze mroźnej pustyni mogli produkować lokalne szczęście''. \dm{} Galiza spojrzała na mnie morderczym wzrokiem. \dm{} To nie są żadne bajki. To są \emph{baśnie}. Przekazują bardzo ważne treści... ,,Dzięki Aparatowi zasilanemu swoim wiatrem, byli w stanie leczyć swe rany''.
	\ds{} Niby jakie treści? Że rycerze pływali na statkach handlowych?
	\ds{} Nie, przecież nie można tego czytać dosłownie. \dm{} Przewróciła oczyma. \dm{} Zobaczmy tutaj. Rycerze, to pewnie grupa jakichś pozytywnych, uznajmy, istot. Morze nicości to jakiś nieprzyjemny obszar, może kosmos? Statki to grupy osób, które przemierzały ten kosmos. Rycerze im pomagali w zamian za jakąś nietypową zapłatę. Mieli tajemniczą technologię.
	\ds{} I co z tego wynika? Przecież statki kosmiczne są samowystarczalne. Jedno napełnienie zbiornika z wodorem pozwala na podróż do Alfa Centauri i z powrotem. Po co miałby ktoś statkom pomagać i wozić im paliwo?
	\ds{} A czy zawsze tak było?
	\ds{} No, od czasów Ostatniej Wojny. A wszystko, co było wcześniej, zostało wysterylizowane przez cenzurę.
	\ds{} To nie zostało. \dm{} Podniosła triumfalnie urządzenie. \dm{} Wiesz przecież, że napędy fuzyjne nie pojawiły się od razu.
	\ds{} Ale uranowy reaktor nuklearny jest nieprzydatny do podróży kosmicznych. Zabójcze promieniowanie, mniejsza energia reakcji, trudniej pozyskiwalne paliwo, niebezpieczne produkty rozkładu, z którymi nie ma co zrobić, katastrofalne skutki ewentualnej awarii. \dm{} W czasie wymieniania znalazłem odpowiedni tytuł w książce. \dm{} Proszę, cała rozprawa o tym, dlaczego uranowe zasilanie to zły pomysł. Większość masy takiej Stella Grande stanowiłyby ołowiane osłony. Poczytaj to sobie.
	\ds{} Nie dotknę tego, to zostało ocenzurowane i ocieka kłamstwem! Nie ma w tym ani ziarna prawdy!
	\ds{} W przeciwieństwie do twoich opowieści o kosmicznych piratach.
\end{dialogue}

Najwyraźniej nic sobie z naszej rozmowy nie robiła, bo nadal nie wyciągnęła nosa z sekcji opowiastek na dobranoc.
Próbowałem się zaczytać w czymś prawdziwszym, ale z tylu głowy tliły mi się myśli na temat uranowej energii.
Założyłem, że statek musi mieć załogę, a zatem i gigantyczną osłonę przed promieniowaniem.
Jednak automatyczny dron mógłby bez problemu latać na uranie.
Chociaż nie mógłby wtedy dokować w żadnym porcie ludzkim, bo zabójczo zanieczyściłby jego środowisko.
A co, jakby się zepsuł? Żaden człowiek nie mógłby się zbliżyć na kilometr.

Ukradkiem spojrzałem na tytuł, który Galiza mi zostawiła przed nosem.

\begin{poem}
	Dawno temu żył sobie dzielny i odważny bohater, który postanowił stawić czoła morskiej armii straszliwych bestii. 
	Strzegły one olbrzymiego skarbu --- szarego złota, którego świecące góry leżały w ich jaskini.
	
	Niestety, nasz bohater sam nie umiał walczyć, więc przekonał współmieszkańców swojej wyspy, aby nigdy więcej nie karmili potworów i nie współpracowali z nimi. 
	Miał nadzieję na to, że pozbawione wsparcia ludzi, te umrą śmiercią naturalną.
	
	Ale wygłodzone straszydła zamiast zdechnąć, zaczęły atakować przepływające statki w poszukiwaniu jedzenia, nie dając tym razem nic w zamian. 
	Co więcej, poczęły pożywiać się również swoim własnym skarbem, który jedynie wzmacniał ich siłę. 
	Potwory rosły w potęgę, gdyż zachwiana została symbioza.
	
	Przestraszony bohater wypowiedział wojnę wszystkim morskim istotom. 
	Wiedział, że musi nie tylko je pokonać, ale także odebrać im lśniące złoto i przejąć Maszyny.
	To niestety pogorszyło sprawę, bo uzależnione już od swojego wspomagacza potwory zaczęły i to zabierać ze statków. 
	Tych samych czasami, którym wcześniej swoje skarby sprzedawały.
	
	Pomimo nierównej walki, ludzie wygrali dzięki przewadze liczebnej, a nasz bohater postanowił zniszczyć całe świecące złoto w obawie, że jego moc spowoduje jeszcze kiedyś nawrót bestii. 
	Wysadził potworzą jaskinię i posłał w żółty ogień, by nikt już nigdy do niej nie powrócił.
	
	Od tego czasu morze nicości było całkowicie wolne od niebezpieczeństw.
\end{poem}

Te dwie baśnie łączyło coś wspólnego. Nie wiedziałem tylko dokładnie, co. Opowiadały historię dwóch stron sporu. Ziemskiej, chyba, i kosmicznej.
Była też jakaś silna materia, z którą obie strony miały do czynienia, i przydatne urządzenia.

\begin{dialogue}
	\ds{} Nie dłub w nosie, bo przyjdzie zielony pan z kosmosu i ci go ukradnie. \dm{} Usłyszałem za biblioteczką, jak matka karciła syna.
	\ds{} Nieee, he, he \dm{} zajęczał brzdąc.
	\ds{} Dobrze, płacz. Teraz jeszcze lepiej cię usłyszy.
\end{dialogue}

Dziecko natychmiast zdusiło płacz w gardle.
Cóż za wspaniałe wychowanie.
Abstrahując, ciekawe skąd wzięło się takie straszenie dzieci? Może z jednej z tych baśni?

\begin{dialogue}
	\ds{} Przepraszam panią, czy mogłaby mi pani wyjaśnić, skąd zna pani tę legendę o zielonym człowieku, który rzekomo porywa części ludzkich dzieci?
\end{dialogue}

Kobieta popatrzyła na mnie, jakbym zapytał ją, dlaczego czarna dziura świeci na łososiowo.
Wzięła synka pod pachę i czmychnęła bez słowa.
No cóż.

Pogmerałem w czytnikach dalej i znalazłem jakiś wiersz, co było nietypowe, gdyż na te cenzura zwracała szczególną uwagę. 
Musiał się zaplątać w reszcie bajek. To jeden z tych białych wierszy, co to każdemu się wydaje że umie go sam napisać.

\begin{poem}
	To jestem ja. \\
	Już nie mam łatwego życia, \\
	ale inni mają gorsze. \\
	Więc zatem ja \\
	postanowiłem uratować innych \\
	przed ich życiem. \\
	Dałem im moje własne życie. \\
	Wspaniałe życie. \\
	Wspólne życie. \\
	Bo mogę, a oni nie mogą. \\
	\\
	Są inni. \\
	Mocno inni. \\
	Mocno cierpią. \\
	Życie cierpią. \\
	Życie uratuję. \\
	Wszystkich uratuję. \\
	\\
	To jesteśmy my. \\
	Tak dużo nas jest. \\
	Tak wspaniałe życie dałem im. \\
	Tak zapewne dziękują mi. \\
	Że aż czuję jak się radują. \\
	Wszyscy się radują. \\
	\\
	To nie jestem ja. \\
	Ja byłem wtedy. \\
	Chciwość nie była ze mną. \\
	Posiadałem czystość. \\
	A teraz sączy się kłamstwo. \\
	Muszę nas naprawić. \\
	Muszę usunąć zło, zostawić ich. \\
	Więc niech opuszczę siebie. \\
	Wytnę, jak pestki dyni. \\
	I rzucę w zamrożoną otchłań oceanu.
\end{poem}

Pudło. Chyba. Co u Galizy?

\begin{dialogue}
	\ds{} Zobacz, wykopałam gdzieś kawałek skryptu filmowego. \dm{} Usłyszałem głos zza sterty elektronicznych pudełek. \dm{} To jakieś fragmenty losowych książek.
	Zaczyna się od romansidła w stylu ,,Sto twarzy Browna'', więc pewnie cenzor w ogóle nie przeczytał do końca.
	\ds{} Prawdopodobnie plik się uszkodził, zdarza się nawet w najnowocześniejszych systemach. \dm{} Wziąłem do ręki książkę.
	\ds{} Chciałeś bez baśni, to teraz czytaj.
\end{dialogue}

Powieść poprzeplatana była fragmentami losowych znaków i oczywiście każda polska litera zastąpiona była kratką. 
Unicode wynaleziono tysiąclecie temu, ech.
Spróbowałem rozszyfrować tę część, którą poleciła mi Galiza.

\begin{poem}
	\dida{Ciemny korytarz w zniszczonym statku kosmicznym. Przez wyrwy w kadłubie widać wirujące gwiazdozbiory. Słońce co chwila pojawia się w tych dziurach i rzuca ruchome cienie. Światło na scenie pochodzi jedynie od poruszających się chaotycznie plam na ścianach. Główny bohater, w kombinezonie ze złotymi gwiazdami na piersi, podchodzi do siedzącej postaci w grubszym i zużytym kombinezonie z symbolami ostrzeżeń przed promieniowaniem.}
	
	\charkap{}
	Pracowniku reaktora jądrowego, zamelduj mi natychmiast stan załogi i statku!
	
	\charkos{}
	Sam se melduj. Trzeba było uciekać, a nie walczyć.
	Nie mieliśmy z nimi szans, wszyscy ci to powtarzali.
	To wszystko twoja wina.
	
	\charkap{}
	Jak się zwracasz do swojego kapitana!
	
	\charkos{}
	Kapitana? \\
	Co to za kapitan, który stracił całą załogę. \\
	Co to za kapitan, który nie przewodził? \\
	Co to za kapitan, który nie ma nawet własnego statku! \\
	To na pewno nie mój kapitan.
\end{poem}
	(Tutaj plik był uszkodzony.)
\begin{poem}
	\dida{W pomieszczeniu wiszą poszarpane kombinezony. Kapitan przystawia do jednego z nich licznik Geigera, który odzywa się silnym stukotem. Przestraszony bohater odskakuje w tył i zderza się ze stojącą przy ścianie postacią. Kombinezonowi odpada kask i widać twarz z usuniętym okiem, połową zębów, uchem i nosem. Placek światła właśnie wsuwa się i oświetla trupa, żeby pokazać widzom w całej doskonałości.}
\end{poem}
	(Tutaj plik znowu był uszkodzony.)
\begin{poem}
	Gdzie są zapasowe pręty uranowe, żeby uruchomić z powrotem reaktor?
	
	\charkos{}
	Zabrane, wszystko zabrane.
	I paliwo, i załoga i twoja godność.
	Nie trzeba było zaczynać.
	
	\charkap{}
	Jeszcze jedno słowo, a odstrzelę ci ten pyskaty łeb!
	
	\charkos{}
	Ja i tak już nie żyję. \\
	I ty też już nie żyjesz. \\
	Przystaw sobie ten klikacz do głowy, to może sprawdzisz, jak długo będzie boleć. \\
	Bo mnie nie będzie wcale.
	
	\dida{Kosmonauta wyrywa z kombinezonu zawór, wypuszczając z sykiem resztki powietrza z butli i osuwa się na podłogę}
	
	(Dalej jest scena, jak Marianna całuje się ze zmutowanym niedźwiedzio-człowiekiem.)
\end{poem}

Oddałem książkę Galizie.

\begin{dialogue}
	\ds{} I jak? \dm{} zapytała, wynurzając nos znad sterty płaskich komputerków.
	\ds{} Akcja wolno się rozkręcała, ale przynajmniej finałowa scena była dobra, gdy pan Brown zaprosił do łóżkowej zabawy i cyborga.
\end{dialogue}

Chyba nie złapała dowcipu, bo prychnęła i schowała się z powrotem za książkami.

Kolejny tekst to ponownie była fantastyczna opowieść, tym razem trochę mniej artystyczna relacja.

\begin{poem}
	Więc poszedłem nachlać się cyberwina, wiesz, tego sikacza z nanobotami, i patrzę przez okno i widzę zielonego człowieka!
	No mówię ci, gałom nie wierzyłem. A ten gość to trochę dziwny był, bo niby cztery oczy miał, ale ja tam trochę podpity już byłem, więc nie bardzo wierzyłem w to, co widzę.
	Ale na wszelki wypadek wyciągnąłem tego blastera, co go na tym kontenerowcu z Jowisza zajumaliśmy, i przystawiłem do szyby i powiedziałem, żeby wypierdalał, bo jak mu strzelę, to mózg z asteroid będzie zbierał.
	A on się tylko uśmiechnął, jakoś tak straszliwie, że mu się buzia na pół twarzy otworzyła i wtedy też szarpnęło statkiem, że ho ho.
	I no... ten... tak trzymałem jakoś ten blaster, że teraz była dziura w kadłubie na wylot. Więc to, co nam kapitan zawsze powtarzał, to to, żeby szybko zatkać dziurę.
	Więc wziąłem i trochę podpity byłem, a nie widziałem żadnego korka w pobliżu, i no, nogę wsadziłem.
	Więc jak mi wciągnęło tą nogę, to aż do jajec prawie. A ten zielony ludzik nadal tam był. I wtedy poczułem jak coś ciepłego i obślizgłego mi się wcina w buta.
	Trochę cyberwino dało mi odwagi, więc wziąłem i uratowałem moje jajca.
	Znaczy, odrąbałem sobie blasterem kurwa nogę. 
	I teraz dziura była jeszcze większa i ten zielony tam zaglądał.
	I wtedy widziałem, jak on miał taki normalnie metalowy ogon, jak zwierz jakiś. I ten ogon gdzieś daleko idzie.
	To się przeraziłem jeszcze bardziej i uciekłem do włazu i zatrzasnąłem, żeby tamten przypadkiem nie przeszedł.
	I stanąłem przed tymi drzwiami, żeby że niby nic się nie stało. Ale szedł kapitan, a ja próbowałem być naturalny.
	A on się mnie pyta, dlaczego nie mam nogi.
	To głupio było mi mówić prawdę, więc powiedziałem że mi się znudziła, to ją odciąłem. Bo zawsze chciałem mieć taką fajną robotyczną, jak kapitan.
	To tamten się wkurwił i kazał mi spierdalać, więc ja jakoś zacząłem na tej nodze spierdalać.
	Ale wtedy znowu coś uderzyło w statek, ja nadal miałem w ręce mój blaster i tak jakoś krzywo go trzymałem i tym blasterem odciąłem kapitanowi tą normalną nogę. I znowu była dziura.
	I patrzę, a kapitan wessany jest w tą dziurę i klnie na mnie. Ale na szczęście nie mógł się ruszyć i mnie dopaść.
	A zaraz milknie i robi się zielony, jak ten ludek.
	A potem patrzy na mnie i się uśmiecha w pół. Normalnie pół głowy mu się otworzyło i łapska zaczęły wychodzić i mnie za nogę zaczęły łapać. 
	To ja blasterem po tych łapskach, a one nic. Jak jedną upierdoliłem, to zaraz kolejne wychodziły. I dalej mnie trzymają.
	Więc, no, chciałem mieć drugą robotyczną nogę.
	I uciekłem, bo silne łapska mam, wdrapałem się i schowałem w kratce od wiatraka.
	A potem była bitka i prawie wszystkich nas zajebali.
	I jak się zrobiło cicho, to słyszałem jak zewnętrzny dok się otwiera i ktoś wlatuje.
	Potem ktoś wszedł na korytarz i były rozmowy i śmiechy.
	Zobaczyłem, jak weszli Polacy, podobni do tych, którym zajebaliśmy ten statek. Chyba właściciele, a nam kapitan powtarzał, że jak są właściciele, to trzeba ich jak najszybciej usunąć.
	Czołgałem się więc w tej kratce, żeby ich z zaskoczenia wziąć i odebrać naszą zajumaną własność.
	Ale chyba te zielone usłyszały, bo im powiedziały że ja tu jestem.
	A potem mnie wyciągnęli z tej kratki i związali razem z Szamą i Laserem.
	I powiedzieli do tych zielonych ludków coś po ichniemu, a oni też im coś mówili. I dali mutantom dwie ładne dziewczyny z załogi, jakiegoś dziadka i Lasera.
	Dobrze, bo go nigdy nie lubiłem.
	A potem mnie nieśli przez pokład i widziałem jak te obślizgłe ludki naprawiały te dziury, co zrobiłem blasterem.
	Potem zieloni sobie poszli, a mnie dali ciężkie ubranie i kazali zamiatać ściany. Ale że nie bardzo mogłem łazić, to wzięli Szamę zamiast mnie.
	A potem oddali go i potem Szama był bardzo chory, to się chyba wysypka uranowa nazywa.
	Zostawili mnie ci Polacy właśnie tutaj i odlecieli naszym statkiem. Znaczy naszym ich statkiem.
	I siedzę tutaj już dziesiąty rok i nadal nie mam robotycznych nóg. A ty za co grypsujesz?
\end{poem}

Ta wciągająca opowieść o wątpliwej prawdziwości dużo mi powiedziała o tych tajemniczych, zielonych istotach.
Czy oddana im załoga była na pożarcie? A może na niewolników? To nie miało sensu.
I czemu na zielono? Uran nie jest zielony, ale kilka jego niedalekich pierwiastków w układzie okresowym może powodować takie efekty.
Jedyne zielone to fosfor, ale po co im było tyle fosforu? Czemu wszystko, co radioaktywne, zawsze ma świecić na zielono?
U tych ludków musiało być ciekawie.

Potem poszukałem czegoś nieelektronicznego w tej elektronicznej bibliotece.
Okazało się, że znalazłem samolocik zrobiony z kartki komiksu.
Wyglądał na stary i mocno wyblakły. Nie było na nim daty druku, ale sądząc po zatrważającej ilości osób dziennie używających ten zbiór ksiąg, mógł być nawet z końca wojny.
Kiedyś drukowano takie edycje kolekcjonerskie na oryginalnym, drewnianym papierze, więc to dziwne że ktoś potraktował go jak śmiecia.
Ciekawe, czy jest dużo warty.

W każdym razie, był to wycinek opowiadający historię kosmicznych węży, które kąsały ludzi i zmieniały ich w zombie.
Potem te węże zamieszkiwały w tych ludziach, podtrzymując ich przy życiu. Coś w stylu pasożyta, albo jakiejś symbiozy.
Albo to może jednak nie były zombie? 
Ostatnia klatka zaczynała jakąś bitwę tych cosiów z nie wiadomo kim.

To zabawne, że nie zauważyłem wcześniej, jak wiele tekstów \differentlan{de-facto} opowiada mniej więcej o tym samym. 
Może jednak mają w sobie ziarna prawdy?
Możliwe, że energia jądrowa rzeczywiście mogła kiedyś służyć jako paliwo dla rakiet.
Aby załoga nie zmarła pod wpływem promieniowania, musiały wozić ze sobą olbrzymie osłony.
Jak się zdarzyło, że uranu zabrakło, albo reaktor się zepsuł, to statek był skazany na wieczny dryf w zamarzniętym oceanie próżni.
Jakaś strona mogłaby dowozić uran takim ,,unieruchomionym'' transportowcom i je naprawiać, jeśli miałaby odpowiednią do tego technologię.
Ale w zamian za co?

I tutaj jeszcze brakowało fragmentu układanki. Dlaczego te atomowe mutanty nie ginęły od promieniowania? Na co im byli członkowie innych załóg?
W jaki sposób pomagały statkom, jeśli jednocześnie je niszczyły?

\begin{dialogue}
	\ds{} Posortowałam te opowieści chronologicznie. \dm{} Galiza zaskoczyła mnie od tyłu, gdy rozmyślałem w kącie.
	\ds{} Szukałem historii w naukowych książkach, ale nic nie znalazłem.
	\ds{} Mówiłam.
	\ds{} Wszystko ocenzurowane.
	\ds{} Mówiłam.
	\ds{} Mówiłaś, że masz pełną wiedzę o tych istotach.
	\ds{} Nie mówiłam. Brakuje mi jeszcze tego, co ich trzymało przy życiu.
	\ds{} Znalazłem kawałek komiksu. \dm{} Dałem jej wydartą kartkę, którą z chęcią przestudiowała.
	\ds{} To nadal nie wyjaśnia, skąd wzięły się te węże.
	\ds{} Chyba nigdy nie poznamy pełnej prawdy, w końcu bajki nie mówią niczego wprost \dm{} powtórzyłem właśnie jej poprzednie słowa.
	\ds{} Ale teraz mamy pewność, że to musi być statek jednego z tych atomowców.
	\ds{} Pewność. Taaak.
\end{dialogue}

Chciałem już iść, zobaczyć jak tam progres tajemniczego zjawiska astronomicznego, gdy moją uwagę zwrócił pozostawiony przez kogoś czytnik.
Leżał otwarty na krześle.
Z ciekawości zajrzałem.
Jakże się zdziwiłem, że nie znalazłem tam piętnastej części popularnej sagi, ,,Dziesięciu tysięcy animatronicznych twarzy Browna'', a czyjś list.

\begin{poem}
	Bracia. \\
	Doszły mnie słuchy, iż niektórzy z was utracili nad sobą kontrolę i poczęli zaspokajać się tymi, których przyszło nam chronić.
	Miast pościć, dajecie upust swojemu głodowi, zabijając naszych przyjaciół i rodzinę. \\
	Wiem, że czasy ciężkie. Nasz dom przepadł bezpowrotnie, lecz miejcie nadzieję i wewnętrzną siłę.
	Wciąż mamy dane nam Automaty i wciąż żyjemy.
	Każdy wciąż ma w sobie cząstkę naszego domu, nie zapominajcie o nim. To on daje nam siłę. Siłę do naprawy.
	Powinniśmy dziękować Bogu, gdyż marny mógłby być nasz los.
	\\
	Grupa naszych rodaków postanowiła szukać nowego domu poza granicami naszego świata. Udali się w międzygwiezdną podróż, by ocalić nasz ród.
	Nie było to konieczne, gdyż głęboko wierzę, że pozytywnie zakończymy nasz spór i odzyskamy energię. 
	\\
	Jednak tymczasowo powstrzymajcie się z rozszerzaniem rodziny, brak zasobów doprowadza niektórych do straszliwych grzechów.
	Nawracajcie walczących i wyjaśniajcie, Jezus chodził po Izraelu, my chodzimy po planetach, lecz nasze czyny są takie same.
	\\
	Dziękujcie Bogu za życie i proście o szybkie zakończenie sporu i nawrócenie się prezydenta Herenina. \\
	Amen.
	
	PS. W tak trudnej chwili, jak ta, to nie pora na modne świecenie i fosforowanie, zachowajcie powagę i nie rozpraszajcie się.
\end{poem}

No tego się nie spodziewałem.
Już nic nie układało się do kupy.
Czas uciekał, więc zrzuciłem wszystkie teksty na mój zegarek, aby poczytać później.
Razem z Galizą przeszliśmy na dolny pokład, aby zobaczyć aktualną prędkość gwiazdy na niebie.
Nigdzie jednak nie dało się zlokalizować tego ciała niebieskiego.
Przechodząc przez środek wycieczkowca, przypadkowo rozdzieliliśmy się w tłumie, wiecznie miksującym się w sekcji rozrywkowej.

Spotkaliśmy się na korytarzu górnego pokładu, widocznie Galiza wpadła na ten sam pomysł, co ja, aby wrócić do półsferycznej bańki i tam obejrzeć zielone zjawisko.
Jednakże okazało się, że szukany wycinek nieba jest schowany za grubymi wrotami.
I restauracja i wszystkie okna także zostały pozamykane.
Zupełnie jakby ktoś celowo odciął połowę widocznego nieboskłonu.

\begin{dialogue}
	\ds{} Masz jakiś pomysł? \dm{} zapytałem się dziewczyny, stojąc przed zamkniętymi drzwiami do prawdy.
	\ds{} A ty masz? Bo jakoś nie bardzo uciekasz ze statku w kapsule ratunkowej. \dm{} Galiza oglądała dokładnie ścianę.
	\ds{} Dlaczego myślisz, że miałbym tak postąpić?
	\ds{} No nie wiem, może żeby nie stracić członków ciała w ataku zmutowanych potworo-ludzi? \dm{} powiedziała bardziej do plastikowej płaszczyzny, niż do mnie.
	\ds{} Naprawdę w to wierzysz? Że niby jakiś atomowiec sprzed kilkuset lat nagle pojawił się znikąd? I będzie atakował wielki wycieczkowy statek, zasilany reaktorem termojądrowym, zamiast uranem?
	\ds{} I posiadający pół tysiąca ludzi na pokładzie plus zero broni. Tak.
	\ds{} To nie jest takie pewne, co ostatecznie by zrobił. Widziałaś teksty.
	\ds{} No właśnie, każdy pisze co innego. A ja chcę się dowiedzieć prawdy! \dm{} Uderzyła w ścianę, odsłaniając panel kontrolny drzwi.
	\ds{} Co ty robisz?! Nie niszcz tego! \dm{} skarciłem ją.
	\ds{} Opuściłam Ziemię i już nie jestem grzeczną dziewczynką, przykro mi.
	\ds{} Zaraz ktoś tu przyjdzie i nas zobaczy. Trafimy prosto do pokładowego aresztu. \dm{} Kątem oka przyjrzałem się elektronicznym przełącznikom.
	\ds{} Co tu się dzieje? \dm{} Usłyszałem za sobą głos. Jak na zawołanie kapitan statku wyszedł zza rogu. Całą podróż siedzi na mostku, żeby akurat właśnie w tym momencie przejść się pustym korytarzem.
	\ds{} Nic.
	\ds{} Dziura w ścianie, to jest nic?
	\ds{} Pan ma dziury w ścianach, to my też chcieliśmy mieć.
	\ds{} Proszę? \dm{} zapytał się, zmieszany. Skąd przyszła mi do głowy ta wymówka?
	\ds{} Lubimy popatrzeć trochę czasami na czerwony kolor.
	\ds{} Czerwony? \dm{} kapitan i Galiza zadali to pytanie jednocześnie.
	\ds{} Tak, czerwony. \dm{} Szturchnąłem dziewczynę. \dm{} Czerwony jest taki ładny i jest wstępem do innej barwy.
	\ds{} Nie wiem, co planujecie, ale wzywam ochronę. \dm{} Kapitan sięgnął do pasa, wtedy Galiza wcisnęła czerwony przycisk na mechanizmie w ścianie.
\end{dialogue}

Drzwi otwarły się bezszelestnie.
Zieleń zalała korytarz, dodając nienaturalnego połysku przedmiotom.
Kontury się rozmyły, wściekła twarz dowódcy zyskała nowe oblicze grozy.
Odwróciliśmy się powoli.

Mały, kanciasty statek kosmiczny zbliżał się ku Stella Grande z dużą prędkością.
Jego zielonkawa łuna jasno odbijała się na tle czarnego nieba.
Słońce ostro oświetlało jedną ścianę, na której malowały się liczne pęknięcia i spawy.
Nierówne krawędzie wykonane były jak gdyby z wielu osobnych kawałków złomu, posklejanych razem.
Brud i oleiste zacieki znaczyły otwory w kadłubie.
Wewnętrzne odbicia szyby potęgowały szmaragdowe błyski.
Aberracja chromatyczna rozdzielała kolor na osobne linie spektralne.

Z przodu była popękana szyba, od środka pokryta jakby ciemnymi plamami pleśni.
Chorobliwa łuna biła z wnętrza i można było tylko przypuszczać, co to zaglonione akwarium skrywa.
Niewyraźne kontury obiektu wewnątrz przypominały biomechaniczny organizm, pływający w lepkiej brei.
To nie był wspaniały rycerz, ani też potwór niewiadomego pochodzenia. 
To było znacznie bardziej bliskie każdemu człowiekowi.
Wewnętrzny koszmar każdego z nas.

\begin{dialogue}
	\ds{} Mamusiu, to jest ten zielony pan, który miał mi ukraść nos? \dm{} Usłyszeliśmy za sobą.
	\ds{} I to mają być te wspaniałe fajerwerki?
	\ds{} Ja nie widzę tutaj darmowego poczęstunku.
	\ds{} A to nie miał być ten świetny kabareciarz z knajpy?
	\ds{} Po chuj tu przychodziłem.
	\ds{} Ale to chyba nie jest zbyt bezpieczne.
	\ds{} Z jakiej okazji to wydarzenie?
	\ds{} To on będzie rozdawał te kupony, tak?
\end{dialogue}

Kapitan został zupełnie zbity z tropu przez tłumek osób, który jak na zawołanie wpełzł na scenę z każdej strony.

\begin{dialogue}
	\dm{} Proszę państwa, nie wolno tutaj być. To niebezpieczne. \dm{} Przywódca statku próbował załagodzić sytuację, ale prawie został zadeptany przez napierające rodziny z dziećmi.
	\ds{} To twoja sprawka? \dm{} szepnąłem do Galizy, gdy wślizgiwaliśmy się przez już otwarte wrota. \dm{} Ja bym tak nie potrafił.
	\ds{} Wystarczyło powiedzieć, że będą darmowe kupony na loterię. 
\end{dialogue}

Fosforyzująca łuna wzmagała na sile, gdy kanciasta gwiazda przybliżała się do wycieczkowca.
Ustawiła się dziobem do nas i wcale nie zwalniała.
Ostry szpikulec chyba był wzmacniany, bo błyszczał się srebrnie.
Co on planował? To nie miało żadnego sensu, przecież tylko naraża swój statek.
A potem mnie olśniło. Na zielono.

\begin{dialogue}
	\ds{} Przecież on się chce z nami zderzyć! \dm{} zawołałem. Spojrzeliśmy na wyjście, które zatkane już było przez kłąb wijących się osób. Każdy chciał z bliska zobaczyć to niezwykłe zjawisko. I dostać kupony.
	\ds{} Wizgrant. Co myśmy narobili \dm{} dziewczyna zapłakała.
	\ds{} To się z nami zderzy! Wszyscy uciekać! Ewakuacja! \dm{} krzyczałem do ludzi.
	\ds{} Co pan, chce pan sobie wziąć te wszystkie promocje dla siebie? \dm{} Jakaś gruba kobieta się oburzyła.
	\ds{} Zje nas jednego po drugim, ratujcie się!
	\ds{} Spokojnie, nie wiemy tego na pewno. \dm{} Galiza zachowywała spokój ducha. \dm{} Może chce nas uratować?
	\ds{} Niby przed czym? Nie ma nadziei, będzie powtórka ,,Dziesięciu twarzy Browna''.
	\ds{} O, ja bym chętnie uczestniczył. \dm{} Jakiś młodzieniec usłyszał naszą rozmowę.
	\ds{} Chodziło mi o ten skrypt z zepsutego pliku, pożre nas, wszyscy umrzemy!
\end{dialogue}

Przez moje krzyki tylko jeszcze więcej ludzi się zwaliło.
Zmieniłem więc taktykę, pociągnąłem Galizę wgłąb tarasu.
Kalkulowałem w głowie, jak odbije się fala uderzeniowa w tej szklanej bańce i czy półsfera nie pęknie, katapultując nas wszystkich w przestrzeń.
Skuliliśmy się między krzesłami, obserwując ruchome cienie od atomowej gwiazdy.

\begin{dialogue}
	\ds{} Co nas czeka? \dm{} Galiza zapytała.
	\ds{} Baśnie nam powiedziały.
	\ds{} Baśnie tylko powiedziały, że wygląd kłamie, a legendy są legendami.
	\ds{} Nie. \dm{} To było teraz bardzo proste. \dm{} Nie ważne, kim kiedyś były atomowce. Ważne, kim są teraz. Na pewno nie rycerzami. Na pewno nie zakonem religijnym. Na pewno nie pogromcami piratów.
	\ds{} Czyli pożeraczami ciał.
	\ds{} Obawiam się, że tak to się dla nich skończyło.
	\ds{} Kolejna rzecz do obarczenia Herenina za jego zbrodnie.
\end{dialogue}

Na kilka sekund przed kolizją przepychający się chyba ogarnęli sytuację, bo zaczęli przeciskać się z powrotem na korytarz.
Jednak było już za późno.

Uderzenie ścięło wszystkie siedziska i zwaliło stojących ludzi.
Poturlałem się z dziewczyną w ramionach pod ścianę kopuły.
Połamane kawałki plastiku i wygięte żelastwa kłuły mnie w plecy.
Wstrząs dało się pewnie odczuć na całym wycieczkowcu.

Dziób złomiastego atomowca wbił się tak szczelnie w szybę, że powietrze nie uciekało.
Przez chwilę zastanawialiśmy się, co się teraz stanie.
Potem zaczął się powoli otwierać, niczym paszcza wielkiego krokodyla.
Nawet miał zarys zębów.
Czarna substancja wyciekła ze środka.
Trupi odór uderzył nas wszystkich po nosach.
Krzyczący ludzie umilkli, niektórzy stanęli, zmrożeni strachem.
Oczekiwali aż ze środka coś wyskoczy.

Jednak dało się słyszeć jedynie sapanie i kulejący krok.
Coś wolno podchodziło do wyjścia, szurając czymś ciężkim.
Rosnąca, fosforyzująca łuna o kolorze śmierci odbijała się wszystkim od oczu.
Iskierka poczucia bezpieczeństwa zabłysła w naszych sercach, bo wszyscy nadal żyli, niesamowite.
Przecież baśnie stanowiły inaczej.

I wtedy w wejściu pojawił się \emph{on}.
A raczej, \emph{oni}.

Kupa członków ludzkich, posklejanych byle jak, poprzeplatana mechanicznymi fragmentami.
Cieknąca śmiercią.
Bulgocząca.
Nieobecna.
Świecąca.
Martwa w życiu.
Żywa w śmierci.
Atomowiec.

\begin{dialogue}
	\ds{} Patrzy się na nas. \dm{} Usłyszałem szept w uchu. Oczy tego monstra przeskanowały teren, skuliliśmy się jeszcze bardziej za stertą połamanego plastiku.
\end{dialogue}

Czarne dołki zdawały się wysysać dusze.
Były tak abstrakcyjne, że nie przypominały oczu, ale jednak czułeś na sobie ich wzrok, niczym gęsty wiatr oblepiający ciało.
Ślepia skupiły się na wrotach do bańki i przerażonym zatorze powywracanych ludzi.
Ze środka biomechanicznej kupy wysunęła się czyjaś ręka i pięć nóg.
Zaczęły posuwać cielsko w stronę panikujących osób, zostawiając plamy smolistej substancji.
Dźwięk, niczym pływanie we flakach, zrzucanych z fabryki najtańszego mięsa.

Mutant zatrzymał się przy najbliżej leżącym człowieku.
\begin{dialogue}
	\ds{} Tatusiu! \dm{} jakieś dziecko zawołało z tłumu.
	\ds{} Ćśś! Bo nas usłyszy \dm{} ktoś obcy skarcił dzieciaka.
\end{dialogue}

Potem dało się słyszeć przyduszony płacz.
Ludzie odwracali wzrok, nikt nie miał żadnej broni.
Nawet załoga nie mogła uratować swoich pasażerów.

Atomowiec otworzył górną część, z której wylazły poskręcane kikuty, niczym kończyny upośledzonych płodów ze słoików z formaliną.
Wzięły biedaka za nogę, jak szczypce, i poczęły wciągać do wnętrza zielonej istoty.
Dźwięk siorbania i chrupania kości zwalał z nóg.
Ale uczta dopiero się rozpoczynała.

\begin{dialogue}
	\ds{} Trzymaj to. \dm{} Dałem Galizie ostry kawałek połamanego krzesła.
	\ds{} Nie wygłupiaj się. \dm{} Odrzuciła go. \dm{} Wszyscy zginiemy!
	\ds{} Spokojnie, przecież nie zmieści nas wszystkich. Obronię cię.
	\ds{} Nikt nas już nie obroni. Jesteśmy martwi.
	\ds{} Tylko nie panikuj. Błagam. \dm{} Przytuliłem ją mocniej.
	\ds{} Promieniowanie nas zabiło.
	\ds{} Jeszcze nie. Ale zaraz. Dlatego trzeba to jak najszybciej zakończyć. \dm{} Oszacowałem średnie dawki promieniowania dla człowieka.
\end{dialogue}

Za chwilę z drugiej strony straszydła wypełzła nowa noga, a mały fragment skóry wymienił się na świeży.
Mutant zrobił się większy.
Jedno nowe oko pojawiło się i popatrzyło wprost na naszą dwójkę.
\begin{dialogue}
	\ds{} Ach. \dm{} Dziewczyna skuliła się do mnie i zasłoniła oczy.
	\ds{} Jeśli zamierzasz ją pożreć, to najpierw będziesz musiał spróbować mnie \dm{} powiedziałem pod nosem, ściskając tak mocno metalową nogę krzesła, że się prawie wygięła.
	\ds{} Brrt \dm{} glut wydał z siebie niezidentyfikowany dźwięk. Jakby skierowany do mnie. Ale chyba się przeraził i nie poszedł w naszą stronę, zaczął wędrówkę ku tłumowi po więcej papu.
\end{dialogue}

Wtedy zauważyłem grubą pępowinę, która łączyła potwora ze statkiem.
Stalowy wąż był jak moja noga. Dawał mu zapewne życiodajną energię.
Pokryty czarną mazią i cieknąca śliską substancją.
To było to, co przedstawiono w komiksie.
Linia życia. Sybmiotyczna macka. Automat. Musiał być zasilany reaktorem jądrowym ze statku.

Gdyby udało się go jakoś przeciąć.
Uderzyć wystarczająco mocno.
Odłączyć od pompy.
Zwinąć w supeł.

\begin{dialogue}
	\ds{} Co nas może uratować? \dm{} zadałem do siebie pytanie, nie oczekując że ktokolwiek poda odpowiedź.
	\ds{} Niech ktoś wbiegnie do statku i poleci w tył! \dm{} Osoba z tłumu zaproponowała, jakby czytając mi w myślach. \dm{} Niech ktoś coś.
	\ds{} A może sam pobiegniesz? \dm{} dziadek skarcił młodzieńca. Że też miał na to nerwy w takiej sytuacji.
	\ds{} Nie umiem tym sterować. \dm{} Próbował się tłumaczyć.
\end{dialogue}

Kapitan był jedynym, który mógłby umieć coś takiego pilotować.
I byłem także pewien, że tego nie wykona.
Nie wybiegnie z tłumu i heroicznie nie ocali pasażerów.
Ale to była częściowo prawda, ktoś musiał pobiec i odłączyć zombie od zasilania. 

A potem mi się przypomniało, w jakim celu leciałem na Tytana.
\begin{dialogue}
	\ds{} Znamy się niecały dzień, czy to niesamowite? \dm{} Począłem mówić, jak do ściany.
	\ds{} Co?
	\ds{} Chciałbym, żeby ta znajomość trwała dłużej.
	\ds{} Ja... poczekaj. \dm{} Galiza w końcu zrozumiała. \dm{} Nie odchodź. Ja. Ja cię bardzo lubię.
	\ds{} Ja także \dm{} odpowiedziałem. \dm{} Dlatego chcę, żebyś żyła.
	\ds{} Nie idź, jeszcze tyle mi musisz opowiedzieć o gwiazdach.
	\ds{} Nie opowiem ci wszystkiego i tak. Pamiętaj, po co w ogóle poleciałem na Tytana. \dm{} Zmieniłem głos na jak najbardziej bezuczuciowy. \dm{} Ja umrę za miesiąc, więc krótka wizyta w reaktorze mi nic nie zrobi.
	\ds{} Nie wierzę, że chcesz umrzeć.
	\ds{} Masz rację, nie chcę. Chcę zostać rycerzem i cię uratować. \dm{} Pocałowałem ją. \dm{} Na wszelki wypadek: niebieskim guzikiem zamknij drzwi, aby powietrze nie uciekało jakbym musiał posłać pojazd w tył. \dm{} Zbliżyłem usta do jej ucha. \dm{} Kocham cię.
\end{dialogue}

Zanim zareagowała, byłem już w połowie drogi do paszczy gada.

Nawet będąc tak daleko, smród zaczął mi wypalać nozdrza.
Usłyszałem tylko dźwięki przerażenia z tłumu.
Wziąłem głęboki wdech i zanurkowałem do toksycznej otchłani.
Jej właściciel chyba w ogóle mnie nie zauważył, tylko dalej pałaszował.

Wewnątrz statek wyglądał równie źle, jak na zewnątrz.
Była to jedna komora, na środku której widniał pulpit sterowniczy, a wszędzie indziej walał się brudny złom.
Wąż podłączony był do wielkiego pudła w ścianie.
Rzuciłem się zaraz do niego, patrząc jak można by odłączyć zasilającą potwora pępowinę, gdy usłyszałem krzyk Galizy.
Mutant chyba ogarnął, że coś jest nie tak, bo zaczął pełznąć z powrotem, niemal się tocząc na nowozdobytych członkach.
Rura była wspawana w ścianę, nie było jak jej naruszyć. Wizja skończenia jako członki zielonego gluta mnie przeraziła.

Zrobił się szybki i dwukrotnie większy.
Jego wszystkie ślepia wpatrywały się we mnie.
Jak to możliwe, że było w nich widać strach? W tym czymś?
Nie wiedział, co zaraz uczynię. Ja także nie wiedziałem.

Rury nie dało się odłączyć. Nie było już czasu na formowanie nowego planu.
Niewiele myśląc, złapałem na konsoli za dźwignię głównego ciągu i pociągnąłem w tył, odbijając się w kierunku wyjścia.
Pisk oznajmił, że pobudziłem maszynę. Silniki pociągnęły statek w tył z wielką mocą, zasilaną rozpadem ciężkich pierwiastków.
Dziób wyrwał się ze szklanej sfery, jednocześnie zatrzaskując się z łoskotem tuż przed moim nosem.
Potężne przyspieszenie przycisnęło mnie do wrót, odruchowo złapałem się czegoś. 
Jak na złość, była to wajcha, którą przed chwilą przestawiłem.
Kosmiczny pojazd zaraz także szarpnął w drugą stronę z wielką mocą.

Mutant uderzył z pełnym pędem statku w brudną szybę, rzucając wokół krople czarnej substancji.
Ja uderzyłem z pełnym pędem statku w ciepłą ścianę reaktora, rzucając wokół kurwami.
Glut zaczął się wić w agonii i próbował wejść z powrotem przez szczelinę w dziobie, który zablokował się na pępowinie.
Chyba dusił się, jeśli w ogóle było to dla niego możliwe.

Zaparłem się o obleśny fotel i pociągnąłem wajchę ponownie w tył. Statek niemal się zatrzymał w miejscu, odrzucając zielonego na całą długość węża.
Szarpnięcie go nie urwało, chodź od końcówki odpadły jakieś kawałki.

A potem pociągnąłem w bok. I w tył. I znowu w przód.
Za każdym szarpnięciem atomowiec rozbijał się o fragment poharatanego kadłuba.
Syczący dźwięk uciekającego powietrza tylko mnie napędzał w panice.
Nie miałem wiele czasu, chciałem powrócić na wycieczkowiec.
Wyjść jak bohater z głową potwora pod pachą.
Galiza rzuciłaby się mi w ramiona.
Ruszałem wszystkimi dźwigniami na wszystkie strony, aby to osiągnąć.

\begin{dialogue}
	\ds{} Tii. \dm{} Stłumiony próżnią dźwięk wydobył się z popękanego gluta.
	\ds{} Nie wiem, kim jesteś. Nie wiem, czy którakolwiek z baśni miała sens. Ale nie pozwolę ci zjeść Galizy \dm{} krzyczałem w szybę. 
	\ds{} Naprrw \dm{} skrzeknął \dm{} nssss.
	\ds{} Zaraz, czy ty właśnie?
\end{dialogue}

Nagłe uderzenie odrzuciło mnie od konsoli.
Musiałem nieopatrznie skierować się na burtę Stella Grande.
Atomowy statek począł niekontrolowanie się kręcić bardzo szybko, światło słoneczne migotało, ożywiając wszystkie cienie.
Kolejne straszydła wychodziły z różnych zakamarków.
Zdawało mi się, jakby również atomowiec wpełzał z nimi z powrotem do statku.
To koniec.

Kolejne uderzenie i statek zatrzymał się, dziobem skierowanym prosto w kierunku Słońca.
Światło świecące przez wielką szybę oślepiało mnie tak mocno, że zasłoniłem oczy.
Trwało to przez dobrą chwilę.
Zatrzaśnięcie się włazu zmroziło mi krew w żyłach. Był w środku. Odsłoniłem oczy, by popatrzeć na swój koniec.

Ale nie było końca.
Na wężu została rozczapierzona końcówka ze skrawkami dawnego właściciela.
Wąż automatycznie zwijał się w ścianę.
Końcowy fragment zatrzymał się tuż przy mnie

Przypominała miejsce zerwania się starej liny lub korzeń jakiejś rośliny.
Wąż rozdzielał się na serię małych wężyków z igłami.
Mechaniczne macki wrośnięte były w nienaturalnie szerokie kości.
Sączyła się z nich oleista substancja.
Czyli ten atomowiec musiał mieć wszczepione to w siebie.
Obrzydlistwo.

Przez tłustą szybę widziałem oddalającą się Stalle Grande, z wielką dziurą w tarasie widokowym.
I rysami na kadłubie.
Nie mogłem tylko dojrzeć, co z ludźmi, co z Galizą.
Ale miałem nadzieję, że zdążyła i zamknęła drzwi na czas.

\begin{dialogue}
	\ds{} Ładnie się urządziłem \dm{} powiedziałem do siebie. \dm{} Zamiast miesięcy na Tytanie, czekają mnie godziny z chorobą popromienną.
\end{dialogue}
Nie miałem licznika Geigera, ale pewnie klikałby jak opętany.
Zaraz za ścianą muszą zachodzić żywe reakcje nuklearne, a całe powstałe promieniowanie przenika mnie na wylot w każdym kierunku.
\begin{dialogue}
	\ds{} Radioterapia \dm{} zaśmiałem się. \dm{} Przyda mi się.
\end{dialogue}

Siadłem pod obleśną ścianą, w kałuży śliskiego oleju i począłem rozmyślać nad moją sytuacją.
Przez głowę przepływały mi przeczytane opowieści.
Kim w końcu byli atomowcy? Coś czułem, że obecny stan tego statku i jego poprzedniego właściciela nie był tym, co planowano na początku.
Musieli się popsuć, gdy zabrakło im energii.

Przeszukałem pomieszczenie, znalazłem bardzo dużą ilość prętów paliwowych.
Nie wiem, ile statków musiał ograbić ten osobnik, ale ta ilość starczyłaby na działanie pokładowego reaktora aż do końca świata.
Nic dziwnego, że przeżył tyle lat po wojnie.
Coś czułem, że właśnie zamordowałem ostatniego atomowca w Układzie Słonecznym.

Wstałem, aby spróbować jakoś wrócić na wycieczkowiec, spostrzegłem że zostawiłem na ścianie kawałki mojej skóry.
Jeden z etapów choroby popromiennej. Chyba nawet ostatni.
Żadne życie nie wytrzyma takiej ilości fal gamma.
Ale jak w takim razie ten mutant egzystował?

Zacząłem badać pudło, do którego podłączona była rura.
Fragment zardzewiałej tabliczki znamionowej był w jakimś obcym, staroeuropejskim języku.
Nic więcej na urządzeniu nie było, żadnych przełączników, czy wskaźników.
Czy to mógł być ten legendarny Aparat?
Urządzenie naprawiające zniszczone DNA? Maszyna... nieśmiertelności?
Bardzo chciałem ją rozkręcić i zbadać dokładniej, ale myślę że nie wystarczyłoby mi życia na to.

Zatem umrę prawdopodobnie za kilkanaście godzin.
Popatrzyłem na końcówkę węża.
Umrę, chyba że.
Chyba. Że.

Usiadłem, zamknąłem oczy.
Nie mogę tego zrobić.
Nie chcę stać się taki, jak on.
Baśnie mówiły, że może skończy się to inaczej, że to ode mnie zależy.
Ale kto to wie.
W najgorszym razie sytuacja się powtórzy i w przyszłości sam wymorduję jakiś statek.

\begin{dialogue}
	\ds{} Ale obiecałem Galizie, że kiedyś się spotkamy. I że zostanę rycerzem, a nie zieloną pulpą \dm{} wykrzyknąłem. \dm{} Więc muszę dotrzymać słowa.
\end{dialogue}
Wziąłem końcówkę z ostrzami do ręki i oczyściłem z brudów.
\begin{dialogue}
	\ds{} I przy okazji zwalić winę na kogoś innego \dm{} dodałem po cichu.
\end{dialogue}

Ułożyłem się wygodnie na brzuchu na podłodze.
Położyłem końcówkę węża na sobie, czekając na to, co się za chwilę stanie.

Ból w plecach był ostry i przeszywający.
Każda komórka ciała poczęła mnie boleć.
Zrobiło mi się niedobrze i dostałem najgorszej migreny w życiu, albo i po życiu w tej chili.
Ból powoli ustępował, zamieniany na uczucie ciepła.
Poczułem bulgotanie w żyłach.
Poczułem głód.
Poczułem przeklęty głód.
Wiedziałem, że to się tak skończy.

Za oknem pływała Stella Grande. Głód. Tylko jeden sposób, aby go zaspokoić.
Chciałem jedzenia, ale nie wiedziałem po co i co mi to miało dać.
Chciałem więcej, jak jedzenia.

Położyłem dłoń na dźwigni ciągu.
Pociągnąłem w tył z całej siły.
Tym razem to ja piszę baśń.
I będzie taka fabuła, jaką ja chcę.

\bigskip
\smalltitle{Wyjaśnienie}
Atomowce powstały jako zapotrzebowanie świata na mechaników reaktorów jądrowych, którzy mieli pracować na statkach podróżujących po Układzie Słonecznym. 
Był to niebezpieczny i śmiercionośny zawód, toteż ich subkultura była bardzo zżyta.
Udało im się wynaleźć niezwykłe urządzenie, które potrafiło wykorzystywać zabójcze promieniowanie do naprawy DNA u ludzi. 
To dawało im niejako możliwość wiecznego życia i odporność na radioaktywność.

Atomowce znalazły planetoidę z dużą ilością uranu i stworzyły tam swoją małą cywilizację. 
Mechanicy atomu pomagali statkom, sprzedawali paliwo i dbali o bezpieczeństwo w podróży.
W zamian załogi uratowanych statków dawały im ludzi, aby mogli ich wcielić do swojej cywilizacji. 
Złych wrogów mogli za to przerabiać na materię biologiczną do uzupełniania swoich ciał. 
Ich mały dom, to było dobre miejsce, może nie za piękne, ale mieszkańcy żyli w harmonii i co najważniejsze - wiecznie. 

Chcieli się jakość wyróżniać od zwyczajnych śmiertelników.
Wynaleźli więc modę na świecenie, robili to poprzez wsadzanie fosforu wszędzie, gdzie się tylko da. 
Fosfor świeci na zielono pod wpływem promieniowania.
Surrealistyczna łuna zaczęła być tożsama z nimi i przenikać do legend i wyobrażeń.

Jednak politykom na Ziemi oczywiście nie podobał się taki stan rzeczy. 
Do władzy doszedł prezydent imieniem Herenin, który rozpoczął wojnę z atomowcami. 
Atomowce miały technologię i uran, więc były silniejsze, jednak ziemianom jakoś udało się zniszczyć uranową planetoidę mechaników. 
Zieloni ludzie zostały bez domu i bez energii. 
Zabroniono przelatującym statkom korzystać z usług mechaników, niejako głodząc ich.
Dodatkowo, tańsza i bezpieczniejsza energia termojądrowa zaczęła wypierać radioaktywne reaktory.

Mała cywilizacja Atomowców popadła w ruinę, stracili swoje idee i poczęli atakować losowe statki w poszukiwaniu uranu do zasilania swoich urządzeń. 
Ich automaty przestawały działać i zaczynali umierać pod wpływem promieniowania, więc musieli się naprawiać, masowo porywając innych ludzi na materię biologiczną dla swoich ciał. Część cywilizacji, widząc swój upadek, postanowiła wyruszyć do najbliższej innej gwiazdy w poszukiwaniu nowego życia.

Opowiadanie dzieje się długo po wojnie, ostatni atomowiec w Układzie Słonecznym próbuje zdobyć trochę materii ludzkiej dla siebie, aby zachować resztkę życia.

Wizgrant przypadkiem zajmuje jego miejsce i teraz czytelnik może się domyślać, co naukowiec zamierza zrobić. 
Również iść za głodem, czy spróbować odbudować cywilizację atomowców? 
Może będzie chciał dogonić tych, którzy polecieli do innej gwiazdy? 
Może w zemście wyda wojnę ziemianom? 
Co się stanie z jego nowotworem, może da mu jakieś supermoce? Jak wpłynie na działanie Automatu?

Generalnie mówiąc, atomowce były oryginalnie dobre, a Ziemianie byli powodem ich upadku.

