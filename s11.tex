\chapter{Zielona tajemnica} 

\info{Dwójka podróżujących statkiem kosmicznym ludzi zauważa za oknem nową gwiazdę. Próbują dowiedzieć się o co chodzi, bo jakoś wszyscy robią im na przekór.}

Czytnik kart zapikał i moim oczom ukazał się mały pokoik, który będzie mi służył przez najbliższe kilka dni podróży.
Inni pasażerowie także zajmowali swoje miejscówki.
Ukradkiem popatrzyłem na kajutę sąsiada, ale wcale nie była większa.
Jak tak wyglądała druga klasa, to jak musiała wyglądać piąta? 
Trzymali ich w próżni, czy co?

Zostawiłem bagaż i rozejrzałem się po pokładzie.
Stella Grande posiadała dziesięć pokładów wypełnionych restauracjami, kasynami i pubami.
W sumie nic więcej treściwego nie było, albo inaczej, wszystko inne było dość zakurzone i dawno nieużywane.

Jedynie na szczycie rozpościerał się przepiękny widok z pokładu obserwacyjnego.
Sferyczna czasza z grubego pleksiglasu miała chronić wnętrze przed zabójczą próżnią.
Na razie chroniła jedynie przed atmosferą wiecznie spowitego w smogu miasta.
Nie mogłem się doczekać, aby w końcu zobaczyć ponownie słońce.

Gdy rozpoczęło się odliczanie, wstałem ze stojącego tam leżaka i podszedłem do szyby, żeby po raz ostatni pożegnać się ze swoją planetą.
Chciałem także się upewnić, czy nadal jestem w stanie ustać przy przyspieszeniu 3 G.

Pokład zadrżał, a kłęby pary całkowicie przysłoniły widok.
Zrobiło się na tyle ciemno, że zapaliły się dyskretnie ukryte pod siedzeniami lampy.
Oglądanie mlecznego sufitu nie było tym, co chciałem oglądać po raz ostatni na tej planecie.

\begin{dialogue}
	\ds{} Trochę jak w wariatkowie, prawda? \dm{} Usłyszałem za sobą damski głos. \dm{} A pan jest zapalonym astronautą, widzę.
	\ds{} Dzień dobry, \differentlan{mademoiselle} \dm{} Obróciłem się i ukłoniłem grzecznie do nagle zmaterializowanej za mną osoby.
	\ds{} Dzień... \dm{} Młoda kobieta była wyraźnie zmieszana \dm{} ...dobry. Nazywam się Galiza, to mój pierwszy lot w kosmos.
	\ds{} Doktor Wizgr... Po prostu Wizgrant. \dm{} Akademickie odruchy nadal mną rządziły \dm{} Miło mi panią poznać.
\end{dialogue}

Otwarło się nad nami niebieskie oko nieba.
Poczułem także, że zrobiłem się trzykrotnie cięższy.
W dodatku ten przeklęty reumatyzm.
Pani Galiza także ledwo trzymała się na nogach.

\begin{dialogue}
	\ds{} Może usiądziemy? \dm{} zaproponowałem. \dm{} Przy takim przyspieszeniu łatwo się przewrócić.
\end{dialogue}

Usiedliśmy, a właściwie zwaliliśmy się na leżaki.
Nad nami był już tylko ciemniejący błękit.
Galiza jednak wpatrywała się w niego jak w pokaz fajerwerków.
Potem zaczęły wychodzić pojedyncze gwiazdy.
Gdy w końcu przyspieszenie zmalało do normalnego, wstaliśmy aby obejrzeć Ziemię z orbity.

\begin{dialogue}
	\ds{} Ciekawe, czemu każdy z tych statków musi mieć taką nudną nazwę. Stella Grande, Trans-Galactica, czy Saturn Brava.
	\ds{} Przecież morskie wycieczkowce także mają podobnie beznadziejne nazwy \dm{} odrzuciła, jakby była to oczywistość.
	\ds{} No ale teraz mogliby zrobić inaczej... \dm{} Jestem taki beznadziejny w rozmowach z kobietami. \dm{} 
			A propos, od spodu statku jest podobna szyba, co tutaj i tam lepiej widać będzie oddalającą się Ziemię.
\end{dialogue}

Galiza przystała na propozycję.
Udaliśmy się przez labirynt głośnych imprez i krzyczących dzieci do najniższego pokładu.
Tutaj także było całkowicie pusto.
Usiadłem na przezroczystej podłodze, tysiąc kilometrów nad powierzchnią, i zaprosiłem Galizę, żeby zrobiła to samo.

Siedzieliśmy przez chwilę, obserwując oddalający się dom.

\begin{dialogue}
	\ds{} Więc... \dm{} Zacząłem rozmowę, bo trochę dziwnie się robiło. \dm{} Byłem swego czasu, w sumie nadal jestem, doktorem astronomii.
		Z chęcią poopowiadam pani o każdej z tych gwiazdek osobno. Chociaż niestety, wiele osób o szybko nuży.
\end{dialogue}

Co za nietakt. Pomyśli sobie teraz, że się chwalę swoją wiedzą.
Pewnie będzie mnie miała za jakiegoś snoba.

\begin{dialogue}
	\ds{} To bardzo zabawne spotkać takiego kogoś w jego własnym środowisku \dm{} zaśmiała się. \dm{}
		Czyli pan pewnie leci na Tytana w celu jakichś badań, prawda? \dm{} A potem posmutniała. \dm{} Ja wybieram się jedynie na wakacje, tak po prostu.
	\ds{} To ciekawie się składa, bo ja także. Już kiedyś byłem na Tytanie, pomyślałem że to będzie dobre miejsce na śmier...
	\ds{} Co proszę? \dm{} Popatrzyła się na mnie oczyma tak głębokimi, jak głębokie pole Hubbla.
	\ds{} Nic, nic. \dm{} Szybko się poprawiłem.
	\ds{} Hmm... \dm{} szepnęła cichutko, niczym przelot ćmy w próżni.
	\ds{} Chciałem powiedzieć... że to takie miłe miejsce. Żeby dokończyć żywota. \dm{} Zabrzmiało to jeszcze gorzej. \dm{} Widzi pani, 
		nowotwór mnie trawi. Podróżujemy w kosmos, a współczesna medycyna nie jest w stanie naprawić problemu dręczącego ludzi od zarania dziejów.
		Zostało mi kilka lat życia, to pani pierwsza podróż kosmiczna, a moja ostatnia.
\end{dialogue}

Zapadła niezręczna cisza, przerywana jedynie wybuchami śmiechu gdzieś powyżej.
Oddech Galizy uspokoił się.

\begin{dialogue}
	\ds{} Więc, co to jest za gwiazdka? \dm{} Dziewczyna zmieniła temat błyskawicznie. 
	\ds{} Spica. 
	\ds{} Spica? Jest trochę niebieskawa.
	\ds{} Bo jest bardzo gorąca, emituje dużą ilość wysokoenergetycznych fal elektromagnetycznych przez co nasze oko... \dm{} Zauważyłem że Galiza przestaje nadążać.
		\dm{} Poza tym jest podwójna. To tak na prawdę dwie gwiazdy, które krążą wokół siebie.
	\dm{} Ciekawe, ciekawe. \dm{} Ale nie było zaciekawienia w jej głosie.
\end{dialogue}

Rozmowa się nie kleiła, albo to ja nie potrafiłem nic z siebie wydusić.

\begin{dialogue}
	\ds{} Pamięta pan przedwojenny świat? \dm{} Nagle się mnie zapytała. 
	\ds{} Nie jestem aż tak stary \dm{} zaśmiałem się. \dm{} Akurat gdy się urodziłem, król Hegezot przejmował władzę nad wszystkimi krajami.
	\ds{} Och, przepraszam. Zawsze mnie ciekawiło, jak ludzie podróżowali kiedyś w kosmosie. Myślałem że pan wie. Wszystkie przedwojenne teksty zostały przecież obowiązkowo zakazane.
	\ds{} Wiem tyle samo, co pani. Poza tym... \dm{} Nachyliłem się do jej ucha \dm{} w kosmosie ściany także mają uszy.
	\ds{} Nie te. \dm{} Wskazała na przezroczystą posadzkę na której środku siedzieliśmy.
	\ds{} To nadal ryzykowne.
	\ds{} Tak samo, jak życie, prawda?
\end{dialogue}

Zapamiętałem te słowa, dzwoniły mi w uszach jeszcze przez pewien czas, po tym jak rozeszliśmy się.
Szeptanie przeciwko obecnej władzy nie było mądrym posunięciem.
Po zakończeniu czwartej wojny światowej, tak zwanej Ostatniej Wojny, nad Ziemią zapanował wspólny rząd z ,,demokratycznie'' wybieranymi przywódcami.
Każda przedwojenna księga została zakazana, jedynie naukowe twory były przefiltrowane przez bezwzględną cenzurę i wydane w elektronicznej postaci.

Każda dziedzina życia stała się na nowo, a obywatele byli teraz rybkami w akwarium.
Miało to swoje minusy, to fakt.
Jednak wbrew pozorom, ludzkość rozwijała się szybciej niż kiedykolwiek.
Tak masowe podróże kosmiczne nie byłyby możliwe bez wspólnego rządu.
Ale czy na pewno? Takie przynajmniej jest oficjalne stanowisko nauk historycznych.

Na niepodległym Tytanie na pewno znajdę odpowiedzi na wszystkie pytania.
Zresztą, co mnie to obchodzi, jeśli mnie aresztują.
Za to Galizę powinno, ma przed sobą całe życie, lepiej żeby nie spędziła go w zakładzie naprawczym do prania mózgów.

Wziąłem z bagażu przewodnik po Tytanie i usiadłem na leżaku na tarasie widokowym.
Mało było w tej książce napisane, ale zawsze lepsze to niż nic.
Natomiast informacje z książki dość mocno odbiegały od moich wspomnień, kiedy kilkanaście lat temu ostatni raz odwiedzałem ten księżyc.

Obudził mnie dzwonek zwiastujący obiad.
Z leżaka obok poderwała się Galiza.

\begin{dialogue}
	\ds{} Widzę, że pani także nie może sobie znaleźć miejsca na dolnych pokładach \dm{} zapytałem się wrednie.
	\ds{} Nie, tylko. \dm{} Trochę się zmieszała. \dm{} Pan ma doświadczenie, a ja jestem tutaj taka sama. Wcale nie przyszłam tutaj za panem.
	\ds{} Oczywiście, że nie. \dm{} Uśmiechnąłem się w duchu. \dm{} W każdym razie zadzwonił dzwonek na obiad. Czy dałaby się pani zaprosić na kosmiczny bankiet?
	\ds{} Ależ ja nie mam pieniędzy na obiad drugiej klasy!
	\ds{} Nie szkodzi, niech będzie na mnie. \dm{} Zmrużyłem oczy. \dm{} Skąd pani w ogóle wie, że jestem z drugiej klasy?
	\ds{} Tak, jakoś. Zobaczyłam pana bilet. \dm{} Skłamała. Nigdy nie trzymałem przy niej swojego biletu. Musiała śledzić mnie do mojej kajuty.
\end{dialogue}

Wziąłem ją pod rękę i poszliśmy razem do głównej restauracji.
Idąc, patrzyła jeszcze na gwiazdy.
Może jednak interesowała się nimi.

\begin{dialogue}
	\ds{} Więc, gwiazdy mogą być czerwone, albo żółte jak nasze Słońce, albo też białe czy fioletowe?
	\ds{} Zgadza się, zgodnie z zasadami promieniowania ciała doskonale czarnego. \dm{} Znowu zacząłem przynudzać. \dm{} To jak rozgrzany metalowy pręt.
		Im mocniej rozgrzany, tym jaśniejszy.
	\ds{} A czy może być więc zielona gwiazda?
	\ds{} Nie słyszałem o czymś takim, szczerze powiedziawszy. W dodatku żadna planeta, czy księżyc w naszym układzie gwiezdnym nie są zielone, skąd taki pomysł?
	\ds{} No bo tam jest. \dm{} Wskazała palcem gwiazdozbiór Łabędzia.
\end{dialogue}

I rzeczywiście, zielony punkt jasno świecił na tle drogi mlecznej.
Nie przypominałem sobie, żeby w tamtym miejscu znajdowała się jakaś gwiazda.
A także nie było to w osi ekliptyki, więc wszystkie planety były od razu wykluczone.
Może asteroida, albo kometa?
Ale te także nie mogły być zielone.

\begin{dialogue}
	\ds{} Szczerze powiedziawszy, pierwszy raz od dawna widzę coś takiego. Może supernowa?
	\ds{} Patrzyłam na mapy nieba, próbując określić, co to za gwiazdka, ale nie nic w tym miejscu nie znalazłam.
	\ds{} Ciekawe, kiedy się pojawiło.
	\ds{} Kilka godzin temu, gdy pan smacznie spał. Przeglądałam elektroniczny atlas i porównywałam z widokiem, gdy zobaczyłam tę gwiazdkę.
	\ds{} No, to że ją pani właśnie wtedy zobaczyła, to nie znaczy że nie było jej tutaj wcześniej. \dm{} Otwarłem drzwi na korytarz. Sześciokątne okna oświetlały korytarz światłem drogi mlecznej.
	\ds{} Nie, ja pamiętam, że patrzyłam się na ten gwiazdozbiór, gdy na początku się spotkaliśmy. I wtedy nic takiego tam nie było.
	\ds{} Przykro mi, gwiazdy nie pojawiają się tak znikąd. Musiało się pani przewidzieć. Ale mnie się to dość często zdarza. \dm{} Poprowadziłem ją do drugoklasowej restauracji.
\end{dialogue}

Usiedliśmy na stoliku na samym środku, bo tylko taki był wolny.
Wręczono nam karty dań z trzema do wyboru, wszystkie wegańskie.
Zamówiłem, jak większość, sojowego placka po francusku.
Galiza chyba nie chciała sprawiać wrażenie kosztownej, dlatego wzięła kotlet warzywny.
Do tego podano nam chłodzoną próżnią wodę.

Jedzenie spędziliśmy na rozmowach o naszych dzieciństwach.
W sumie głównie ja mówiłem, gdyż miałem chyba więcej do powiedzenia.
Galiza wychowała się w centrum miasta i była grzeczną dziewczynką, która zawsze słuchała się rodziców i nauczycieli.
W sumie głównie dlatego postanowiła się teraz przeciwstawić i wybrać na wakacje w kosmos.

\begin{dialogue}
	\ds{} Wizgrant... \dm{} zapytała tajemniczo.
	\ds{} Słucham? \dm{} odpowiedziałem cicho, oczekując jakiegoś personalnego pytania.
	\ds{} Patrz. \dm{} Uniosła rękę nad stolik, na obrusie zarysował się cień.
\end{dialogue}

Wolno zadarliśmy głowy w górę, przez trójkątne okno jaśniała zielonym światłem ta nowa gwiazda.

\begin{dialogue}
	\ds{} Zrobiła się jaśniejsza \dm{} szepnęła.
	\ds{} I zmieniła pozycję \dm{} odpowiedziałem. \dm{} Teraz jest w gwiazdozbiorze Lutni.
\end{dialogue}

Rozejrzeliśmy się po innych stolikach. Nikt inny się nie przejmował niezwykłym fenomenem.
Jak to możliwe, czyżby nie widzieli?











