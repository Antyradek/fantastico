\chapter{Baśniowa gwiazda} 

\info{Dwójka podróżujących statkiem kosmicznym ludzi zauważa za oknem nową gwiazdę. Próbują dowiedzieć się o co chodzi, bo jakoś wszyscy robią im na przekór.}

Czytnik kart zapikał i moim oczom ukazał się mały pokoik. 
Ta mała klitka będzie mi służyć przez najbliższe kilka dni podróży.
Inni pasażerowie statu także zajmowali swoje miejscówki w innych pokojach, podobnych do mojego.
Ukradkiem popatrzyłem na kajutę sąsiada, czy aby nie była większa. 
Na szczęście nie była.

Jeśli tak wyglądała druga klasa, to jak musiała wyglądać piąta? 
Trzymali ich na zewnątrz w próżni? W namiotach?

Zostawiłem bagaż, dokładnie zamknąłem drzwi i rozejrzałem się po pokładzie.
Stella Grande posiadała dziesięć pięter, wypełnionych restauracjami, kasynami i pubami.
W sumie nic więcej treściwego nie było, albo inaczej, wszystko inne było dość zakurzone i dawno nieużywane.
Trafiłem na jakąś salę kinowo-teatralną, której nikt nie sprzątał od kilku kursów.
Była i siłownia w której brakowało większość sprzętu.

Poszedłem najwyżej jak się dało.
Stąd rozpościerał się przepiękny widok z pokładu obserwacyjnego.
Sferyczna czasza z grubego pleksiglasu miała chronić wnętrze przed zabójczą próżnią.
Na razie chroniła jedynie przed atmosferą wiecznie spowitego w smogu miasta.
Nie mogłem się doczekać, aby w końcu zobaczyć ponownie słońce.

Gdy rozpoczęło się odliczanie, wstałem ze stojącego tam leżaka i podszedłem do szyby, żeby po raz ostatni pożegnać się ze swoją planetą.
Chciałem także się upewnić, czy nadal jestem w stanie ustać przy przyspieszeniu 3 G.

Pokład zadrżał, a kłęby pary całkowicie przysłoniły widok.
Zrobiło się na tyle ciemno, że zapaliły się dyskretnie ukryte pod siedzeniami lampy.
Oglądanie mlecznego sufitu nie było tym, co chciałem oglądać po raz ostatni na tej planecie.

\begin{dialogue}
	\ds{} Trochę jak w wariatkowie, prawda? \dm{} Usłyszałem za sobą damski głos. \dm{} A pan jest zapalonym astronautą, widzę.
	\ds{} Dzień dobry, \differentlan{mademoiselle} \dm{} Obróciłem się i ukłoniłem grzecznie do nagle zmaterializowanej za mną osoby.
	\ds{} Dzień... \dm{} Młoda kobieta była wyraźnie zmieszana \dm{} ...dobry. Nazywam się Galiza, to mój pierwszy lot w kosmos.
	\ds{} Doktor Wizgr... Po prostu Wizgrant. \dm{} Akademickie odruchy nadal mną rządziły \dm{} Miło mi panią poznać.
\end{dialogue}

Otwarło się nad nami niebieskie oko nieba.
Poczułem także, że zrobiłem się trzykrotnie cięższy.
W dodatku ten przeklęty reumatyzm.
Pani Galiza także ledwo trzymała się na nogach.

\begin{dialogue}
	\ds{} Może usiądziemy? \dm{} zaproponowałem. \dm{} Przy takim przyspieszeniu łatwo się przewrócić.
\end{dialogue}

Usiedliśmy, a właściwie zwaliliśmy się na leżaki.
Nad nami był już tylko ciemniejący błękit.
Galiza jednak wpatrywała się w niego jak w pokaz fajerwerków.
Potem zaczęły wychodzić pojedyncze gwiazdy.
Gdy w końcu przyspieszenie zmalało do normalnego, wstaliśmy aby obejrzeć Ziemię z orbity.

\begin{dialogue}
	\ds{} Ciekawe, czemu każdy z tych statków musi mieć taką nudną nazwę. Stella Grande, Trans-Galactica, czy Saturn Brava.
	\ds{} Przecież morskie wycieczkowce także mają podobnie beznadziejne nazwy \dm{} odrzuciła, jakby była to oczywistość.
	\ds{} No ale teraz mogliby zrobić inaczej... \dm{} Jestem taki beznadziejny w rozmowach z kobietami. \dm{} 
			A propos, od spodu statku jest podobna szyba, co tutaj i tam lepiej widać będzie oddalającą się Ziemię.
\end{dialogue}

Galiza przystała na propozycję.
Udaliśmy się przez labirynt głośnych imprez i krzyczących dzieci do najniższego pokładu.
Tutaj także było całkowicie pusto.
Usiadłem na przezroczystej podłodze, tysiąc kilometrów nad powierzchnią, i zaprosiłem Galizę, żeby zrobiła to samo.

Siedzieliśmy przez chwilę, obserwując oddalający się dom.

\begin{dialogue}
	\ds{} Więc... \dm{} Zacząłem rozmowę, bo trochę dziwnie się robiło. \dm{} Byłem swego czasu, w sumie nadal jestem, doktorem astronomii.
		Z chęcią poopowiadam pani o każdej z tych gwiazdek osobno. Chociaż niestety, wiele osób o szybko nuży.
\end{dialogue}

Co za nietakt. Pomyśli sobie teraz, że się chwalę swoją wiedzą.
Pewnie będzie mnie miała za jakiegoś snoba.

\begin{dialogue}
	\ds{} To bardzo zabawne spotkać takiego kogoś w jego własnym środowisku \dm{} zaśmiała się. \dm{}
		Czyli pan pewnie leci na Tytana w celu jakichś badań, prawda? \dm{} A potem posmutniała. \dm{} Ja wybieram się jedynie na wakacje, tak po prostu.
	\ds{} To ciekawie się składa, bo ja także. Już kiedyś byłem na Tytanie, pomyślałem że to będzie dobre miejsce na śmier...
	\ds{} Co proszę? \dm{} Popatrzyła się na mnie oczyma tak głębokimi, jak głębokie pole Hubbla.
	\ds{} Nic, nic. \dm{} Szybko się poprawiłem.
	\ds{} Hmm... \dm{} szepnęła cichutko, niczym przelot ćmy w próżni.
	\ds{} Chciałem powiedzieć... że to takie miłe miejsce. Żeby dokończyć żywota. \dm{} Zabrzmiało to jeszcze gorzej. \dm{} Widzi pani, 
		nowotwór mnie trawi. Podróżujemy w kosmos, a współczesna medycyna nie jest w stanie naprawić problemu dręczącego ludzi od zarania dziejów.
		Zostało mi kilka lat życia, to pani pierwsza podróż kosmiczna, a moja ostatnia.
\end{dialogue}

Zapadła niezręczna cisza, przerywana jedynie wybuchami śmiechu gdzieś powyżej.
Oddech Galizy uspokoił się.

\begin{dialogue}
	\ds{} Więc, co to jest za gwiazdka? \dm{} Dziewczyna zmieniła temat błyskawicznie. 
	\ds{} Spica. 
	\ds{} Spica? Jest trochę niebieskawa.
	\ds{} Bo jest bardzo gorąca, emituje dużą ilość wysokoenergetycznych fal elektromagnetycznych przez co nasze oko... \dm{} Zauważyłem że Galiza przestaje nadążać.
		\dm{} Poza tym jest podwójna. To tak na prawdę dwie gwiazdy, które krążą wokół siebie.
	\dm{} Ciekawe, ciekawe. \dm{} Ale nie było zaciekawienia w jej głosie.
\end{dialogue}

Rozmowa się nie kleiła, albo to ja nie potrafiłem nic z siebie wydusić.

\begin{dialogue}
	\ds{} Pamięta pan przedwojenny świat? \dm{} Nagle się mnie zapytała. 
	\ds{} Nie jestem aż tak stary \dm{} zaśmiałem się. \dm{} Akurat gdy się urodziłem, król Hegezot przejmował władzę nad wszystkimi krajami.
	\ds{} Och, przepraszam. Zawsze mnie ciekawiło, jak ludzie podróżowali kiedyś w kosmosie. Myślałem że pan wie. Wszystkie przedwojenne teksty zostały przecież obowiązkowo zakazane.
	\ds{} Wiem tyle samo, co pani. Poza tym... \dm{} Nachyliłem się do jej ucha \dm{} w kosmosie ściany także mają uszy.
	\ds{} Nie te. \dm{} Wskazała na przezroczystą posadzkę na której środku siedzieliśmy.
	\ds{} To nadal ryzykowne.
	\ds{} Tak samo, jak życie, prawda?
\end{dialogue}

Zapamiętałem te słowa, dzwoniły mi w uszach jeszcze przez pewien czas, po tym jak rozeszliśmy się.
Szeptanie przeciwko obecnej władzy nie było mądrym posunięciem.
Po zakończeniu czwartej wojny światowej, tak zwanej Ostatniej Wojny, nad Ziemią zapanował wspólny rząd z ,,demokratycznie'' wybieranymi przywódcami.
Każda przedwojenna księga została zakazana, jedynie naukowe twory były przefiltrowane przez bezwzględną cenzurę i wydane w elektronicznej postaci.

Każda dziedzina życia stała się na nowo, a obywatele byli teraz rybkami w akwarium.
Miało to swoje minusy, to fakt.
Jednak wbrew pozorom, ludzkość rozwijała się szybciej niż kiedykolwiek.
Tak masowe podróże kosmiczne nie byłyby możliwe bez wspólnego rządu.
Ale czy na pewno? Takie przynajmniej jest oficjalne stanowisko nauk historycznych.

Na niepodległym Tytanie na pewno znajdę odpowiedzi na wszystkie pytania.
Zresztą, co mnie to obchodzi, jeśli mnie aresztują, to przecież w aktualnym stanie nic nie tracę.
Za to Galizę powinno, ma przed sobą całe życie, lepiej żeby nie spędziła go w zakładzie naprawczym do prania mózgów.

Wziąłem z bagażu przewodnik po Tytanie i usiadłem na leżaku na tarasie widokowym.
Mało było w tej książce napisane, ale zawsze lepsze to niż nic.
Natomiast informacje z książki dość mocno odbiegały od moich wspomnień, kiedy kilkanaście lat temu ostatni raz odwiedzałem ten księżyc.

Obudził mnie dzwonek zwiastujący obiad.
Z leżaka obok poderwała się Galiza.

\begin{dialogue}
	\ds{} Widzę, że pani także nie może sobie znaleźć miejsca na dolnych pokładach \dm{} zapytałem się wrednie.
	\ds{} Nie, tylko. \dm{} Trochę się zmieszała. \dm{} Pan ma doświadczenie, a ja jestem tutaj taka sama. Wcale nie przyszłam tutaj za panem.
	\ds{} Oczywiście, że nie. \dm{} Uśmiechnąłem się w duchu. \dm{} W każdym razie zadzwonił dzwonek na obiad. Czy dałaby się pani zaprosić na kosmiczny bankiet?
	\ds{} Ależ, ja nie mam pieniędzy na obiad drugiej klasy!
	\ds{} Nie szkodzi, niech będzie na mnie. \dm{} Zmrużyłem oczy. \dm{} Skąd pani w ogóle wie, że jestem z drugiej klasy?
	\ds{} Tak, jakoś. Zobaczyłam pana bilet. \dm{} Skłamała. Nigdy nie trzymałem przy niej swojego biletu. Musiała śledzić mnie do mojej kajuty.
\end{dialogue}

Wziąłem ją pod rękę i poszliśmy razem do głównej restauracji.
Idąc, patrzyła jeszcze na gwiazdy.
Może jednak interesowała się nimi.

\begin{dialogue}
	\ds{} Więc, gwiazdy mogą być czerwone, albo żółte jak nasze Słońce, albo też białe czy fioletowe?
	\ds{} Zgadza się, zgodnie z zasadami promieniowania ciała doskonale czarnego. \dm{} Znowu zacząłem przynudzać. \dm{} To jak rozgrzany metalowy pręt.
		Im mocniej rozgrzany, tym jaśniejszy.
	\ds{} A czy może być więc zielona gwiazda?
	\ds{} Nie słyszałem o czymś takim, szczerze powiedziawszy. W dodatku żadna planeta, czy księżyc w naszym układzie gwiezdnym nie są zielone, skąd taki pomysł?
	\ds{} No bo tam jest. \dm{} Wskazała palcem gwiazdozbiór Łabędzia.
\end{dialogue}

I rzeczywiście, zielony punkt jasno świecił na tle drogi mlecznej.
Nie przypominałem sobie, żeby w tamtym miejscu znajdowała się jakaś gwiazda.
A także nie było to w osi ekliptyki, więc wszystkie planety były od razu wykluczone.
Może asteroida, albo kometa?
Ale te także nie mogły być zielone.

\begin{dialogue}
	\ds{} Szczerze powiedziawszy, pierwszy raz od dawna widzę coś takiego. Może supernowa?
	\ds{} Patrzyłam na mapy nieba, próbując określić, co to za gwiazdka, ale nie nic w tym miejscu nie znalazłam.
	\ds{} Ciekawe, kiedy się pojawiło.
	\ds{} Kilka godzin temu, gdy pan smacznie spał. Przeglądałam elektroniczny atlas i porównywałam z widokiem, gdy zobaczyłam tę gwiazdkę.
	\ds{} No, to że ją pani właśnie wtedy zobaczyła, to nie znaczy że nie było jej tutaj wcześniej. \dm{} Otwarłem drzwi na korytarz. Sześciokątne okna oświetlały korytarz światłem drogi mlecznej.
	\ds{} Nie, ja pamiętam, że patrzyłam się na ten gwiazdozbiór, gdy na początku się spotkaliśmy. I wtedy nic takiego tam nie było.
	\ds{} Przykro mi, gwiazdy nie pojawiają się tak znikąd. Musiało się pani przewidzieć. Ale mnie się to dość często zdarza. \dm{} Poprowadziłem ją do drugoklasowej restauracji.
\end{dialogue}

Usiedliśmy na stoliku na samym środku, bo tylko taki był wolny.
Wręczono nam karty dań z trzema do wyboru, wszystkie wegańskie.
Zamówiłem, jak większość, sojowego placka po francusku.
Galiza chyba nie chciała sprawiać wrażenie kosztownej, dlatego wzięła kotlet warzywny.
Do tego podano nam chłodzoną próżnią wodę.

Jedzenie spędziliśmy na rozmowach o naszych dzieciństwach.
W sumie głównie ja mówiłem, gdyż miałem chyba więcej do powiedzenia.
Galiza wychowała się w centrum miasta i była grzeczną dziewczynką, która zawsze słuchała się rodziców i nauczycieli.
W sumie głównie dlatego postanowiła się teraz przeciwstawić i wybrać na wakacje w kosmos.

\begin{dialogue}
	\ds{} Wizgrant... \dm{} zapytała tajemniczo.
	\ds{} Słucham? \dm{} odpowiedziałem cicho, oczekując jakiegoś personalnego pytania.
	\ds{} Patrz. \dm{} Ostrożnie uniosła rękę nad stolik, na obrusie zarysował się cień.
\end{dialogue}

Wolno zadarliśmy głowy w górę, przez trójkątne okno jaśniała zielonym światłem ta nowa gwiazda.

\begin{dialogue}
	\ds{} Zrobiła się jaśniejsza \dm{} szepnęła.
	\ds{} I zmieniła pozycję \dm{} dopowiedziałem. \dm{} Teraz jest w gwiazdozbiorze Lutni.
\end{dialogue}

Rozejrzeliśmy się po innych stolikach. Nikt inny się nie przejmował niezwykłym fenomenem.
Jak to możliwe, czyżby nie widzieli?

Wtedy cały pokój zajaśniał na zielono.
Reflektory skupiły swój kolor na małej scenie w rodu, na którą wyszedł rudy człowieczek w stroju Leprechauna. 
\begin{dialogue}
	\ds{} Czym różni się kosmonauta, od astronauty? \dm{} zaczął swój występ bez słowa powitania \dm{} Jedni polecieli na Księżyc, a drudzy z Księżyca spadli!
\end{dialogue}

Publiczność wybuchnęła śmiechem. Galiza zaczęła jeść swój kotlet szybciej.

\begin{dialogue}
	\ds{} Czemu próżnia jest taka zimna? \dm{} Na kolejny dowcip poszedłem w ślady mojej pary. \dm{} Bo się przestraszyła czarnej dziury!
\end{dialogue}

Tym razem to zielony ludek śmiał się najbardziej ze wszystkich.

\begin{dialogue}
	\ds{} Dlaczego UFO jest okrągłe?
	\ds{} ...chłodź-my \dm{} Galiza wskazała wyjście, plując jedzeniem. Pokiwałem głową, zatykając usta palcem, żeby mój placek też nie wyleciał.
\end{dialogue}

Uciekliśmy w samą porę, jak już byliśmy w bezpiecznym korytarzu, cała sala zaryczała ze śmiechu.

Skierowaliśmy swe kroki w stronę ogólnodostępnego teleskopu.
Można na nim dokładniej oglądać kosmos, przez który właśnie przedziera się Stella Grande.
Mgławice w rzeczywistości są znacznie bardziej kolorowe, niż jak się je ogląda z Ziemi, bo atmosfera pochłania część kolorów.
Liczyłem na to, że uda nam się bliżej przyjrzeć tajemniczej gwiazdce.

Jednak tym razem zastała nas informacja, że teleskop jest w renowacji.
To było dziwne, gdyż jeszcze przed startem był otwarty.

Udałem się do biblioteki, z nadzieją na znalezienie jakiejś informacji o przelatujących w pobliżu kometach.
Cóż innego mogło to być? Takie jasne.
Zostawiłem Galizę w korytarzu prowadzącym do kajut piątej klasy.
Nie chciała, żebym widział, gdzie spała.

Biblioteka stylizowana była na starodawny zbiór książek.
Elektroniczne ekrany oprawione były w papierowe, postarzane okładki i poustawiane na drewnianych półkach.
Każda z takich ,,książek'' łączyła się z głównym komputerem i mogła wyświetlić wszystko to, co każda inna.

Wziąłem jakąś taką z najmniej zużytą okładką i zacząłem przeglądać dostępne tytuły.
Tylko naukowe i jakieś bełkotliwe śmieci.
Ani tu, ani tam nie było żadnej wzmianki o zielonych gwiazdach, kometach, czy innych podobnych zjawiskach.
Nie chciało mi się wierzyć, że to zjawisko zaszło pierwszy raz w historii, na moich oczach.
Dlatego uznałem, że informacja o tej komecie musiała znajdować się w ocenzurowanej części literatury.

Poszedłem poszukać dziewczyny, od razu znalazłem ją na dolnym pokładzie, centralnie na środku jej ulubionej, przezroczystej podłogi.
Ziemia spadała pod nami, Galiza tęsknie patrzyła się na niebieskawą kropkę.

Wyjaśniłem sytuację z biblioteką, jednak nim dokończyłem opowieść, otrzymałem karteczkę z gwiazdozbiorem Łabędzia i zaznaczoną na niej kreską.
Wokół niej zapisane były godziny.
Moja teoria o naturalnym pochodzeniu tego obiektu legła w gruzach.

\begin{dialogue}
	\ds{} Wierzysz w kosmitów? \dm{} zapytała się tajemniczo \dm{} bo ja wierzę.
	\ds{} Nie spotkaliśmy jeszcze żadnych, a poznaliśmy wszystkie zakamarki Układu Słonecznego. W każdej chwili w kosmos patrzą się tysiące teleskopów, monitorujących każdy skrawek nieba. To trochę niemożliwe, żeby się tak nagle przybysze z zewnątrz pojawili niezauważeni.
	\ds{} Weź poprawkę na rząd światowy.
	\ds{} Wziąłem, nie wszystkie teleskopy przecież są pod ich kontrolą. Na samym Tytanie pewnie pracuje z tuzin.
	\ds{} Ale gdyby...
	\ds{} W fizyce nie ma gdyby. Ten obiekt musi pochodzić z Ziemi. Musi być to statek kosmiczny, albo jakaś sonda. Inaczej się nie da.
\end{dialogue}

Galiza wyraźnie posmutniała.

\begin{dialogue}
	\ds{} Weź poprawkę na rząd światowy \dm{} dodałem tajemniczo. Galiza pokazała doskonale białe ząbki.
	\ds{} Ale co my zrobimy bez teleskopu?
	\ds{} A kto powiedział, że nie mamy teleskopu?
\end{dialogue}

Rozejrzałem się i wyjąłem ze ściennego lichtarza szkiełko w kształcie małej soczewki skupiającej.
Ustawiłem się tak, aby zielone UFO było w linii prostej ze mną i ozdobnym kawałkiem podłogi, który także okazał się być soczewką.
Spojrzałem przez szkiełko i zacząłem się posuwać w przód i w tył. W końcu zobaczyłem wyraźniejszy obraz kanciastego stateczka.
Dałem popatrzyć dziewczynie.

\begin{dialogue}
	\ds{} Dlaczego obraz jest odwrócony? \dm{} Zadała to pytanie, które zadaje 99\% osób pierwszy raz spoglądających przez teleskop.
	\ds{} Bo soczewka załamuje promienie, które przecinają się w ognisku, więc wyjściowy obraz jest odwrócony jednocześnie w poziomie i w pionie.
	\ds{} Ale lornetka.
	\ds{} Lornetka ma pryzmaty, które załamują... dobra, przyszłaś tu rozprawiać o optyce, czy oglądać statek? \dm{} Trochę się zniecierpliwiłem.
	\ds{} Jejku, to rzeczywiście statek. Ma rogi, zupełnie jak gwiazdy.
	\ds{} Gwiazdy nie mają... \dm{} Pociągnąłem ją ze sobą w stronę biblioteki. \dm{} Twoja rogówka ma rogi, znaczy żyły na których zachodzi dyfrakcja...
\end{dialogue}

Oczywiście w bibliotece nie było także nic o zielonych statkach kosmicznych.
Z naukowych pism dowiedziałem się jedynie tyle, że statek tej wielkości nie mógłby długo przeżyć w próżni bez jakiegoś wodorowego zasilania.
Zasilanie fuzyjne Stella Grande zabiera większość pojemności statku, a i tak jest jedno z najmniejszych możliwych. Co więc zasilało tamten statek?
Reakcja rozpadu? Kto jeszcze używał tej brudnej i drogiej technologii w dzisiejszych czasach?
Pomyślałem, że rząd światowy przecież wyklął i zabronił używać uranowych reaktorów, więc też ciężko by było, żeby zastosował je w swoich tajnych projektach.

\begin{dialogue}
	\ds{} ,,...ratowali statki, w których żagle wiatr dąć przestał, i dawali im trochę swojego wiatru, który wcześniej razem łapali na wietrznej wyspie.''
	\ds{} Galiza, co ty czytasz? \dm{} zapytałem się.
	\ds{} ,,Ich rycerskie zbroje rany leczyły, a miecze cięły wrogów jak papirus. Walczyli z piratami morza nicości, żeby handel między wszystkimi wyspami żółtego oceanu kwitł.''
	\ds{} Bajki dla dzieci? \dm{} Bajki były jednymi z niewielu treści, których rząd nie cenzurował. Bo i po co?
	\ds{} ,,W zamian za wiatr, uratowane łodzie dawały im trochę niepotrzebności, aby rycerze mroźnej pustyni mogli wyczyścić swoje zbroje do połysku.'' \dm{} Galiza spojrzała na mnie morderczym wzrokiem. \dm{} To nie są żadne bajki. To są \emph{baśnie}. Przekazują bardzo ważne treści. ,,Dzięki Maszynie byli w stanie leczyć swe rany.''
	\ds{} Niby jakie treści? Że rycerze pływali na statkach handlowych?
	\ds{} Nie, przecież nie można tego czytać dosłownie. \dm{} Przewróciła oczyma. \dm{} Zobaczmy tutaj. Rycerze, to pewnie grupa jakichś pozytywnych, uznajmy, istot. Morze nicości to jakiś nieprzyjemny obszar, może kosmos? Statki to grupy, które przemierzały ten obszar, a tamci im w tym pomagali. I brali od nich jakąś nietypową zapłatę. Chyba chodzi o zasilanie.
	\ds{} I co z tego wynika? Przecież statki kosmiczne są samowystarczalne. Jedno napełnienie zbiornika z wodorem pozwoli na podróż do Alfa Centauri i z powrotem. Po co miałby ktoś statkom pomagać i wozić im paliwo?
	\ds{} A czy zawsze tak było?
	\ds{} No, od czasów Ostatniej Wojny. A wszystko, co było wcześniej, zniknęło w cenzurze.
	\ds{} To nie zniknęło. \dm{} Podniosła triumfalnie urządzenie. \dm{} Wiesz przecież, że napędy fuzyjne nie pojawiły się od razu.
	\ds{} Ale napęd nuklearny jest nieprzydatny do podróży kosmicznych. Zabójcze promieniowanie, mniejsza energia reakcji, niebezpieczne pozostałości, z którymi nie ma co zrobić, katastrofalne skutki ewentualnej awarii. \dm{} W czasie wymieniania znalazłem odpowiedni tytuł w książce. \dm{} Proszę, cała rozprawa o tym, dlaczego uranowe zasilanie to zły pomysł. Weź sobie to.
	\ds{} Nie dotknę się tego, to zostało ocenzurowane! Nie ma w tym ani ziarna prawdy.
	\ds{} W przeciwieństwie do twoich opowieści o kosmicznych piratach.
\end{dialogue}

Najwyraźniej nic sobie z naszej rozmowy nie robiła, bo nadal nie wyciągnęła nosa z sekcji opowiastek na dobranoc.
Próbowałem się zaczytać, ale z tylu głowy tliły się myśli na temat prawdziwości baśni.

Ukradkiem spojrzałem na tytuł, który Galiza mi zostawiła przed nosem.

\begin{sl}
\begin{quote}
	Dawno temu żył sobie dzielny i odważny bohater, który postanowił stawić czoła morskiej armii straszliwych bestii, które strzegły olbrzymiego skarbu --- zielonego złota.
	
	Niestety, sam nie umiał walczyć, więc przekonał ludzi ze swojej wyspy, aby nigdy więcej nie karmili potworów i nie współpracowali z nimi, w nadziei na to, że pozbawione wsparcia ludzi, te umrą naturalną śmiercią.
	Ale wygłodzone straszydła zamiast zdechnąć, zaczęły atakować przepływające statki w poszukiwaniu jedzenia, nie dając nic w zamian. Co więcej, zaczęły pożywiać się również swoim własnym skarbem, który tylko wzmacniał ich siłę. Potwory rosły w potęgę, gdyż zachwiana została symbioza.
	
	Przestraszony bohater wypowiedział wojnę wszystkim morskim istotom. Wiedział, że musi nie tylko je pokonać, ale także zabrać im ich zielone złoto.
	Sprawa była utrudniona, że pozbawione zielonego wspomagacza potwory zaczęły i to zabierać ze statków.
	Pomimo nierównej walki, żółci ludzie wygrali dzięki przewadze liczebnej, a nasz bohater postanowił zniszczyć całe złoto w obawie, że jego moc spowoduje jeszcze kiedyś nawrót bestii.
	
	Od tego czasu morze nicości było całkowicie wolne od niebezpieczeństw.
\end{quote}
\end{sl}

Te dwie baśnie miały ze sobą coś wspólnego. Nie wiedziałem tylko dokładnie, co. Opowiadały historię dwóch stron sporu. Ziemskiej, chyba, i kosmicznej.
Była też jakaś silna materia z którą obie strony miały do czynienia.

\begin{dialogue}
	\ds{} Nie dłub w nosie, bo przyjdzie zielony pan z kosmosu i ci go ukradnie. \dm{} Usłyszałem za biblioteczką jak matka karciła syna.
\end{dialogue}

Zwróciłem uwagę na to, tylko dlatego, że byłem wyczulony na kolor zielony.
Ciekawe, skąd wzięło się takie straszenie dzieci? Może z jednej z tych baśni? Ale raczej nie ma związku z gwiazdą, którą studiujemy.
Nie chciałem się pytać, skąd wzięła się legenda o zielonym człowieku, ale wątpię, czy zatroskana matka sama to wiedziała.

O, tu jest jakiś wiersz, takie rzeczy nie były pobłażliwie przepuszczane. Może się zaplątał w reszcie bajek.

\begin{poem}
	To jestem ja. \\
	Nie mam łatwego życia, \\
	ale inni mają gorsze. \\
	Więc zatem ja. \\
	Postanowiłem uratować innych. \\
	Przed ich życiem. \\
	Dałem im moje własne życie. \\
	Wspaniałe życie. \\
	Wspólne życie. \\
	Bo mogę, a oni nie mogą. \\
	\\
	To są inni. \\
	Mocno inni. \\
	Mocno cierpią. \\
	Życie cierpią. \\
	Życie uratuję. \\
	Wszystkich uratuję. \\
	\\
	To jesteśmy my. \\
	Tak dużo nas jest. \\
	Tak wspaniałe życie dałem im. \\
	Tak zapewne dziękują mi. \\
	Że aż czuję jak się radują. \\
	Wszyscy się radują. \\
	\\
	To nie jestem ja. \\
	Ja byłem wtedy. \\
	Nie byłem chciwy. \\
	Miałem rację. \\
	A teraz tylko kłamstwo. \\
	Muszę naprawić. \\
	Muszę usunąć zło, zostawić ich. \\
	Więc niech opuszczę siebie. \\
	Wytnę jak pestki dyni. \\
	I rzucę w zamrożoną otchłań oceanu.
\end{poem}

Pudło. Chyba. Ciekawe, co u Galizy.

\begin{dialogue}
	\ds{} Zobacz, wykopałam gdzieś kawałek skryptu filmowego. \dm{} Usłyszałem głos zza sterty czytników. \dm{} To jakieś fragmenty losowych książek.
	Zaczyna się od romansidła w stylu ,,100 twarzy Browna'', więc pewnie cenzor nawet nie przeczytał do końca.
	\ds{} Pewnie plik się uszkodził, zdarza się w nawet najnowocześniejszych systemach. \dm{} Wziąłem do ręki książkę.
	\ds{} Chciałeś bez baśni, to teraz nie marudź.
\end{dialogue}

Powieść poprzeplatana była fragmentami losowych znaków i oczywiście każda polska litera zastąpiona była kratką. Unicode wynaleziono tysiąclecie temu, ech.
Spróbowałem rozszyfrować fragment, który poleciała mi Galiza.

\begin{poem}
	\dida{Ciemny korytarz w zniszczonym statku kosmicznym. Przez wyrwy w kadłubie widać wirujące gwiazdozbiory. Słońce co chwila pojawia się w tych dziurach i rzuca ruchome cienie. Światło na scenie pochodzi jedynie od poruszających się chaotycznie plam na ścianach. Główny bohater, w kombinezonie z gwiazdami, podchodzi do siedzącej postaci w grubszym kombinezonie z logami ostrzeżeń przed promieniowaniem.}
	
	\charkap{}
	Pracowniku reaktora jądrowego, zamelduj mi natychmiast stan załogi i statku!
	
	\charkos{}
	Sam se melduj. To przez ciebie wszystko. Trzeba było uciekać, a nie walczyć.
	Nie mieliśmy z nimi szans, wszyscy ci to powtarzali.
	
	\charkap{}
	Jak się zwracasz do swojego kapitana!
	
	\charkos{}
	Co to za kapitan, który stracił całą załogę. \\
	Co to za kapitan, który nie ma nawet swojego statku!
\end{poem}
	(Tutaj plik był uszkodzony.)
\begin{poem}
	\dida{W pomieszczeniu wiszą poszarpane kombinezony. Kapitan przystawia do jednego z nich licznik Geigera, który odzywa się silnym stukotem. Przestraszony odskakuje w tył i zderza się ze stojącym przy ścianie kosmonautą. Kombinezonowi odpada kask i widać twarz z usuniętym okiem, połową zębów, uchem i nosem. Placek światła właśnie wsuwa się i oświetla trupa, żeby pokazać widzom w całej doskonałości.}
\end{poem}
	(Tutaj plik znowu był uszkodzony.)
\begin{poem}
	Gdzie są zapasowe pręty uranowe, żeby uruchomić z powrotem reaktor?
	
	\charkos{}
	Zabrane, wszystko zabrane.
	I paliwo, i załoga i twoja godność.
	
	\charkap{}
	Jeszcze jedno słowo, a odstrzelę ci łeb!
	
	\charkos{}
	Ja i tak już nie żyję. \\
	I ty też już nie żyjesz. \\
	Przystaw sobie ten klikacz do głowy, przystaw do ściany, przystaw do mnie. \\
	To może sprawdzisz, jak długo będzie boleć. \\
	Bo mnie nie będzie wcale.
	
	\dida{Kosmonauta wypuszcza z sykiem resztki powietrza z butli i osuwa się na podłogę}
	
	(Dalej jest scena jak Marianna całuje się ze zmutowanym niedźwiedzio-człowiekiem.)
\end{poem}

Oddałem książkę Galizie.

\begin{dialogue}
	\ds{} I jak? \dm{} zapytała, wynurzając nos znad sterty elektronicznych komputerków.
	\ds{} Akcja wolno się rozkręcała, ale przynajmniej finałowa scena seksu była dobra, gdy pan Brown zaprosił do zabawy cyborga.
\end{dialogue}

Chyba nie złapała dowcipu, bo prychnęła i schowała się z powrotem za książkami.

Kolejny tekst to ponownie była fantastyczna opowieść, tym razem trochę mniej artystyczna relacja.

\begin{poem}
	Więc poszedłem nachlać się cyberwina, wiesz, tego sikacza z nanobotami, i patrzę przez okno i widzę zielonego człowieka!
	No mówię ci, gałom nie wierzyłem. A ten gość to trochę dziwny był, bo niby cztery oczy miał, ale ja tam trochę podpity już byłem, więc nie bardzo wierzyłem w to, co widzę.
	Ale na wszelki wypadek wyciągnąłem tego blastera, co go na tym kontenerowcu z Jowisza zajumaliśmy i przystawiłem do szyby i powiedziałem, żeby wypierdalał, bo jak mu strzelę, to mózg z asteroid będzie zbierał.
	A on się tylko uśmiechnął jakoś tak straszliwie, że mu się buzia na pół twarzy otworzyła i wtedy tak szarpnęło statkiem, że ho ho.
	I no, ten, tak trzymałem ten blaster że była dziura w kadłubie. Więc co nam kapitan powtarzał to to, żeby szybko zatkać dziurę.
	Więc wziąłem i trochę podpity byłem i no, nogę wsadziłem.
	Więc jak mi wciągnęło tą nogę, to aż do jajec prawie. A ten zielony ludzik nadal tam był. I wtedy poczułem jak coś ciepłego i obślizgłego mi się wcina.
	Trochę cyberwino dało mi odwagi, więc wziąłem i uratowałem moje jajca.
	Znaczy, odrąbałem sobie blasterem kurwa nogę. 
	I teraz dziura była jeszcze większa i ten zielony tam zaglądał.
	I wtedy widziałem, jak on miał taki normalnie metalowy ogon, jak zwierz jakiś. I ten ogon gdzieś daleko idzie.
	To się przeraziłem jeszcze bardziej i uciekłem do włazu i zatrzasnąłem żeby już tlen nie uciekał.
	I stanąłem przed tymi drzwiami, żeby że nic się nie stało. Ale szedł kapitan, a ja próbowałem być naturalny.
	A on się mnie pyta, dlaczego nie mam nogi.
	To głupio było mi mówić, więc powiedziałem, że mi się znudziła, to ją odciąłem. Bo zawsze chciałem mieć taką fajną robotyczną, jak on.
	To tamten się wkurwił i kazał mi spierdalać, więc ja jakoś zacząłem na tej nodze spierdalać.
	Ale wtedy znowu coś uderzyło w statek i tak jakoś tym blasterem odciąłem kapitanowi tą normalną nogę. I znowu była dziura.
	I patrzę, a kapitan wessany jest w tą dziurę i klnie na mnie. I robi się zielony jak ten ludek.
	A potem patrzy na mnie i się uśmiecha w pół. Normalnie pół głowy mu się otworzyło i łapska zaczęły wychodzić i mnie za nogę zaczęły łapać. 
	To ja blasterem po tych łapskach, a one nic. Jak jedną upierdoliłem, to zaraz kolejne wychodziły. I dalej mnie trzymają.
	Więc, no, chciałem mieć drugą robotyczną nogę.
	I uciekłem, bo silne łapska mam i schowałem się w kratce od wiatraka.
	A potem była bitka i prawie wszystkich nas zajebali.
	I jak się zrobiło cicho, to weszli Polacy, podobni do tych, którym zajebaliśmy ten statek.
	A potem mnie wyciągnęli z tej kratki i postawili razem z Szamą i Laserem.
	I powiedzieli do tych zielonych ludków coś po ichniemu, a oni też im coś mówili. I dali im trzy ładne dziewczyny i Lasera.
	Dobrze, bo go nigdy nie lubiłem.
	A potem mnie prowadzili przez pokład i widziałem jak te ludki naprawiały te dziury, co zrobiłem blasterem.
	Potem zieloni sobie poszli, a mnie dali ciężkie ubranie i kazali zamiatać ściany. Ale że nie bardzo mogłem łazić, to wzięli Szamę zamiast tego.
	A potem oddali go i potem Szama był bardzo chory, to się chyba wysypka uranowa nazywa.
	Zostawili mnie ci Polacy właśnie tutaj i odlecieli naszym statkiem. Znaczy ich statkiem.
	I siedzę tutaj już dziesiąty rok i nadal nie mam robotycznych nóg. A ty za co grypsujesz?
\end{poem}

Ta wciągająca opowieść o wątpliwej prawdziwości dużo mi powiedziała o tych tajemniczych zielonych istotach.

Ja tymczasem poszukałem czegoś nieelektronicznego w tej elektronicznej bibliotece.
Okazało się, że znalazłem samolocik zrobiony z kartki komiksu.
Wyglądał na stary i mocno wyblakły. Nie było na nim daty druku, ale sądząc po ilości osób używających bibliotekę, mógł być nawet z końca wojny.
Kiedyś drukowano takie edycje kolekcjonerskie na oryginalnym drewnianym papierze, więc to dziwne że ktoś potraktował go jak śmiecia.

W każdym razie był to wycinek opowiadający historię kosmicznych węży, które kąsały ludzi i zmieniały ich w zombie.
W skrypcie filmowym była mowa o martwych ludziach, więc to chyba nie o to chodzi.
Potem te węże zamieszkiwały w tych ludziach, podtrzymując ich przy życiu. Coś w stylu pasożyta, albo jakiejś symbiozy.
Albo to może jednak nie były zombie? 
Ostatnia klatka zaczynała jakąś wojnę tych par z nie wiadomo kim.

Przypomniała mi się opowieść o dzielnym bohaterze, który walczył z jakimiś potworami. 
A potem był wiersz z punktu widzenia węża? 
Ale wąż nie mógł chyba zostawić tego zombie, żeby żył bez niego, prawda?

To zabawne, że nie zauważyłem wcześniej, jak wiele tekstów de-facto opowiada mniej więcej o tym samym. 
Wróciłem więc do moich naukowych tekstów i postanowiłem wyciągnąć niektóre liczby i powtórzyć zapisane tam obliczenia.

I wyszło mi, że Galiza miała rację, bo wiele wyników nijak się miało do tego, co autorzy książek, albo cenzorzy, chcieli mi przekazać.
Możliwe, że energia jądrowa rzeczywiście mogła kiedyś służyć jako paliwo dla rakiet. Jednak ponieważ była tak niebezpieczna i tak szkodliwa, ewentualna awaria w pustce kosmosu mogła równać się ze śmiercią. 
A pewnie i nie zawsze wystarczało uranu na całą podróż. Może na przykład jeśli statek musiał nieoczekiwanie walczyć z piratami, to zwiększony koszt energetyczny powodował, że tracił zasilanie i zaczynał dryfować w nieznane. Jakaś strona mogłaby właśnie dowozić uran takim ,,unieruchomionym'' statkom, w zamian za?

I tutaj jeszcze brakowało fragmentu układanki. Dlaczego te atomowe mutanty nie ginęły od promieniowania? Na co im były kawałki załóg?
Niby jak pomagały statkom, jeśli jednocześnie je niszczyły?

\begin{dialogue}
	\ds{} Posortowałam te opowieści chronologicznie. \dm{} Galiza zaskoczyła mnie od tyłu, gdy rozmyślałem w kącie.
	\ds{} No tak, chronologia. 
	\ds{} Tak... \dm{} zmieszała się. \dm{} To znaczy po kolei, od najstarszego do najnowszego. Historię atomowców.
	\ds{} Czy mogłabyś mi ją streścić?
	\ds{} Niby taki oczytany jesteś, ojoj.
	\ds{} Szukałem w naukowych książkach, ale nic nie znalazłem.
	\ds{} Mówiłam.
	\ds{} Wszystko ocenzurowane.
	\ds{} Mówiłam.
	\ds{} Mówiłaś, że masz pełną wiedzę o tych atomowcach.
	\ds{} Nie mówiłam. Brakuje mi jeszcze wiedzy, co ich trzymało przy życiu.
	\ds{} Znalazłem kawałek komiksu. \dm{} Dałem jej wydartą kartkę, którą z chęcią przestudiowała.
	\ds{} To nadal nie wyjaśnia za dużo.
	\ds{} Chyba nigdy nie poznamy pełnej prawdy, w końcu baśnie nie mówią niczego wprost.
	\ds{} Ale przynajmniej poznałam prawdę na temat tej gwiazdki. To musi być statek jednego z tych atomowców.
	\ds{} Myślisz? Przecież to jedynie zielony stateczek, nie ma sensu od razu wyciągać takich wniosków.
\end{dialogue}

Zrzuciłem wszystkie teksty na mój zegarek, aby poczytać później.
Razem z Galizą przeszliśmy się na dolny pokład, aby zobaczyć progres zielonego statku kosmicznego.
Nigdzie jednak nie dało się go zobaczyć.
Przechodząc w górę, przypadkowo się rozdzieliliśmy w tłumie, wiecznie miotającym się w sekcji rozrywkowej.

Spotkaliśmy się przypadkowo na korytarzu górnego pokładu, okazało się, że szukany widok jest zamknięty za grubymi wrotami.
Restauracja z szybami również była zamknięta.
Zupełnie jakby ktoś celowo odciął połowę widocznego nieba.

\begin{dialogue}
	\ds{} Masz jakiś pomysł? \dm{} zapytałem się dziewczyny, stojąc przed zamkniętymi wrotami na półsferyczny taras widokowy.
	\ds{} A ty masz? Bo jakoś nie bardzo uciekasz ze statku w kapsule ratunkowej. \dm{} Galiza oglądała dokładnie ścianę.
	\ds{} Dlaczego myślisz, że miałbym tak postąpić?
	\ds{} No nie wiem, może żeby nie stracić członków ciała w ataku zmutowanych ludzi? \dm{} powiedziała bardziej do ściany, niż do mnie.
	\ds{} Na prawdę w to wierzysz? Że niby jakiś atomowiec sprzed kilkuset lat nagle pojawił się znikąd? I będzie atakował wielki wycieczkowy statek zasilany reaktorem fuzyjnym?
	\ds{} I posiadający pół tysiąca ludzi na pokładzie. Tak.
	\ds{} Twoje bajki, nie ważne, jak logiczne by były, nadal pozostaną bajkami. Musisz zweryfikować swoje wierzenia.
	\ds{} A co myślisz, że właśnie robię? \dm{} Uderzyła w ścianę, odsłaniając panel kontrolny drzwi.
	\ds{} Co ty robisz?! Nie niszcz tego! \dm{} skarciłem ją.
	\ds{} Nie jestem grzeczną dziewczynką, przykro mi.
	\ds{} Zaraz ktoś tu przyjdzie i nas zobaczy. Trafimy prosto do tytanowego więzienia. \dm{} Kątem oka przyjrzałem się elektronicznym przełącznikom.
	\ds{} Co tu się dzieje? \dm{} Jak na zawołanie kapitan statku wyszedł zza rogu. Całą podróż siedzi na mostku żeby akurat przejść się właśnie w tym momencie pustym korytarzem.
	\ds{} Nic.
	\ds{} Dziura w ścianie, to jest nic?
	\ds{} Pan ma dziury w ścianach, to my też chcieliśmy mieć. 
	\ds{} Proszę? \dm{} Zapytał się, zmieszany. Skąd przyszedł mi do głowy pomysł, że to by wypaliło?
	\ds{} Chcieliśmy popatrzeć trochę na czerwony kolor.
	\ds{} Czerwony? \dm{} Kapitan i Galiza zapytali się jednocześnie.
	\ds{} Tak, czerwony. \dm{} Szturchnąłem dziewczynę. Czerwony jest taki ładny i jest wstępem do innego koloru.
	\ds{} Nie wiem, co planujecie, ale wzywam ochronę. \dm{} Kapitan sięgnął do pasa, wtedy Galiza wcisnęła czerwony przycisk na tablicy.
\end{dialogue}

Drzwi otwarły się z łoskotem.
Zieleń zalała korytarz, dodając nienaturalnego połysku przedmiotom.
Kontury się rozmyły, wściekła twarz dowódczy zyskała nowe oblicze grozy.
Odwróciliśmy się powoli.

Mały, kanciasty statek kosmiczny zbliżał się ku Stella Grande z dużą prędkością.
Jego zielonkawa łuna jasno odbijała się na tle czarnego nieba.
Słońce ostro oświetlało jedną ścianę, na której malowały się liczne pęknięcia i spawy.
Nierówne krawędzie wykonane były jak gdyby z wielu osobnych kawałków złomu, posklejanych razem.
Brud i oleiste zacieki znaczyły otwory w kadłubie.

Z przodu była popękana szyba, od środka pokryta jakby ciemnymi plamami pleśni.
Zielonkawa łuna biła ze środka i można było tylko przypuszczać, co to zaglonione akwarium skrywa.
Niewyraźne kontury poruszały się w środku w biomechaniczny sposób.
To nie był wspaniały rycerz, ani też potwór niewiadomego pochodzenia. 
To było znacznie bardziej bliskie każdemu człowiekowi.

\begin{dialogue}
	\ds{} Mamusiu, to jest ten zielony człowiek, który miał mi ukraść nos? \dm{} Usłyszeliśmy za sobą.
	\ds{} I to mają być te wspaniałe fajerwerki?
	\ds{} Ja nie widzę tutaj darmowego poczęstunku.
	\ds{} A to nie miał być ten świetny kabareciarz z knajpy?
	\ds{} Po chuj tu przychodziłem.
	\ds{} Ale to chyba nie jest zbyt bezpieczne.
	\ds{} Z jakiej okazji to wydarzenie?
\end{dialogue}

Kapitan został zupełnie zbity z tropu przez tłumek osób, który jak na zawołanie wpełzł na scenę z każdej strony.

\begin{dialogue}
	\dm{} Proszę państwa, nie wolno tutaj być. To niebezpieczne. \dm{} Kapitan próbował załagodzić sytuację, ale prawie został zadeptany przed napierające rodziny z dziećmi.
	\ds{} To twoja sprawka? \dm{} szepnąłem do Galizy gdy wślizgiwaliśmy się przez już otwarte wrota. \dm{} Ja bym tak nie umiał.
	\ds{} Wystarczyło powiedzieć, że będą darmowe kupony. 
\end{dialogue}

Zieleń wzmagała na sile, gdy kanciasta gwiazda przybliżała się do wycieczkowca.
Ustawiła się dziobem do nas i wcale nie zwalniała.

\begin{dialogue}
	\ds{} Przecież on się chce z nami zderzyć! \dm zawołałem. Spojrzeliśmy na wyjście, które zatkany było przez kłębiące się osoby. Każdy chciał z bliska zobaczyć to niezwykłe zjawisko.
	\ds{} Wizgrant. Co myśmy narobili. \dm{} dziewczyna szepnęła.
	\ds{} To się z nami zderzy! Ewakuacja! \dm{} krzyczałem do ludzi.
	\ds{} Co pan, chce sobie wziąć wszystkie kupony dla siebie? \dm{} Jakaś gruba kobieta się oburzyła.
	\ds{} Zje nas wszystkich.
\end{dialogue}

Pociągnąłem Galizę między krzesła z boku, kalkulowałem, jak odbije się fala uderzeniowa w tej szklanej bańce i czy nie pęknie, katapultując nas wszystkich w przestrzeń.
Skuliliśmy się, obserwując ruchome cienie od zielonej gwiazdy.
Na kilka sekund przed kolizją ludzie chyba się ogarnęli, bo zaczęli przeciskać się z powrotem.

Uderzenie ścięło wszystkie krzesła i zwaliło ludzi.
Dziób złomiastego atomowca wbił się w kopułę tak, że powietrze nawet nie uciekało.
Zaczął się powoli otwierać, wyciekła ze środka jakaś czarna substancja.
Trupi odór uderzył nas wszystkich po nosach.
Wszyscy ludzie umilkli, niektórzy stanęli, zmrożeni strachem.
Dało się słyszeć sapanie i kulejący krok.

I wtedy w wejściu pojawił się \emph{on}.
A raczej. Oni.

Kupa członków ludzkich, posklejanych byle jak, poprzeplatana mechanicznymi fragmentami.
Cieknąca śmiercią.
Bulgocząca.
Nieobecna.

Oczy tego monstra skanowały teren, skuliliśmy się jeszcze bardziej.
Skupiły się na wejściu i przerażonym zatorze powywracanych ludzi.
Ze środka kupy wysunęła się czyjaś ręka i pięć nóg.
Zaczęły posuwać cielsko w stronę panikujących osób.

Mutant zatrzymał się przy najbliżej leżącym człowieku.
Otworzył górną część, z której wylazły poskręcane ręce.
Wzięły biedaka za nogę, jak szczypce, i poczęły wciągać do wnętrza zielonej istoty.
Za chwilę z drugiej strony wypełzła nowa noga, a fragment skóry wymienił się na świeży.
Nienasycony atomowiec zaczął wędrówkę do reszty osób.

Wtedy zauważyłem grubą pępowinę, która łączyła potwora ze statkiem.
Stalowy wąż był jak życiodajna rura.
Pokryta czarną mazią i cieknąca śliską substancją.
To właśnie był brakujący fragment układanki.
Gdyby udało się ją jakoś przeciąć.

Potem mi się przypomniało, po co leciałem na Tytana.
\begin{dialogue}
	\dm{} Niebieski zamyka wrota. \dm{} szepnąłem Galizie na ucho i pocałowałem.
\end{dialogue}

Zanim zareagowała, już byłem w połowie drogi do statku.

Smród zaczął wypalać mi nozdrza.
Wziąłem głęboki wdech i zanurkowałem do kosmicznej rakiety.
Właściciel chyba nawet mnie nie zauważył, tylko dalej pałaszował tłum.

W środku statek wyglądał równie źle, jak na zewnątrz.
Była to jedna komora, na środku której widniał pulpit sterowniczy, a wszędzie indziej walał się brudny złom.
Wąż podłączony był do wielkiego pudła w ścianie.
Przypadłem zaraz do niego, patrząc jak można by odłączyć pępowinę, gdy usłyszałem krzyk Galizy.
Mutant chyba ogarnął, że coś jest nie tak, bo zaczął pełzać w moją stronę, niemal się tocząc na nowozdobytych członkach.

Nie znajdując dobrego rozwiązania, złapałem za dźwignię ciągu na konsoli i pociągnąłem w tył.
Statek wyrwał się z dziury, a dziób zatrzasnął się z łoskotem, przyciśnięty szklaną ścianą.
Potężna siła ciągu odrzuciła mnie w przód, a że nie puściłem się dźwigni, statek powtórzył i szarpnął w drugą stronę.

Mutant uderzył z całą siłą w brudną szybę, rozplaskując czarną substancję. Zaczął się wić w agonii i próbować wejść z powrotem przez szczelinę w dziobie, który zablokował się na pępowinie.

Zaparłem się o obleśny fotel i pociągnąłem wajchę zaraz w tył, odrzucając zielonego na całą długość węża.
A potem w bok. I w tył. I znowu w przód.
Za każdym szarpnięciem rozbijał się o kadłub statku.
Syczący dźwięk uciekającego powietrza tylko mnie poganiał.
Ruszałem wszystkimi dźwigniami na wszystkie strony.

Zatrzaśnięcie się włazu do końca wybiło mnie z rytmu.
Na końcu węża została rozczapierzona końcówka ze skrawkami dawnego właściciela.
Przez tłustą szybę widziałem oddalającą się Stalle Grande z wielką dziurą w tarasie widokowym.
Nie mogłem dojrzeć, co z ludźmi, co z Galizą.
Ale miałem nadzieję, że zamknęła drzwi na czas.

Siadłem pod obleśną ścianą, w kałuży śliskiego oleju i począłem rozmyślać nad moją sytuacją.
Przez moją głowę przepływały przeczytane opowieści.
Sięgnąłem do zegarka, żeby przeczytać książki. Tym razem prawdziwe.
Te, które Galiza mi znalazła. Nieocenzurowane.

Jak trochę mi przeszło, wstałem, zostawiając na ścianie kawałki mojej skóry.
Gdybym miał licznik Geigera, pewnie wyleciałby poza skalę.
Żadne życie nie wytrzyma takiego promieniowania.
Ale jak w takim razie ten mutant...?

Zacząłem badać urządzenie, do którego podłączona była rura. 
Pismo w jednym z przedwojennych alfabetów.
Tabliczki znamionowe z datami i numerami potwierdzały tezy Galizy.
To było zbudowane w erze, w której wszystkie statki latały na uran.
Samo jest zasilane tą energią.
Czy to mógł być ten legendarny Aparat?
Urządzenie, które naprawiało zniszczone DNA? Maszyna... nieśmiertelności?
Bardzo chciałem je rozkręcić i zbadać, ale chyba nie wystarczyłoby mi życia na to.
Umrę prawdopodobnie za kilkanaście godzin. 

Popatrzyłem na końcówkę węża.
Chyba, że.

Usiadłem, zamknąłem oczy.
Ból był ostry i przeszywający.
Powoli ustępował.
Poczułem bulgotanie w żyłach.
Poczułem głód.
Ten przeklęty głód.

Za oknem nadal świeciła się Stella Grande. Ten głód.
Położyłem dłoń na dźwigni ciągu. Wiedziałem, co muszę zrobić.

Pociągnąłem w tył z całej siły.



