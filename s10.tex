\chapter{Bicie dzwonu, śmierć komu?}

\info{Bombastyczna rodzinka spotyka na swojej drodze tajemniczy zegar.}

Przylecieli swoim małym statkiem pod wieczór.
Zwabieni dużą aktywnością uniwersalności w tej galaktyce.
Ukryli rakietę w pobliskich górkach, wzięli najpotrzebniejsze narzędzia i poszli rozejrzeć się po zabudowaniach.
Ta cywilizacja nie powinna była istnieć.

Ale istniała i należało to zbadać.

Było pochmurno.
Mglista i zimna noc wgryzała się w grube, ciemne ubrania, w jakie się dla niepoznaki ubrali.
Duży, wczesnowiosenny i niebieski księżyc oświetlał świat trupią łuną, niczym wielka lampa nad stołem w prosektorium.
Będąc tam, wolałeś trzymać się w cieniu, wydawało się jakby to światło mogłoby wyssać z ciebie życie.
Ale i cień wydawał się nieprzyjemnym atramentem, który może złapać za nogę i wciągnąć w siebie na zawsze, jak czarna dziura. Nigdzie nie było ratunku.

Był też drugi księżyc, mniejszy i cały czerwony.
Nadawał miastu krwisty poblask. 
Z prosektorium przenosił do ciemni fotograficznej, gdzie wisiały kościste zdjęcia rentgenowskie.
Była wczesna wiosna, ale odczucie towarzyszyło jednak późnej jesieni.
Leżące gdzieniegdzie kupy śniegu, wyglądały w tym świetle jak zwłoki rozszarpanych zwierząt.
Tam, gdzie światła księżyców się spotykały, panował nieprzyjemny fiolet, niczym żałobna szata rozpostarta na ziemi, ciągnąca się od horyzontu po horyzont.

Źródło uniwersalności, nieprawidłowej materii wszechświata, znajdowało się w centrum miasteczka.
Czwórka przybyszów posnuła się zabłoconymi uliczkami, omijając kałuże o niezbadanej głębokości.
Starali się pływać w cienistym atramencie, aby nie zwracać na siebie zbytniej uwagi.
Elektroniczny czujnik uniwersalności pikał coraz szybciej, niczym EKG leżącego na stole operacyjnym pacjenta.
Napawało to nadzieją, pomimo że każda zdrowa na umyśle osoba omijałaby uniwersalność o lata świetlne, gdyby wiedziała czym jest i czym może być.
Nie dotyczyło to przybyszy.

Najmocniejsze odczyty były na głównym placyku miasteczka.
To jedyne miejsce, gdzie błoto było tak zadeptane, jakby przeryło je stado ziemskich dzików.
Połyskujące kałuże patrzyły się na rodzinkę swoimi czerwonymi, księżycowymi źrenicami.
Ale nie wyglądały strasznie, bowiem i tak przyćmione były czernią, bijącą od wielkiej wieży.
Wieży zegarowej, górującej nad półkolistymi budynkami, niczym sęp nad swoją przyszłą ofiarą.

Zegar jaśniał żółtawym światłem, niczym trzeci księżyc.
Miał osiem godzin i jedną wskazówkę, która poruszała się w lewo, odwrotnie niż na ziemskich zegarach.
Złoty okrąg odbijał księżyce, niwelując ich nieprzyjemność.
Pomimo złowrogiego zarysu wieży wokół, żółte oko emanowało duchowym ciepłem i domowością.
Ale nauczona doświadczeniem rodzinka Nocnych wolała stać w pełnym fiolecie, zamiast ufać tej dziwacznej i niebezpiecznej materii, jaką była uniwersalność.

Na plac powoli zaczęli schodzić się mieszkańcy.
Ubrani byli w podobne stroje do gości, snuli się, omijając błotne pułapki.
Nie rozmawiając ze sobą, szli niczym na śmierć.
Wpatrywali się w słoneczny cyferblat, jak w jakiegoś boga.
Wskazówka wskazywała niemal na dolną godzinę, niczym palec kostuchy wybierający swoją następną ofiarę.
Czekali.

Architektura miasta składała się z jednopoziomowych, smukłych budynków z ciemnego metalu.
Spiczaste zdobienia przypominały na myśl pazury stworzeń spod łóżka.
Małe okienka, przepuszczające czerwone światło pochodni, wyglądały jak oczy czających się w cieniu drapieżników.
Wszystkim buchała para z ust i unosiła się bardzo wolno, prawie się nie rozmywając, niczym uchodzące w zaświaty dusze.
Podobne dymne kolumny sterczały z wierzchołka każdego domu, jak nitki marionetek sterowane przez wielkiego lalkarza, albo raczej, zegarmistrza.
Widać było, że ci ludzie zbierali się na tym placu co noc.

Przestraszone dziewczynki przytuliły się do kolan swojego taty, a żona gładziła je po ukrytych pod ciemnymi kapturami głowach.
\begin{dialogue}
	\ds{} Boję się \dm{} wyszeptała siedmioletnia Żywia do swojej mamusi.
	\ds{} Ale ty jesteś strachliwa \dm{} odpowiedziała jej starcza o rok Nadzieja.
	\ds{} Ciszej, bo nas odkryją \dm{} szepnął im ojciec rodziny. \dm{} Zobaczcie, chyba coś się dzieje.
\end{dialogue}

Wskazówka przesunęła się lekko w prawo, lądując dokładnie na dolnym znaku. 
Coś zgrzytnęło, coś metalicznie stuknęło we wnętrzu wieży.
Złoty okrąg tarczy błysnął.
Wszyscy ludzie umilkli.
Wtedy dał się słyszeć mechaniczny głos, niczym wybijany na mechanizmie.

\begin{sl}
\begin{quote}
Posłuchajcie uważnie moje dziatki \\
wybitej o północy zagadki. \\
Przed wschodem słońca znajdźcie rozwiązanie, \\
albo komuś coś się dzisiaj stanie. \\
Bim-bom. \\
Z piecem dom. \\
Trucizna w pudełku. \\
Obok kowadełka. \\
Dodana do soli. \\
Nie zrobi tego powoli. \\
Ofiara się nie ukryje. \\
Zabije to zabije. \\
\end{quote}
\end{sl}

I była tam wyróżniająca się grupka mieszkańców.
Mieli mniej eleganckie stroje i pałki w rękach, pobiegli w jakimś kierunku, większość osób podążyło za nimi, co także uczynili kosmiczni przybysze.
Początkowa cisza zniknęła, zrobiło się głośniej.
Ludzie rozprawiali o tym, czego może dzisiaj dotyczyć dzisiejsza zagadka.
Fioletowy całun śmierci został rozdarty przez chaotyczne biegi, głośne rozmowy i kłótnie o rozwiązania.

Rodzice uważnie podsłuchiwali wszystkiego wokół siebie.
Wygląda na to, że zegar zadawał zagadkę codziennie i codziennie losowa osoba umierała pod wpływem magicznej siły mechanizmu.
Nie było widać śladów morderstwa, a w prawej ręce ofiary pojawiał się zwój papieru z odpowiedzią.
Nigdy w historii miasta nie udało się rozwiązać zagadki na czas.

Maria Nocna robiła notatki ze wszystkiego na swoim elektronicznym komunikatorze.
Uniwersalność była nieprawidłową materią, coś jak rak wszechświata.
Nie słuchała się zasad fizyki, nie można było nią sterować, nie miała sensu.
Przyjmowała bardzo różne formy, od tortów weselnych, po latające po wszechświecie dziadki w wannach pełnych wody.
A tutaj był zegar zadający zagadki.

Dom piekarza otoczony był szczelnym kordonem gapiów.
W błocie, na kolanach, siedział związany piekarz.
\begin{dialogue}
	\ds{} Gdzie jest ta trucizna, którą chcesz dodać rano do chleba? \dm{} Przywódca bandy groził mu nożem. \dm{} Wiem, że to ty, tym razem rozwiążemy tę zagadkę.
	\ds{} Przysięgam! \dm{} płakał piekarz. \dm{} To nie o mnie tej nocy chodzi! Nikogo nie zamierzam otruć.
	\ds{} To poszukamy jej za ciebie. \dm{} Pstryknął palcami i reszta jego bandy zaczęła przewracać mu mieszkanie do góry nogami. \ds{} Zapytam się jeszcze raz...
	\ds{} ...nic trującego nie mam, tylko rzeczy do wyrobu chleba. Zagadka przecież nigdy nie jest oczywista, to nie ja.
	\ds{} Czyżby? Jest piec? Jest. Jest kowadełko do krojenia? Jest. Są ofiary jedzące rano twój chleb? Są.
	\ds{} Wystarczy że nie upiekę dzisiaj rano niczego i nikt nie będzie mógł się otruć.
	\ds{} A więc potwierdzasz, że gdybyś upiekł, to ktoś by się otruł, tak?
\end{dialogue}
Więzień przewrócił oczyma, to będzie długa noc.

Tymczasem ktoś przyniósł znalezione pudełko.
Przywódca otworzył, powąchał i uśmiechnął się.

\begin{dialogue}
	\ds{} A co to w takim razie jest, proszę piekarza? \dm{} Podsunął mu pudełko pod nos.
	\ds{} Sól.
	\ds{} A słyszałeś może także nazwę ,,biała śmierć?''
	\ds{} Jak się wszystkiego zje za dużo, to zaszkodzi.
	\ds{} Zagadka mówiła, że trucizna miała być dodana do soli, zobaczymy zaraz ile w tym prawdy. \dm{} Wsadził mu pudełko w twarz. \dm{} Jedz tą swoją truciznę!
\end{dialogue}

Szepty przeszły po ludziach.
Ktoś podniósł rękę, ktoś się cofnął o pakę kroków, ktoś zemdlał.
Chyba nie za bardzo ludziom się to podobało.
\begin{dialogue}
	\ds{} Dzisiejsza noc jest specjalna. Dzisiaj bowiem rozwiązujemy Zagadkę. Dzisiaj uwalniamy się spod władzy Zegara.
\end{dialogue}
Kilka osób popatrzyło się na wieżę, wieża popatrzyła się z powrotem na nich. Błysnęła złotym okręgiem tarczy. Mam was na oku.
\begin{dialogue}
	\ds{} Jeśli tego nie zrobimy, zginie dzisiaj więcej niż jedna osoba, zginęliby wszyscy, którzy kupiliby poranny chleb od tego truciciela.
\end{dialogue}
Ktoś chciał protestować, ktoś chciał ocalić torturowanego biedaka.
Ale jak kilka osób odwróciło się w jego kierunku z takim samym wyrazem twarzy, jak u przywódcy bandy, to zaraz uciekł od zbiorowiska.
I tym sposobem biedaczysko zjadło kilogram soli.

Zostawili go w zimnym błocie, a księżyce przykryły go fioletowym całunem.
Jeszcze chwilę trząsł się, aż wkrótce umarł od zimna, albo nadmiaru soli.
Chwalili się potem, że znaleźli rozwiązanie, że pokonali Zegar.
Ale jakoś nikt nie potrafił tego potwierdzić. 
Zegar tykał, jak dawniej ale i nikt nie wiedział co miałby zrobić po rozwiązaniu jego zagadki.
Zatrzymać się? Wybić pochwalną melodyjkę? Zapaść pod ziemię? A może dać punkt i kontynuować zagadki następnej nocy?

Słońce wstawało, zalewając miasto bielą.
Biały karzeł był małą gwiazdką, przebijającą się ledwo przez chmury.
Dzień osuszał atrament i zwijał w kłębek fioletową kołdrę.
Błoto przestawało patrzeć się na ciebie, a domy mieniły się tęczowo.
Noc nawet nie chowała się jak zwykle po kątach, uciekła cała za horyzont.
Jedynie zamarznięty piekarz na środku ulicy był trochę nie na miejscu.

I wtedy zegar uderzył dzwonem tak mocno, że gdyby ludzie mieli szyby w oknach, to już by ich nie mieli.
Nadzieja przewróciła się w błoto w akompaniamencie śmiechów siostry.
Za karę wciągnęła ją razem ze sobą.
Ludzie przystanęli na chwilę, pomacali się po sobie, jakby chcieli sprawdzić czy nadal żyją.
Ktoś sobie sprawdzał puls.
Po kilku sekundach, rozluźnieni, kontynuowali swoje wędrówki.






