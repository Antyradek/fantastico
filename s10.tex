\chapter{Bicie dzwonu, śmierć komu?}

\info{Bombastyczna rodzinka spotyka na swojej drodze tajemniczy zegar.}

Przylecieli swoim małym statkiem pod wieczór.
Zwabieni dużą aktywnością uniwersalności w tej galaktyce.
Ukryli rakietę w pobliskich górkach, wzięli najpotrzebniejsze narzędzia i poszli rozejrzeć się po zabudowaniach.
Ta cywilizacja nie powinna była istnieć.

Ale istniała i należało to zbadać.

Było pochmurno.
Mglista i zimna noc wgryzała się w grube, ciemne ubrania, w jakie się dla niepoznaki ubrali.
Duży, wczesnowiosenny i niebieski księżyc oświetlał świat trupią łuną, niczym wielka lampa nad stołem w prosektorium.
Będąc tam, wolałeś trzymać się w cieniu, wydawało się jakby to światło mogłoby wyssać z ciebie życie.
Ale i cień wydawał się nieprzyjemnym atramentem, który może złapać za nogę i wciągnąć w siebie na zawsze, jak czarna dziura. Nigdzie nie było ratunku.

Był też drugi księżyc, mniejszy i cały czerwony.
Nadawał miastu krwisty poblask. 
Z prosektorium przenosił do ciemni fotograficznej, gdzie wisiały kościste zdjęcia rentgenowskie.
Była wczesna wiosna, ale odczucie towarzyszyło jednak późnej jesieni.
Leżące gdzieniegdzie kupy śniegu, wyglądały w tym świetle jak zwłoki rozszarpanych zwierząt.
Tam, gdzie światła księżyców się spotykały, panował nieprzyjemny fiolet, niczym żałobna szata rozpostarta na ziemi, ciągnąca się od horyzontu po horyzont.

Źródło uniwersalności, nieprawidłowej materii wszechświata, znajdowało się w centrum miasteczka.
Czwórka przybyszów posuwała się zabłoconymi uliczkami, omijając kałuże o niezbadanej głębokości.
Starali się pływać w cienistym atramencie, aby nie zwracać na siebie zbytniej uwagi.
Elektroniczny czujnik uniwersalności pikał coraz szybciej, niczym EKG leżącego na stole operacyjnym pacjenta.
Napawało to nadzieją, pomimo że każda zdrowa na umyśle osoba omijałaby uniwersalność o lata świetlne, gdyby wiedziała czym jest i czym może być.
Nie dotyczyło to przybyszy.

Najmocniejsze odczyty były na głównym placyku miasteczka.
To jedyne miejsce, gdzie błoto było tak zadeptane, jakby przeryło je stado ziemskich dzików.
Połyskujące kałuże patrzyły się na rodzinkę swoimi czerwonymi, księżycowymi źrenicami.
Ale nie wyglądały strasznie, bowiem i tak przyćmione były czernią, bijącą od wielkiej wieży.
Wieży zegarowej, górującej nad półkolistymi budynkami, niczym sęp nad swoją przyszłą ofiarą.

Zegar jaśniał żółtawym światłem, niczym trzeci księżyc.
Miał osiem godzin i jedną wskazówkę, która poruszała się w lewo, odwrotnie niż na ziemskich zegarach.
Złoty okrąg odbijał księżyce, niwelując ich nieprzyjemność.
Pomimo złowrogiego zarysu wieży wokół, żółte oko emanowało duchowym ciepłem i domowością.
Ale nauczona doświadczeniem rodzinka Nocnych wolała stać w pełnym fiolecie, zamiast ufać tej dziwacznej i niebezpiecznej materii, jaką była uniwersalność.

Na plac powoli zaczęli schodzić się mieszkańcy.
Ubrani byli w podobne stroje do gości, snuli się, omijając błotne pułapki.
Nie rozmawiając ze sobą, szli niczym na śmierć.
Wpatrywali się w słoneczny cyferblat, jak w jakiegoś boga.
Wskazówka wskazywała niemal na dolną godzinę, niczym palec kostuchy wybierający swoją następną ofiarę.
Czekali.

Architektura miasta składała się z jednopoziomowych, smukłych budynków z ciemnego metalu.
Spiczaste zdobienia przypominały na myśl pazury stworzeń spod łóżka.
Małe okienka, przepuszczające czerwone światło pochodni, wyglądały jak oczy czających się w cieniu drapieżników.
Wszystkim buchała para z ust i unosiła się bardzo wolno, prawie się nie rozmywając, niczym uchodzące w zaświaty dusze.
Podobne dymne kolumny sterczały z wierzchołka każdego domu, jak nitki marionetek sterowane przez wielkiego lalkarza, albo raczej, zegarmistrza.
Widać było, że ci ludzie zbierali się na tym placu co noc.

Przestraszone dziewczynki przytuliły się do kolan swojego taty, a żona gładziła je po ukrytych pod ciemnymi kapturami głowach.
\begin{dialogue}
	\ds{} Boję się \dm{} wyszeptała siedmioletnia Żywia do swojej mamusi.
	\ds{} Ale ty jesteś strachliwa \dm{} odpowiedziała jej starcza o rok Nadzieja.
	\ds{} Ciszej, bo nas odkryją \dm{} szepnął im ojciec rodziny. \dm{} Zobaczcie, chyba coś się dzieje.
\end{dialogue}

Wskazówka przesunęła się lekko w prawo, lądując dokładnie na dolnym znaku. 
Coś zgrzytnęło, coś metalicznie stuknęło we wnętrzu wieży.
Złoty okrąg tarczy błysnął.
Wszyscy ludzie umilkli.
Wtedy dał się słyszeć mechaniczny głos, niczym wybijany na mechanizmie.

\begin{sl}
\begin{quote}
Posłuchajcie uważnie moje dziatki \\
wybitej o północy zagadki. \\
Przed wschodem słońca znajdźcie rozwiązanie, \\
albo komuś coś się dzisiaj stanie. \\
Bim-bom. \\
Z piecem dom. \\
Trucizna w pudełku. \\
Obok kowadełka. \\
Dodana do soli. \\
Nie zrobi tego powoli. \\
Ofiara się nie ukryje. \\
Zabije to zabije. \\
\end{quote}
\end{sl}

I była tam wyróżniająca się grupka mieszkańców.
Mieli mniej eleganckie stroje, chusty na twarzach i pałki w rękach, pobiegli w jakimś kierunku, większość osób podążyło za nimi, co także uczynili kosmiczni przybysze.
Początkowa cisza zniknęła, zrobiło się głośniej.
Ludzie rozprawiali o tym, czego może dzisiaj dotyczyć dzisiejsza zagadka.
Fioletowy całun śmierci został rozdarty przez chaotyczne biegi, głośne rozmowy i kłótnie o rozwiązania.

Rodzice uważnie podsłuchiwali wszystkiego wokół siebie.
Wygląda na to, że zegar zadawał zagadkę codziennie i codziennie losowa osoba umierała pod wpływem magicznej siły mechanizmu.
Nie było widać śladów morderstwa, a w prawej ręce ofiary pojawiał się zwój papieru z odpowiedzią.
Nigdy w historii miasta nie udało się rozwiązać zagadki na czas.

Maria Nocna robiła notatki ze wszystkiego na swoim elektronicznym komunikatorze.
Uniwersalność była nieprawidłową materią, coś jak rak wszechświata.
Nie słuchała się zasad fizyki, nie można było nią sterować, nie miała sensu.
Przyjmowała bardzo różne formy, od tortów weselnych, po latające po wszechświecie dziadki w wannach pełnych wody.
A tutaj był zegar zadający zagadki.

Dom piekarza otoczony był szczelnym kordonem gapiów.
W błocie, na kolanach, siedział związany piekarz.
\begin{dialogue}
	\ds{} Gdzie jest ta trucizna, którą chcesz dodać rano do chleba? \dm{} Przywódca bandy groził mu nożem. \dm{} Wiem, że to ty, tym razem rozwiążemy tę zagadkę.
	\ds{} Przysięgam! \dm{} płakał piekarz. \dm{} To nie o mnie tej nocy chodzi! Nikogo nie zamierzam otruć.
	\ds{} To poszukamy jej za ciebie. \dm{} Pstryknął palcami i reszta jego bandy zaczęła przewracać mu mieszkanie do góry nogami. \ds{} Zapytam się jeszcze raz...
	\ds{} ...nic trującego nie mam, tylko rzeczy do wyrobu chleba. Zagadka przecież nigdy nie jest oczywista, to nie ja.
	\ds{} Czyżby? Jest piec? Jest. Jest kowadełko do krojenia? Jest. Są ofiary jedzące rano twój chleb? Są.
	\ds{} Wystarczy że nie upiekę dzisiaj rano niczego i nikt nie będzie mógł się otruć.
	\ds{} A więc potwierdzasz, że gdybyś upiekł, to ktoś by się otruł, tak?
\end{dialogue}
Więzień przewrócił oczyma, to będzie długa noc.

Tymczasem ktoś przyniósł znalezione pudełko.
Przywódca otworzył, powąchał i uśmiechnął się.

\begin{dialogue}
	\ds{} A co to w takim razie jest, proszę piekarza? \dm{} Podsunął mu pudełko pod nos.
	\ds{} Sól.
	\ds{} A słyszałeś może także nazwę ,,biała śmierć?''
	\ds{} Jak się wszystkiego zje za dużo, to zaszkodzi.
	\ds{} Zagadka mówiła, że trucizna miała być dodana do soli, zobaczymy zaraz ile w tym prawdy. \dm{} Wsadził mu pudełko w twarz. \dm{} Jedz tę swoją truciznę!
\end{dialogue}

Szepty przeszły po ludziach.
Ktoś podniósł rękę, ktoś się cofnął o parę kroków, ktoś zemdlał.
Chyba nie za bardzo ludziom się to podobało.
\begin{dialogue}
	\ds{} Dzisiejsza noc jest specjalna. Dzisiaj bowiem rozwiązujemy Zagadkę. Dzisiaj uwalniamy się spod władzy Zegara.
\end{dialogue}
Kilka osób popatrzyło się na wieżę, wieża popatrzyła się z powrotem na nich. Błysnęła złotym okręgiem tarczy. Mam was na oku.
\begin{dialogue}
	\ds{} Jeśli tego nie zrobimy, zginie dzisiaj więcej niż jedna osoba, zginęliby wszyscy, którzy kupiliby poranny chleb od tego truciciela.
\end{dialogue}
Ktoś chciał protestować, ktoś chciał ocalić torturowanego biedaka.
Ale jak kilka osób odwróciło się w jego kierunku z takim samym wyrazem twarzy, jak u przywódcy bandy, to zaraz uciekł od zbiorowiska.
I tym sposobem biedaczysko zjadło kilogram soli.

Zostawili go w zimnym błocie, a księżyce przykryły go fioletowym całunem.
Jeszcze chwilę trząsł się, aż wkrótce umarł od zimna, albo nadmiaru soli.
Chwalili się potem, że znaleźli rozwiązanie, że pokonali Zegar.
Ale jakoś nikt nie potrafił tego potwierdzić. 
Zegar tykał, jak dawniej ale i nikt nie wiedział co miałby zrobić po rozwiązaniu jego zagadki.
Zatrzymać się? Wybić pochwalną melodyjkę? Zapaść pod ziemię? A może dać punkt i kontynuować zagadki następnej nocy?

Słońce wstawało, zalewając miasto bielą.
Biały karzeł był małą gwiazdką, przebijającą się ledwo przez chmury.
Dzień osuszał atrament i zwijał w kłębek fioletową kołdrę.
Błoto przestawało patrzeć się na ciebie, a domy mieniły się tęczowo.
Noc nie chowała się, jak zwykle, po kątach, uciekła cała za horyzont.
Jedynie zamarznięty piekarz na środku ulicy był trochę nie na miejscu.

I wtedy zegar uderzył dzwonem tak mocno, że gdyby ludzie mieli w oknach szyby, to już by ich nie mieli.
Nadzieja przewróciła się w kupę, jeszcze niedawno krwistego, śniegu w akompaniamencie śmiechów siostry.
Za karę wciągnęła ją razem ze sobą.
Ludzie przystanęli na chwilę, pomacali się po sobie, jakby chcieli sprawdzić czy nadal żyją.
Ktoś sobie sprawdzał puls.
Po kilku sekundach, rozluźnieni, kontynuowali swoje wędrówki.

Grupka osób biegła niespiesznie, jakby chcieli zdążyć na pociąg, który i tak ma opóźnienie.
Pierwsza osoba trzymała w ręce zwój papieru.
Do grupki dołączali się też inni ludzie, zaciekawieni rozwiązaniem zagadki.
Rodzinka Nocnych także poszła za nimi.
Zegar swoim uderzeniem zabił jakąś prostytutkę z domu publicznego na skraju miasta.
W jej ręce znaleziono ten zwój, tak było co ranek.

Przybiegli do domu kowala.
Kowal siedział przed domem i popatrzył się na przybyszów w taki sposób, jak trener patrzy się na swojego zawodnika, który przybiegł ostatni.
\begin{dialogue}
	\ds{} Moja żona umarła dzisiaj rano \dm{} zaczął czytać z papirusu, pomimo że nawet go nie trzymał. \dm{} Została otruta naszyjnikiem z masy solnej, który ktoś jej podarował. \dm{}
		Przebił właściciela papirusu wzrokiem. \dm{} Kto? Na pewno nie piekarz, którego zamordowaliście dzisiaj w nocy.
	\ds{} Ja to zrobiłem. \dm{} Aptekarz wyszedł z domu obok. \dm{} Ta jędza uczyła nasze dzieci, że Zegar jest jakimś rodzajem boga. Że trzeba mu oddawać cześć.
		Bóg jest tylko jeden, a ten zegar jest jego dokładnym przeciwieństwem! \dm{} Kilka osób mimowolnie popatrzyło na wieżę, złoty okrąg wokół tarczy błysnął, jakby chciał zaprzeczyć jego słowa.
	\ds{} A kowadło? A pudełko? \dm{} dopytywali się ludzie.
	\ds{} Kowadełko. \dm{} Aptekarz wskazał palcem na ucho. \dm{} Pudełko. \dm{} Wskazał na swoją głowę. \dm{} I trucizna w środku.
\end{dialogue}
Ludzie czytali z papirusu i milcząco przytakiwali.
\begin{dialogue}
	\ds{} Zaczynają mu budować kapliczki. Składają dary na głównym placu. Nie możemy na to pozwolić! To on zabił wszystkie trzy osoby. Waszymi i moimi rękami \dm{} kontynuował.
	\ds{} Zabiłeś tak samo, jak zegar, jak śmiesz nas pouczać? \dm{} ktoś z tłumu wykrzyknął.
	\ds{} Zabiłem, żeby nasze dzieci nie zabijały w imię mechanicznego szatana, śmierć za więcej śmierci. Każdy z was zrobiłby to na moim miejscu.
	\ds{} Sam jesteś szatanem. 
	\ds{} Przynajmniej nie torturowałem piekarza, zrobiłem to humanitarnie, dla dobra nas wszystkich.
	\ds{} Jakiego dobra?
\end{dialogue}
I zaczęli się przepychać.
Nauczona doświadczeniem rodzinka Nocnych powoli się wycofała.
Krzyki było słychać na całej ulicy, ciekawe czy czwarta osoba teraz zginie.

Rafał Nocny umieścił w urządzeniu różaniec, buteleczkę wody święconej i kawałek Jana Pawła II.
Bogofon zatrzeszczał, zawibrował i na ekranie ukazała się świetlista postać.
Młodzieniec miał jasne włosy, błyszczące w ostrym świetle oraz białą szatę ze srebrnymi akcentami z pereł i diamentów.
Za nim znajdowały się rzędy kolorowych kul, wokół których chodzili inni aniołowie.
\begin{dialogue}
	\ds{} Niebiański departament symulacji alternatywnych wersji wszechświata, w czym mogę pomóc? \dm{} Zamrugał kilka razy.
		\dm{} Och, to wy. I jak tam codzienna egzystencja, moi dzielni wojownicy uniwersalności?
	\ds{} Znaleźliśmy zegar \dm{} wypaliła niespodziewanie Nadzieja.
	\ds{} O, to miło. \dm{} Uśmiechnął się, pokazując idealnie proste i białe zęby. \dm{} Na samej Ziemi znajduje się kilka miliardów zegarów, mam je wymienić?
	\ds{} Ale ten zadaje zagadki \dm{} dodała Maria Nocna.
	\ds{} I morduje ludzi \dm{} wspomniał Rafał.
	\ds{} I gada po polsku. \dm{} Żywia się ożywiła.
	\ds{} Chwileczkę... \dm{} Anioł gdzieś zniknął. Dał się słyszeć tylko jego głos zza bogofonu.
	\ds{} Hmm... to by było... nie... jeden... drugi... trzeci...
\end{dialogue}
Rodzinka popatrzyła się po sobie pytająco.
\begin{dialogue}
	\ds{} Hitler wygrywa wojnę, to nie.
	\ds{} Ludzie znoszą jaja? To też nie.
	\ds{} Derdenole asuktują fizultanowe bugysty? Listo.
	\ds{} ...wygrywa wybory. Brrr.
	\ds{} O, zegar zadaje zagadki. To będzie to.
\end{dialogue}
Tym razem to nie jest uniwersalność, tylko wyciekły eksperyment.
Anioł pokazał kamerą na jeden z roztrzaskanych słojów do symulacji wszechświatów.
Uniwersalność jest losowością, która jest umieszczana w tych słojach przed symulacją, coś jak komórki macierzyste.
Kiedy główny zbiornik uniwersalności wyciekł i losowość zalała wszechświat, trochę normalnych symulacji też popękało i uciekło.
To była jedna z nich, na szczęście. Polski język jest w Niebie popularny.
\begin{dialogue}
	\ds{} Celem tego eksperymentu było zbadać, jak się żyje w świecie w którym co noc wymagana jest ścisła współpraca pomiędzy obcymi ludźmi, inaczej będą ginąć losowe osoby.
	\ds{} Jaki był wynik? \dm{} Rafał się zapytał.
	\ds{} Wyciek wszystko rozwalił przed zakończeniem eksperymentu. Zresztą, jesteście tam. Co sądzicie o tym pomyśle?
	\ds{} Beznadziejny.
	\ds{} Czy te noce serio muszą być takie straszne?
	\ds{} Boję się tego zegara.
	\ds{} Dzisiaj zginęły trzy, a może cztery, osoby. Tylko jedna jest ofiarą zegara. Odpowiedz sobie na to pytanie.
	\ds{} I gdyby ludzie współpracowali ze sobą, to nikt by nie zginął. \dm{} Anioł stawał przy swoim. \dm{} A jak przećwiczyliby sobie umysły, pomyślcie.
	\ds{} Nie wynaleźli jeszcze elektryczności, a w błocie na ulicach można utonąć. Słuchają się grupy jakichś dresów, zamiast sami rozwiązywać zagadki.
	\ds{} Klasycznie. Chcieliście poznania dobra i zła, to macie.
	\ds{} To było wiek wszechświata temu.
	\ds{} I czy przez ten czas coś się zmieniło? \dm{} Anioł popatrzył na ludzi z wyrzutem. \dm{} Zapisuję jako nieudany. Jak zwykle. Możecie zniszczyć zegar.
	\ds{} Rozwiązując zagadkę?
	\ds{} Myślałem o bombie sacroteriowej, która wysadziłaby cały system gwiezdny, ale tak też można.
	\ds{} Zabić tych wszystkich ludzi?
	\ds{} Nie mają dusz, mieli żyć w symulacji. Nie mają potrzeby istnieć. To tylko trochę uniwersalności. 
	\ds{} I to niby my jesteśmy tymi złymi?
	\ds{} Każda z tych kul posiada miliardy istnień. \dm{} Odwrócił się, wskazując na okrągłe słoje. \dm{} Nie mówiąc o tej, która symuluje aktualny wszechświat. Na prawdę chcesz kolejny raz przez to przechodzić?
		\dm{} Westchnął anielskim śpiewem. \dm{} Zresztą, w tej galaktyce nie ma innego życia, róbcie z nimi co chcecie, mogą zostać, co za różnica. 
		Macie przecież wolność, którą tak chętnie sobie zerwaliście z drzewa.
\end{dialogue}

Wychodząc ze ślepej uliczki, Rafał zapytał się przechodnia, czy wie czy ktoś jeszcze poległ w kłótni przed domem kowala. 
Okazało się, że zginęły trzy osoby, razem z kowalem za to że miał taką złą żonę. To było razem sześć osób.
Kobieta się zaśmiała, bo było to mocno poniżej codziennej średniej.

Wyglądało na to, że to nie zegar był największym mordercą w tym mieście.
Tarcza na wieży przytakująco błysnęła światłem białej gwiazdy.

Kolejnej nocy ponownie miasto zebrało się na głównym placu.
Niebieski sierp księżyca świecił znacznie słabiej, teraz wszystko było czerwone.
Ta czerwień już nie była łuną krwi, jak poprzednio, była ostrzeżeniem.
Była alarmem na tonącej łodzi podwodnej.
Uwaga, dzisiaj zginie jeszcze więcej osób!

\begin{sl}
\begin{quote}
Posłuchajcie uważnie moje dziatki \\
wybitej o północy zagadki. \\
Przed wschodem słońca znajdźcie rozwiązanie, \\
albo komuś coś się dzisiaj stanie. \\
Ding-dong. \\
Budowniczy wsiąkł. \\
Szukacie, a znajdziecie. \\
Wszystkich jego dzieci. \\
Dokąd sobie poszedł? \\
Wziął ze sobą kalosze. \\
I oto historia jest cała. \\
Znajdźcie tego bałwana. \\
\end{quote}
\end{sl}

Banda dresów stała w miejscu.
Trzeba było znaleźć jakiegoś budowniczego, który sobie gdzieś poszedł.
Zawołano więc wszystkie rodziny budowniczych, pytając się, czy nie nie brakuje im jakichś osób.
Zaprzeczyli.

Ktoś podsunął pomysł, że budowniczy nie musiał mieć rodziny.
Więc poproszono, właściwie zagoniono, wszystkie dzieci w miasteczku na plac, wraz z rodzicami.
Ale zegar nieubłaganie tykał dalej.

A może chodziło o dzieci jako konstrukcje? Ale jak znaleźć budowniczych każdej rzeczy w mieście?

A może dałoby się znaleźć brakujące kalosze. Ale jak znaleźć coś, czego nie ma?
I tak sprzeczali się i sprzeczali.
I nic z tego nie wychodziło.
Jednak tym razem nie dochodziło do rękoczynów.

Nocni poszukali w mieście jakiegoś muzeum.
Znaleźli jedno, które robiło także za bibliotekę.
Każdej nocy wiele osób przychodziło tutaj, szukając odpowiedzi na własną rękę.
Był spis zagadek i rozwiązań od początku istnienia, czyli przez jakieś dwa tysiące lat, kiedy nastąpił wyciek.
Także wczorajszy problem z trucizną. Cała historia wyglądała bardzo podobnie.
Nieprawdopodobne, że do tego czasu nie potrafili się zjednać.
Opisy miasta nie różniły się od tych sprzed kilkuset lat.
Może jeśli wiesz, że każda noc może być twoją ostatnią, to nie ma sensu nic robić?

Wieża jest niezniszczalna, nie ma wejść, a każdy, kto spróbuje się na nią wdrapać, doznaje zawału serca.
Próbowano już atakować, pertraktować, złożyczyć, straszyć zegar, ale nic to nie dało.

\begin{dialogue}
	\ds{} A może chodzi o prawdziwego bałwana? \dm{} zaproponowała Nadzieja, widząc jak inne dzieci lepią na ulicy bałwany z brudnego i czerwonego śniegu.
	\ds{} Ale kto miałby być budowniczym bałwana? Każdy mógłby by być \dm{} Maria Nocna się zaśmiała.
	\ds{} Nie, bałwan budowniczym.
	\ds{} Bałwany nie budują bałwanów, głuptasku... chociaż.
	\ds{} Ale wsiąkają, prawda?
	\ds{} Gdyby topiący się bałwan rozpadł się na kilka kawałków, to były by jego dzieci, racja? \dm{} Do rozmowy przyłączył się Rafał.
	\ds{} Musiałby mieć kalosze, czasami wsadzają takie rzeczy bałwanom.
	\ds{} I pewnie masę innych rzeczy, bo to musiał być bardzo duży i bardzo pieczołowicie zbudowany bałwan.
	\ds{} No to szukamy wielkiej kałuży w której pływa garnek, miotły, węgle, szalik i kalosze, a wokół sterczą grudy śniegu. Powodzenia.
\end{dialogue}







