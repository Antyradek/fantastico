\chapter{Bicie dzwonu, śmierć komu?}

\info{Bombastyczna rodzinka spotyka na swojej drodze tajemniczy zegar.}

Przylecieli swoim małym statkiem pod wieczór.
Zwabieni dużą aktywnością uniwersalności w tej galaktyce.
Ukryli rakietę w pobliskich górkach, wzięli najpotrzebniejsze narzędzia i poszli rozejrzeć się po zabudowaniach.
Ta cywilizacja nie powinna była istnieć.

Ale istniała i należało to zbadać.

Było pochmurno.
Mglista i zimna noc wgryzała się w grube, ciemne ubrania, w jakie się dla niepoznaki odziali.
Wczesnowiosenny i niebieski księżyc oświetlał świat trupią łuną, niczym wielka lampa nad stołem w prosektorium.
Będąc tam, wolałeś trzymać się w cieniu, bo wydawało się jakby to światło mogłoby wyssać z ciebie życie.
Ale i cień wydawał się nieprzyjemnym atramentem, który mógłby w każdej chwili złapać za nogę i wciągnąć w siebie na zawsze. 
Przeskakując między strachem i koszmarem, nie mógłbyś znaleźć bezpiecznej wyspy.

Był też drugi księżyc, mniejszy i cały czerwony.
Nadawał miastu krwisty poblask. 
Z prosektorium przenosił do ciemni fotograficznej, gdzie wisiały kościste zdjęcia rentgenowskie.
Była wczesna wiosna, ale odczucie towarzyszyło jednak późnej jesieni.
Leżące gdzieniegdzie kupy śniegu, wyglądały w tym świetle jak zwłoki rozszarpanych zwierząt.
Tam, gdzie światła księżyców się spotykały, panował nieprzyjemny fiolet, niczym żałobna szata rozpostarta na ziemi, ciągnąca się od horyzontu po horyzont.

Źródło uniwersalności, nieprawidłowej materii wszechświata, znajdowało się w centrum miasteczka.
Czwórka przybyszów posuwała się zabłoconymi uliczkami, omijając kałuże o niezbadanej głębokości.
Starali się pływać w cienistym atramencie, aby nie zwracać na siebie zbytniej uwagi.
Elektroniczny czujnik uniwersalności pikał coraz szybciej, niczym EKG leżącego na stole operacyjnym pacjenta.
Napawało to nadzieją, pomimo że każda zdrowa na umyśle osoba omijałaby uniwersalność o lata świetlne, gdyby wiedziała czym jest. I czym może być.

Gdy serce ich pacjenta wyrwało się z klatki piersiowej, spostrzegli że są na głównym placu. 
To jedyne miejsce, gdzie błoto było tak zadeptane, jakby przeryło je stado ziemskich dzików.
Połyskujące kałuże patrzyły się na rodzinkę swoimi czerwonymi, księżycowymi źrenicami.
Nie wyglądały jednak strasznie, bowiem i tak przyćmione były czernią, bijącą od wielkiej wieży.
Wieży zegarowej, górującej nad półkolistymi budynkami, niczym sęp nad swoją przyszłą ofiarą.

Zegar jaśniał żółtawym światłem, niczym trzeci księżyc na niebie.
Miał osiem godzin i jedną wskazówkę, która poruszała się w lewo, odwrotnie niż na ziemskich zegarach.
Złoty okrąg wokół tarczy odbijał krew i trupa, pozornie niwelując ich złowrogość.
Pomimo mrocznego zarysu wieży wokół, żółte oko emanowało sercowym ciepłem i domowością.
Jednak nauczona doświadczeniem rodzinka Nocnych wolała pływać w czarnych dziurach, zamiast ufać tej dziwacznej i niebezpiecznej strukturze, jaką była uniwersalność.

Na plac powoli zaczęli schodzić się mieszkańcy.
Ubrani byli podobnie do gości, snuli się, omijając błotne pułapki.
Zwiewne, czarne stroje wydawały się meduzami, unoszącymi się bezwładnie po dnie oceanu.
Nie rozmawiając ze sobą, sunęli niczym na śmierć.
Wpatrywali się w słoneczny cyferblat, jak skazaniec wpatruje się w swojego kata.
Wskazówka sterczała w kierunku dolnej godziny, niczym palec kostuchy wybierający swoją następną ofiarę.
Czekali.

Smukłe budynki z ciemnego metalu.
Spiczaste zdobienia były jak sterczące pazury potworów z głębi atramentowych kałuż.
Małe okienka, przepuszczające czerwone światło pochodni, wyglądały jak oczy czyhających w cieniu drapieżników.
Z ust buchała para i unosiła się bardzo wolno, prawie się nie rozmywając, niczym uchodzące w zaświaty dusze.
Dymne kolumny sterczały z wierzchołka każdego domu, jak nitki marionetek sterowane przez wielkiego lalkarza, albo raczej, zegarmistrza.
Widać było, że ci ludzie zbierali się na tym placu co noc.
Co noc oczekiwali na śmierć.

Przestraszone dziewczynki przytuliły się do kolan swojego taty, a żona gładziła je po zakapturzonych głowach.
\begin{dialogue}
	\ds{} Boję się \dm{} wyszeptała siedmioletnia Żywia do swojej mamusi.
	\ds{} Ale ty jesteś strachliwa \dm{} odpowiedziała jej starcza o rok Nadzieja.
	\ds{} Ciszej, bo nas odkryją \dm{} szepnął im ojciec rodziny. \dm{} Zobaczcie, chyba coś się dzieje.
\end{dialogue}

Wskazówka przesunęła się lekko w prawo, lądując dokładnie na dolnym znaku. 
Coś zgrzytnęło, coś metalicznie stuknęło we wnętrzu wieży.
Złoty okrąg tarczy błysnął.
Wszyscy ludzie umilkli.
Wtedy dał się słyszeć mechaniczny głos, niczym wybijany na stalowym mechanizmie.
Był pozbawiony nawet iskierki uczucia.

\begin{sl}
\begin{quote}
Posłuchajcie uważnie moje dziatki \\
wybitej o północy zagadki. \\
Przed wschodem słońca znajdźcie rozwiązanie, \\
albo komuś coś się dzisiaj stanie. \\
Bim-bom. \\
Z piecem dom. \\
Trucizna w pudełku. \\
Obok kowadełka. \\
Dodana do soli. \\
Nie zrobi tego powoli. \\
Ofiara się nie ukryje. \\
Zabije to zabije. \\
\end{quote}
\end{sl}

I była tam wyróżniająca się grupka mieszkańców.
Mieli brudnawe stroje, chusty na twarzach i pałki w rękach. Trochę śmierdzieli.
Wpasowywali się w mroczny krajobraz aż za dobrze.
Jakoś pozostali ludzie trzymali się od nich na dystans.
Po wymianie krótkich, niezrozumiałych zdań, pobiegli wszyscy w jakimś kierunku, a raczej podążyli za sobą nawzajem, bo nie widać było, żeby mieli jakąkolwiek zbiorową świadomość.
Większość trzymających się z dala obywateli mimowolnie podążyło za nimi, co uczynili także kosmiczni przybysze.

Prosektoryjna cisza zniknęła, trupy ożyły, rozpoczęły się rozmowy i dyskusje.
Ludzie rozprawiali o tym, czego może dotyczyć dzisiejsza zagadka.
Fioletowy całun śmierci został rozdarty przez chaotyczne biegi, głośne rozmowy i przepychanki wśród dzieci.
Czerwone oko na nieboskłonie usilnie próbowało emanować złowrogością, ale nikt już nie zwracał na nie uwagi.

Członkowie rodziny z kosmosu uważnie podsłuchiwali rozmowy wokół siebie.
Wyglądało na to, że zegar zadawał zagadkę codziennie i codziennie losowa osoba umierała pod wpływem tajemniczej siły mechanizmu.
Śladów morderstwa nigdy nikt nie znalazł, a w prawej ręce ofiary pojawiał się zwój papieru z rozwiązaniem.
Oczywiście zapisanym tym upiornym rymowanym tekstem.
Nigdy w historii miasta nie udało się rozwiązać zagadki na czas i zawsze ktoś przypłacał to swoim życiem.
Sam nie wiedziałeś, czy ta noc nie będzie twoją ostatnią. 
Żywia i Nadzieja bały się, że którąś z nich może trafić zegar, albo gorzej, ich rodziców.

Maria Nocna bazgrała notatki na swoim elektronicznym komunikatorze.
Uniwersalność to była nieprawidłowa materia, coś jak nowotwór wszechświata.
Ten twór nie słuchał się zasad fizyki, nie można było nim sterować, nie miał sensu.
Uniwersalność przyjmowała bardzo różne formy, od tortów weselnych, po latające po wszechświecie dziadki w wannach pełnych wody.
A tutaj podróżnicy trafili na zegar zadający zagadki.

Największe zagęszczenie gapiów wypadło przed czymś, co mogło być domem piekarza.
Piec i sól. Musiało się zgadzać.
W błocie, na kolanach, siedział związany właściciel przybytku.
\begin{dialogue}
	\ds{} Gdzie jest ta trucizna, którą chcesz dodać rano do chleba?
\end{dialogue}

	Przywódca bandy groził mu wałkiem do ciasta. 
	Był smukły, ubrany lepiej od reszty, jego twarz pozbawiona była wyrazu i pomimo światła księżyców,
	zawsze wyglądała jakby była złowieszczo oświetlona od dołu.
	Szrama tu i ówdzie potęgowała ten efekt.
	Łysa głowa błyskała bielą jak kość nagiej czaszki.
	Brakowało mu jeszcze kosy.
	
\begin{dialogue}		
	\ds{} Nie wmówisz nam, że zagadka nie jest o tobie. Tym razem pokonamy Zegar.
	\ds{} Przysięgam! \dm{} 
		piekarz płakał. \dm{} 
		To nie o mnie tej nocy chodzi! Nikogo nie zamierzam otruć.
	\ds{} Nie przyznasz się? To zaraz znajdziemy dowód. \dm{} 
		Pstryknął palcami i reszta jego bandy zaczęła przewracać mu mieszkanie do góry nogami. \ds{} 
		Zapytam się jeszcze raz...
	\ds{} ...nic trującego nie mam, tylko składniki do wypieków. Zegar przecież nigdy nie mówi wprost, to nie ja.
	\ds{} Czyżby? Jest piec? Jest. Jest kowadełko do krojenia? Jest. Są ofiary jedzące rano twój chleb? Są.
	\ds{} Wystarczy że nie upiekę dzisiaj rano niczego i nikt nie będzie mógł się otruć.
	\ds{} A więc potwierdzasz, że gdybyś coś upiekł, to ktoś by się otruł, tak?
	\ds{} Przekręcasz moje słowa.
\end{dialogue}
Szepty przeszły po zbiorowisku i bandzie.
\begin{dialogue}
	\ds{} Nie pyskuj. \dm{} 
		Rozejrzał się nerwowo. \dm{} 
		Bo zaraz spłaszczę ci tym facjatę! \dm{} 
		Zagroził. \dm{} 
		A wy co? Dalej szukać tej trucizny.
\end{dialogue}
Tymczasem ktoś przyniósł jakieś znalezione pudełko.
Przywódca otworzył, powąchał i uśmiechnął się, odsłaniając szereg migoczących czerwienią zębów.
\begin{dialogue}
	\ds{} A co to w takim razie jest, proszę piekarza? \dm{} Podsunął mu znalezisko pod nos.
	\ds{} Sól.
	\ds{} Sól? A słyszałeś może także określenie ,,biała śmierć?''
	\ds{} Wszystkiego jak się za dużo zje, to szkodzi.
	\ds{} Zagadka mówiła, że trucizna miała być dodana do soli, zobaczymy zaraz ile w tym prawdy. \dm{} Wsadził mu pudełko w twarz. \dm{} Jedz tę swoją truciznę!
\end{dialogue}
Poruszenie wśród ludzi.
Ktoś podniósł rękę, ktoś chrząknął, ktoś się cofnął o parę kroków, ktoś zemdlał.
Chyba nie za bardzo im się to podobało. Bo następnej nocy sami mogli skończyć na miejscu piekarza.
\begin{dialogue}
	\ds{} Dzisiejsza noc jest specjalna. 
		\dm{} Kostucha sączyła ludziom truciznę do uszu. 
		\dm{} Dzisiaj bowiem rozwiązujemy Zagadkę. Dzisiaj uwalniamy się spod władzy Zegara.
\end{dialogue}
Kilka osób popatrzyło się na wieżę, wieża popatrzyła się z powrotem na nich. Błysnęła złotym okręgiem tarczy. Mam was na oku.
\begin{dialogue}
	\ds{} Jeśli puścimy piekarza wolno, zginie dzisiaj więcej niż jedna osoba, zginą wszyscy, którzy kupują poranny chleb od tego truciciela.
\end{dialogue}
Ktoś chciał protestować, ktoś chciał ocalić torturowanego biedaka i swoje poranne śniadanie.
Ale jak kilka osób odwróciło się w jego kierunku z takim samym wyrazem twarzy, jak u przywódcy bandy, to zaraz ulotnił się ze zbiorowiska.

I tym sposobem torturowany zjadł na raz cały kilogram soli.
Niedługo potem dostał drgawek, co triumfalnie ogłoszono jako dowód, że w soli była trucizna.
Nie wszyscy uwierzyli, ale rzekomy truciciel pilnowany był do końca, żeby nikt przypadkiem nie mógł zweryfikować niepowtarzalnego zdania białogłowego.

A jak już dowiódł, dlaczego sól odmierza się w szczyptach, a nie w łyżkach, to zostawili jego truchło w zimnym błocie, a księżyce przykryły go fioletowym całunem.
Zegar nadal tykał, jak zawsze, ale też nikt nie wiedział jak miałby się zachować po rozwiązaniu jego zagadki.
Zatrzymać się? Wybić pochwalną melodyjkę? Zapaść się pod ziemię? A może dać punkt i kontynuować zagadki następnej nocy?
Nie przeszkadzało to bandzie w ogłoszeniu sukcesu i świętowaniu.

Słońce wstawało, zalewając miasto bielą.
Biały karzeł tego systemu był małą gwiazdką, przebijającą się ledwo przez chmury.
Dzień osuszał atrament i zwijał w kłębek fioletową kołdrę.
Błoto zamykało swe ślepia, a metalowe domy poczęły się skrzyć.
Noc nie chowała się, jak zwykle, po kątach. Uciekła w całości za horyzont, wygoniona jasną nadzieją.
Nawet mroczna wieża zrobiła się mniej mroczna, chociaż rzucany na miasto cień wszystkim przypominał o jej istnieniu.
Jedynie piekarz na środku ulicy był trochę nie na miejscu.
Ale czy warto było się nim, przejmować? 
Dzisiaj on, jutro ktoś inny.

I wtedy zegar uderzył dzwonem tak mocno, że gdyby ludzie mieli w oknach szyby, to już by ich nie mieli.
Ludzie przystanęli na chwilę, pomacali się po sobie, jakby chcieli sprawdzić czy nadal żyją.
Ktoś sobie zbadał puls.
Po kilku sekundach, rozluźnieni, kontynuowali swoje wędrówki.
Nawet słychać było gdzieniegdzie jakieś śmiechy.
To była ta spokojniejsza część doby.

Grupka osób biegła niespiesznie, jakby chciała zdążyć na pociąg, który i tak ma opóźnienie.
Pierwsza osoba trzymała w ręce zwój papieru.
Do grupki dołączali się też inni, zaciekawieni rozwiązaniem zagadki.
Rodzinka Nocnych skorzystała z okazji.

Tego poranka zegar swoim uderzeniem zabił jakąś prostytutkę z domu publicznego na skraju miasta.
W jej ręce znaleziono ten zwój.
Nagle osunęła się na ziemię, już jej nie dobudzili.
I tak ginął ktoś co ranek.

Przybiegli do domu kowala.
Kowal siedział przed domem i popatrzył się na przybyszów w taki sposób, jak trener patrzy się na swojego zawodnika, który przybiegł na metę ostatni.
\begin{dialogue}
	\ds{} Moja żona umarła dzisiaj rano. \dm{} 
		Nie widział papirusu, a jednak doskonale znał jego treść. \dm{} 
		Została otruta naszyjnikiem z masy solnej, który ktoś jej podarował. \dm{}
		Skupił swój wzrok na właścicielu kartki, tak jak ogniskuje się światło lupy w celu podpalenia czyjegoś domu. \dm{} 
		Na pewno nie piekarz, którego zamordowaliście dzisiaj w nocy. Więc kto?
	\ds{} Ja to zrobiłem. \dm{} 
		Aptekarz wyszedł z domu obok. \dm{} 
		Ta jędza uczyła nasze dzieci, że Zegar jest jakimś rodzajem boga. Że trzeba mu oddawać cześć. Bóg jest tylko jeden, a ten zegar jest jego dokładnym przeciwieństwem! \dm{} 
		Kilka osób mimowolnie popatrzyło na wieżę, złoty okrąg wokół tarczy błysnął, jakby chciał zaprzeczyć jego słowa.
	\ds{} A kowadło? A pudełko? \dm{} 
		dopytywał się tłum.
	\ds{} Kowadełko. \dm{} Aptekarz wskazał palcem na ucho. \dm{} Pudełko. \dm{} Wskazał na swoją głowę. \dm{} I trucizna w środku.
\end{dialogue}
Ludzie czytali z papirusu i milcząco przytakiwali.
\begin{dialogue}
	\ds{} Zaczynają mu budować kapliczki. \dm{} 
		Nakręcał się dalej. \dm{}
		Składają dary na głównym placu, prosząc o łatwą zagadkę. Nie możemy na to pozwolić! To on. To Zegar zabił wszystkie trzy osoby. Waszymi i moimi rękami. Nakręcany diabeł.
	\ds{} Zabiłeś w taki sam sposób. Jak śmiesz nas pouczać? \dm{} 
		ktoś z tłumu wykrzyknął.
	\ds{} Zabiłem, żeby nasze dzieci nie zabijały w imię mechanicznego szatana, śmierć za więcej śmierci. Każdy z was zrobiłby to na moim miejscu.
	\ds{} Sam jesteś szatanem. 
	\ds{} Przynajmniej nie torturowałem piekarza, zrobiłem to humanitarnie, dla dobra nas wszystkich.
	\ds{} Jakiego dobra?
\end{dialogue}
I zaczęli się przepychać.
Nauczona doświadczeniem rodzinka Nocnych powoli się wycofała.
Krzyki było słychać na całej ulicy, ciekawe czy kolejna osoba znowu zginie.

%%% TODO Pierwszy przelot

Rafał Nocny umieścił w urządzeniu różaniec, buteleczkę wody święconej i kawałek Jana Pawła II.
Bogofon zatrzeszczał, zawibrował i na ekranie ukazała się świetlista postać.
Młodzieniec miał jasne włosy, błyszczące w ostrym świetle oraz białą szatę ze srebrnymi akcentami z pereł i diamentów.
Za nim znajdowały się rzędy kolorowych kul, wokół których chodzili inni aniołowie.
\begin{dialogue}
	\ds{} Niebiański departament symulacji alternatywnych wersji wszechświata, w czym mogę pomóc? \dm{} Zamrugał kilka razy.
		\dm{} Och, to wy. I jak tam codzienna egzystencja, moi dzielni wojownicy uniwersalności?
	\ds{} Znaleźliśmy zegar \dm{} wypaliła niespodziewanie Nadzieja.
	\ds{} O, to miło. \dm{} Uśmiechnął się, pokazując idealnie proste i białe zęby. \dm{} Na samej Ziemi znajduje się kilka miliardów zegarów, mam je wymienić?
	\ds{} Ale ten zadaje zagadki \dm{} dodała Maria Nocna.
	\ds{} I morduje ludzi \dm{} wspomniał Rafał.
	\ds{} I gada po polsku. \dm{} Żywia się ożywiła.
	\ds{} Chwileczkę... \dm{} Anioł gdzieś zniknął. Dał się słyszeć tylko jego głos zza bogofonu.
	\ds{} Hmm... to by było... nie... jeden... drugi... trzeci...
\end{dialogue}
Rodzinka popatrzyła się po sobie pytająco.
\begin{dialogue}
	\ds{} Hitler wygrywa wojnę, to nie.
	\ds{} Ludzie znoszą jaja? To też nie.
	\ds{} Derdenole asuktują fizultanowe bugysty? Listo.
	\ds{} ...wygrywa wybory. Brrr.
	\ds{} O, zegar zadaje zagadki. To będzie to.
\end{dialogue}
Tym razem to nie jest uniwersalność, tylko wyciekły eksperyment.
Anioł pokazał kamerą na jeden z roztrzaskanych słojów do symulacji wszechświatów.
Uniwersalność jest losowością, która jest umieszczana w tych słojach przed symulacją, coś jak komórki macierzyste.
Kiedy główny zbiornik uniwersalności wyciekł i losowość zalała wszechświat, trochę normalnych symulacji też popękało i uciekło.
To była jedna z nich, na szczęście. Polski język jest w Niebie popularny.
\begin{dialogue}
	\ds{} Celem tego eksperymentu było zbadać, jak się żyje w świecie w którym co noc wymagana jest ścisła współpraca pomiędzy obcymi ludźmi, inaczej będą ginąć losowe osoby.
	\ds{} Jaki był wynik? \dm{} Rafał się zapytał.
	\ds{} Wyciek wszystko rozwalił przed zakończeniem eksperymentu. Zresztą, jesteście tam. Co sądzicie o tym pomyśle?
	\ds{} Beznadziejny.
	\ds{} Czy te noce serio muszą być takie straszne?
	\ds{} Boję się tego zegara.
	\ds{} Dzisiaj zginęły trzy, a może cztery, osoby. Tylko jedna jest ofiarą zegara. Odpowiedz sobie na to pytanie.
	\ds{} I gdyby ludzie współpracowali ze sobą, to nikt by nie zginął. \dm{} Anioł stawał przy swoim. \dm{} A jak przećwiczyliby sobie umysły, pomyślcie.
	\ds{} Nie wynaleźli jeszcze elektryczności, a w błocie na ulicach można utonąć. Słuchają się grupy jakichś dresów, zamiast sami rozwiązywać zagadki.
	\ds{} Klasycznie. Chcieliście poznania dobra i zła, to macie.
	\ds{} To było wiek wszechświata temu.
	\ds{} I czy przez ten czas coś się zmieniło? \dm{} Anioł popatrzył na ludzi z wyrzutem. \dm{} Zapisuję jako nieudany. Jak zwykle. Możecie zniszczyć zegar.
	\ds{} Rozwiązując zagadkę?
	\ds{} Myślałem o bombie sacroteriowej, która wysadziłaby cały system gwiezdny, ale tak też można.
	\ds{} Zabić tych wszystkich ludzi?
	\ds{} Nie mają dusz, mieli żyć w symulacji. Nie mają potrzeby istnieć. To tylko trochę uniwersalności. 
	\ds{} I to niby my jesteśmy tymi złymi?
	\ds{} Każda z tych kul posiada miliardy istnień. \dm{} Odwrócił się, wskazując na okrągłe słoje. \dm{} Nie mówiąc o tej, która symuluje aktualny wszechświat. Na prawdę chcesz kolejny raz przez to przechodzić?
		\dm{} Westchnął anielskim śpiewem. \dm{} Zresztą, w tej galaktyce nie ma innego życia, róbcie z nimi co chcecie, mogą zostać, co za różnica. 
		Macie przecież wolność, którą tak chętnie sobie zerwaliście z drzewa.
\end{dialogue}

Wychodząc ze ślepej uliczki, Rafał zapytał się przechodnia, czy wie czy ktoś jeszcze poległ w kłótni przed domem kowala. 
Okazało się, że zginęły trzy osoby, razem z kowalem za to że miał taką złą żonę. To było razem sześć osób.
Kobieta się zaśmiała, bo było to mocno poniżej codziennej średniej.

Wyglądało na to, że to nie zegar był największym mordercą w tym mieście.
Tarcza na wieży przytakująco błysnęła światłem białej gwiazdy.

Kolejnej nocy ponownie miasto zebrało się na głównym placu.
Niebieski sierp księżyca świecił znacznie słabiej, teraz wszystko było czerwone.
Ta czerwień już nie była łuną krwi, jak poprzednio, była ostrzeżeniem.
Była alarmem na tonącej łodzi podwodnej.
Uwaga, dzisiaj zginie jeszcze więcej osób!

\begin{sl}
\begin{quote}
Posłuchajcie uważnie moje dziatki \\
wybitej o północy zagadki. \\
Przed wschodem słońca znajdźcie rozwiązanie, \\
albo komuś coś się dzisiaj stanie. \\
Ding-dong. \\
Budowniczy wsiąkł. \\
Szukacie, a znajdziecie. \\
Wszystkich jego dzieci. \\
Dokąd sobie poszedł? \\
Wziął ze sobą kalosze. \\
I oto historia jest cała. \\
Znajdźcie tego bałwana. \\
\end{quote}
\end{sl}

Banda dresów stała w miejscu.
Trzeba było znaleźć jakiegoś budowniczego, który sobie gdzieś poszedł.
Zawołano więc wszystkie rodziny budowniczych, pytając się, czy nie nie brakuje im jakichś osób.
Zaprzeczyli.

Ktoś podsunął pomysł, że budowniczy nie musiał mieć rodziny.
Więc poproszono, właściwie zagoniono, wszystkie dzieci w miasteczku na plac, wraz z rodzicami.
Ale zegar nieubłaganie tykał dalej.

A może chodziło o dzieci jako konstrukcje? Ale jak znaleźć budowniczych każdej rzeczy w mieście?

A może dałoby się znaleźć brakujące kalosze. Ale jak znaleźć coś, czego nie ma?
I tak sprzeczali się i sprzeczali.
I nic z tego nie wychodziło.
Jednak tym razem nie dochodziło do rękoczynów.

Nocni poszukali w mieście jakiegoś muzeum.
Znaleźli jedno, które robiło także za bibliotekę.
Każdej nocy wiele osób przychodziło tutaj, szukając odpowiedzi na własną rękę.
Był spis zagadek i rozwiązań od początku istnienia, czyli przez jakieś dwa tysiące lat, kiedy nastąpił wyciek.
Także wczorajszy problem z trucizną. Cała historia wyglądała bardzo podobnie.
Nieprawdopodobne, że do tego czasu nie potrafili się zjednać.
Opisy miasta nie różniły się od tych sprzed kilkuset lat.
Może jeśli wiesz, że każda noc może być twoją ostatnią, to nie ma sensu nic robić?

Wieża jest niezniszczalna, nie ma wejść, a każdy, kto spróbuje się na nią wdrapać, doznaje zawału serca.
Próbowano już atakować, pertraktować, złożyczyć, straszyć zegar, ale nic to nie dawało.

\begin{dialogue}
	\ds{} A może chodzi o prawdziwego bałwana? \dm{} zaproponowała Nadzieja, widząc jak inne dzieci lepią na ulicy bałwany z kawałków trucheł.
	\ds{} Ale kto miałby być budowniczym bałwana, córciu? Każdy mógłby być \dm{} Maria Nocna się zaśmiała.
	\ds{} Nie, bałwan budowniczym.
	\ds{} Bałwany nie budują bałwanów, głuptasku... chociaż.
	\ds{} Ale wsiąkają, prawda?
	\ds{} Gdyby topiący się bałwan rozpadł się na kilka kawałków, to byłyby jego dzieci, racja? \dm{} Do rozmowy przyłączył się Rafał.
	\ds{} Musiałby mieć kalosze, lecz czasami wsadzają takie rzeczy bałwanom.
	\ds{} I pewnie masę innych rzeczy, bo to musiał być bardzo duży i bardzo pieczołowicie zbudowany bałwan.
	\ds{} No to szukamy wielkiej kałuży w której pływa garnek, miotły, węgle, szalik i kalosze, a wokół sterczą grudy śniegu. Powodzenia.
\end{dialogue}

Ale że noc chyliła się ku zachodowi, a rozwiązania nie było nawet na horyzoncie, postanowili poszukać tego nietypowego zrządzenia losu.
Dopiero, gdy się rozjaśniło, znaleźli kałużę, która idealnie odpowiadała ich przewidywaniom.
Rafał zanurzył rękę w błocie i w tej samej chwili zegar oznajmił koniec czasu.
Wyciągnął parę brudnych kaloszy.
Westchnął i usiadł na jednej ze śnieżnych grud, które odpadły od głównego tworu.
Nie długo trwało aż przybiegł ktoś ze zwojem papirusu.
\begin{dialogue}
	\ds{} Nie zdążyłem \dm{} oznajmił. \dm{} Prawdopodobnie nie da się zdążyć.
\end{dialogue}
Jednak przybysz pobiegł uliczkami i głosił, że znalazł kogoś, kto rozwiązał zagadkę.
Rodzinka wolała więc w tym momencie zniknąć, ale jak na złość, w dzień już nie było atramentowych cieni.

Nauczeni doświadczeniem, Nocni wzięli ze swojej rakiety worek nanobotów i rozesłali po całym mieście, aby zmapowały teren.
Tej nocy zegar nie będzie miał z nimi szans.
Będą wiedzieć wszystko i będą jednocześnie wszędzie.
Żadne sztuczki z szukaniem igieł w stogach siana, albo kaloszy w kałużach błota nie przejdą.

Trzecia noc była tak nieprzyjemna, jak iloczyn dwóch poprzednich.
Gęste chmury nie przepuszczały żadnego światła, nawet tej ciemniowej żarówki.
Jedynie żółta tarcza zegara delikatnie oświetlała miasto.
Teraz to ja jestem waszym słońcem.

Deszcz padał, błotne ulice zamieniły się w potoki.
Mglista firana zasłoniła miasto, niczym pajęczyna zasłaniała oplatywaną ofiarę pająka.
Prawie wszystkie nanoboty ugrzęzły w ziemi lub zostały spłukane w żarłocznym, wszechobecnym cieniu.
Nie zemną takie numery.

Drgające połyski okręgu zdawały się wskazywać, że rechocze.
Nie było widać zarysów wieży, była tylko świecąca tarcza, unosząca się nad ludźmi niczym ich nowa i jedyna gwiazda.
Ludzie stali na głównym placu jak co noc, przyklejeni do deszczowej pajęczyny, i nie spodziewali się, że i tym razem będzie jakkolwiek inaczej.
Legenda o rozwiązanej zagadce zgasła jak świeczka rzucona na ulicę. Zgasła za horyzontem zdarzeń.
Nikt nie uwierzył, nikt nie miał nadziei.
Nawet Rafał Nocny.

Zegar odezwał się, jak co nocy.

\begin{sl}
\begin{quote}
Posłuchajcie uważnie moje dziatki \\
wybitej o północy zagadki. \\
Przed wschodem słońca znajdźcie rozwiązanie, \\
albo komuś coś się dzisiaj stanie. \\
Dzyń-dzoń. \\
Nie dostrzeżecie go. \\
Jest pewien byt, \\
co wszystkich morduje w mig. \\
I to nie jestem ja. \\
Chociaż reszta się zgadza. \\
Dzwoni dzwonkami. \\
Zagadki rozdaje. \\
\end{quote}
\end{sl}

\begin{dialogue}
	\ds{} Inni ludzie! \dm{} inni ludzie powiedzieli to jednocześnie.
	\ds{} Bóg?
	\ds{} Inny zegar?
	\ds{} Strach?
	\ds{} Wiatr?
	\ds{} My sami?
\end{dialogue}
Ale kolejne pomysły grzęzły w ludzkiej pajęczynie bezsensowności coraz bardziej.
Ktoś usiadł w błocie.
Ktoś zemdlał z beznadziejności i się utopił.
Ktoś celowo sobie żyły podciął.
Widać śmierć mogła przyjść od ciebie samego.

I musieli tak stać.
I nic nie mogli zrobić.
Bo wszyscy byli splątani pajęczyną, która sklejała ich mózgi.
Oko wielkiego pająka jednak bacznie się przyglądało.
Nie pozwolę wam umrzeć, cierpcie.

I tylko Nocni potrafili się z tej pajęczyny zerwać.
Nie dlatego, że mieli doświadczenie, inteligencję, czy chęć do życia.
A dlatego że za rogiem czekała na nich rakieta i kilka sacroteriowych bomb.
Oni jedyni mogli po prostu sobie pójść i zostawić wszystko w cholerę.
I między innymi dlatego postanowili zostać i się nie poddawać.
Bo jakby poszli, to dzisiejsza pajęczyna zmieniłaby się w jutrzejsze skamieliny.
A zegar wkrótce zostałby sam w wymordowanym mieście i sam sobie zadawał zagadki.

Nawet nie było widać, kiedy przyszedł poranek.
Gdy każdy zwymiotował wszystkie swoje pomysły i nie było już kwasu żołądko...
\begin{dialogue}
	\ds{} ...pisarz opowiadania.
\end{dialogue}
Wszystkie oczy zwróciły się na małą Żywię, która była zaskoczona tak samo, jak oni.
\begin{dialogue}
	\ds{} No bo to pisarz pisze zagadki i dzwoni dzwonem, jak pisze ,,bim-bom''.
\end{dialogue}
Ludzie trochę się zaśmiali, trochę przez łzy, a trochę przez deszcz.
Jednak dopiero po minucie zauważyli, że nie ma nad nimi ich słońca.
Rozglądali się, szukając tak znienawidzonej przez wszystkich tarczy, lecz niebo było całkowicie czarne.
Lekkie buczenie coraz bardziej zagłuszało szum deszczu.
\begin{dialogue}
	\ds{} Wszyscy uciekać! \dm{} krzyknął Nocny, ciągnąc za sobą swoją rodzinę.
\end{dialogue}
Nie wiadomo dlaczego, ale inni ludzie również zaczęli w pośpiechu opuszczać plac.
Było nadal tak samo ciemno.
Wtem wielkie uderzenie i fala błota znikąd porwała biegnących.
Rzuciła w wąskie uliczki, wpadła do domów, gasząc latarnie, rozerwała pajęczynę na strzępy.
Spadająca wieża ostatni raz zadzwoniła przytłumionym dzwonem, zadając swoją ostatnią śmierć.


















