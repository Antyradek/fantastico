\chapter{Bicie dzwonu, śmierć komu?}

\info{Bombastyczna rodzinka spotyka na swojej drodze tajemniczy zegar.}

Przylecieli swoim małym statkiem pod wieczór.
Zwabieni dużą aktywnością uniwersalności w tej galaktyce.
Ukryli rakietę w pobliskich górkach, wzięli najpotrzebniejsze narzędzia i poszli rozejrzeć się po zabudowaniach.
Ta cywilizacja nie powinna była istnieć.

Ale istniała i należało to zbadać.

Było pochmurno.
Mglista i zimna noc wgryzała się w grube, ciemne ubrania, w jakie się dla niepoznaki odziali.
Wczesnowiosenny i niebieski księżyc oświetlał świat trupią łuną, niczym wielka lampa nad stołem w prosektorium.
Będąc tam, wolałeś trzymać się w cieniu, bo wydawało się jakby to światło mogłoby wyssać z ciebie życie.
Ale i cień wydawał się nieprzyjemnym atramentem, który mógłby w każdej chwili złapać za nogę i wciągnąć w siebie na zawsze. 
Przeskakując między strachem i koszmarem, nie mógłbyś znaleźć bezpiecznej wyspy.

Był też drugi księżyc, mniejszy i cały czerwony.
Nadawał miastu krwisty poblask. 
Z prosektorium przenosił do ciemni fotograficznej, gdzie wisiały kościste zdjęcia rentgenowskie.
Była wczesna wiosna, ale odczucie towarzyszyło jednak późnej jesieni.
Leżące gdzieniegdzie kupy śniegu, wyglądały w tym świetle jak zwłoki rozszarpanych zwierząt.
Tam, gdzie światła księżyców się spotykały, panował nieprzyjemny fiolet, niczym żałobna szata rozpostarta na ziemi, ciągnąca się od horyzontu po horyzont.

Źródło uniwersalności, nieprawidłowej materii wszechświata, znajdowało się w centrum miasteczka.
Czwórka przybyszów posuwała się zabłoconymi uliczkami, omijając kałuże o niezbadanej głębokości.
Starali się pływać w cienistym atramencie, aby nie zwracać na siebie zbytniej uwagi.
Elektroniczny czujnik uniwersalności pikał coraz szybciej, niczym EKG leżącego na stole operacyjnym pacjenta.
Napawało to nadzieją, pomimo że każda zdrowa na umyśle osoba omijałaby uniwersalność o lata świetlne, gdyby wiedziała czym jest. I czym może być.

Gdy serce ich pacjenta wyrwało się z klatki piersiowej, spostrzegli że są na głównym placu. 
To jedyne miejsce, gdzie błoto było tak zadeptane, jakby przeryło je stado ziemskich dzików.
Połyskujące kałuże patrzyły się na rodzinkę swoimi czerwonymi, księżycowymi źrenicami.
Nie wyglądały jednak strasznie, bowiem i tak przyćmione były czernią, bijącą od wielkiej wieży.
Wieży zegarowej, górującej nad półkolistymi budynkami, niczym sęp nad swoją przyszłą ofiarą.

Zegar jaśniał żółtawym światłem, niczym trzeci księżyc na niebie.
Miał osiem godzin i jedną wskazówkę, która poruszała się w lewo, odwrotnie niż na ziemskich zegarach.
Złoty okrąg wokół tarczy odbijał krew i trupa, pozornie niwelując ich złowrogość.
Pomimo mrocznego zarysu wieży wokół, żółte oko emanowało sercowym ciepłem i domowością.
Jednak nauczona doświadczeniem rodzinka Nocnych wolała pływać w czarnych dziurach, zamiast ufać tej dziwacznej i niebezpiecznej strukturze, jaką była uniwersalność.

Na plac powoli zaczęli schodzić się mieszkańcy.
Ubrani byli podobnie do gości, snuli się, omijając błotne pułapki.
Zwiewne, czarne stroje wydawały się meduzami, unoszącymi się bezwładnie po dnie oceanu.
Nie rozmawiając ze sobą, sunęli niczym na śmierć.
Wpatrywali się w słoneczny cyferblat, jak skazaniec wpatruje się w swojego kata.
Wskazówka sterczała w kierunku dolnej godziny, niczym palec kostuchy wybierający swoją następną ofiarę.
Czekali.

Smukłe budynki z ciemnego metalu.
Spiczaste zdobienia były jak sterczące pazury potworów z głębi atramentowych kałuż.
Małe okienka, przepuszczające czerwone światło pochodni, wyglądały jak oczy czyhających w cieniu drapieżników.
Z ust buchała para i unosiła się bardzo wolno, prawie się nie rozmywając, niczym uchodzące w zaświaty dusze.
Dymne kolumny sterczały z wierzchołka każdego domu, jak nitki marionetek sterowane przez wielkiego lalkarza, albo raczej, zegarmistrza.
Widać było, że ci ludzie zbierali się na tym placu co noc.
Co noc oczekiwali na śmierć.

Przestraszone dziewczynki przytuliły się do kolan swojego taty, a żona gładziła je po zakapturzonych głowach.
\begin{dialogue}
	\ds{} Boję się \dm{} wyszeptała siedmioletnia Żywia do swojej mamusi.
	\ds{} Ale ty jesteś strachliwa \dm{} odpowiedziała jej starcza o rok Nadzieja.
	\ds{} Ciszej, bo nas odkryją \dm{} szepnął im ojciec rodziny. \dm{} Zobaczcie, chyba coś się dzieje.
\end{dialogue}

Wskazówka przesunęła się lekko w prawo, lądując dokładnie na dolnym znaku. 
Coś zgrzytnęło, coś metalicznie stuknęło we wnętrzu wieży.
Złoty okrąg tarczy błysnął.
Wszyscy ludzie umilkli.
Wtedy dał się słyszeć mechaniczny głos, niczym wybijany na stalowym mechanizmie.
Był pozbawiony nawet iskierki uczucia.

\begin{sl}
\begin{quote}
Posłuchajcie uważnie moje dziatki \\
wybitej o północy zagadki. \\
Przed wschodem słońca znajdźcie rozwiązanie, \\
albo komuś coś się dzisiaj stanie. \\
Bim-bom. \\
Z piecem dom. \\
Trucizna w pudełku. \\
Obok kowadełka. \\
Dodana do soli. \\
Nie zrobi tego powoli. \\
Ofiara się nie ukryje. \\
Zabije to zabije. \\
\end{quote}
\end{sl}

I była tam wyróżniająca się grupka mieszkańców.
Mieli brudnawe stroje, chusty na twarzach i pałki w rękach. Trochę śmierdzieli.
Wpasowywali się w mroczny krajobraz aż za dobrze.
Jakoś pozostali ludzie trzymali się od nich na dystans.
Po wymianie krótkich, niezrozumiałych zdań, pobiegli wszyscy w jakimś kierunku, a raczej podążyli za sobą nawzajem, bo nie widać było, żeby mieli jakąkolwiek zbiorową świadomość.
Większość trzymających się z dala obywateli mimowolnie podążyło za nimi, co uczynili także kosmiczni przybysze.

Prosektoryjna cisza zniknęła, trupy ożyły, rozpoczęły się rozmowy i dyskusje.
Ludzie rozprawiali o tym, czego może dotyczyć dzisiejsza zagadka.
Fioletowy całun śmierci został rozdarty przez chaotyczne biegi, głośne rozmowy i przepychanki wśród dzieci.
Czerwone oko na nieboskłonie usilnie próbowało emanować złowrogością, ale nikt już nie zwracał na nie uwagi.

Członkowie rodziny z kosmosu uważnie podsłuchiwali rozmowy wokół siebie.
Wyglądało na to, że zegar zadawał zagadkę codziennie i codziennie losowa osoba umierała pod wpływem tajemniczej siły mechanizmu.
Śladów morderstwa nigdy nikt nie znalazł, a w prawej ręce ofiary pojawiał się zwój papieru z rozwiązaniem.
Oczywiście zapisanym tym upiornym rymowanym tekstem.
Nigdy w historii miasta nie udało się rozwiązać zagadki na czas i zawsze ktoś przypłacał to swoim życiem.
Sam nie wiedziałeś, czy ta noc nie będzie twoją ostatnią. 
Żywia i Nadzieja bały się, że którąś z nich może trafić zegar, albo gorzej, ich rodziców.

Maria Nocna bazgrała notatki na swoim elektronicznym komunikatorze.
Uniwersalność to była nieprawidłowa materia, coś jak nowotwór wszechświata.
Ten twór nie słuchał się zasad fizyki, nie można było nim sterować, nie miał sensu.
Uniwersalność przyjmowała bardzo różne formy, od tortów weselnych, po latające po wszechświecie dziadki w wannach pełnych wody.
A tutaj podróżnicy trafili na zegar zadający zagadki.

Największe zagęszczenie gapiów wypadło przed czymś, co mogło być domem piekarza.
Piec i sól. Musiało się zgadzać.
W błocie, na kolanach, siedział związany właściciel przybytku.
\begin{dialogue}
	\ds{} Gdzie jest ta trucizna, którą chcesz dodać rano do chleba?
\end{dialogue}

	Przywódca bandy groził mu wałkiem do ciasta. 
	Był smukły, ubrany lepiej od reszty, jego twarz pozbawiona była wyrazu i pomimo światła księżyców,
	zawsze wyglądała jakby była złowieszczo oświetlona od dołu.
	Szrama tu i ówdzie potęgowała ten efekt.
	Łysa głowa błyskała bielą jak kość nagiej czaszki.
	Brakowało mu jeszcze kosy.
	
\begin{dialogue}		
	\ds{} Nie wmówisz nam, że zagadka nie jest o tobie. Tym razem pokonamy Zegar.
	\ds{} Przysięgam! \dm{} 
		piekarz płakał. \dm{} 
		To nie o mnie tej nocy chodzi! Nikogo nie zamierzam otruć.
	\ds{} Nie przyznasz się? To zaraz znajdziemy dowód. \dm{} 
		Pstryknął palcami i reszta jego bandy zaczęła przewracać mu mieszkanie do góry nogami. \ds{} 
		Zapytam się jeszcze raz...
	\ds{} ...nic trującego nie mam, tylko składniki do wypieków. Zegar przecież nigdy nie mówi wprost, to nie ja.
	\ds{} Czyżby? Jest piec? Jest. Jest kowadełko do krojenia? Jest. Są ofiary jedzące rano twój chleb? Są.
	\ds{} Wystarczy że nie upiekę dzisiaj rano niczego i nikt nie będzie mógł się otruć.
	\ds{} A więc potwierdzasz, że gdybyś coś upiekł, to ktoś by się otruł, tak?
	\ds{} Przekręcasz moje słowa.
\end{dialogue}
Szepty przeszły po zbiorowisku i bandzie.
\begin{dialogue}
	\ds{} Nie pyskuj. \dm{} 
		Rozejrzał się nerwowo. \dm{} 
		Bo zaraz spłaszczę ci tym facjatę! \dm{} 
		Zagroził. \dm{} 
		A wy co? Dalej szukać tej trucizny.
\end{dialogue}
Tymczasem ktoś przyniósł jakieś znalezione pudełko.
Przywódca otworzył, powąchał i uśmiechnął się, odsłaniając szereg migoczących czerwienią zębów.
\begin{dialogue}
	\ds{} A co to w takim razie jest, proszę piekarza? \dm{} Podsunął mu znalezisko pod nos.
	\ds{} Sól.
	\ds{} Sól? A słyszałeś może także określenie ,,biała śmierć?''
	\ds{} Wszystkiego jak się za dużo zje, to szkodzi.
	\ds{} Zagadka mówiła, że trucizna miała być dodana do soli, zobaczymy zaraz ile w tym prawdy. \dm{} Wsadził mu pudełko w twarz. \dm{} Jedz tę swoją truciznę!
\end{dialogue}
Poruszenie wśród ludzi.
Ktoś podniósł rękę, ktoś chrząknął, ktoś się cofnął o parę kroków, ktoś zemdlał.
Chyba nie za bardzo im się to podobało. Bo następnej nocy sami mogli skończyć na miejscu piekarza.
\begin{dialogue}
	\ds{} Dzisiejsza noc jest specjalna. 
		\dm{} Kostucha sączyła ludziom truciznę do uszu. 
		\dm{} Dzisiaj bowiem rozwiązujemy Zagadkę. Dzisiaj uwalniamy się spod władzy Zegara.
\end{dialogue}
Kilka osób popatrzyło się na wieżę, wieża popatrzyła się z powrotem na nich. Błysnęła złotym okręgiem tarczy. Mam was na oku.
\begin{dialogue}
	\ds{} Jeśli puścimy piekarza wolno, zginie dzisiaj więcej niż jedna osoba, zginą wszyscy, którzy kupują poranny chleb od tego truciciela.
\end{dialogue}
Ktoś chciał protestować, ktoś chciał ocalić torturowanego biedaka i swoje poranne śniadanie.
Ale jak kilka osób odwróciło się w jego kierunku z takim samym wyrazem twarzy, jak u przywódcy bandy, to zaraz ulotnił się ze zbiorowiska.

I tym sposobem torturowany zjadł na raz cały kilogram soli.
Niedługo potem dostał drgawek, co triumfalnie ogłoszono jako dowód, że w soli była trucizna.
Nie wszyscy uwierzyli, ale rzekomy truciciel pilnowany był do końca, żeby nikt przypadkiem nie mógł zweryfikować niepowtarzalnego zdania białogłowego.

A jak już dowiódł, dlaczego sól odmierza się w szczyptach, a nie w łyżkach, to zostawili jego truchło w zimnym błocie, a księżyce przykryły go fioletowym całunem.
Zegar nadal tykał, jak zawsze, ale też nikt nie wiedział jak miałby się zachować po rozwiązaniu jego zagadki.
Zatrzymać się? Wybić pochwalną melodyjkę? Zapaść się pod ziemię? A może dać punkt i kontynuować zagadki następnej nocy?
Nie przeszkadzało to bandzie w ogłoszeniu sukcesu i świętowaniu.

Słońce wstawało, zalewając miasto bielą.
Biały karzeł tego systemu był małą gwiazdką, przebijającą się ledwo przez chmury.
Dzień osuszał atrament i zwijał w kłębek fioletową kołdrę.
Błoto zamykało swe ślepia, a metalowe domy poczęły się skrzyć.
Noc nie chowała się, jak zwykle, po kątach. Uciekła w całości za horyzont, wygoniona jasną nadzieją.
Nawet mroczna wieża zrobiła się mniej mroczna, chociaż rzucany na miasto cień wszystkim przypominał o jej istnieniu.
Jedynie piekarz na środku ulicy był trochę nie na miejscu.
Ale czy warto było się nim, przejmować? 
Dzisiaj on, jutro ktoś inny.

I wtedy zegar uderzył dzwonem tak mocno, że gdyby ludzie mieli w oknach szyby, to już by ich nie mieli.
Ludzie przystanęli na chwilę, pomacali się po sobie, jakby chcieli sprawdzić czy nadal żyją.
Ktoś sobie zbadał puls.
Po kilku sekundach, rozluźnieni, kontynuowali swoje wędrówki.
Nawet słychać było gdzieniegdzie jakieś śmiechy.
To była ta spokojniejsza część doby.

Grupka osób biegła niespiesznie, jakby chciała zdążyć na pociąg, który i tak ma opóźnienie.
Pierwsza osoba trzymała w ręce zwój papieru.
Do grupki dołączali się też inni, zaciekawieni rozwiązaniem zagadki.
Rodzinka Nocnych skorzystała z okazji.

Tego poranka zegar swoim uderzeniem zabił jakąś prostytutkę z domu publicznego na skraju miasta.
W jej ręce znaleziono ten zwój.
Nagle osunęła się na ziemię, już jej nie dobudzili.
I tak ginął ktoś co ranek.

Przybiegli do domu kowala.
Kowal siedział przed domem i popatrzył się na przybyszów w taki sposób, jak trener patrzy się na swojego zawodnika, który przybiegł na metę ostatni.
\begin{dialogue}
	\ds{} Moja żona umarła dzisiaj rano. \dm{} 
		Nie widział papirusu, a jednak doskonale znał jego treść. \dm{} 
		Została otruta naszyjnikiem z masy solnej, który ktoś jej podarował. \dm{}
		Skupił swój wzrok na właścicielu kartki, tak jak ogniskuje się światło lupy w celu podpalenia czyjegoś domu. \dm{} 
		Na pewno nie piekarz, którego zamordowaliście dzisiaj w nocy. Więc kto?
	\ds{} Ja to zrobiłem. \dm{} 
		Aptekarz wyszedł z domu obok. \dm{} 
		Ta jędza uczyła nasze dzieci, że Zegar jest jakimś rodzajem boga. Że trzeba mu oddawać cześć. Bóg jest tylko jeden, a ten zegar jest jego dokładnym przeciwieństwem! \dm{} 
		Kilka osób mimowolnie popatrzyło na wieżę, złoty okrąg wokół tarczy błysnął, jakby chciał zaprzeczyć jego słowa.
	\ds{} A kowadło? A pudełko? \dm{} 
		dopytywał się tłum.
	\ds{} Kowadełko. \dm{} Aptekarz wskazał palcem na ucho. \dm{} Pudełko. \dm{} Wskazał na swoją głowę. \dm{} I trucizna w środku.
\end{dialogue}
Ludzie czytali z papirusu i milcząco przytakiwali.
\begin{dialogue}
	\ds{} Zaczynają mu budować kapliczki. \dm{} 
		Nakręcał się dalej. \dm{}
		Składają dary na głównym placu, prosząc o łatwą zagadkę. Nie możemy na to pozwolić! To on. To Zegar zabił wszystkie trzy osoby. Waszymi i moimi rękami. Nakręcany diabeł.
	\ds{} Zabiłeś w taki sam sposób. Jak śmiesz nas pouczać? \dm{} 
		ktoś z tłumu wykrzyknął.
	\ds{} Zabiłem, żeby nasze dzieci nie zabijały w imię mechanicznego szatana, śmierć za więcej śmierci. Każdy z was zrobiłby to na moim miejscu.
	\ds{} Sam jesteś szatanem. 
	\ds{} Przynajmniej nie torturowałem piekarza, zrobiłem to humanitarnie, dla dobra nas wszystkich.
	\ds{} Jakiego dobra?
\end{dialogue}
I zaczęli się przepychać.
Nauczona doświadczeniem rodzinka Nocnych powoli się wycofała.
Krzyki było słychać na całej ulicy, ciekawe czy kolejna osoba znowu zginie.

Rafał Nocny umieścił w kwadratowym urządzeniu różaniec, buteleczkę wody święconej i kawałek Jana Pawła II.
Bogofon zatrzeszczał, zawibrował i na ekranie ukazała się trójwymiarowa postać.
Młodzieniec miał jasne włosy, błyszczące w ostrym świetle, oraz białą szatę ze srebrnymi akcentami z pereł i diamentów.
Wyglądał jak doskonały człowiek, może nawet aż za bardzo doskonały.
Za nim znajdowały się rzędy kolorowych kul, wokół których chodzili inni aniołowie.
\begin{dialogue}
	\ds{} Niebiański departament symulacji alternatywnych wersji wszechświata, w czym mogę pomóc? 
		\dm{} Rozległ się anielski głos, niczym grany na trąbach w akompaniamencie tysiącosobowego chóru. Anioł popatrzył się swoimi niebieskimi oczyma wprost na ich dusze.
		\dm{} Och, to wy, hejka. I jak tam wasza ludzka egzystencja, moi dzielni wojownicy uniwersalności? Nasze perpetuum mobile w waszej rakiecie nie zawodzi?
	\ds{} Znaleźliśmy zegar 
		\dm{} wypaliła niespodziewanie Nadzieja.
	\ds{} O, to miło. 
		\dm{} Uśmiechnął się nieco drwiąco, pokazując idealnie proste i białe zęby. 
		\dm{} Na samej Ziemi znajduje się kilka miliardów różnych zegarów, chciałabyś może doprecyzować?
	\ds{} Ale ten zadaje zagadki 
		\dm{} dodała Maria Nocna.
	\ds{} I morduje ludzi 
		\dm{} wspomniał Rafał.
	\ds{} I gada po polsku. 
		\dm{} Żywia się ożywiła.
	\ds{} Hmmm... 
		\dm{} Anioł zmarszczył czoło na którym nie pojawiła się ani jedna zmarszczka.
		\dm{} Strasznie nieuniwersalna ta wasza uniwersalność. I jeszcze ten polski język. Spauzujcie na kilka pulsów, coś mi świta.
\end{dialogue}
Rodzinka popatrzyła się po sobie pytająco. Zza kamery dał się słyszeć dźwięk przerzucanych ksiąg.
\begin{dialogue}
	\ds{} Hitler wygrywa wojnę, to nie.
	\ds{} Ludzie znoszą jaja? To też nie.
	\ds{} Derdenole chomipują fizultanowe bugysty? Listo.
	\ds{} Nasz ulubieniec wygrywa wybory. Może innym razem.
	\ds{} Atomy poruszają się pod wpływem miłości? Nic z tego nie wynikło.
	\ds{} Wymuszone ćwiczenia inteligencji. O.
\end{dialogue}
Na co trafiła rodzina Nocnych tym razem, to nie była surowa uniwersalność, tylko jeden z wyciekłych eksperymentów.
Anioł obrócił swój bogofon na reszki słoju do symulacji wszechświatów.
Jeszcze nie posprzątali do końca po katastrofie roku zerowego.
Uniwersalność jest umieszczana w tych słojach przed rozpoczęciem symulacji, to coś jak komórki macierzyste.
Kiedy główny zbiornik tej surowej losowości wyciekł i zalał wszechświat, trochę normalnych symulacji też popękało.
\begin{dialogue}
	\ds{} Na trop nakierował mnie ten wasz język. W Niebie wielu posługuje się słownictwem Narodu Zapasowego. Dużo bardziej przyszłościowy niż łacina, wiecie? I tak fajnie trzeszczy, odstrasza diabły.
		\dm{} Niebiański urzędnik trochę zboczył z tematu.
		\dm{} W każdym razie... ,,celem tego eksperymentu było zbadanie, jak się żyje w świecie w którym co noc wymuszona jest ścisła współpraca pomiędzy obcymi ludźmi w rozwiązywaniu codziennej zagadki''
		\dm{} czytał.
		\dm{} No tak, zawsze najprościej poprzez wiecznie wiszącą nad nimi groźbę śmierci.
	\ds{} I jaki był wynik? 
		\dm{} Rafał się zapytał.
	\ds{} To ja powinienem się was zapytać.
\end{dialogue}
Rafał odwrócił bogofon w stronę miasta.
\begin{dialogue}
	\ds{} Do stu Judaszy, to się na mnie patrzy...
	\ds{} A no. Na wszystkich się patrzy. I wszystkich może na równi zamordować.
	\ds{} Jak w takim zagadkowym świecie radzą sobie mieszkańcy? 
		\dm{} Dało się słyszeć tarcie pióra o papirus.
	\ds{} Dzisiaj zginęły trzy osoby. Może cztery. Tylko jedna była ofiarą zegara. Odpowiedz sobie na to pytanie.
	\ds{} No a gdyby współpracowaliby ze sobą, zamiast się zabijać, to może rozwiązaliby zagadkę na czas i nikt by nie zginął, co nie?
		\dm{} Chrobotanie zwiększyło wysokość. \dm{} A jak wspaniale przećwiczyliby sobie przy okazji umysły, pomyślcie. Świat geniuszy.
	\ds{} Nie wynaleźli jeszcze elektryczności, a błota jest tu więcej niż twardego gruntu. Jest też jakaś grupa dresów, co robi za zbawicieli i wczoraj zatorturowali niewinnego piekarza na oczach niewzruszonych gapiów w imię rozwiązania zagadki.
	\ds{} Tak. Była możliwość, że tak się będzie działo. \dm{} Anioł popatrzył na wszystkich ludzi z wyrzutem. \dm{} Zapisuję jako nieudany. Jak zwykle. Teraz najlepsza część. Możecie zniszczyć zegar.
	\ds{} Rozwiązując zagadkę?
		\dm{} Żywia miała siostrę w głosie.
	\ds{} Myślałem o bombie sacroteriowej, tej co na raz zniknęłaby cały system gwiezdny. Ale manualnie też można, jak wam się aż tak nudzi.
	\ds{} Mielibyśmy zabić tych wszystkich ludzi?
	\ds{} Nie posiadają dusz, celem ich życia była symulacja. Nie mają potrzeby istnienia. To tylko trochę skonsolidowanej uniwersalności. Wielkie mi halo.
	\ds{} I to niby my jesteśmy tymi złymi?
	\ds{} Ech... 
		\dm{} Westchnął organową muzyką.
		\dm{} Każda z tych kul za moimi plecami posiada septyliardy istnień. 
		\dm{} Nawet się nie odwrócił. 
		\dm{} Nie mówiąc o tej jednej symulacji, która odtwarza nasz aktualny wszechświat. Na prawdę chcesz kolejny raz przez to mentalnie przechodzić?
		\dm{} Przeczytał mu odpowiedź w myśli, że nie.
		\dm{} Zresztą, w tej galaktyce nie ma innego życia, róbcie z nimi co chcecie, dla nas już nie istnieją.
		\dm{} Zatrzasnął księgę. 
		\dm{} Macie przecież wolność, którą tak chętnie sobie zerwaliście z drzewa. Pobawcie się w Boga.
\end{dialogue}

Wychodząc ze ślepej uliczki, Rafał zapytał się przechodnia, czy wie może kto jeszcze poległ w ostatniej kłótni przed domem kowala. 
Okazało się, że zadźgano dwie osoby, a jedną zadeptano. I na koniec kowal powiesił się ze smutku. To było razem siedem osób, dużo.
Kobieta się zaśmiała, dzisiaj i tak wypadło mocno poniżej tygodniowego rekordu.

Wyglądało na to, że to wcale nie zegar był największym mordercą w tym mieście.
Tarcza na wieży przytakująco odbiła światło białej gwiazdy.

Kolejnej nocy odbyło się ponowne zebranie na głównym placu.
Niebieski sierp księżyca świecił teraz znacznie słabiej, dawał na niebie miejsce dla swojego czerwonego przeciwieństwa.
Ta czerwień nie była już łuną krwi, jak poprzednio, była ostrzeżeniem.
Była alarmem na tonącej łodzi podwodnej.
Uwaga, dzisiaj zginie jeszcze więcej osób!

\begin{sl}
\begin{quote}
Posłuchajcie uważnie moje dziatki \\
wybitej o północy zagadki. \\
Przed wschodem słońca znajdźcie rozwiązanie, \\
albo komuś coś się dzisiaj stanie. \\
Ding-dong. \\
Budowniczy wsiąkł. \\
Szukacie, a znajdziecie. \\
Wszystkich jego dzieci. \\
Dokąd sobie poszedł? \\
Wziął ze sobą kalosze. \\
I oto historia jest cała. \\
Znajdźcie tego bałwana. \\
\end{quote}
\end{sl}

Banda dresów nadal stała w miejscu, czekając na to co powie im ich jedyny mózg.
W tej zagadce trzeba było znaleźć jakiegoś budowniczego, który sobie gdzieś poszedł.
Zawołano więc wszystkie rodziny w których byli jacyś konstruktorzy, pytając się, czy grupy są w kompletach.
Jak jeden potwierdzili.

Ktoś podsunął pomysł, że budowniczy nie musiał mieć rodziny, mógł sam wychowywać dziecko.
Więc poproszono, właściwie zagoniono, wszystkie dzieci w miasteczku na główny plac, wraz z rodzicami.
Ale szybko porzucono ten pomysł gdy okazało się, że musiałaby przyjść cała wioska.
A zegar nieubłaganie tykał dalej.

A może symbolicznie chodziło o dzieci jako o zbudowane konstrukcje? 
No ale jak znaleźć budowniczych każdej, nawet najmniejszej rzeczy w mieście?

Trzeba znaleźć kalosze. Jak znajdziemy kalosze, to budowniczy będzie obok nich. Wspaniałe rozwiązanie.
I tak sprzeczali się i sprzeczali.
I nic z tego nie wychodziło.

%%% TODO Pierwszy przelot

Nocni poszukali w mieście jakiegoś muzeum.
Znaleźli jedno, które robiło także za bibliotekę.
Było zatoczone, bowiem każdej nocy wiele osób przychodziło tutaj, próbując wynajdywać swoją własną broń na zegar.
Posiadali tutaj spis wszystkich zagadek i ich rozwiązań od początku istnienia miasta, to jest przez jakieś dwa tysiące lat, od czasu katastrofalnego w skutkach wycieku uniwersalności i uwolnienia się tej symulacji ze słoja.
I jakoś nie widać było, żeby ten świat w jakikolwiek sposób się zmienił. 
Był do bólu statyczny, może dlatego że mieszkańcy co noc tracili czas na walkę z zagadkami zamiast zajmować się nauką i samodoskonaleniem?
Może jeśli wiesz, że każda noc może być twoją ostatnią, to nie ma sensu nic robić?
Zabawne, że przez tyle lat ci ludzie nie potrafili się zjednać.
Ale z drugiej strony, czy jakakolwiek cywilizacja we wszechświecie tego dokonała? Przecież nie.

Rodzice wczytali się w księgi historyczne.
\begin{dialogue}
	\ds{} A może chodzi o prawdziwego bałwana? \dm{} przerwała im znudzona Nadzieja, widząc jak inne dzieci lepią na ulicy struktury z wszechobecnych kawałków rozszarpanych trucheł.
	\ds{} Ale kto konkretnie miałby być budowniczym bałwana, córciu? Każdy mógłby być, zobacz ile tutaj dzieci \dm{} Maria Nocna się zaśmiała.
	\ds{} Nie, bałwan ma być budowniczym.
		\dm{} Głupi dorośli, znowu nie rozumieją.
	\ds{} Bałwany nie budują bałwanów, głuptasku...
	\ds{} Mogą budować jak są duże. I mogą też wsiąkać.
	\ds{} Ale o co ci chodzi?
	\ds{} Chyba wiem o co chodzi Nadziei. 
		\dm{} Tata zamknął kilkusetletni tom. 
		\dm{} Załóżmy tak. Gdyby topiący się bałwan rozpadł się na kilka kawałków, to te odpadłe bryły to byłyby jak jego dzieci. Prawda? A ponieważ jak się topi, to robi pod sobą wielką kałużę i może całkowicie sam zniknąć, zostawiając swoje niestopione fragmenty.
	\ds{} I kalosze! \dm{} Żywia zawołała. \dm{} I miotłę, i garnek, i marchewkę!
	\ds{} Nie wiem, czy na tej planecie rosną marchewki.
	\ds{} Utopiły się w błocie, musimy znaleźć i odbudować bałwana! \dm{} Dzieci wybiegły na ulicę, rodzice za nimi. Tyle razy powtarzali żeby nie biegać po obcych planetach, ale jak do ściany.
\end{dialogue}

Przez resztę nocy dziewczynki szukały truchła wspomnianego bałwana, a ich rodzice szukali dziewczynek.
Dopiero, gdy się rozjaśniło, spotkali się zupełnie przypadkowo nad kałużą, która idealnie odpowiadała ich przewidywaniom.
\begin{dialogue}
	\ds{} Co my mówiliśmy o oddalaniu się w taki sposób?
	\ds{} A gdyby was ta banda dopadła? Pomyśleliście o tym? \dm{} Rodzice robili to, co zwykle.
	\ds{} Nie ma czasu, trzeba rozwiązać zagadkę. \dm{} Żywia próbowała się tłumaczyć.
	\ds{} A gdybyście nas nie znaleźli, to co byście zrobiły na tej obcej planecie?
	\ds{} Nadzieja, zobacz, tu wystaje garnek.
	\ds{} Byliśmy na tylu niebezpiecznych światach a wy się niczego nie nauczyliście.
	\ds{} Nie mogę dosięgnąć, ty spróbuj, masz dłuższą rękę.
	\ds{} Możecie zapomnieć o wizycie w Pałacu Nadiru, powiemy królowi Freonowi, jak się zachowywałyście.
	\ds{} A patykiem?
	\ds{} I nie dostaniecie już więcej lodów od niego.
	\ds{} Czekaj, coś czuję.
	\ds{} Ferro już nie będzie z wami latał w...
\end{dialogue}

Uderzenie dzwonu zatrzęsło światem.
Koniec czasu wszystkich uciszył.
Rodzice spojrzeli na swoje pociechy, a potem na kałużę otoczoną kilkoma białymi grudkami.
Rafał zanurzył rękę w błocie.
Wyciągnął parę brudnych kaloszy.
Westchnął.
Niedługo to trwało, aż przybiegła do nich jakaś obca osoba. W ręku trzymała zwój papieru.
Studiowała długi czas to miejsce, patrząc to na rodzinę, to na kałużę, to na ubłocone kalosze, to na swój zwój.
Potem gdzieś pobiegła.
A nasza rodzinka, nie mając ochoty na bycie celebrytami, ulotniła się, jak to dobrze już wcześniej miała wyćwiczone.

Nauczeni porannym doświadczeniem, Nocni wzięli ze swojej rakiety worek nanobotów i rozesłali po całym mieście, aby zmapowały im teren. Każdą kałużę, każe zagłębienie miało być wpisane do pokładowego komputera.
Zegar także próbowały zbadać, ale nie znalazły w nim jakiegokolwiek otworu. 
Wieża okazała się całkowicie zamkniętą bryłą.
Na tę zabawę zszedł im cały dzień.

Zobaczymy, jak nędzna mechaniczność poradzi sobie z doskonałą formą elektroniki.
Teraz Nocni będą wiedzieć wszystko i będą jednocześnie wszędzie.
Żadne sztuczki z szukaniem igieł w stogach siana, albo kaloszy w kałużach błota nie przejdą.

Trzecia noc była tak nieprzyjemna, jak iloczyn dwóch poprzednich.
Gęste chmury nie przepuszczały żadnego światła, nawet tej wściekłej żarówki na niebie.
Jedynie żółta tarcza zegara delikatnie oświetlała miasto.
Teraz to ja jestem waszym słońcem.

Deszcz padał, błotne ulice zamieniły się w potoki.
Mglista firana zasłoniła miasto, niczym pająk oplotujący siecią swoją ofiarę.
Prawie wszystkie nanoboty ugrzęzły w bagnie lub zostały spłukane w żarłocznym, wszechobecnym cieniu.
Nie zemną takie numery.

Czas się zbliżał.
Drgające połyski okręgu zdawały się wskazywać, że rechocze.
Nie było widać zarysów wieży, była tylko świecąca tarcza, unosząca się nad ludźmi niczym ich nowa i jedyna gwiazda.
Na zawsze.
Mieszkańcy stali na głównym placu jak co noc, przyklejeni do deszczowej pajęczyny, i nie spodziewali się, że i tym razem będzie jakkolwiek inaczej.
Legenda o rozwiązanej zagadce zgasła jak świeczka rzucona na ulicę. 
Zgasła, połknięta przez horyzont zdarzeń cienia.
Nikt nie uwierzył, nikt nie miał nadziei.
Nawet Rafał Nocny.

Zegar zadzwonił jak co nocy.

\begin{sl}
\begin{quote}
Posłuchajcie uważnie moje dziatki \\
wybitej o północy zagadki. \\
Przed wschodem słońca znajdźcie rozwiązanie, \\
albo komuś coś się dzisiaj stanie. \\
Dzyń-dzoń. \\
Nie dostrzeżecie go. \\
Jest pewien byt, \\
co wszystkich morduje w mig. \\
I to nie jestem ja. \\
Chociaż reszta się zgadza. \\
Dzwoni dzwonkami. \\
Zagadki rozdaje. \\
\end{quote}
\end{sl}

\begin{dialogue}
	\ds{} Inni ludzie! \dm{} inni ludzie powiedzieli to jednocześnie.
	\ds{} Bóg? \dm{} ktoś zaproponował.
	\ds{} Inny zegar?
	\ds{} Strach?
	\ds{} Wiatr.
	\ds{} A my sami?
\end{dialogue}
Ale kolejne pomysły grzęzły w ludzkiej pajęczynie bezsensowności coraz bardziej.
Ktoś zrezygnowanie usiadł w błocie.
Ktoś zemdlał z beznadziejności i się utopił.
Ktoś celowo sobie żyły podciął.
Widać śmierć mogła przyjść nie tylko od innych ludzi, czy zegara, ale także od ciebie samego.

I musieli wszyscy tak stać.
I nic nie mogli zrobić.
I nawet jak próbowali, to nie mogli.
Bo wszyscy byli splątani pajęczyną, która sklejała ich mózgi.
A oko wielkiego pająka bacznie się przyglądało.
Nie pozwolę wam umrzeć, jeszcze nie, cierpcie.

Tylko Nocni potrafili się z tej pajęczyny zerwać.
Nie dlatego, że mieli doświadczenie, inteligencję, czy chęć do życia.
A dlatego że byli oszustami.
Bo za rogiem czekała na nich rakieta i kilka sacroteriowych bomb.
Bo ich historia w każdej chwili mogła się zakończyć widowiskowym \emph{deus ex machina}.
Oni jedyni byli w stanie po prostu sobie pójść i zostawić wszystko w cholerę.

I między innymi dlatego postanowili zostać i się nie poddawać.
Bo jakby poszli, to dzisiejsza pajęczyna zmieniłaby się w jutrzejsze skamieliny.
A zegar wkrótce zostałby sam, w wymordowanym mieście, i sam sobie zadawał zagadki.

Nawet nie było widać, kiedy przyszedł poranek.
Z cebra nie przestawało się lać.
Tylko wieża pokazywała poranną godzinę. Albo chciała nam ją pokazać.

Gdy każdy zwymiotował wszystkie swoje pomysły i u nikogo nie było już kwasu żołądko...
\begin{dialogue}
	\ds{} ...pisarz opowiadania.
\end{dialogue}
Wszystkie oczy zwróciły się na małą Żywię, która była zaskoczona tak samo, jak oni.
\begin{dialogue}
	\ds{} No bo to pisarz to pisze zagadki i dzwoni dzwonem, jak pisze ,,bim-bom''.
\end{dialogue}
Ludzie trochę się zaśmiali, trochę przez łzy, a trochę przez deszcz.
Jednak dopiero po minucie zauważyli, że nie ma nad nimi ich mechanicznego słońca.
Rozglądali się, szukając tak znienawidzonej przez wszystkich tarczy, lecz niebo było całkowicie czarne.
Lekkie buczenie coraz bardziej zagłuszało szum deszczu.
\begin{dialogue}
	\ds{} Wszyscy uciekać! \dm{} krzyknął Rafał, ciągnąc za sobą swoją rodzinę.
\end{dialogue}
Nie wiadomo dlaczego, ale inni ludzie również zaczęli w pośpiechu opuszczać plac.
Potykali się o swoje nogi.
Gubili kalosze w kałużach.
Nawoływali się nawzajem.
Próbowali świecić pochodniami, które natychmiast gasły.

Wtem wielkie uderzenie i fala błota znikąd.  
Porwała wszystkich biegnących.
Rzuciła w wąskie uliczki, wpadła do domów, gasząc latarnie, rozerwała pajęczynę na strzępy.
Przytłumiony dźwięk dzwonu rozszedł się, po raz pierwszy, po ziemi.
I tym razem nikogo nie zabił.


















